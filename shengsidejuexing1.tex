%!TEX TS-program = xelatex
%! TEX encoding = UTF-8 Unicode

%========================================全文布局
\documentclass[twoside,openany]{book}
\usepackage[screen,paperheight=14.4cm,paperwidth=10.8cm,
left=2mm,right=2mm,top=2mm,bottom=5mm]{geometry}

\usepackage[]{microtype}
\usepackage{graphicx}
\usepackage{amssymb,amsmath}
\usepackage{booktabs}
\usepackage{titletoc}
\usepackage{titlesec}
\usepackage{tikz}
\usepackage{enumerate}
\usepackage{wallpaper}
\usepackage{indentfirst}
%========================================设置字体
\usepackage{ctex}
%\usepackage[CJKnumber]{xeCJK}
\usepackage{xpinyin}
\setCJKmainfont[BoldFont={Adobe Heiti Std R}]{Hiragino Sans GB W3}
\setCJKfamilyfont{kai}{Adobe Kaiti Std R}
\setCJKfamilyfont{hei}{Adobe Heiti Std R}
\setCJKfamilyfont{fsong}{Adobe Fangsong Std R}

\newcommand{\kai}[1]{{\CJKfamily{kai}#1}}
\newcommand{\hei}[1]{{\CJKfamily{hei}#1}}
\newcommand{\fsong}[1]{{\CJKfamily{fsong}#1}}

\renewcommand\contentsname{目~录~}
\renewcommand\listfigurename{图~列~表~}
\renewcommand\listtablename{表~目~录~}
\usepackage{romannum}
%========================================章节样式
\titlecontents{chapter}
[0em]
{}
{\large\CJKfamily{hei}{}}
{}{\dotfill\contentspage}%用点填充
%
\titlecontents{section}
[4em]
{}
{\thecontentslabel\quad}
{}{\titlerule*{.}\contentspage}

\titleformat{\chapter}[display]
	{\CJKfamily{fsong}\large\centering}
	{\titlerule[1pt]%
	 \filleft%
	}
	{-7ex}
	{\Huge
	 \filright}
	[{\titlerule[1pt]}]

%========================================设置目录
\usepackage[setpagesize=false,
            linkcolor=black,
            colorlinks, %注释掉此项则交叉引用为彩色边框(将colorlinks和pdfborder同时注释掉)
            pdfborder=001   %注释掉此项则交叉引用为彩色边框
            ]{hyperref}

\setlength{\parindent}{2em} %首行缩进
\linespread{1.2}              %行距
\setlength{\parskip}{15pt}    %段距

%========================================页眉页脚
\usepackage{fancyhdr}
\pagestyle{fancy}
\fancyhf{}
\fancyfoot{}
\fancyfoot[LE,RO]{\scriptsize\thepage}
\setlength{\footskip}{6pt}
%========================================标题作者
\title{生死的觉醒(上)}
\author{高桥信次\ 著\\ \xpinyin*{慰宣}\ 译}
\date{}
\newcommand{\mt}[1]{\textbullet \textbf{#1}}
%========================================正文
\begin{document}
\TileSquareWallPaper{1}{TGTamber}%背景图片
\pagenumbering{arabic}
\maketitle
\tableofcontents
%\newpage

\zihao{4}
\noindent
\chapter*{封面语}\addcontentsline{toc}{chapter}{\large\CJKfamily{hei}封面语}
全集传记故事,读来令人如面观如来一般,对于释迦牟尼佛成道之由来,从出家、求道、悟道、传道等心路历程,栩栩如生。不仅让现代人了解佛陀亦是与凡人一般有血肉之躯,而能劝化众生本身亦具佛性。

佛陀教导我们如何与人相处,如何过合理的生活。惟有万世不移的法理,方能给予我们真正的安宁与平静。读者有缘阅读本书---必可同沾佛陀慈悲之丰润,和平与安祥之情操油然而生。

\chapter*{作者简介}\addcontentsline{toc}{chapter}{\large\CJKfamily{hei}作者简介}
作者高桥信次先生自幼即经历多次灵的体验,于电子工学、物理、天文、医学等皆有涉猎……对人类心灵与肉体的关系有精辟独特的理解,经一再深入探究的结果,发现了灵魂的转生轮回一事。

著作:《般若心经的真义》、《心的原点》、《心的指针》、《超越的爱》、《恶灵---破除躁郁症》、《心灵的对话》、《缘生之舟\Romannum{1}神理篇》、《缘生之舟\Romannum{2}科学篇》、《缘生之舟\Romannum{3}现证篇》。

高桥信次大师生于长野县,如他事前的预言般,于1976年6月蒙佛接引,往生极乐。

\chapter*{原序}\addcontentsline{toc}{chapter}{\large\CJKfamily{hei}原序}
老实说,我所看过的小说、传记和传奇等,简直少得可以数出来,而且对于故事中情节的安排及人物的描绘,可以说是一窍不通,据我所知,大凡一部著作的产生,都要经过一段漫长的酝酿时期,亦即作者除了要有累积的人生体验外,还要能做精密的构思,以及勤奋的搜集工作,此外,还必须具备一枝生花妙笔。

我在此要敬告各位读者的,则是我写作这本书,几乎把上述的这些过程都省略了,说来也许你们不信,这本书全是凭着一种不可思议的灵的启示而完成的,或者你可去问那些学过佛理或读过圣经的人,他必告诉你这本书解开了他对宗教的疑窦,帮助他了解经典中艰涩冷奥的词义,在我身边并有许多亲身经历过转生轮回且因此而悟道的人,他们能对我文中所述及的一切道理加以印证,在华严经十地品、新约圣经使徒行传第二章中,记载有关弟子赞叹教主以及自身表现的奇迹,直至今天,所谓的「奇迹」仍层出不穷,基于这个因素,更坚定了我著作本书的意念。

现在市面上所看到的释迦牟尼佛传,多半是描述佛陀和弟子们之间的事情,以及他度化众生的种种,对于他自出家到悟道的一段遭遇和心路历程,则付之\xpinyin*{阙}如。

佛理不是光用头脑和眼睛就可以想见的,必须用自己的身与心来体验个中的道理,日久之后,一些往日百思不得其解的道理自然能在心中豁然开朗,而一切的疑惑也能在瞬息间烟消云散,佛陀当初在怎样的情况下使自己达于这种不可名状的境界,引起了我莫大的兴趣。

本书的用语力求浅显,譬如「诸法无我」,一般的解释是:「宇宙间的万事万物,皆是缘聚而生,缘散而灭,没有一个『自我实体』的存在。」我把「法」解释为「秩序」。我们纵观大自然的井然秩序,四时的推移,星体的运转等,可以从中获得一些启示。

我们今天所面临的能源危机、空气污染等问题,难道不是因为人类自己已违背了大自然的法则,破坏了大自然的秩序所遭到的一种后果吗?这也就是「秩序」中包含着可贵的个体意志,而不是个体的\xpinyin*{恣}意任性。简单地说,「诸法无我」,就是一种不偏不倚的中道。

「佛陀」者,由梵文音译而来,即「觉者」也,我们尊称释迦牟尼为「佛陀」,因为他是人世间一位彻悟生死,洞悉人生真相的大智者,以其完美的人格与言行被世人尊奉为神明,时至今日,人们忘记了他也曾是人的事实,甚至有人说,释迦牟尼佛又不是人,怎能了解到我们「人」的烦恼?他能给我们人类什么样的启示呢?可别忘了,释迦牟尼佛曾以其人身,抛弃了荣华富贵与天伦之乐,经历了许多修炼的历程,忍受了许多身心上的煎熬与痛苦,我们不可抹煞这些事实。

总之,我想告诉读者的是,本书绝不同于时下一般流传的佛陀传记,希望读者读完本书后,能将佛陀的启示发挥到日常生活中,这才是我著作本书最大的心愿。


\chapter*{译序}\addcontentsline{toc}{chapter}{\large\CJKfamily{hei}译序}
宗教不应造成隔阂,因为宗教本同源,我想这是本书原作者高桥信次先生著作本书的最大宗旨。

有了以上的认识,当您看到释迦牟尼佛在定中曾与阿蒙(耶稣基督的前身)谈话,亦曾与摩西见面的这类情节,就不会感到诧异了。

当我译作本书时,曾一度很感惶惑,以为本书只是日本坊间的一类随兴之作,差一点就此搁笔停译了。可是,我又怀着好奇的心情想一见分晓,渐渐地,整本书的脉络分明了,以致我更加喜爱本书,曾日以继夜地玩味及翻译,虽倍尝译程的艰辛,可是意念却非常的虔诚,亦曾参阅引据佛教经典,并多次请益法师,不敢稍有踰越。

本书的奇特,在于揭示了许多新观念,打破佛教思想上固有的框子,作者用极浅显易懂的笔调带领读者接近真理,令人读来亲切可喜,惟书中或有伏笔,或有无稽处,但是请不要\xpinyin*{遽}下断语,不妨继续的追踪下去,就像笔者译作时,虽曾数度停笔深思,终究会有收获的。

作者虽藉释迦牟尼成佛的过程写成此书,却未曾打着任何宗教的旗帜来传教,关于这一点,读者自己可以体会到,作者旨在凭着一己的超觉能力(见作者简介)把人生的价值、生死的真相以及宇宙的本源等,以一般人最易接受的方式阐述出来,相信本书对迫切想探讨人生问题的人士必有莫大的助益。

的确,佛陀未曾以「教主」自居,因为他本就肯定人人都能成佛,他只是为我们指引一条道路罢了。他常对奉他如神明的弟子们说:「我亦是大众中的一分子,莫令我异于众人」。

在本国佛教改制的议论声中,本书正可帮助读者诸君了解日本人对佛理的诠释理路,对于改制与否,您或许也可以表示一点意见!容我卖个关子,请诸君先行阅读全文,再请批评赐教译者于后记中做成的几点讨论。

\begin{flushright}
	识于永和宅中\hspace{.5cm}

慰萱\hspace{1.5cm}

  六十六年十一月十日
\end{flushright}

\chapter*{再版序言}\addcontentsline{toc}{chapter}{\large\CJKfamily{hei}再版序言}
《生死的觉醒》一书具有涤除心垢和提升精神领域的作用,经由读者的热烈反应而获得了证实。

读者们除自己购阅外,或辗转推介友人购阅,或大批购置以为赠书,足以说明本书真正的价值所在,更蒙有心人士之不弃,不惮烦琐地就书中的某一问题与译者做书面讨论,这种对真理热切期盼与追求的心情,真教人感动不已。

在日本,高桥信次先生一系列的心灵类书籍被图书馆协会列为各级学校的图书馆藏书,并将之视为不可多得的品格教育的教学资材,市面上心灵方面的著作很多,为何独独高桥先生的书能获得此一殊荣?因为高桥先生在中和的言行中让人接触到真理的精髓,也可说他本身就是真理的显现,自然说出来的话就格外具有影响力。

或有人说高桥先生虽具有超人的灵力,却未能以文字将体悟的境界描述周详,以致文中有为人诟病之处。今蒙数位佛界\xpinyin*{耆宿}的指正,译者谨在此列出三点,以供读者参考定夺:

一、本书将佛陀的境界列同梵天的层次,是一种错误。「佛陀」是超乎天界的一种存在。佛陀经常在天界说法化渡人,即是最好的证明。

二、文中有关梵天斥责佛陀的一段叙述,显出错倒的现象。依据佛经记载,天众当时是劝请佛陀住世说法的,其景象才更合乎调和之道。

三、佛陀入涅槃不能视如自杀,两者在意义与境界上有天壤之别。

以上三点,参见原书第一章第十七节。该节文字固然与佛学的一贯道理相悖,然并不影响作者在阐述灵性不灭这一点上的真实性,诚如高桥先生在《心灵的对话》一书中所说:「关于另一世界的境界,我若不以大家现有的概念来引申,则将是无法想象的,甚或是不能存在的。由于此一事例,望读者在阅读本书之际,当以更自由的心灵来接触书中的道理,既不一定要肯定它,也不一定要否定它,因肯定与否定恰是两个极端,都有偏执之嫌,而偏执正是探讨真理的障道,存疑方是治学的基本态度。」

本书自出版以来业已引起读者对人生问题的研究与兴趣,尔后第二、三、四集亦将陆续出版,希望读者能继续本着爱顾的热忱,不吝赐下高见,则于真理,必能收越辩越明之效。

\begin{flushright}
慰萱\hspace{1.5cm}

六十七年五月八日	
\end{flushright}






%
%目录生死的觉醒(上集)
%
%原序
%
%一本奇特的书——译序	
%
%再版序言	
%
%第一章成佛之道
%
%一、从诞生到出家	
%
%二、探访名师.	
%
%三、五位比丘	
%
%四、净饭王与释迦族的悲叹	
%
%五、一个妇人的布施	
%
%六、遍涉疑问与答案	
%
%七、暗中摸索	
%
%八、迈向开悟之道	
%九、
%九、一口牛乳		
%
%十、内心的搏斗	
%
%十一、光明在望——发现心中量度的尺
%
%十二、除去心中的阴霾——童年与出家
%
%十三、出家后	
%
%十四、梦幻的世界	
%
%十五、跟恶魔决斗	
%
%十六、伟大的时刻	
%
%十七、与梵天的一席谈	
%
%十八、神游梵天界	
%
%十九、布法之旅	
%
%第二章 五位阿罗汉
%
%一、观自在力
%
%二、向鹿野苑进发	
%
%三、怪异的景象——修道场上的妖气
%
%四、最早的弟子		
%
%五、启蒙与光明
%
%六、涤除心垢,迈向阿罗汉之道
%
%第三章佛陀和弟子们
%
%一、耶萨的苦恼
%
%二、传道之旅	
%
%三、治病	
%
%四、拜火	
%
%五、迦叶尊者归投佛门
%
%六、三宝
%
%七、僧团的形成
%
%八、出家必备的条件	
%
%十、婆罗门僧来寻衅	
%十一、
%十二、十字架上的爱	
%十三、
%十一、奉献竹林精舍	
%
%十二、布施的功德	
%
%十三、迦毗罗卫国的期盼	
%
%十四、机缘		
%
%十五、皈依	
%
%十六、盂兰盆会和供养	
%
%十七、舍利弗与目犍连	
%
%用语诠释
%
%后记	请续阅生死的觉醒下集







\chapter{第一章\ 成佛之道}\label{ch1}
纪元前六五四年的印度,是由十六个大小不同的王国所组成。王国间常为己利彼此征战不休。其中的\xpinyin*{迦毗\xpinyin{罗}{luo2}卫国},是由一位释迦族的净饭王所统治。

\xpinyin*{释迦牟尼\xpinyin{佛}{fo2}},在未成佛之前在这样的环境中诞生,称为悉达多太子。
	
\newpage
\section{从诞生到出家}\label{sec1.1}

\mt{诞生的背景}

纪元前六五四年的印度,以中印度为中心,由于贵族、武将的权势很大,约有十六个大小国家形成了经年争霸不休的混乱局面。释迦牟尼\textperiodcentered 悉达多,释迦牟尼佛在这样的环境中诞生了。

身为迦毗罗卫国王子的悉达多,会\xpinyin*{唾}弃王位,抛开享乐,不是没有原因的,迦毗罗卫国,是\xpinyin*{憍萨罗国}的属国,由憍萨罗国来看,迦毗罗卫国有如一个城寨。

释迦牟尼\textperiodcentered 悉达多太子降生在摩耶夫人回娘家的途中,照当时印度的习俗,女子多半返回娘家生产,而摩耶夫人也依俗回她位于喜马拉雅山\xpinyin*{麓}的娘家拘利族的天臂城,她是该城城主喜觉大王的妹妹,岂料行经\xpinyin*{蓝毗尼园}时,夫人即感周身不适,就在休\xpinyin*{憩}时生下了一个男婴。虽然迦毗罗卫国上下为这一件喜事通宵达旦地狂欢庆祝,但王后摩耶夫人却因产后体虚,而于太子降生后的第七天\xpinyin*{溘}然长逝,净饭王一下子\xpinyin*{堕}入了深沉的悲哀中,此时,只有他最能领会得失之间的无奈。

摩耶夫人的妹妹\xpinyin*{摩诃波\xpinyin{阇}{she2}波提},当初为服侍待产的姐姐而来到迦毗罗卫国,如今义不容辞地留在宫中代姐姐抚育幼儿,也顺理成章地成为太子的继母。

\mt{阶级森严的种姓制度}

三千年前的印度社会,由于种族的尊卑,形成了征服者和被征服者的关系,严格划分阶级的种姓制度,事实上就是征服者所想出来的愚民政策,以控制大多数被征服者的思想和行为,制定这套制度的当然就是自居第一阶级的婆罗门,他们把人分成四等:

第一等是婆罗门,亦即婆罗门教的传教士,是知识阶层,自称由他们所崇奉的「梵天王」的嘴巴转化而来,所以他们负有用嘴说教的责任,以领导下面三个阶层。

第二等是\xpinyin*{刹帝利},亦即王家贵族,是统治阶层。

第三等是\xpinyin*{吠舍},是农工商阶层。

第四等是首陀罗,亦即奴隶,是被征服者,注定要为上面三个阶层劳碌奔波,是最下等的人,限制也特别多,过的是非人的生活。

各阶级间严格地划定了界限,在物质生活上有极不平等的待遇,即使在行住坐卧间,也不可稍有逾越,当然更不能互通婚姻了,不管你是多么有才干,只要你是奴隶,就世世代代都不得翻身,也不能受教育,求知与传道是上层阶级的事,即使连吠舍这一阶层的人,也在限制之列。

偏偏那第三第四等人在社会中占了绝大多数,所以当时的印度,潜伏着很严重的社会问题,到处充斥着悲观与厌世的人,再不然就是一些今朝有酒今朝醉的享乐分子,企图以酒色来麻\xpinyin*{痹}痛苦的感觉,伟大的宗教发源于此,不是没有原因的。

\mt{悲天悯人的太子}

很多人不明白,为什么悉达多太子过的是人人所钦羡的皇宫生活,竟还会有厌世的心理呢?他深得父王与母后的宠爱,住的是琼楼玉宇,吃的是山珍海味,穿的是绫罗绸缎,出游时,护卫侍从,前呼后拥,烦闷时,自有乐师奏起悠扬的音乐,宫女摆起美妙的舞姿,因天资聪颖,很早就已博览群籍,深通义理。又体力超群,精通武艺,有过一箭射穿七鼓的辉煌纪录,世人难以企及的福与慧,他都兼备了,他还希冀什么呢?

奇怪的是,他本身这些优\xpinyin*{渥}的条件,似乎就是他引以为忧的最大源头,在他那英挺俊秀的脸庞上,总有一层抹不掉的淡淡忧郁,没有人了解他心中期盼着什么,就如没有人知道悉达多太子竟对他在\xpinyin*{襁褓}中失去的母亲非常思念,照理说,他对这位母亲既不复记忆其容貌,也谈不上有什么深厚的母子之情,也许就是一种天性上的联系所使然吧,他年龄愈长,思念愈切,常想:「母亲因生我而遭不幸,我也何其不幸,不能承欢膝下,不能报答她的生育之恩,人死不能复生,这是多么无可奈何的事啊!」

他甚至不断地思索着一些不可思议的问题,诸如是否有再见母亲的机会?人如何与死去的亲友接触?

再回过头来看看太子自己所处的时代与环境。

迦毗罗卫国的四周,几乎每天都扮演着「争城以战,杀人盈野」的惨剧,到处是弱肉强食的凄惨景象,他很清楚,现在迦毗罗卫国虽然因防卫森严,国内暂时安定繁华,可是邻国憍萨罗国正虎视眈眈,随时会趁虚来袭。他身为一国的太子,在政治的紧张情势下,也许侍从中就有混入的间谍随时\xpinyin{俟}{si4}机加害于他也未可知,所以他吃的东西要先行经过试食,他走到哪里,都有侍从紧随护卫,每天都必须忍受着这种\xpinyin*{桎梏}般的生活,他对生的真相与生的束缚,有了深一层的体认和感受,也因此悲哀和痛苦不断地滋扰着他的心。

\mt{生命的觉醒}

由于母后摩耶夫人的死,在年轻的悉达多太子心中产生了很大的激荡,他不时反问自己:「人为什么要生?人又为什么会死?」

出外郊游时,无意中\xpinyin*{瞥}见拄杖而行的\xpinyin*{偻佝}老人和满身烂\xpinyin*{疮},痛苦呻吟的病人,自此以后,他的脑中又时常盘旋着另一些问题:「人为什么会老?为什么会病?」

从今天心理学的观点来看,悉达多太子是个非常敏感的人,他对一般人司空见惯而毫不以为怪的现象,会沉思竟日,茶不思饭不想的,尤有甚者,他看到社会上四种阶级间的不平等待遇,更不断在心中打着问号,诸如:

「同样是人,为什么有尊卑的区别?同样有喜怒哀乐的情绪,为什么有的人长年生活在痛苦与悲哀中?」

他远离宫中的\xpinyin*{烦嚣},终日\xpinyin*{盘桓}在幽静的花园或树林深处沉思,他越是处身于享乐的环境中,就越感到不安,盯视着陶醉在歌舞中的女子,环绕在身边的仆役侍从,一种莫名的空虚感袭上心头,他当下感觉到眼前的一切荣华富贵,其本质不过是个空,愚眛的人们,为了争取它,付出的代价是什么?到头来又得到了什么?

\mt{疑惧不安的父王}

净饭王早已注意到悉达多太子的情绪变化,他所疑惧的事实,显然有了征兆,原来早先净饭王庆获传人时,一位有名望的婆罗门阿私陀仙人曾来宫中预言:「太子的诞生是人类的一件大事,他将为谋人类的幸福而离开王宫,他将为苦难的众生宣说人生大道,这是可喜可贺的事,请陛下不要难过,太子将来就是继承王位,做了一国之王,或各国的盟主,其意义也远不及做一位全天下人的『转轮圣王』啊!」

净饭王听了此话,一则以喜,一则以忧。喜的是,有子不凡,与有荣焉;忧的是,太子离宫,他将痛失爱子,而迦毗罗卫国将痛失英明的领导者。出家修行意味着抛弃一切人间的享乐,而独自过着清苦禁欲的日子。他的爱子有朝一日要走这条路径,是他无论如何难以面对的事实,如今他接触到太子不同寻常的言行,见他郁郁寡欢地在宫中摒弃一切享乐,为此非常着慌,整日坐立难安,到处访求可安定太子的良方,他为太子布置了寒、暑、温三时宫殿,精选了许多宫娥和美女陪侍左右。

「请尽快为太子选一位妃子吧!妻子儿女可以束缚一个男人的心。」

净饭王接受了这个建议,因此美丽而高贵的天臂城公主耶输陀罗被娶进宫来。当时悉达多太子十九岁,太子虽很感谢父王的关爱之情,也很欣赏妻子的美丽与娴淑,但是他深知越是美好的东西,越会束缚一个人的心,美丽的妻子将造成他心里的负荷,因为他知道,人的欲望犹如无底的深渊,而人若一任欲望的摆布,势将迷失自己。因此,悉达多太子益发不快乐了。

他变得更喜孤独,几乎每晚都避到地窖中沉思冥想,他想着人类的四大问题,也就是说,他要寻求人类生老病死的原因及解脱之道。

\mt{胜败之间}

十年的岁月匆匆而过,二十九岁的悉达多太子,对于国事更关切,也做了更深入的了解,他协助父王处理许多政事,深得父王的赞赏。唯其如此,他为迦毗罗卫国的安危情势更担忧了,眼见许多国家为了巩固自己的势力,耍出了各种手段,政治联姻是其中之一。问题就在于即使是政治联姻,也仍难确保一个国家的安定与生存,更何况迦毗罗卫国只是一个小国呢?他就这样一直生活在不安与焦虑中,苦苦思索着解决之道。

有一天,净饭王实在忍不住了,就对他说:「你也是一个快当父亲的人了,为什么还这么闷闷不乐呢?你若离开迦毗罗卫国,我怎么办呢?你就是不替我想,也要替全国的老百姓想,你既是我的继承人,就要像个继承人的样子为国家做事。」

净饭王的意思,就是要悉达多太子不要老是想着自己的事,净饭王的一番话固然很有道理,但是他哪里知道悉达多太子一直想着的,正是天下人的事情。

当然,一个国家受到外侮,若不奋力抵抗,一旦战败了,全国的人即将沦为奴隶,并将\xpinyin*{贻}祸子孙,同时为了维持国内的安宁,还必须不让敌国有可乘之机,这些都是一国的领袖所要肩负的责任,做父王的,从这个角度来看,就觉得身为王位继承人的悉达多太子,未免太怠忽职守了,父王的这番道理,太子当然都明白,只是,他想得更多。

固然敌人来攻城,是件令人痛恨的事,但是两方交战造成的死伤,不论是敌我,这都是人类的损失,在陈尸遍野的战场上,你不会因为这是敌人的死尸就不悲哀了,同样是人,为何一定要争个你死我活?这些人无非是少数野心家手下的牺牲者罢了,而那些统治者的野心又是因何而起的呢?怎样杜绝战祸,才是一个为政者应考虑的根本之计,再说,「胜败乃兵家常事」,今天打胜仗了,并不表示明天一定胜利,这么说来,\xpinyin*{汲汲}于胜利的努力与工夫,都是很虚无飘渺的,它们不能保证你一生的胜利与安稳。一个人若没入了胜败的漩涡中,就注定不会有真正的平安,你的心灵也不复平静了,又因为身陷胜败漩涡中,就「胜」来说,不难产生胜则骄,骄则穷兵黩武,继则恃武轻敌,以致溃败沦亡,同理,败则怨,怨则俟机报复。因为这样,胜与败循环不已,战乱与死亡则连绵不绝。

悉达多太子下一个结论:「我即使保得迦毗罗卫国于一时,却难保它于永远,我要设法寻出根本的解决之道!」

\mt{连夜奔离}

由于父王丝毫未能触及他心中烦恼的心弦,悉达多太子感到无比的寂寞与孤独,悉达多太子面现忧凄的说:「请父王谅察我的苦心,我一心想解除的是人类根本的烦恼,您若能对我保证三件事情,我就不再想出家的事。」

净饭王见状,急切地回答说:「儿啊!只要你不离宫,什么事我都可以答应你。」

「这三件事是人绝对不生病,绝对不年老,也绝对不会死。父王啊,您能对儿保证这三件事吗?」

净饭王听后,\xpinyin*{瞠}目结舌,低下头来沉吟了良久,他很困扰,世上哪有绝对的事情,即使一个国王,也有不自由的时候,他觉得太子的这三个要求太荒谬了。于是他嘱咐摩诃波阇波提夫人来说服太子,希望他打消这种怪\xpinyin*{诞}的念头,但是悉达多太子心意已定,犹如磐石,丝毫不为所动,最后净饭王想出了一个下下之策,干脆把他软禁起来,不准他离城一步,还加派了侍卫。

悉达多太子只是直觉到,要想解决\xpinyin*{萦}绕脑际的问题,唯有\xpinyin*{摒}除一切享乐,包括父母妻子的天伦之乐,否则他将一无所获,至于如何离城,以及离城何往,倒是并未做过\xpinyin*{缜}密周详的计划,一天晚宴上,他照例兴味索然地坐在宝座上,眼前晃动着五颜六色的绫罗彩缎,一群妖冶艳丽的宫女正故作娇态地扭动着腰肢,耳畔不时传来杯瓶的碰触声和男士们酒后纵情的调笑声。

「这一切有什么意义呢?」他不禁自问。

就在此时,他\xpinyin*{憬}悟到,不宜再拖延了,他惊觉到这会儿是离城的好机会,眼见宫廷上下因晚宴的纵乐狂饮,都已陆续沉在睡梦中了。他看到那些疲倦不堪而倒卧于地的舞女们,纵横杂陈,睡姿狼狈,因热舞而沿着面\xpinyin*{颊}流下的汗水,在敷满白粉的面庞上划下了一条条的细纹,似乎预现了老人的斑驳皱纹。

「再美的女子,隐藏在背后的,仍免不了是老丑与死亡。」太子叹息道。

他回到后宫,向沉睡中的亲人一一告别,注视他们,心中涌起无限感慨,因为他对骨肉亲情的感受很深,才触发他对世事无常的觉醒,他不得不\xpinyin*{暂}时忍痛离去,他反身急速地穿越长廊,来到马厩,命车匿备马。

「快,快把马牵出来!」

车匿因为谨守净饭王的吩咐,本不想听从,但是太子的声音中透着无可抗拒的威严,使他忙不迭地牵出马来,并依言开启城门,车匿也跟着骑了一匹马随侍在侧,两人于是快马加鞭,在黑夜中,直往城外奔去,一路来到离城一百公尺处,悉达多太子陡地勒住马缰,他掉转马头,依依不舍地望着星光下的迦毗罗卫城。城堡此时静静地躺在夜幕中,城中高低的屋宇,与星光辉映,明灭着慈和的光,那里是他生活了二十九年的地方,一旦离开,百感交集,几乎按捺不住内心的激动,二十九年来的生活,在他心中交织着,闪现着。

他必须抛下这一切回忆!于是他再掉转马头,挥起马鞭,两匹马旋即又在夜空下奔腾起来,太子此时虽仍频频回头,因为心志坚定,说也奇怪,此身似已能超然物外,迦毗罗卫城似乎与他无关地不再羁绊住他了,二十九年的岁月,也能如烟云般从他心头消逝。

悉达多太子俯身驱策着高俊的白马,奔腾在星空中,后来在佛陀的画传上是这样描述的:「人像轮转的明月,马像飘动的白云。」

车匿紧随在后,脑中不断出现净饭王震怒的容色,除了逃跑,除了跟定太子,别无他途,两人就这样马不停蹄地奔驰了约五个钟头,这种赶法,一般人早就要不支倒地了,在中途稍作休\xpinyin*{憩}后,又继续向前奔,奔至苦行林时,天边已微露鱼肚白了。

从迦毗罗卫国到这里,步行约四天的行程,亦即约一百四十公里,想想看,他俩只花了一个晚上。既到此地,就不用担心会有追兵了,悉达多太子舒了一口气。

这个林子位于摩竭陀国的北部,他记得几年前,自己曾路经此地,林间溪水\xpinyin*{珲琮},草树丛生,有芒果、龙眼等可以果腹,倒不失为一个修行的好场所。

他找到一处可坐的地方,那是大树被砍伐后所遗留的根部,当他落座后,抬起头,从林间的空隙处望进夜空,空中闪烁的点点星光,犹如耀眼的钻石,他此刻所看到的夜空,竟与在御花园中所见的大异其趣呢!

这其间的差异在那里?为什么他现在有着与从前\xpinyin*{迥}然不同的感受,自己也很感诧异,他看着明灭有致的星星,好像只要一伸手,就可抓下一把来,此时,自己与大地间好像还有一个我,当他的脚踩踏大地时,立即有一股震力直撼身髓。

这真是奇妙无比的感觉。

当太阳从东边冉冉升起时,悉达多太子抬起眼,嘱咐身边的车匿将东西收拾好立刻赶回城去,车匿恐惧不安,扑地跪在地上,恳求太子收留他,悉达多太子虽很怜悯他,但并未允准,太子当下脱掉身上的王服,要与车匿调换,至此车匿已忍不住\xpinyin{啜}{chuo4}泣起来,哀求太子打消出家的念头,悉达多太子同情地看着他说:「王位和华丽的衣服,都将妨碍我的修行,你以后自会明白的。」

\section{探访明师}\label{sec1.2}

\mt{苦行林中}

悉达多让车匿回去后,他就往林间走去,他要去叩访一位人称\xpinyin*{跋}\xpinyin{伽}{qie2}仙人的长者,这是一座有名的苦行林,林间有许多躲避战乱的士兵,也有许多修道的僧侣。悉达多问跋伽仙人:「人们在此修苦行,为的是什么?」

仙人回答:「图的是修个来世,你看,当今战祸连绵,死伤无数,心灵一刻也不能得到安宁,我们相信,只有在肉体尝尽了苦头之后,也就是在痛苦受完之后,自然能于来世生在和平而快乐的天堂里。」

悉达多满脸狐疑,不解地说:「这么说来,如果在天堂的福享尽了,岂不是又要堕到地面来修严格的苦行吗?这并不是彻底的解决之道啊!」

「你是受不了这种严酷的苦行吧!」跋伽讥讽他。

他继续向前行,所经之处,不时出现一些修炼苦行的人,他们再三重复着非常恐怖的苦行,有的人挨在熊熊烈火边,不断烘烤自己的身体,以致满身焦黑;有的人裸露着身子躺卧在满地的玫瑰刺上,肌肤因刺伤而滴着血,身体到处留着瘀痕和肿块;也有人将自己倒挂在树上,忍受着难以言喻的痛苦,他们相信,肉体上的极度痛苦,可以断绝心灵上的痛苦;他们也相信,今生把苦受尽了,来生就有福可享,为求心灵的永久安宁,肉体暂时的折磨是必需的。像这样惨不忍睹的酷刑,能使人产生平静安宁的心?悉达多不得不怀疑。

悉达多以为,问题的重心不是来世,而是现世,到底人是因什么目的而诞生的呢?是追求那不可企盼的来世?还是在现实中愉悦地度完一生?无疑的,悉达多二十九年来一直苦苦追寻的是后者。而这充满苦行者的苦行林不能解决他的问题。

\mt{内省}

悉达多在苦行林中待了三天,就失望地出发到另一处修行者聚集的山区,在那里他先后遇到了几位有名的仙人和修行者,有一位名郁陀伽的婆罗门,很受时人崇敬,但是悉达多听他对生老病死的解释,也如跋伽仙人般不着边际。悉达多在山区逗留了一个礼拜,也是不得要领,他真的失望极了,打从出家的第一天碰到跋伽仙人起,他就一直在黑暗中摸索,不禁感到前途茫茫,满心的晦涩。

「我的路是不是走错了?」

他回想在迦毗罗卫国的生活,以及促使他产生出城动机的种种情景,他不断在心中自问自答着。

「如果我这步路走错的话,是不是该再回到宫中?在那里重新再发掘途径?」

他虽这么想着,但是此刻的心意和一个多星期前出城时的心意,全无二致,求道的心依然坚定卓绝,他看得很清楚,任何人一旦高居于宝座上,而要想悟到人生的哲理,那比登天还难。人的心是很难驾\xpinyin*{驭}控制的,它会攀援外境而迷失正道,譬如我们明知某个人平时说话很虚假,但是偶尔被他赞赏两句时,雀跃之情,油然而起,深信他对自己的赞词是出自真心的;又如对方不管多么可钦可佩,一旦有违背自己意愿的地方,也会恨他一辈子,诸如此类,这本是人之常情。

悉达多想,是这样与天地为伍容易悟道呢?还是沉迷于声色欲乐中较容易呢?答案当然是前者,经过了七天的沉思与反省,悉达多重拾起信心,于是离座而起,迈步向毗舍离前进。

\section{五位比丘}\label{sec1.3}

\mt{震惊的消息}

悉达多太子连夜奔离王宫的事,一下子传遍了整个迦毗罗卫国,全国莫不为之黯然,净饭王的悲哀之情,我们可以想见,当他听完车匿的报告,眼前一片黑暗,令他寝食难安的预言,竟然还是实现了。一旦事实呈现眼前,他又如身陷一场噩梦般地,觉得虚无飘渺,他希望如以往无数噩梦一样,眼睛一睁,一切梦境都不存在了,而仍看见太子在清晨的御花园中\xpinyin*{踟蹰低徊}。

他长久以来就预感事情会发生,只是他常自我安慰地不愿去面对,现在他想把事实否定,也只像是把刀的锋口隐藏而否认它的锋利,无非是自欺欺人,他当下派出大队人马去四处追寻,一面深自思索让太子回心转意的计策,他虽想出了很稳妥的办法,但是却不能下定决心去付诸实行,如果是在战场上,他随时能临机应变,指挥若定,但是对悉达多的事,却不知为什么在该做决定的时候,又举棋不定,\xpinyin*{踌躇}不前了。

殿堂上,一时鸦雀无声,摩诃波阇波提夫人、文武百官以及车匿等,都屏息静待净饭王开口,时间一分一秒地慢慢推移,不,似乎是静止了,差不多一个钟头后,净饭王才抬起头来,打破了周遭的寂静,用着像是自言自语的低弱声音说道:「……我想没有办法可以追回他了,今天这件事,我早就料到了,悉达多一向对自己决定的事,从不反悔的,就是把他追回来了,也追不回他的心。好了,车匿,你也够辛苦的了,好好地去歇一会儿吧!千万不可寻短见,这件事不怪你,我知道,你也是奉命行事,现在没什么事了,大家都去休息吧!」

净饭王很\xpinyin*{颓}丧地站起来,慢慢地拖着步子回到寝宫。

\mt{\xpinyin*{舐犊}情深}

\xpinyin*{翌}晨,净饭王传唤了五位健壮优秀的年轻武士来到庭院中,他们知道一定是为悉达多太子的事,只是不知道将要担负何种重任,净饭王所吩咐的事情非常简单,他对这五个人说:「你们去陪护太子,同时负责随时补\xpinyin{给}{ji3}他的衣食所需。」

所谓陪护太子,就是去做太子的护身侍卫,穿戴与太子一样的衣饰,并随侍在侧,若遇不测当即时挺身替死,陪护是自古以来保护王族的一种职务。

这五个人,彼此面面相\xpinyin*{觑},虽然净饭王因心情沉重而严肃沉默,但是对太子所表现的宽大胸怀,使他们感动不已,有许多国家则不然,在平时就传出父子争权夺利的事情,任何一方都可能牺牲在权力争夺的血战下,照理说,身为太子,私离宫廷,背弃王朝,早已犯下了\xpinyin*{忤逆}不孝的重罪,任何一位居心\xpinyin*{叵}测的父王,都可能以此为借口而除掉他,但是眼前这位父王,不但没有惩罚的念头,反而怀着悲痛怜悯的心情来关切太子的未来,最后还嘱咐他们要侍奉太子到老死,这五位武士打心底里感佩净饭王,互相点头示意,很诚意地秉持了王的圣旨,就立刻启程了。

\mt{阿罗汉与六神通力}

后来这五位武士都成了悉达多的弟子,也是最先听他说法的人。他们都证得了阿罗汉果,阿罗汉就是悟道的意思,也是修行者已超脱生死的一种果位,当然其间也有许多阶段,一个得到阿罗汉果的人,具有测知他人心意的本事,这就称为他心通。

讲到这里,不妨先来谈一谈心的问题,这样或可帮助我们了解什么是修行。人心的活动范围,有所谓的表面意识和潜在意识(亦即意识和潜意识)。一个人如果悟道了,就是因为他的表面意识和潜在意识达到了统一和谐的境地,不用藉助五官,就能对万事万物了然于心,像有的预言家,就凭这种能力预知未发生的事,那些未来的影像犹如映现在电视荧光幕上一般清晰明朗,同理,对于他人的内心活动,也能了如指掌。这种能力,一般都称之为神通力。

佛教中所说的「六神通力」(简称六通),也就是在意识和潜意识达于统一和谐的情况下所产生的六种超能力,此六种超能力即指天眼通、天耳通、他心通、宿命通、神足通和漏尽通;天眼就是心眼,能见一般肉眼所不能见的境界,如至坟地时,能见种种幽灵活动其间;又如能放眼望至无穷远处;或见到极细微处。天耳也非一般肉耳所能及,它有如俗下所说的顺风耳,能听到极遥远的或极微弱的声音,甚至听到另一世界千奇百怪的声音,或美妙悦耳的天使歌声;他心通在前面已述及,一个人若有他心通,就能掌握人心,控制他人的行为了;宿命通在于能明了一个人的命运,就好像看电影,把一个人的来龙去脉摸得一清二楚。佛教中肯定人不止活一世,人有前世,也有来世,而且生生世世在生死中流转,这都是具有宿命通者所昭示给我们的道理,释迦牟尼佛就具有很高超的宿命能力,故能宣说宇宙中精深奥妙的哲理,当然他的另五种神通力更不用说了。神足通,就是一种能神游四方的能力,譬如你人在此地,但是一瞬间已到了美国的白宫,并清清楚楚地看到白宫内的情景,二楼的走廊上有某一总统的肖像挂着,总统的房内有华盛顿的半身像。不想看的时候,一下子又可回到原地;漏尽通是一种心灵上了无挂碍的能力,我们一般人很容易受自己周围的环境所束缚,\xpinyin*{衍}生许多的烦恼,但是具漏尽通的人能超然物外,逍遥自在,荣辱不系于心,得失不存乎身,他能洞见事物的真相,过着自信满足的生活。自然这一种神通,是要凌驾其他五种神通之上的。

就以上六种神通力来说,各有其境界,依人修炼的功夫而分成好几个阶段,所谓「欲穷千里目,更上一层楼」,有人只能略施小技,有人却是神通广大,这都是功夫精不精深的差别,自然是功夫越精深,所悟得的境界就越高妙也越接近真理。

阿罗汉本是梵语音译而来,因具有许多种含义,故保留它的原音,阿罗汉是开悟的第一个阶段,也是独善其身的时期。如果再进一步地经过严格的修行,就能达到「菩萨」(梵文「\xpinyin*{菩提萨埵}」的简称)的境界。「菩萨」意即「觉有情」,也就是说,一位菩萨本身是已悟道的有情众生,悟道后启导其他的有情众生一同走上觉悟之道,这是与阿罗汉的境界不同的,我们也可称之为兼善天下的阶段。菩萨分五十一个等位,也各有其境界。第五十一个等位是到达佛境的最后一个阶段,要修到这一地步,要历经千辛万苦,甚至穷生生世世的精力。

释迦牟尼成佛以后,所到之处,徒众风从云涌,其中修证阿罗汉果位的,不下数百多人。陪护太子的五卫士,憍陈如、阿舍婆誓、摩诃跋提、十力迦叶及摩男俱利,最后都证得阿罗汉果,也有人进一步修「菩萨行」的。

\mt{五个现实问题}

净饭王之所以派遣这五个人去追随太子,是因为他还抱着一线希望。他总认为悉达多还年轻,涉世未深,只是一时的冲动,自有回心转意的一天,当然,净饭王也知道超脱「生老病死」的重大意义,但是他觉得这对一个凡人来说,根本是不能实现的梦想,等悉达多猎取真理的箭矢没有了,他自会回头的。净饭王派遣五人的意思是,要太子知道父王正时时刻刻等他回宫,而不是对他的事漠不关心,所以他派去的五位武士,实际上是太子和迦毗罗卫国间的一个联系,永远不会有断的一天,于是,净饭王朝盼暮望,期盼有这么一天,五个人轻松愉快地陪侍太子回宫,就这样,岁月在期待中一天天过去。

至于悉达多呢?迫使他下定决心出城的,有五种原因,第一,他一直耿耿于怀母后摩耶夫人的死。第二,迦毗罗卫城内(事实上是全印度的社会)阶级贫富的差异所造成的不平等现象。第三,对人类自相残杀的战乱深恶痛绝。第四,摩诃波阇波提夫人生下难陀后,引起他对历来王朝争权夺位的警觉。第五,以耶输陀罗为首的\xpinyin*{嫔}妃之间所引起的争宠风波。

他对生老病死的问题如此关切,完全种因于这些现实生活中的恼人问题,如果他对现实问题一点也不疑虑,他也就不会问:人是什么?他若毫无忧惧地沉溺在声色犬马之中,则终其一生将不过是迦毗罗卫城的一个平庸的小城主罢了,悉达多太子的苦恼,以及出家的意图,并不如净饭王所想象的那么单纯。

\mt{情感的羁绊}

奉了国王的命令,憍陈如等五人,为寻悉达多太子的行踪,沿途不断地向人探听,也曾寻遍了整个苦行林,悉达多太子在离开苦行林后,被许多本族的人认出来了。他会被发现,是因为那一身不相衬的马伕装扮,憍陈如等五人终于依路人透露的线索,来到悉达多沉思冥想的地方。他们在距离二十公尺处的大树下,偷\xpinyin*{觑}到正在静坐沉思的太子,于是他们按捺住焦灼的心,在一旁守候。此时,在场的还有马歇大臣随后派出的侍卫及文武大臣等数十人,包括车匿在内,他们悄悄向四处散开,围住悉达多太子,并渐渐向悉达多太子挨近,他们未曾惊扰太子,却很纳闷,不知太子什么时候会睁开眼睛。他们自从奉命出宫寻见太子,至今已整整花了六天功夫。车匿轻步走到太子座前,跪在地上,伸出双臂向前屈至地面,并把额头伏在地上,悲切无声地哀求悉达多太子回宫,就这样匍伏着过了好长一段时间。

体格较魁梧的憍陈如,这时也跪近悉达多太子,忍不住开口道:「禀呈太子,大王及宫中文武百官都为您的出家感到无比的沉痛,请您能体念我们的心意,再回到宫中,我们希望能像以前一样侍候您,大王日夜祈祷,就请您让大王的愿望实现吧!太子,请您多想一想啊!」

他垂着头,哀凄地对禅定中的悉达多太子轻声慢语地恳求着。

本来站在憍陈如旁边的阿舍婆誓、跋提、迦叶和摩男俱利,也都一齐跪倒在太子面前苦苦哀求,悉达多太子好早就已察觉到他们的出现,也感觉到他们都环绕在四周,只是他并不立刻睁开眼睛,他暗自思\xpinyin*{忖}:「这些我不想见到的人,竟然还是找到我了。」

于是他在脑中盘旋着该如何答覆他们的问题,又如何拒绝他们的要求。当他隐约听到\xpinyin*{窸窣}的脚步声时,悄悄抬起眼睑,看到瘦骨嶙峋的车匿,正匍伏在面前。他忍不住一阵心酸,想那可怜的马伕,一定是为了他出家的事受到良心的谴责,即使父王不追究了,他自己也不会原谅自己,他若再这样折磨下去,要需多少个身体来支撑呢?现在看到他跪在自己面前哀伤\xpinyin*{恸}哭的样子,悉达多为之神伤。

悉达多太子沉缓地睁开眼睛,环顾了一下四周,然后轻轻地叹了口气说:「我为了想断绝一切的苦恼,才决定离宫的,我要设法救助一切跟我有同样苦恼的人,所以我必须先修行,才能悟道度众,我很了解家人此刻所感受到的悲痛,这本是骨肉间至情的表露,但是,有朝一日,我要去渡脱比这惨重的痛苦,你们带来的食物和衣服,对我没什么用处,等一会儿希望你们再带回去,我很感激你们对我的爱戴,但是这一切都不能解决问题,我真的心领了,现在希望你们赶快回去,不要再为我操心,你们就代我向父王母后禀明我的心意,并代为问候大家。」

悉达多太子说话时,语气柔和而坚定,态度谦恭而沉稳,在场的人都低垂着头,他们感觉到悉达多太子的轩昂气势,以致感到自己的软弱与渺小。

憍陈如等很了解悉达多太子的为人,太子从来不食言,他们深知说服不易,也深深感佩太子那颗深沉的同情心以及尊重他人的心,所以对悉达多太子那一番十分寡情的话,丝毫也不介意。他们因为秉承了净饭王的命令,故不断恳求太子说:「我们要永远守护太子您,请准许我们留在您的身边吧!」

说完,五人同时将身子仆至地面,并把右手举至胸前,表明了誓死相随的意愿。

但是悉达多太子用着很严峻的语气说:「不行,你们一定要回去,快回去,不得停留。」他们还想说什么,但是被太子坚决的神色给震慑住了。

\mt{与孔雀同悲}

净饭王可称得上是一位难得的英明之主。他处事决断,从无滞碍,巧妙地掌握了全迦毗罗卫国的人心,对外又能发挥独到的政治手腕,采行怀柔政策,故能以一个小国而安立于群雄间。反观太子悉达多,就没有他父王那种治国的气魄了,可以说,他还是非常稚气,且十分率直,他相信天地间的正气,最不屑政治上的花样与把戏,他坚信凡事站在理上,没有不成功的事,对于必须杀身才能成仁的一类事,他一定是义无反顾的,宫廷上下的人都对他心悦诚服,就因为他并不恃太子的权势来牵制他们,相反的,他对宫中的奴仆,常会不自觉地流露出同情与怜悯,而不大使唤他们。

有一次,一只美丽非凡的孔雀误闯御花园,净饭王看了十分喜爱,就嘱宫女小心地饲养。几天后,当宫女来喂食时,见\xpinyin*{槛}中的孔雀已不知去向。净饭王大发雷霆,立刻下令把该宫女幽禁起来。

悉达多得知这个消息,立刻晋见父王。

「这不关宫女的事,孔雀是我放走的,父王请罚我好了。」

「什么?你明知我非常喜欢这只孔雀,却为何还要放走\xpinyin*{牠}?你这不孝的孩子,既背叛了我的命令,是不能留你活命了。」

净饭王在宝座上,高耸着肩,眼中如燃烧着火焰般瞪视着悉达多太子,大有罪不可赦的愤恨之情,但是随即暗自叫道:「糟了!」他在盛怒下,竟以王子的命来抵偿,这是很失算的事,但是一个君王,不能食言而肥,他后悔莫及,希望有个人出来为太子说情,好给他一个台阶下来。他怀着一颗悔恨交集的心,却仍必须僵硬着怒目圆睁的脸,他偷觑一下周围的人,希望有人站出来。

在净饭王身边的摩诃波阇波提夫人,这时开口了:「大王犯不着为了一只孔雀的事生那么大的气。您不妨问问他放走孔雀的原因,问好再依情况处置也不迟啊!您就请息怒吧!」

这时,摩诃波阇波提夫人转向太子问道:「你一定有放走孔雀的理由吧?」

悉达多太子凝视母后的眼睛,音调很沉稳地说:「母后,我想先问您,如果您在邻国迷路了,被邻国的人抓去关起来,您会怎么样?」

「……」摩诃波阇波提夫人无词以对。

「虽说是一只孔雀,也是一个有血有肉的生命,它是半途迷路而进入花园的,如果是野生的,它家里必还有兄弟姐妹在等他归去;如果有母鸟,母鸟必在等它这个爱子回去;如果是有人饲养的,那么它也该回到主人那里,这才是道理,我见父王把它抓来以后,似乎并没有释放的意思,所以我就擅作主张地把它给放了,请原谅我没有向父王禀告一声。」

其实这就是悉达多长久以来积历在心头的不满,他对人与人间互相倾轧的现象深恶痛绝,如今就藉着孔雀的事情,悠悠地抒发出来。

净饭王只是闷坐一旁,皱着眉头,摩诃波阇波提夫人语气温和地说:「我很了解你的看法,这只孔雀自有它的归处,我们不该强留,我会联络我的娘家,过些天送一只孔雀来好了。大王请看我的面子,就原谅太子这一次吧。」

孔雀事件就此平息了,我们也能由此事件窥知悉达多太子为人之一二。

\mt{可悯的生命}

马歇大臣派来的人马依言先行回宫了,而憍陈如等也知道多说无益,悉达多太子坚决的态度回答了一切,他们只好退下来,悄悄与车匿商量日后的大计。商量的结果是,憍陈如等五人决定找一处不妨碍太子的地方修行,并一面守护着太子,他们决定不回迦毗罗卫城了,这以后,他们有好长一段时间都是远远地守护太子,未曾去惊扰他。所以,回到迦毗罗卫国传达消息的任务,就落在车匿的身上,车匿虽然坚拒不受,也再三恳求他们留他下来,可是,一定得有个人回去复命啊,车匿在万般无奈的情况下来向悉达多告别。

「你要回去了吗?你就代我向大家说声好,相信来日还有再见面的一天,请大家要多保重啊!不见憍陈如他们,是怎么回事?」

悉达多突然问到五个人。他们六人刚刚商议的事,车匿是不敢说出来的,慌张间,他结结巴巴地说:「是......是这样子的,他们早一步回去了,我......我一定把太子您的决定禀报国王王后知道。太子,请多保重,我---走了。」

悉达多看着频频向自己行礼致意的车匿,心中升起了无限的同情,同情他那常被责罚打骂的地位和身份,此人何辜?他一定要努力求道,来改善种种不合理的现象,此刻他内心虽感怜悯,也只得暂且抛开感情的包袱,假装无动于衷地扳着面孔,一句话也不说,只是默默看着车匿收拾行囊。

在这傍晚时分,林间树下的一天又将过去了。对面的山坡上,稀落的树影勾勒出细长有致的斑纹。在斑纹间,冉冉升起缕缕轻烟。烟儿带着生命的讯息幽缓地、连绵不绝地轻轻\xpinyin*{翳}入天庭,对山的修行者必在忙碌地做着入晚的准备吧!星影尚未出现的天空,\xpinyin*{湛蓝澄澈},这里那里地打着如棉球似的云朵,西边,被抹上了一片嫣红,夕阳的景致与色彩,千变万化,大自然岂不正在讴歌天堂的美妙?

车匿好不容易地收拾好他那其实很简单的行囊,长跪着向悉达多告辞,然后才站起身来,迎着夕阳走去。他低垂着双肩拖着蹒跚而无力的步子,手中的行囊和脚下的步伐,似乎都成了他极沉重的负荷,这样一个沮丧而疲惫的身影,悄然消失于树海的那一边,因着地势起伏,影子也忽隐忽现。悉达多就定睛凝望那渐去渐远的影子,久久不忍离去。

\section{净饭王与释迦族的悲叹}\label{sec1.4}

\mt{桥上的石头}

车匿在接近伽毗罗卫城时,一路飞奔起来,当他抵达时,城内为之骚动不已,有人以为太子已随同归城了,也有人想知道事情的真相,城内人民的情绪越激昂,车匿的步履越畏怯。假如他带回来的是太子归城的捷报,他也会涌现无比轻快的心境,但是,此刻带回来的竟是太子不再归城的噩耗,他怎忍得下心来呈报呢?尤其是如何面对望眼欲穿的净饭老王和初为人母的耶输陀罗夫人呢?他的心苦涩不堪。

说到耶输陀罗「初为人母」,我们来谈谈悉达多的新生儿——罗睺罗。「罗睺罗」原意为「桥上的石头」。「桥上的石头」在当时的印度来说,起码具有两种意思。当时印度,在桥上架设的多为吊桥,吊桥上若有大石,会增加桥的负担而十分危险;另一方面,在人人渡河时所必经的桥上放置石块,令人难于通行而多所不便,所以「桥上的石头」意味着危险和障碍。罗睺罗的来临,加重了悉达多感情的负荷,他非常喜爱这个可爱伶俐的小生命,唯其如此,他才更警觉到快乐是短暂的一种痛苦,罗睺罗将是他出家的另一个感情羁绊,所以他用心良苦地把孩子取名为「罗睺罗」。

净饭王对罗睺罗寄予很大的希望,他原以为罗睺罗一定可以断掉悉达多出城念头,如今这一个希望也已落空了。

\mt{诸行无常}

当时有很多人不谅解悉达多的行为,因为他无端端地离开父母,所以背上了不孝的恶名,因为他撇下了襁褓中的新生命,有人责他是狠心的父亲。悉达多这么忍心地抛掉亲情的温暖和天伦的快乐,更愚蠢地放弃了万人翘首的权势地位和荣华富贵,他到底追求的是什么?他追求的是,心灵上永恒的宁静以及生命上不朽的存在。

「花无百日红,人无千日好」,他早已看出世事无常的本质。他知道人所胶着的情感促使无常变成刽子手,扼杀了人们完美的心灵。人如果不脱开感情和无常的缠缚,人心永远得不到安宁,在人世间种种相对的情境中,人很难冷静地看清真实的自我,这些相对的情境,指的是生与死,寿与夭,穷与通,荣与辱,贵与贱……,而人,就迷失在其间,要想使自己的心,不受这相对情境的牵绊,唯一的出路,是走向「绝对」。简单地说,就是我们要以平等的心对待相对的事物,我们不在心中制造「差别」,不刻意想着某人是穷人,某人是贵人;我在失意中,他在得意中……依此类推,如果你的心中有了差别的形相,你必产生差别的感受而被套在相对的情境中,喜怒哀乐于其间,悲欢离合于其间,永无了时。

保持一颗对无常警觉的心,人就不会在一些相对的事物上产生无谓的烦恼,做无意义的奔驰,自然也就不那么容易被外界的事物与现象所左右了,悉达多在宫中生活了二十多年,在不断地思维与观察中,了悟了这一点,他的感情丰富而细腻,所以他感受到千百倍于他人的痛苦,人在生活中或许尝到了无比的快乐滋味,相对地,到头来也会觉悟到自己曾付出了多少的代价。悉达多之所以痛苦,因为他在事物上预见了一切的过程与结果,短暂的欲乐,不过像涂了蜜汁的刀刃,能给人最痛苦的伤害,有这种自觉的人,在人世间的烦嚣中寻不到快乐,只有舍离世间的一切,因为舍离,他获得了更丰富的东西,也给世人带来精神上最大的安慰,我们在悉达多的得失间,可有过什么权衡吗?

\mt{割舍之情}

车匿的脚步,在越接近宫门时,越感到沉甸甸的,他脑中浮现出无数张泪脸儿,穿越宫中的长廊,想想自己过去来往于这条长廊,前后也不下十数次了,为什么今天对这个长廊竟有着说不上来的异样感觉呢?这个廊,似乎又短又冷,走在其间,好像有阵阵冷风,正飕飕地袭来,他不想那么快见到净饭王,却不一会儿工夫已到了净饭王焦灼等待着的大殿了。

净饭王远远望见\xpinyin*{佝偻}着的车匿,见车匿面现憔悴与不安,他的心痛苦地往下沉,车匿语无伦次地还未说完,耶输陀罗已禁不住哭出声来,第三夫人也抑制不住地抽\xpinyin*{搐}起来,第二夫人呢?自从悉达多离宫以后,就一直病卧床上,对她来说,悉达多是她的一切,往日悉达多跟她说过的每一句话,都深深地印在她脑海,并时时浮现于脑际。她常自豪地认为,世上没有谁再能像她这样爱悉达多了,她跻身在耶输陀罗和第三夫人之间,为不能独占悉达多而诅咒自己身为印度女性的命运,但是,当悉达多一出现在她眼前,这一切不满都遁隐无形了,在欢乐之余,她看到了自己身为女人的光明面。

随着悉达多的隐去,她的美梦破灭了,好像一个空壳一样,意识已远远地离开她的躯体,她模模糊糊地似乎听到悉达多归城的消息,那远离的意识一下子飘回了现实,于是她迫不及待地,在宫女扶持下来到大殿,视线模糊中,她望见颠危危的马伕的影子,以为就是悉达多,便手足无措地心乱了起来,心魂甫定,却又听到马伕报告,至此,她希望的线溃断了,在声嘶力竭中昏厥过去。

净饭王倾听车匿的报告,听到一半,就举手示意说:「好了,我知道了。你也辛苦很累了吧!快下去休息吧!」

大厅里,弥漫了浓浓的悲伤的气氛,只有年老的净饭王,冷静地沉思着,依悉达多的个性,目前是不可能回来,但是他不放弃希望,他想,悉达多是自己的孩子,是一个人,怎么可能解决上天才能解决的生老病死四大问题?总之,日久之后,悉达多总会空着手回来的,到那时候,他必会好好地做迦毗罗卫国的国王。

「还有那五个武士,可以做我和他之间的桥梁。我这么做,他必能了解我的心。……」净饭王满怀希望地想。

\section{一个妇人的布施}\label{sec1.5}

\mt{林间的饮食}

虽然悉达多已渐渐习惯了林间树下的生活,但那生活真是无法想象地困苦,口渴时,只能喝河水,印度的河流大部分都很混浊,若想直接就着水喝,需要相当大的勇气。这种水喝下去,要忍受腹痛和下痢的痛苦,这是与过去宫中\xpinyin*{迥}然不同的生活,凡事都得亲自动手,为了想喝较干净的水,还必须用鹿皮自制\xpinyin*{漉水囊},再将水仔细地过滤,然而因为太阳的热晒,细菌蚊虫杂然繁殖,即使一再过滤,恶臭仍阵阵逼来。每次喝水,都是闭着眼睛,屏住呼吸,再一口气咽下去,那难受的滋味,使他不期然怀念起宫中纯净甘美的水,饥饿时,就采\xpinyin*{撷}山中林间的果实来充饥,此外,他也如其他的修行者一样,有时到人烟聚集的地方去化一点食物来。

在印度,修行者的修行具有很崇高的意义,人们经常以无比虔敬的心来供养他们,只要僧侣一站在屋檐下,主妇们会立刻把饭菜捧出来。食物虽然很粗陋,但是当你接触一张张温婉善良的脸,感谢之情就会油然从心田升起。

\mt{忘我的布施}

有一次,悉达多驻足在一间简陋矮小的茅舍前,屋子里静悄悄的。因为房舍矮小,悉达多站在屋檐下,对室内的情景,一目了然。他看到一个主人模样的男子\xpinyin*{憔悴}不堪地躺卧在\xpinyin*{蓆}垫上,由于久病不起,身形消瘦,露出了骨节,正静静地仰视着天花板。地上围坐了五个孩子,想必就是病人的儿女。一个头发散乱看来像是母亲的女人,正一瓢瓢地把稀饭分在五个孩子的碗里,只见稀饭就那么半瓢浅浅地沉在碗底。悉达多看了很不忍心,正想转身离去时,他听到其中一个孩子问道:「妈,谁来了?」

正在舀稀饭的妇人停下来望着悉达多。接着,妇人急速走到屋檐下,把手中自己要吃的一点点稀饭送到悉达多面前,说:「请拿去吃吧!」

在屋檐下,因为强光的照射,这个女子的模样很清楚的映入悉达多的眼中,她的衣服沾满了油垢;蓬乱的头发,因蒙上了厚厚的灰尘,而赤褐斑斑,灰黑的脸上,布满了困厄的痕迹,只有眼睛,澄清明亮,闪着坚毅果敢的光芒。

悉达多身子微微一欠说:「谢谢您,您的心意就是最好的布施,我非常感激。」

但是妇人并不将手缩回,仍对悉达多说:「吃一口也好,我希望您不嫌弃。」

悉达多心想,如果此刻掉头离去,就辜负了这妇人的美意,她那「希望你吃一口」的心意,完全离开了私我的欲念,将一个人内在的虔敬完全表露在言行上了,这是一种极致的表现,她此刻那种供养苦行者的神态,似乎连自己的体肤都可以割舍一般,她那一贫如洗的家庭,蓆上的病人,挨饿的孩子,都在眼底消失了。此时此景,悉达多不忍辜负妇人的心意,于是在妇人的面前,当下就一口一口地把碗中的稀饭吃了下去,妇人那被生活折磨得有些僵硬的脸庞,顿时\xpinyin*{绽}露出春阳般的微笑,悉达多也笑了。悉达多对布施的妇人深深地作\xpinyin*{揖}后,就径自回到山中。

那位妇人供养时的情景,一时萦绕在悉达多的心田,竭尽心力的布施,深深地感动人心,重所鼓起他人面对生活的勇气,能拂去人们心灵的邪恶,能使人拾回初生时的赤子之心……。如果人人拥有一颗童稚的心,这个世界将和平安乐,到处充满欢乐的笑声,人的悲哀没有了,喜悦增加了......这样的环境,这样的心境,本应属于每一个人的。

可是反观现实,满目\xpinyin*{疮痍},人究竟为了什么?你争我夺地永无休止,一些野心家们,打着胜利的旗帜,登高一呼,万众\xpinyin*{翕}从,悉听摆布,悉达多看不出这一现象能带给个人或他人什么样的幸福,在宫中,他未曾体会过真心所激发出来的喜悦,强邻会来攻的噩梦永远没有醒的时候,他的神经被不安和恐惧旋得紧紧地,宫中的食物的确甘美,但是不知道间谍什么时候会在上面布下毒药,虽有试毒的仆役,但是能够保住自己的性命,却保不住仆役的性命,到头来,这终究是一场悲剧,此情此景,虽然美食当前,却尝不出它们真正的味道。

就人的\xpinyin*{癖}性来说,食物再丰盛,再鲜美,天天吃,总有吃腻的一天,往日在宫中,他也有过华服美食是享乐的念头,如今,随着岁月的增长,他发现了粗茶淡饭的美妙,妇人的这一布施,使悉达多接触到人性的光辉,激起了他心中的一丝信念,他决定挺起胸膛,为证悟真理而付出最大的努力。

\mt{适应}

悉达多在行乞化缘的当儿,不但体验到人性的微妙,而且尝到了粗食的滋味,自从习惯了行乞生活,内心不再挣扎和痛苦,回想出城的第四天,除了污水和野菜,他的腹中没有其他的东西,他想仿照其他的僧侣到村中乞食,却因与他向来熟悉的生活大相迳庭,很难鼓起勇气,从一个王子变成一个乞丐,两者是两个极端,这中间的适应过程是相当艰辛的,因为不能战胜饥饿而不得不下定决心去乞食时,他感到步伐不稳,又似乎时时有人出现在背后盯视着他,使他行动迟缓呆滞,然而这样辛苦,是有代价的,他可以携回米和菜,到林间煮稀饭,只是悉达多没有烹饪的经验,经常吃到夹生的饭粒,即使是如此,他仍甘之如\xpinyin*{饴}而不觉\xpinyin{莞}{wan3}尔。

以往在城里,常为了怕城池被攻陷而终日\xpinyin*{忐忑}不安,现在在野地里,代之而起的是野兽随时会来袭的恐惧,猛虎、恶狼、毒蛇等随时都有扑过来的可能。所以,悉达多在禅定时或晚间入眠前,必定先生起火,选择大树下或洞穴中落榻,为防御蚊子和毒虫的侵袭,并采集草药制蚊虫药液,涂在身上,入夜以后的森林,更是恐怖阴森,野兽的咆哮不时震撼着林木,闻之毛骨\xpinyin*{悚}然,风穿林梢,直如庞然怪物呼啸而过。初至森林时,悉达多一夜都无法合眼,深感痛苦,然而这本是修行生活中必然的现象,若不能克服这一关,还谈什么修行呢?于是他把心一横,准备承受一切。说也奇怪,他这样想了之后,内心反而静定下来,心一静,一切畏惧都自然远离了。

森林中的生活,仍然比宫廷中的生活安适多了。白天,他藉着树荫移动的位置来判定时刻,并安排游化和禅定的时间。晚间,则看着天空的星星来决定作息的时间。



\section{遍涉疑问与答案}\label{sec1.6}

\mt{婆罗门的生活}

在悉达多离开迦毗罗卫国的第六天晚上,他遇见一位婆罗门,这位年约七十多岁的长者,进入森林后,发现了一处幽静的场所,就兀自沉思冥想起来。

在婆罗门家庭,小孩到六、七岁,就被送到一位师父家中求学,约需十二年,学祭事,学吠陀,然后再回到家中,过着一般婆罗门所过的生活,如祭祀亡灵,授业传道……等。\xpinyin{俟}{si4}达到一定的年龄,就把家事托付给下一代,他则进入森林,去过冥想和苦行的生活,以修来世。婆罗门在森林中过苦行生活的,并不限于男子,有时夫妇一起苦修的也有,这一点看来,当时印度社会在男女平等的观念上,虽不像现在,但是比诸当时其他的社会,是要开明得多。

在初从事修行生活时的僧侣,一般称之为「沙弥」,年老后进入森林,做进一步的禅定与冥想,并企图从种种现世的苦痛中解脱出来,则称为「沙门」。婆罗门出家修行的习俗,起源于伍巴尼夏都时代。分幼年、青年、壮年、老年四个阶段。这些修行支配了他们的一生,形成了他们很特殊的生活形态。大体说来,由于婆罗门一生起码有四个阶段的修炼生活,使他们在才艺方面,的确能超人一等。像悉达多年幼时,就有教他文武艺事的婆罗门教师,年龄稍长后,又有一位专门教导他运用听力来学习的教师。也许就因为这一类的学习,悉达多后来在神佛的信仰方面,能够无师自通。

前面已谈过当时印度社会的种姓制度,我们知道婆罗门掌握了最高的知识权力,地位非常崇高,在每一阶级之中又细分了无数阶级,即连婆罗门中,居最下等的婆罗门则永远是最下等的。

\mt{修行的路}

悉达多舍弃了王子的尊贵地位,毅然进入林中修行,他此刻所要修的,是婆罗门最后一阶段所要达到的「沙门」的境界。

「修行」代表行为和思想的超越,有许多人并不真正去探求什么真理,只想陶醉在修行者的优越感中,因此,在宗教的思想方面也\xpinyin{参}{cen1}\xpinyin{差}{ci1}而杂驳,就悉达多所知,许多婆罗门修行者,由于\xpinyin*{囿}于习俗,只在一种森林生活的特殊仪式下,投身于一种修行的气氛中,一点也没有悟道的上进心,悉达多巧遇这位婆罗门长者,虔敬地等待长者从禅定中醒来,只见长者出定后,就站起身来,走到河边浴身,这位长者神态自若,似乎对修行生活早就习以为常了,悉达多谦恭地请教他禅定的问题,他的回答是很浅显的,他禅定的目的,在于\xpinyin*{袪}除心中的一切杂念,使自己达到「无」的境界。他认为人在「无」中,就能见佛,佛我则能达于一体。

从杂念中超拔而达于「无念」的三眛境界,自古以来,有过这样的例子,而且能立刻与佛境取得联系,在剑道武术中,也有无念无想的修炼窍门,当自己抛弃了自我观念,完全置身于虚无的境界时,心底深处所产生的强大力量,能自然窥得虚实而予对方迎头痛击。可是,要控制一己杂乱的心绪,谈何容易,人在呼吸之间,意识随着流转,也许能控制于一时,却很难持续于久远,人若是任由心思驰逐于外在的事物,则那一颗奔驰的心就与悟道无缘了,悉达多决定把迦亚达那选为长期的修行场所,没有必要的话,他不离开此地,如此过了四个月后,他对沙门的修行生活,可说已很习惯了。

他之所似周游各地,主要在于寻访师父。但是,没有一位堪称他的师父,悉达多曾在苦行林的南方探访过跋伽仙人,在西南方探访过阿罗蓝仙人,他曾眺望过这两处修道的地方,深觉有如一幅美丽的风景图画,山水相依,明洁柔和,\xpinyin*{刹}时洗涤了他茫然已久的心灵。当这一片景色在悉达多的眼前开展,他不禁为之屏住呼吸,惊讶不已,这么迷人的画面后面,隐藏着\xpinyin*{腥羶}与污垢,一批修行者废置当前的美景,却在心上制造一片凶残的景象。前面说过,悉达多并不苟同跋伽仙人所主张的修道理论。

解脱的原意,本在脱离人生的烦恼与痛苦,他很难想象人在苦恼的连锁中会脱离苦恼,他不赞同跋伽仙人的看法。

至于阿罗蓝仙人的看法又是如何呢?

「人有生老病死的苦恼,是因为人们执着一个『我』。」

因此,人应舍弃「我」的观念,用禅定的力量来使自己从「非想」达到「非非想」的境地,最后终能悟到宇宙的至理。当然,要发挥「心」的大用,一定要先使这颗心坚韧,在烦嚣的尘世间,我们的心易于攀附外境,沉沦而不自知,当我们在深山野外,人迹罕至的地方,心才易于收敛而达于静定,亦得以反躬自省而心无旁鹜,能够持戒、谦逊、忍辱、离恶、近善......专心致志地进入禅定之境。

当心体合一而忘我时,心中会涌现欢悦,但是不能执着一时的欢悦,因为这只是其中的一个过程,我们应舍离这个境界,这以后才可达于「正念」。「正念」是一种分别邪正的工夫,有此工夫则不再被外在的形相所骚扰,所以说,正念也是「乐受」的根基。当你能不被环境所左右时,本心自如自在的,又如何不乐呢?然而此乐也必须从心中除去,凡是会使你的心攀附着一切事相,都要弃之如敝\xpinyin*{屣},为的是使自己的心无着无碍,唯有在禅定中我们能获受这种力量和利益,行之于日常生活间,我们就不再役于自己喜爱的事相,难舍能舍,正是「无想」所赐的工夫,也就是一种解脱之道。

再说生命本身,在宇宙太初之时,本身混沌而来去自如的,由于混沌的生命一动,生出了「我」(的观念),由「我」又生出了贪着愚眛的心,此心生爱执而产生肉体,有了肉体,为满足肉体所需而产生种种烦恼,又因烦恼而衍生无穷的痛苦,以致伤神,以致老、病、死……如此流转不息。

悉达多觉得这种说法,有些道理,但是阿罗蓝仙人所强调的从「非想」到「非非想」的境界,就是在以「无想」来切断「想」。事实上,仍在执着一个「想」字。不可否认的,人的存在,就是一种意识的动态存在,这一股「意识的流」想要立刻加以阻\xpinyin*{遏},就必须用另一股力量来达成这任务,而另一股阻遏的力量,充其量也不过是一个「想头」,到头来,我们也许虽曾抑制了一股意识的流,但另一种新的「意识的流」,却不可避免地产生了,这是「意念」的一种回转形态。

\mt{走自己的路}

悉达多大惑不解的是,从这两位仙人的行径来看,似乎人是为了修行而诞生人间的,为了要悟道,而刻意地虐待肉体,为了要求真理,深入山林,修三眛禅定,假使人果真是为这两个原因而做人的话,那世上没有比人更可怜更悲惨的生物了。显然,他们的修行方法,都泯灭了人道,人生来就有眼耳鼻口,并且有手有脚,还有思想的能力,却为什么把上天赐予的大好官能废置蔽\xpinyin{塞}{se4}呢?为什么我们既有了手和脚,却要摆到火上去受烤刑呢?人在大地生存时,自然赋予人的头脑和五官,是应付生活的最低保障。我们为何不但不善加运用,还要明显地摒弃这个保障呢?悉达多对于违反人道的修道方式,很不以为然。

依悉达多的看法,像婆罗门所追求的天堂的心,从这种残酷的修行方式中是很难证得的,更别提什么生老病死的超脱了。那么,有没有另一条路可以走呢?悉达多不住地思索着,他现在身为一个沙门,也像阿罗蓝仙人一般,乞食化缘,以河水解渴,以河水浴身,他采行了阿罗蓝仙人的禅定方法,在修禅的内容上,却自己有着清晰的想法和做法,心田内不断涌现问题,而他对问题也都能一一地用\xpinyin*{缜}密的思虑找出合理的解答,虽说他自己能设法解答自己心中的疑难,却没有把握这些答案的真实性会在何年何月,在什么地方显彰出来。

好像一艘已出航的船,不知目的地,徒然在大洋中飘泊不定,若是碰到了狂风巨浪,还有被吞没的可能。

没有人协助而能安然无事地到达目的地,这一希望是很渺茫的。他虽立下决心独自去求道,但是犹如猎人,在无路的山林间追逐鸟兽,既艰辛,又无多大希望。

\mt{与\xpinyin*{摩竭陀}国王结缘}

摩竭陀国的国王\xpinyin*{频婆娑罗}王,对宗教的信仰很虔诚,很礼遇山林间的修道者。因为国王如此,老百姓们也因此对修道者的布施都很虔敬而丰厚,国王更从修道者中选举道行高深的来担任武士,供给他们最优\xpinyin*{渥}的生活,频婆娑罗王风闻悉达多的德行,也备了车马亲自到悉达多修行的地方,国王问:「您从哪一国来的?」

悉达多恭敬地回答说:「我来自毗萨罗国的属国释迦王族,叫释迦牟尼\textperiodcentered 悉达多。」「喔?您既有这样高贵的身份,又为什么要出家呢?实在很可惜,我希望您能留下来,您需要什么,我都会尽心尽力地供养您。」

王很温和地说。他真希望悉达多能成为他的武士。

悉达多婉谢了,他说:「我就是为了抛弃一切荣华富贵才出家的,您对我的厚爱,我很感谢,为追求人生最高的境界,我必须修行。」

国王从悉达多的神情中看出他的坚决意志,也就不再强留,他俩就人世间的苦乐状况做了一番讨论,最后频婆娑罗王对悉达多说:「迦亚达耶有一个乌鲁维拉\xpinyin*{喀}萨巴尊者。」

悉达多却想,国王虽然好意介绍一位尊者给我,但是我访求师父至今,一直都在失望中,他怕这一位尊者将是他的另一个失望,甚至延\xpinyin*{宕}了他悟道的时间,想着,就对国王表示了他的感谢之情:「相信有缘的话,我们还会再见面。」

频婆娑罗王激动地走过来,双手搭着他的肩膀,诚挚亲切地说:「当你得道的时候,请务必来看我。」

就这样,彼此许下了诺言,在往后的六年中,悉达多未曾与乌鲁维拉喀萨巴尊者谋面。后来悉达多成佛后,回忆起与频婆娑罗王的相逢,以及他交代的话,曾往访乌鲁维拉喀萨巴尊者。尊者和他的全体弟子最后都归依了他。

\section{暗中摸索}\label{sec1.7}

\mt{夜空下的感悟}

在黑暗的山中,悉达多的心也如山中的夜一样,一线光明都没有。问题,答案,问题,答案……一天二十四小时,就这样不停地探究着。他不能这样无限制地追究下去了,问题离核心越来越远了,于是,他把某一件事归纳成一点,再从各种不同的角度来探讨这一点。每当这样动着脑筋的时候,他就头痛欲裂,眼睛布满了血丝,思考问题时所耗的精力竟数倍于体力的劳动,就好像一团杂乱无绪的丝线,要一一解开的话,得花多少心神呢?

自出城以来,他一点头绪都没有,为此,他心神不宁,情绪紧张,曾经过严格锻炼的结实肌肤也渐松弛了,出城仅数月,他原本丰润如满月般的脸凹陷下去了,胸背间的肋骨历历现出,整个人笼罩在瘦弱的阴影中。悉达多时常躺卧在地上,仰视夜晚的苍穹,然后深深地吸气和吐气。他这样做,是为了松弛全身紧张的筋肉和神经。

万里无云的夜空,那慈祥、伟大而神奇的景致,真美得叫人透不过气来。满天细细密密的繁星,好像金砂撒落在蓝黑的缎子上。不管悉达多此时是多么苦闷,那一粒粒的星星,在天空中却自得其乐地眨着笑眼,大的,小的,亮晶晶的。满天星空,犹如一个有生命的体,正关爱地俯视着悉达多。

悉达多也目不转睛地凝视着天空,那闪烁的星群好似纷纷从夜空拾级而下,悄悄地走近他身边,像要跟他说什么似地,却又不说什么,那一明一灭的光渐渐近了,带着无限的爱和包容,越来越近了……。悉达多静静地闭着眼睛,感到这些星星,一一都变成了好大好大的光体,把他的身和心温暖地揽在怀里了。他的意识旋转起来,飞快地在光的圆塔中回转,回转跟着光,飞到这里,又滚到那里。刚刚还陷在烦恼中的自己,不知隐\xpinyin*{遯}到哪里去了。悉达多的心识,早已融入了支配天地的慈悲光芒中了。

\xpinyin*{陡}地醒转过来,那道温暖的,满含慈爱的宇宙之光已经消失无踪,留下的仍是衰弱枯竭的肉体。

「那一颗颗的星,原来也有生命啊!」

悉达多被闪在夜空中的星星,深深地吸引住了。他感慨而激动,久久不能自己,回想他三十年来的生活,若说有过平安的时刻,那就是孩提时绕在母亲身边戏耍的时候,自懂事以后,因为身负了未来保卫国家的使命,跟着师父习武,小小的心灵承受着王太子的重担,所以,孩提时能够真正从心灵深处笑出声来的日子,想去苦苦追忆却不复可得。

五位奉命而来的武士,因护卫王太子的名义而留在森林里,起先都未曾惊扰悉达多,后来却都跟着他修行了。有一个时期,悉达多时常与他们攀谈,他们可算是十足的修行者了。抛掉了长矛,丢弃了匕首,身心上感到无比的轻松,他们今后不必再\xpinyin*{驰骋}疆场,也不必忍受着杀人和被杀的极度不安了,这五个人出身于释迦族,他们的教育得自婆罗门,为了保家卫国,文武双全的婆罗门负责实施私塾教育,有关社会的知识,则吸收自外来的\xpinyin{贾}{gu3},至于宗教,工商各业的常识,则有专门的学者来指导他们。当然,武术的课程是不可或缺的,不过在农忙时期,他们也必须加入播种收割的行列,有时更与工商阶级、奴隶阶级并肩保卫国家,守护社稷。

释迦族本名「拘利族」,在印度属名门豪族,本支与分支间的关系与名位划分得很清楚,采世袭制度,国王则是族中举拔出来的长者,但并不表示政体属专制君主,族中每有要事议定时,都是由族中的长者聚集票决,投票时,以削尖之木棒,分别涂以红色或蓝色,赞成时,投出蓝色之棒;反对时,投出红色之棒。而开会时,多半围坐大形的圆桌边,所决议的事情,就是国王也没有否决的权力,所以,我们可以了解,释迦族所采的,其实是一种共和政体,当然,跟今天所谓的议会制,还是有出入的,然就当时而言,这真是极其理想而进步的一种政体。

有一时期,迦毗罗卫国的出入境是很自由开放的,因此有了间谍渗入的混乱事件发生,像扰乱社会的安宁,无故地杀人放火等,使得迦毗罗卫国一时间紧张纷乱而加强了戒备,憍陈如等在这样的环境下成长,所以他们也迫切地想要追求心灵的平静,当他们决定放下刀剑,跟随悉达多专心修行时,心境比在迦毗罗卫城时更明朗更轻快了,以往每日入夜后,总是怀着\xpinyin*{惴惴}不安的心情,深恐第二天就身首异处,睡眠时刀剑不离身的那段日子,是不堪回首的。

现在,他们不再有朝夕对垒的敌人,当然就不必准备什么防御工事了,与他人彼此信赖,得到了心灵上最的大安慰,因为志同道合,他们有了切磋\xpinyin*{砥砺}的好朋友。有时候,因为六根(眼、耳、鼻、舌、身、意)的贪着外境,就在即将失足坠落之际,总有一位善解人意的好朋友会来拉他一把,给他最大的扶持与鼓励,悉达多时常于午夜梦回之际,在他们休\xpinyin*{憩}的地方踱来踱去。他们围着营火,或成「大」字睡着,或躬身曲\xpinyin*{肱},发出了如野兽\xpinyin*{噑}叫的\xpinyin*{鼾}声,安定而深沉地融在大地的夜中。

「这样真好……」

悉达多很感安慰地喃喃自语。

\mt{悟道的起点}

经常,在夜里,往日宫廷中的景象,会不期然地涌现脑际,色彩缤纷,彷彿他的夫人耶输陀罗就近在\xpinyin*{咫}尺,对着他嫣然微笑着。悉达多为那迷人的笑\xpinyin*{靥}神往不已,不禁想起了他俩恩爱的生活,欲念与烦恼如火焰般从心底燃烧起来,他沉浸在温馨的往事中,突然,他猛地抬起头。

「呀,这样不行。」

他痛苦地鞭\xpinyin*{笞}自己的心灵,因心系故乡,意志变得薄弱而不堪一击,为此,他不禁感到汗颜,当初出城的目的是什么呢?把国家、妻儿都抛弃的目的又在那里?他原是为了彻底断绝烦恼以解开人生之谜才出家的,为什么因为见到宫廷来的五个武士,就勾起了\xpinyin*{遐}思?常此以往,后面的路程无疑是险峻而难行的,悉达多认为自己的心变得如此脆弱,一定是有原因的,他静下来反观自身。

人的烦恼,经常盘根错节地纠结在心灵深处,潜伏在某种行为、姿态或形相中,待机缘成熟,它们就在意识中活动起来,使你犹如置身于现实境界般,影像鲜明、生动逼真,令人莫辨虚实,只赖五官和六根生存的人,所接触到的外境,往往被潜在意识所蒙蔽,人们易忽略行为前的念头,而人真正的「自我」却潜藏在无数的念头中生存、呼吸。

人的外在所能表现的形象和作为,犹如冰山所露出的一小角,藏在水底看不见的部分占了整座冰山的绝大部分,就拿对女人看法来说,人因心境与观念的不同,女人存在的意义就各有不同,同样一个女人,有的人觉得美若天仙,有的人想入非非,有的人兴起杀机,有的人觉得\xpinyin*{龌龊}可耻,有的人\xpinyin*{怦}然心动,有的人面对她,就\xpinyin*{赧然消沮},有的人联想到动物,有的人视之如需人照顾的小羔羊,有的人愿意为她奉献一生……。这种种念头,不一而足,这一个个体,以不同的姿态活在每一位接触到她的人心中。

可笑的是,人们就执意于自己的所见与所想,认为那才是真实,有的人终其一生就活在自己的想象中,这种心甘情愿地被自己的想象所拘限起来的人,将永远无法从心田跨出一步,去欣赏更辽阔的天空。整日为国事奔波的政治家,每天为保国而备战的武士,为营利而操劳的商贾,游手好闲的浪人等,哪一个人不是受了自己「一念之动」的驱使而度过他的一生?

人之有苦闷、烦恼、憎恨和悲伤等情绪,就\xpinyin*{肇}因于念头所系的诸般事相,所以当悉达多思念耶输陀罗时,虽然身在山野,但是心却在迦毗罗卫城的王宫内,好像成了宫中的一分子。憍陈如等五人同悉达多一样,年轻力壮,所以聚在一起谈天时,很轻易地就会把话题集中在女人身上,他们常围着一堆柴火,回想在迦毗罗卫城多彩多姿的生活,谈到有趣的地方,也会禁不住捧腹大笑,气氛非常热闹。

这是他们修行生活中轻松的一面,这样谈天的时候,村中晃动的灯火会使他们动心,也有人觉得脱离山林未尝不是好事,在他们中间就有了这种意见纷歧的现象,他们并不怕日复一日地饮食粗茶淡饭,对年轻人来说,他们承受得了这些,但是要克服本能上对女子的欲望,却需要相当大的勇气和毅力。

每天清晨,当公鸡啼叫第一声时,他们就到尼莲禅河沐浴,然后准备上街乞食,每站在屋檐下,就能得到主妇们刚拔来的蔬菜,连同米饭,他们愉悦地一路谈着回到山中。这种单调刻板的日子,过得也颇习惯了,他们已能体会到乞食的要领,已具备了一个沙门的条件了。之后,他们也依阿罗蓝仙人修道的方式来禅定冥想,间或分配些许时间行肉体的苦行与锻炼,或者,再到别的修行场所观摩探访,寻求悟道的端绪,参考他人的说法和修行方法。

修行场上,有些传道的人神气活现,五个人感到浑身不自在,本来大自然是很明朗而开阔的,没有一点拘束,而人,生存于其间,但是这些人用着很不自然的神态,以神的名义来传教,看在一般人的眼里,深感厌恶。他们传教时,或满脸怒容,或口出恶言中伤外教,有时更刻意地标新立异,或傲慢,或\xpinyin{阿}{e1}\xpinyin{谀}{yu2},像这样极反常的言行,徒然使人感觉他们不正常而连带怀疑他们所传的教义。那种不平衡的心理所导致的失常行径,无非是「私我」的意念太强而失去了自我。五个人遇到这种光怪陆离的现象,心里很不是滋味地掉头而去了,悉达多也一直很失望没有一位可依靠的师父。

「也许悟道本就是要靠自己的力量吧!」他这样安慰自己,可是他感到,人终究是一个弱者,沉浮于大海中,急于抓住可依附的东西。

当时的印度,将神和佛一律称为「梵天」,而梵天界是高于菩萨界的一个世界。梵天的心,处于阿罗汉及菩萨界之上的境界,近于如来的心境,梵天也被视为一个能修成如来果位的修行场,如来是「佛陀」的十号之一,具有广大智慧的意义。梵天则负有做如来左右手的任务,而在人世间,负有传布佛法的职责,因此,他拥有如泉涌般的智识和智慧。

总之,在当时的印度梵天的地位是崇高的,而修行者的一致目标,都在于能修得梵天的心,处于梵天的境界。

\section{迈向开悟之道}\label{sec1.8}

\mt{艰苦的道路}

岁月悠悠,悉达多出城至今,一幌眼四年过去了。现在正是雨季。悉达多正聚精会神地沉入冥想中,洞穴阴暗而潮湿。今天不知怎的,心灵异常平静,感到周遭未曾有过的清朗,当他沉得越深,眼前反而越明亮,竟至金光闪闪起来。悉达多感到一道温暖的光,把自己层层围抱起来了,耳边响起了银铃似的美妙旋律,旋即美妙的声音似乎没入金色的光中,变成金色的音符上下跳动着。光越来越强了,不一会儿,汇成一个人的形状。

「梵天,啊——梵天出现了。」悉达多在心中发出了惊喜的喊叫声,四年来的辛勤终于有了报偿,他正惊异不止的一瞬间,金光突然消失了。刚刚回绕耳畔的金色旋律也\xpinyin*{戛}然而止,剩下的只是洞外\xpinyin*{淅淅沥沥}的雨声。悉达多又一次失去了梵天的影子。这四年间,有过数次相同的境遇,每当身临美妙的境界时,他的心都为之一动,而心一动,人又立刻回到现实,金色的光旋即没入黑暗。在光中,他清晰地感受到那种神魂脱离躯体的美妙,宁\xpinyin*{谧}安适。

每遇雨季,他们就得留在洞内。虽然算好了每人一天三个的份量,储藏了不少的芒果和苹果,但是因为天气潮湿,水果还是一个接着一个地腐烂了。他们仍必须冒雨在泥泞的山林间觅取果食,而当雨季过后,往往食物奇缺,他们则必须在几近绝食的状态下过日子。此时,悉达多已与五个武士共同生活在一起了,平时,憍陈如等远远地围绕着悉达多,一边尽守护之责,一边自行修炼,在一年半载后,就相聚在一起,或谈笑,或休\xpinyin*{憩}。悉达多对他们温和而有礼,想破除他们之间的主仆关系,但是这五个人,始终保持着侍从应有的礼节。

雨季将届时,跋提采\xpinyin*{撷}蜂蜜并储存于大钵之中,以供滋补之用,偶而悉达多也会尝食一些,蜂蜜在当时是弥足珍贵的食物,不像现在有养蜂场,有时遍寻山林而不可得,有的人利用蜂鸟寻觅蜂巢,连带地,蜂鸟也非常珍奇,将它放入森林前,先在它的脚上绑上块小红布片,以便于区别,对修行的僧侣而言,每天粗食糊口之余,蜂蜜可以补充一些养分。

洞穴外滴嗒不停的雨,把悉达多等六人阻在洞内,他们就利用这恼人的雨天来修禅定,禅定的次数增多了,绝食的时间也加长了,他们倒要看看这种情况能持续多久,他们往往整个礼拜都用盐和水来度日子。在这种饥馑的状况下,肉体的痛苦最先出现,使得禅定无法继续,口腹的欲望难以克服,就是勉强入定,定中也不时出现美食佳肴,忽隐忽现地,搅乱了他们的禅定。万般无奈下,他们仍挣扎着继续修行,有时在一种屏住气息的修行法中,虽仅忍了几分钟,耳内就轰隆作响,几至神志不清起来。

摩男俱利忍不住说:「殿下,如果我们失去了肉体,那里还能悟什么道呢?若说死就是悟道,那么诞生不就是一种错误吗?」

「你说得很对。我们不可以这样虐待自己的肉体,人既生下来了,就应该好好地活下去,人活着,一定有一个道理。」

悉达多说完,又若有所思地对自己说:「这个谜一定要设法解开来。」

摩男俱利本是悉达多的堂兄弟,他已粗具修行者的风范了。由于经年累月的严格修炼,本很健壮结实的躯体已日形消瘦,倒是两眼\xpinyin*{炯}炯有神,射出奇特的光芒,虽然不过二十岁刚出头,却老气横秋地看似四五十岁的模样。然而跟他交谈两三句话,你仍能看到留在他眉宇间的稚气与天真。

迦毗罗卫国仍时常有人来探望他们,其中车匿是最常来走动的一个,这一次他还是带了衣服,食物和日用品前来,连同净饭王的叮咛以及摩诃波阇波提夫人、耶输陀罗的馈赠,但是也仍被悉达多拒绝了。净饭王这种三番两次,耗尽心血的深情厚意,悉达多为之热泪盈眶。他深知一旦屈服于物质的欲望,他们离真理的路就会越来越远了,他当初就是在不得答案誓不回城的决心下离开迦毗罗卫城的,如果此刻接受父王的厚意,当初又何必远走丛林,自甘清苦?而他们坚定的信念又何以维系?

每当车匿背负粮食和物品前来时,就是悉达多和憍陈如面临最大考验的时候,他们饥肠辘辘,衣衫\xpinyin*{褴褛},面对满目的美食,有股伸手取食的冲动,干净华丽的衣服,也迫不及待地想换上身,然而悉达多都强忍着不敢这么做,他的面庞因内心的交战而呈现不安的容色,自悉达多出城以来,车匿奔波于他和净饭王之间,因为未能达成连系的任务而哀伤不已,这一次,他又不得不收拾好行囊,踏上归程。只见他又带着一颗失望的心,消失在无情的大雨中。

雨季终了的第二天,悉达多想:「跟憍陈如他们共同生活,虽然有个伴,但是在修道方面却没什么进展,这好像是皇宫生活的延续,我们之间存在的仍是王子和属臣的关系,他们仍以往日那种心情来伺候我,要想悟道,非得靠个人的力量不可,别人肉体上的痛苦,你无法承当下来。孩子的病,做父母的也不能代替,同理,悟道也应如此……。」

悉达多于是决心暂时远离他们,准备到另一修行场去。后来,悉达多遇到一位回教的老僧侣,跟他谈到人生的问题,他的论调跟跋伽仙人大同小异,也是主张由肉体的苦刑而消灭身心的烦恼,接着又先后遇到几位婆罗门的僧侣,对于「心」的问题,他们都说不出个所以然来,且比之阿罗蓝仙人所说的,都相差太远了。他只得又回到原地,跟憍陈如等谈他此行的感想,并探讨其他的修行方法。悉达多不在的时候,憍陈如等亦到其他修道场听道,且跟传教人讨论许多有关生老病死的问题,讨论的结果是,除了自身的体悟,别无他途。

\mt{心扉初开}

第五年的雨季又来临了。悉达多在洞穴中修禅定,金色的光再度降临,这一次,金光带来悉达多未曾得见的一种鲜明的绿,在眼前一大片一大片地泛开了,轻快的牧歌由远处飘来,在鲜绿中荡漾,悉达多置身于这一片绿中,放眼静观,只见枝桠上翡翠般的叶,色彩明艳,跃动着绿色的生命,翠绿深深沁入悉达多的心扉。回想往日在城中,跟着父亲在原野狩猎,在疆场驰骋,于林间小憩时,也曾接触过许多种绿,但是都比不上眼前这一片鲜嫩可爱的绿,在绿色的画中,自然在欢笑,绿叶在呼吸,他融入其间,感到无比的亲切,一切都那么真实,这是他生平第一遭的体验。

「好美啊!」他从内心发出赞叹,因感动而全身颤抖,泪如泉涌般沿着面颊\xpinyin*{簌簌}地流下来,他静静地睁开眼,这一次,刚刚禅定中的美妙景致仍然回荡眸底,久久不去。历尽千辛万苦,他的心扉终于开启了,他感到自己在开悟的路途上迈进了一大步。

雨歇了,他仍至各处游化,他的心在欢唱,积压已久的块垒倏忽不见了。他在充满光的世界中\xpinyin*{徜徉},光中的境界随着他的心念变幻无穷,而他也能一次一次品尝到新的喜悦,日子一天天地在他静寂的心湖上溜过去,心湖平静无波。悉达多连日来,如乘游于太虚的流波中,轻灵自在,他深信这种心境若持续不断,终有悟道的一天。

曾几何时,这明镜般的心灵,竟然蒙上了黯淡的烟雾,想来这多半起因于憍陈如等五人和婆罗门僧侣的争论与口角。婆罗门的僧侣惯常带着阶级意识来鄙视其他的修行者,憍陈如等不甘受他们这种屈辱的嘲笑,群起抵御他们的非难。听完憍陈如等的叙述,悉达多的心湖起了波澜。他很激动,且充满了斗志,他也要来评评理。

「婆罗门长者,您们自称是婆罗门种,是神佛的使者,又口口声声指他们是释迦族修行者的仆役。我想请问,这种区别是神佛定的,还是您们的意思?」

婆罗门僧侣很不客气地回道:「在古经典中早就载明了这一点,婆罗门本来就应支配下面三个阶层,总而言之,武士就是武士,奴隶就是奴隶,想要负起神佛之子的使命,根本就是不可能的。」

「如此说来,我想再请教您,大自然是为了您们而存在的啰?太阳的照射,只是为了给婆罗门温暖吗?我们既是不平等的,为什么也感受得到同样的温暖呢?」

婆罗门僧侣为之语\xpinyin{塞}{se4}。憍陈如等围坐一旁,脸色苍白地瞪视着这位僧侣,僧侣无言地离开了。憍陈如气愤填膺地说:「殿下,要不是因为跟您在一起,我真想把他的嘴撕裂。」

在修行者之中,像这个僧侣一样喜欢标榜自己而洋洋得意的人,比比皆是。为了这个争论,悉达多那好不容易敞开的心扉,此时好像又被自己的手给掩上了,感情的激动,能把人带离正道。

「唉---」悉达多仰望星空,长长地叹了一口气。

为了要争长短,把辛苦获得的平静给失落了,像小石落在湖面,泛起层层的涟漪,若想使湖面重新归于平静,要等待良久,……悉达多回想刚刚那一幕。

「只想凭内在的智慧就要悟道,真是异想天开。不修苦行,就想一步登天。呵呵……」婆罗门僧侣讥讽的言词,不屑的神情不时浮现,悉达多的心越来越黯淡了。

「不行,这样不行。」他深深自责。然而因为自责太深,他的心益发不平静了。他越急于悟道,就越不安稳。那颗心好像拍打岸边的水,一波紧似一波,挤迫得烦闷不堪。他的心,僵硬沉重。

\mt{更上一层楼}

我们通常被他人毁谤时,自我\xpinyin*{庇}护的心会油然而生,气愤的火焰从心底燃烧起来。我们很难得冷静地想到,他人毁谤自己,自有他毁谤的理由,有时,你周围的\xpinyin*{勃溪\footnote{意为:吵架、争斗。}},也可能是上天的试炼。上天既要试炼我们,自有他试炼的理由,我们应勇于面对这种试炼,如果说我们没有接受试炼的必要,我们自不会遇到逆境。这一切,都在于自己如何自处,当我们冷静地放宽心怀的时候,我们的心就能包容一切。如果没有宽宏的心,就不可能求得至高的真理。

生气的情\xpinyin*{愫}是心灵平安的大敌,而心的平安,是人的感情最高最深奥的境界。人是情感的具体存在,试想人若失去了情感、情绪或情操,将会变成什么面目,那跟机器人或木偶又有何不同?如果我们对身边发生的杀人、放火、抢劫等行为无动于衷的话,那么同样的,对东西的怜惜心、对人的慈爱心也就无从萌芽。

人因为有情感、情绪和情操,所以特别能感受到大自然的慈悲以及人与人之间爱的行为,人们受了爱的影响,也同样在心底激起爱的回响,如果我们忽视了人的这一层根本的精神作用,就无法继续探讨人的问题了。我们也可以说,感情事实上就是人之所以为人的一个重要因素,同样是感情,有愤怒、悲伤、嫉妒,也有慈悲、仁爱、欢乐等,人若一味任由外在的喜怒哀乐之情来闭塞内在和平宁\xpinyin*{谧}的情感,要想悟道,将遥遥无期。

悉达多一时间受不住婆罗门的毁谤,起而生气地做自我辩护,并企图以犀利的言词来击败对方,让对方接受自己的看法。再没有比想胜过他人更背道的了。自我辩护无非是自我安慰的行径,以为非如此,就无法达到心灵的平静,非自我辩护,就没有人会真正了解自己。其实,想辩护的,是另一个不实的、在相对情境中生存的「我」,这个「我」很虚荣地想站在自己的立场来与他人对抗,如果我们能深入内在那个一直与天地合一的实体,就能发现,它自满自足而无惑于外在的荣辱得失。所以,若站在相对的立场,硬把自己不见得正确的观念与感受传递给对方,当对方不妥协时,这不但刺伤了自己的心,也在对方的心上留下了难以磨灭的烙印。

要行看得透、想得正的「中庸之道」,一定要拥有一颗绝对的、宽广的心,并过着以此心为轴的生活。换句话说,我们当善养一颗无私的,包容万事万物的心,悉达多因为经过一场与婆罗门的论战而一度失去了心灵的平静,然而在反面,他体悟到一颗更难能可贵的「中道」之心。

\section{一口牛乳}\label{sec1.9}

\mt{不松不紧的\xpinyin*{弦}}

出城以来,第六个年头快过去了。悉达多感到一无所获,这种严格的戒律所加诸身体的痛苦,经过了这段漫长岁月的修炼,并未能给心灵带来恒常的平静,在感觉上,好像往日在皇宫地\xpinyin*{窖}里所做的冥想,要比现在的冥想丰实多了。现在是出家第六年的年尾,六年来,时间、精力都抛下去了,然而这两者都无法与「道」的本质相契会:「修行真的有益吗?」他偶而会产生这样的疑问,也可说这个疑问好像是横\xpinyin*{亘}在面前的障碍物,悉达多本很魁梧,然因长期清苦的生活,体重和体形都减至六年前的一半,好像一位\xpinyin*{羸}弱多病的老人,正陷入皮骨相连以待死亡的境地。

缓缓流动着的尼莲禅河,不知洗清过多少修行者的身体与心灵,悉达多在水中,俯视及腰的水面,水面映现出一个陌生的影子。仅仅六年的光景,自己恍如变了一个人,想这六年间,自己的心性始终如一,丝毫未有任何迁易,然而他的躯体,竟让他有着异样陌生的感觉,这个尖\xpinyin{削}{xiao1}嶙峋的外貌,教人看了全无好感,若自己心存恐怖的念头,这种形象的确使他害怕,但是若视之当然,也就能泰然处之,悉达多领悟到这一点。

有一天清晨,悉达多在尼莲禅河沐浴,一只只的鸟儿轻快愉悦地从他头顶掠过,喜孜孜地叫着,随之飞入对岸的森林中。他们好像没有烦恼,也没有痛苦缭绕的余音,使悉达多神往不已,他从河中起来,希望自己此刻也能像那些快乐的鸟儿一样,自由地飞,欢乐地唱,若能一飞冲入太虚,那该有多美妙啊!这样想着,不禁有些嫉妒小鸟了。悉达多舒畅地坐在满植牧草的砂石地上,风中飘来了一位女子的歌声。哦,是歌声乘着风在飘荡,轻柔的音符,疑似天外的仙乐。悉达多竖起耳朵,浑然陶醉在歌声中。歌词是这样的:

琴的弦,转得紧,弦丝寸寸断;

琴的弦,调得松,弦音难为听;

不松不紧,恰恰好,妙音响天边;

随着节奏与旋律,大家来跳舞;

大家围个圆圈来跳舞,来跳舞……

悉达多惊叹不已地迎着朝霞送过来的美妙歌声。那嘹亮的歌喉深深打动了他的心,他的每一根汗毛彷彿都竖着耳朵在倾听,听得他忘记了自己,也忘记了身在何处,冬天的晨曦,将山野染成粉色,壮丽雄伟的旭日,即将冉冉上升了。

「不松不紧,恰恰好,妙音响天边……」这一段歌词在悉达多的心里回荡再三,久久不去。

「我明白了!噢,我总算明白了。」这首歌,好像是上天的启示,他的一切疑虑,顷刻间都化作一个个音符,徜徉在优美的旋律中。

在迦毗罗卫城时,他也曾从宫女处听过这首抒情歌曲,歌词也知道,为何到了出城六年后的今天,才领悟了歌中的大道理,而终于解开了六年来苦思不得其解的谜,对悉达多而言,这位唱歌的女子不\xpinyin*{啻}是一位天女。

\mt{拂然而去}

太阳温\xpinyin*{煦}地俯视大地,照亮了悉达多的心,然后默默地,姿态雍容地登上云座。悉达多站起来,径自向歌声飘来的方向跑去,歌唱的少女,正在挤牛乳,牛儿就拴在树荫下。她年约十六、七岁,虽然打扮很粗陋,但是掩不住长圆的脸庞所透出的高贵气质,悉达多深怕惊动了她,便轻步走过去,在她身后静静地听她唱完。等少女一曲终了,悉达多便轻咳一声开口道:「让我享受到这么美妙的歌声,真谢谢你。」

悉达多轻轻向女孩子行了个礼。那女孩儿很惊讶,但是见悉达多的样子既庄重又诚恳,这才安下心来,旋即却又羞怯地望着地面。

「这是刚挤出来的牛奶,如果您喜欢的话,要不要喝一口?」她柔声地说,双\xpinyin*{颊}随即泛起红晕,不等回答就蹲下身拿起瓶子往悉达多的钵倒去,满满地倒了一钵。

「谢谢你!」悉达多打心底里道了一声谢,而后他在心中不断重复着这句话。不只感谢钵里的牛奶,同时也是感谢那首歌。

「你能不能告诉我你叫什么名字?」

「我叫难陀婆罗。」她很率真地回答悉达多。

「你几岁了?」

「十七岁。」

说到这里,少女抬起头来,认真地望着悉达多,突然间,她像被什么震了一下倒退了两步,悉达多微着慈蔼的笑容,女孩子惶恐地放下手中的瓶子,\xpinyin*{倏}地蹲\xpinyin*{跪}下来。

「师父,您是一位了不起的人。请原谅我的无礼。」女孩子像受到了什么惊吓,慌乱地匍匐在悉达多的脚边。

「难陀婆罗,你不要怕。我从释迦族迦毗罗卫国来,我不是婆罗门,我叫释迦牟尼\textperiodcentered 悉达多,是个出家修行的人,请快站起来吧!」

悉达多说完,难陀婆罗仰起头望着他。

「悉达多尊者,您身上发出一种奇异的光,好像是梵天的样子。」她双手合掌,恭敬而颤抖地说,在远远的那一头,憍陈如等像发现了什么似地频频往这里看。

悉达多向难陀婆罗微微欠个身,就回到五个人的地方,然后喝下少女所供养的牛奶,这一钵味道鲜美无比的牛奶,如一股暖流,流遍了全身,舒畅了筋骨。牛奶好像沙漠中的甘泉,滋养了疲惫而枯竭的身心。看到这情形,憍陈如很不以为然地提高了声音说:「殿下,您是不是不想修道了?修行的人不能吃喝腥\xpinyin*{羶}的东西,您忘记了吗?」

悉达多看着五个人说:「你们想,如果再这样虐待身体,那么还没有悟道,身体己经不存在了。我决心把这不成人形的身体改善过来。」

悉达多的态度十分坚决,五个人很惊讶地面面相觑。

「这么说,您是真想放弃修行了,您原来竟是一个意志薄弱的人哪,近来,我们都感觉您好像离开了我们,总是一个人想事情,如果您决定放弃修行,那我们只好离开您,自己到别的地方修行了。」接着,五个人窃窃私议。

好一会儿,憍陈如站起身来说:「我们打算从今天起就离开您了。过去我们一直像弟子或随从一样地守护您,不过,现在您既不是王子,也不是我们的师父了,您走您的吧!」

五个人突然对悉达多像是陌路人般,说走就走了,而且表示此生不再听他的话了。看着他们这种\xpinyin*{忤}逆凶蛮的态度,悉达多哑然无语。

\section{内心的搏斗}\label{sec1.10}

\mt{侣鱼虾而友糜鹿}

五个人头也不回地从悉达多那儿离开后,沿着尼莲禅河向北走去,悉达多静静地望着五个渐去渐远的背影。他跟他们朝夕相处,同起居,同甘苦;或讨论,或谈笑,既融洽,又亲密。如今他们却为了一口牛奶,而绝断地掉头离去,撕裂了彼此之间的情谊,这是悉达多万万没有想到的。本来「道」是必须自己参悟的,修行也应独自为之,当初五个人来依附他时,虽在许多方面大家都有个照应,少去了许多不便,然而他曾一度有过离去之意,如今见他们绝情地求去,他也只好默默地承受这份落寞之情。他无意干涉他们的行径,也不便束缚他们,只是,单单为了牛乳一事,就不能沟通彼此的意见,他感到非常遗憾。甚至可说,一种莫名的空虚感又向他袭来,忍辱的处境,是多么寂寞。

悉达多起身往乌鲁维拉森林走去。他发现一棵菩提树,树下有根座,于是他立下誓言,从今以后,若不悟道,誓不离此座,这棵大树,树龄约数百年,树荫下可容数人遮荫,在此修行,或可免去被猛兽袭击的恐惧和烦恼,菩提树的叶子硕大而繁茂,可防朝露和雨水,就是在艳阳高照的大白天,也透不进光来,对禅定来说,没有比这更理想的地方了,四周寂静无声,他有种无牵无挂的感觉,在三十公尺外的地方,视界就开展了,明媚的风光,尽收眼底,尼莲禅河静静地躺在他脚下,轻轻缓缓地流着,流着……这里解决了他沐浴和饮食的问题。悉达多坚定的想,在这一生中,这土地将是上苍赐给他的最后修道场了。

他用鹿皮所制的\xpinyin*{漉}水囊放进水中\xpinyin*{汲}水,并收集了些许的盐和水果,一并置于树根旁。然后正对着太阳上升的方向正坐,一面回忆难陀婆罗所唱的民謡,一面沉入冥想。就在昨天,他还和憍陈如等一起修行,一起谈话,今天却已分道扬\xpinyin*{镳},留下他独来独往了,与人相处时,为了顾及相处时的种种问题,他的心神不无旁\xpinyin*{鹜},现在,\xpinyin*{孑}然一身,他不必再顾忌什么,好似刀剑入鞘般,安心自在。

当初逃离迦毗罗卫城的那种坚决求道的磊落心情再度来临,他的心思不再任意被杂念所\xpinyin*{攫}了,这种奇妙的感受,使他觉得就是立刻身亡也是值得的,回想憍陈如等离去前,他还曾抱着未悟道不能死的念头,现在当他从执着中解脱出来而找到自我时,他肩膀上的负荷减轻了,胸襟奔放而开朗了。

当他盘坐在菩提树下冥想时,枝头的鸟儿不时会飞绕他的身旁,有时干脆停在他的肩膀,扯着嘹亮清脆的嗓音吱吱喳喳,他张开眼睛,把放着食物的手掌平伸出去,小鸟就飞过来,用尖尖小小的\xpinyin*{喙}啄食着,鸟儿对悉达多丝毫不存警戒和防范的心,而悉达多也很忘情地呵护着她们,但是,当悉达多猛然惊觉到这奇妙的景象时,鸟儿也似有所觉地立刻飞离了,有时候,野猴子也来了,它爬到悉达多的背上,骨碌碌的大眼睛东张西望的,或从悉达多的背上,纵身一跳,跳至身后的枝干,这样来回跳着,好玩极了。

现在的悉达多已不同于昨天的悉连多了,他离开了一切的执着妄念,找回了真正的自我,也就是说,他从人世间的杂\xpinyin{沓}{ta4}纷扰中解脱出来,真正投入了大自然宽宏慈悲的心怀中。

天色入晚后,他用树枝和落叶烧起薪火,薪火本是用来防范老虎、野狼等猛兽的,然而现在,当他听到响彻山林的吼叫声,已能镇定如恒,不再像从前,每有动静,总是蓦地睁眼,心怀瑟缩,显露随时拔腿逃跑的动摇姿态。如往日一样,当他冥想越深,眼前越亮,最后呈现灿烂的金光,他想:这黄金般的慈光会不会是他的错觉?所以他缓缓张开眼睛,奇怪的是,慈光依旧不变,而且能清晰地看到面前薪火的光,那是和慈光截然不同的。从前他虽在空中见到金光,但是眼睛一睁,金光就遁入黑暗,现在,不论睁眼还是闭眼,他都感到真真实实地身临其境了。

(要推展这颗谐和宁谧的心,该……?如何……?)

悉达多做更深沉的冥想。

当他心中出现许多问号时,围绕在周身的金光亮得更耀眼了,然而又见黑暗由远而近地慢慢拢过来。最后,剩下熊熊的薪火,在黑暗中亮着。

「哎呀,还是跟从前一样。」悉达多失望地自言自语。

\mt{心魔出现}

悉达多重行沉入冥想,他慢慢轻轻地\xpinyin*{阖}上双眼,急于想从先前激动的心波中脱开,不意耳边响起一个女人轻轻柔柔的声音:「悉达多殿下,……」

好熟悉的声音,他不自觉地抬起眼,越过火堆,在黑暗的前方,幌动着耶输陀罗的影子,耶输陀罗满脸思念的神情,定睛地望着他。她穿着透明薄纱质的裙衫,微动着身躯,面露逗人的微笑,并伸出双手想迎接悉达多。

「夜这么深了,耶输陀罗怎么会到这里来?她又怎么知道我在这里?」悉达多困惑地想着,他想,自己跟妻子分别已六年了,从出城\xpinyin*{迄}今,对迦毗罗卫城内的情形已感渺茫,自与五个人分手后,也不知他们的下落,就算耶输陀罗巧遇他们,他们又怎么会知道我在这里呢?

「这真是不可思议!」

「我是不是在做梦?」悉达多满怀疑惑,他用手捏压自己的腿,有着疼痛的感觉,他再放眼仔细地看,看到耶输陀罗的左边也有个人。

「那是歌玛!」第二夫人歌玛,也正伸着手在召唤悉达多。他转移视线,这回看到两人的周围,还有一些跟他说过话的宫女。

「到底是怎么回事?难道说耶输陀罗将离人间,特来向我道别的吗?」

悉达多直视前方,无意站起来。只见耶输陀罗扭动娇躯,极尽挑逗之能事,俨然一个卖笑女郎,姿态妖冶。

「——这是恶魔。」悉达多猛然地惊觉到事态的\xpinyin*{蹊跷}。

就在他这样想着的同时,耶输陀罗、歌玛以及宫女们的影子倏地不见了。悉达多的周围重归于寂静。当他憬悟到对方可能是魔障时,他的身躯立刻被梵天的光层层包围,迫使女人们的影子没入了黑暗。

魔鬼,这不知是什么东西的怪物,会在何时、何地突然出现,也是令人茫茫无所知的。魔鬼犹如寄生于人体的蛔虫,蛔虫在人体内,随时等待\xpinyin*{攫}取食物中的养分,使人体衰弱至死。魔鬼也是如此,它常驻在人的心中,把人心搅和得失去了平静,也偏离了正道。譬如它会增强你的斗狠心。当你与某人作对时,会顿失同情与宽厚的心,只一心一意想见对方倒下去;看到他人流血,就有种莫名的快感;看到别人的不幸遭遇,亦无动于衷。等到回复到本来的意识,又会悔恨自己的残酷而\xpinyin*{惴}惴不安,但是往往已于事无补了。这就是魔鬼在作\xpinyin*{祟}。

照理说,人都有一颗推己之心,谁也不愿意看到他人的悲哀与不幸。但是一旦面对压迫自己的对方,那颗推己之心就会藏匿起来,看到对方的窘状,心中会不禁大喊「快哉」!具有这种幸灾乐祸心理的,大有人在。魔鬼就专找这种人的心去筑起它的窝巢,而恒常地逗留其间,当你自陷这种玩弄他人不幸的境地时,自己的罪业也跟着深重,使身处的环境益加浊乱,当情势造成时,想要脱出你自制的牢笼,已困难重重,或者说是沉沦而不能自拔了。

恶魔是人心系缚所造的恶行中产生的一种恶念,是罪恶的\xpinyin*{渊薮\footnote{渊:深水,鱼住的地方;薮:水边的草地,兽住的地方。比喻人或事物集中的地方。}},当人世混乱,人心糜烂之时,恶魔得以趁隙而入,造成一种恶魔的现象世界,人在此现象界中,益加迷惑,更起而造作罪行,于是境乱心,心造境,如此生生不息。

自有人类以来,恶魔常思统治人心,如杀人、放火、奸淫、掳掠等不一而足,视人的生命如草芥,如果战乱与争夺继续衍生,恶魔的降临,是不问今昔的,它,无所不在,无时不现,它就跟着人的心随伺左右,掀风作浪,暴动战乱。悉达多自知自己之所以看到暗处的幻影,是因为自己内心的一角,还潜伏着男女欲念的残骸,所以他不能受魔鬼的诱惑,及时拉住了自己。这是生平第一次的体验,在克制了欲念之后,他安然地躺卧菩提树下。

薪火发出劈哩啪啦的声音,余烬还在残弱地燃烧着,黑夜过去了,远处鸡啼阵阵,悉达多在鸡鸣中睁开眼睛。他孑然一身地渡过了漫长的夜,一早醒来,有着异样清朗的感觉,心灵上的重担卸下来了,清晨的空气凉凉的,沁人心扉。他就这样仰卧着,凝望顶上繁茂的枝叶。小鸟吱喳的争鸣着,忙碌地从这个枝枒跳到那个枝枒,有的拍翅穿梭于枝叶间。昨晚的那一幕,使悉达多心有余悸,不过今早,那先前困扰心底的薄雾,似乎正一层层地揭开了,此心已无滞碍,感觉不到一丝丝的拘泥。

\section{光明在望——发现心中量度的尺}\label{sec1.11}

\mt{「中道」之理}

跟憍陈如等离别后的第二夜。

悉达多自前一天起已开始了与大自然为伍的生活,他以开阔而自在的心不断反省自身,想到六年来同甘共苦的五个同伴,心中不禁黯然,并怀着些许忧虑。他们那五个人如今隐没在何处?现在在干什么?还是那样固执吗?还在认真地修着苦行吗?

他把现在的自己,和过去三十年来的自己做了一个比较,过去在宫中,茶来伸手,饭来张口,\xpinyin*{睏}了,自有侍婢服侍着睡在柔软的卧垫上,自出城后,严守修行的戒律,忌食\xpinyin*{腥羶}的食物,杜绝豪奢的享受,昨日,还存着一种「悟道前绝不能死」的念头;今天,这一切的心理过程和曾经执持的想法,一日之间,都离他而去了。他此刻正享受着心灵上未曾有过的轻松和愉快,抛弃了一切人心的执着,这种心情,是昨天前一直很陌生的,现在这种来去自在有如菩萨的心情,与过去那谨守「少食、少睡、多禅定」的心境,简直有如天壤云泥之别,悉达多有一股想把这美妙的体验说给那五个人听的冲动。他记得在洞穴中摩男俱利所发的疑问:「如果说死就是悟道,那么诞生不就是一种错误吗?」

当时悉达多只知他这个问题有反驳的余地,然而他不能做具体而明确的答覆。如果是现在,他知道该怎么答覆了。

「肉体有肉体的职分,如果忽视这个职分而想悟道的话,那根本就是邪道,这只不过是一种观念上的游戏罢了。悟道要靠心,此心应存在于健全的体魄之中,体弱多病,意识模糊的人,用什么条件来接触神佛的心呢?我们放眼看着周围的大自然,哪一样生物不是跃动着健康的生命?太阳的光与热,不就是生气与健康的象征吗?太阳不会大声吼叫,也不会横眉竖目,跟神佛一样,热切地照耀大地抚育万物,所以,宽厚仁慈的心胸寄托在健康强壮的体魄中。悟道的大前提,即在于精神与肉体的谐和交融上。……」

悉达多想象自己这样回答五个人。当初摩男俱利虽对苦行表示了异议,但是为了容不下那一口牛乳的行径,也跟着其他四个人远离了他。悉达多想,现在多说也无益了,当时若耐心地以「中道」之理来说服他们,或有挽留他们的可能。那该多好!

\mt{光明与黑暗}

悉达多继续聚精会神地冥想着,在他冥想的世界中,不同于外在的黑暗,而是一片风和日丽景致,黄金色的,圆圆的太阳,放出柔和的光芒,照亮了悉达多的心,悉达多的周遭被金光包围着,与现实世界不同的是,这金光不断地膨胀,充满了无限的平安。受到太阳慈和的光,大自然健康地透出喜气,坦然的斜坡上,布满了新绿的嫩草和细芽,像要对悉达多说话似的,而悉达多也不知不觉间报以温馨的微笑,鸟儿在他耳畔初试啼声,扬起生命的乐章,充满和\xpinyin*{煦}朝阳的小丘上,明亮的光向前延伸,越伸越远,就在宽阔的远方,你仍能见到洒现一地的光,这就是自由,是永恒,也是真实。悉达多仔细审顾自己和自己所站的地方。

睁开眼睛,森林的黑暗满布眼前。对着冥想中的光明和此刻眼前的黑暗,悉达多惊讶不已。冥想中的光明和现实中的黑暗,是\xpinyin*{迥}然不同的存在。这意味着一切事物都有明与暗两面,也可说我们每个人的内心有明暗两部份,同时象征我们人心善恶的两面,对善恶要追根究底,是悟道的本质,悉达多感到脸上热热的,那是泪水!如泉涌般不停流下来的泪水,坚定了他探究明暗的心,他轻轻用手把泪拭去。

太阳是圆圆大大的,在高空,把它的热和光平等博爱地分赐给万生万物,充满金色的心的世界,太阳毫不吝惜地平分慈光,然而为什么心中的太阳也有被黑暗取代的时候?天上的乌云出现时,太阳的光被遮蔽了,同理,心中的太阳,被自己心中的黑暗给遮蔽了,人的困苦和悲哀是心中的黑暗所造成,同时两者也造成了心灵的黑暗,就好像掩映湖面的月亮本是很美的,但是湖面突起波纹,把月亮美丽的影子歪扭得变形了。

人的心,本也可反映明媚的月影,平静的心湖上,散发深秋浓浓的诗意,如果长期保有心湖的平静,不使泛起纷乱的波澜,此心将永浴在大自然的恩泽和美丽中。在母亲身边天真嬉戏的孩子很可爱,既无痛苦又无悲伤,然而跟着岁月的增长,因环境、教育和交友的各种因素下,他的自我意识萌芽了,他纯真的心灵蒙上了一层层的阴影,他的心湖受着周围的影响而泛起一波波的水纹,「自我保存」的念头因而塑成了他的身与心。而困厄,就产生于自我意识的觉醒,自我意识觉醒了,心湖不再平静了,人在不如意时,常会意气用事\xpinyin*{悖}于事理,而在听到他人赞美自己时,又会喜不自胜洋洋得意,原因很简单,一般人很少低下头来看看自己,世间的许多混乱,源于问题的症结被隐埋,人人以自我为中心来思想来行动,各种脱了轨的现象就是这样产生出来的。

大自然替自己涂上了各种颜色,这样山水草木的色彩调和了,但是人世间呢?到处充满了不调和的色彩,人与人之间的纷争,自私自利的行为,种族间的歧视,婆罗门族的优越等,处处都可看到矛盾的景象,如果我们不诞生,我们就可免去这些苦痛,可是我们既诞生了,这中间一定存着「生」的目的和使命。那么,生的目的是什么?生的使命又是什么呢?

每个活着的人,免不了病痛,也终将老死,想要逃避这些,不\xpinyin*{啻}是异想天开,不管你是贵如一国之主,还是贱如受人驱使的奴隶,都是赤裸裸地来到人间,又双手空空地离开人间,什么名位,权势,财产……都得留在世上。虽说如此,人类对世间的一切所抱持的欲望,仍如无底的深渊,永远无法填满,透过五官而能感知的现象世界,是无常而不实的,一个人就是拥有世界上一切的东西,他仍然挣脱不出烦恼,说穿了,人都是欲望的玩物。人生不过是痛苦的连锁,如果有谁自认自己的人生没有困苦,那他充其量不过是向现实低头了,或逃避了现实,再不然就是自己遏制了欲望的横流,一个能知足的人,自然能自得其乐,不过,他们的内心深处,终究抹不掉苦闷烦恼的阴影。

「人的诞生是一种错误……?」你不禁会发出这样的疑问。人带着欲望和苦恼度过他的一生,一件苦恼往往引发另一件苦恼,人就不自觉地陷入苦恼的\xpinyin*{窠臼},有时竟是反覆着同一件苦恼的事,这是人生最大的不幸,解脱之道是什么?万人可共同遵循的解脱之道又是怎样的道?人们在一清早,打从睁开眼睛起,就开始思这想那的,烦恼就在每一动念间,事实上,即使是在睡梦中,仍免不了被日间的问题与烦恼所缠缚,固然有时候,人在睡梦中,浑然忘记了自己,就连眼、耳、口、鼻等也失去了日间的作用,有时候在睡觉时,或可逃避烦恼于一时,不过,待第二天一觉醒来,一切现实(或说残酷的现实)又都争赴眼前,使你不自觉地再度烦恼起来。

就是这一点,苦恼并非来自肉体,而是完全系于一心,由于「念头要动」这一种心的作用,产生了无尽的烦恼与苦闷。这颗心,藉着肉眼接触到外界,心性有了群体生活的体验和知识哲理的学习,以致对身边事物产生了美丑善恶的分别与评价。

有一次,有两个武士因迷路而来到山中,这两个吃了败仗而忍着饥饿的武士,为了寻找吃的东西,想不到竟迷了路,两个人饿眼昏花地走着,突然看到一棵很大的芒果树,可惜上面只剩下一大一小的两个芒果。其中一个武士伸出手正想摘下那个大芒果,不料另一个武士说:「是我先看到的!」说着就抢先摘下来,并且忙不迭地把小芒果也一并摘下来占为己有,两个人因此发生争执而大打出手,那个抢芒果的武士体力和武术都略胜一筹,所以不费吹灰之力就扭住另一个武士的脖子,把他压倒在树边。打输了的那个武士虽然被制服了,可是很不甘心地握紧了拳头斜瞪着眼睛,看那个胜利者狼吞虎咽地吃着芒果。

这整件事,欲望是指使者,如果这两个人都能稍稍为对方着想,不要太过贪婪,两个芒果可以平分而一同充饥,这不是很调和的景象吗?然而两个人都争先恐后地想要独占芒果,结果引起一场打斗。

「不松不紧,恰恰好,妙音响天边。」如果两个武士都具有一颗合乎「中道」的心,这场争夺战就可免除了,两人都想活命,偏偏两个人都只把自己的性命看得最重要,而忘了另一个人同样想活命,如果你阻碍了对方的求生之道,对方焉有不起而抗争的道理?虽然对方一时屈居下风,他怀恨或俟机报复,都将造成你人生的缺憾,试想,你一人独占两个芒果而把饥饿待毙的同伴留在一边,这是多么不合情理。

\mt{八正道}

我们透过五官得到自身的体验和知识,所以容易依自己的欲望来评鉴一般事物,因为如此,人类的社会产生了矛盾和纠纷,跟大自然所昭示的「中道」越离越远了,是非不明,歪曲事实,国与国之间的争战,推究根本的原因,还是在统治者的欲望与野心上,为了发展本国的机运,或拓展本国的领土,而不惜大动干戈侵略他国,由于争战,免不了被套进胜败的圈子里,若不实时醒悟,将注定沉浮于苦痛的漩涡之中。

人不经历谐和的生活,就无法矢志于「中道」,也就不能抓住内在那颗真正的心。因此,我们必须养成对事理做正确判断的习惯,要具备「正见」的眼光,努力地使自我超脱于私欲之上,就事态的真实情况来做评价,如果不是这样,人人都站在自己的立场发言或行事,就难免有失正道,我们若能抛掉男与女的立场,老与幼的立场,得与失的立场,依此类推,就能以一个超然的我和一个大自然一分子的我来看事理,看对方,看周遭的一切,显赫一时的官场人物,一旦退休在野,他的感觉会如何?有人为了公司的利益,不择手段地设计出一套骗术,使后进人员无所适从;也有人从一个嫉恶如仇的检察官,一变而为一个替罪犯辩解的律师,在法庭上变换了主客的地位,他们的心究竟在何处,又如何转变?这些情况我们就不得而知了。

事相和谐的基本要件,在于对事理有正确的评价。每一事相的背后,都存在着导致事相的因子,如果是跟自己有直接关系的因子,就应该省察自己的心。这是很重要的一点,因为透过肉眼所观察到的外界,其形象的曲直皆取决于内在心眼的量度,心眼不澄明,未能从各种角度来透澈分析,难免会被私欲牵着走,终致「昧于良心」,把外在的现象歪曲了。

人的心就像一面镜子,如果灰蒙蒙的,景物都照得不清楚,所以要靠「反省」来拭亮它。而「反省」时,仍需要一把「中道」的量尺。一个人如果每一念头之起,都以自我为中心,就不能避免与他人的冲突,心中的任何一个念头,待机缘成熟时,便会发展成事实,或成为人格的一部份,一个仁慈和蔼的人,能使与他接触的人都感受到他的特质,而他也能得到相同的回应,我们的食物、食器、房子、家具、衣饰、车船、道路、火箭、宇宙飞船等,哪一样不是由人的「念」制造出来的呢?所以「想」之一事,作用是很大的,如果一味以自我为出发点,就无暇顾及他人,那么会与他人发生\xpinyin*{龃龉}是可想见的,会过孤独不悦的日子也是必然的。

说话也是一样。

「葫芦是长不出骏马来的……」这是古人的明训。换句话说,你种什么,会收什么,从来没有例外,你若用轻蔑的语气对人说话,或是出之以粗陋的态度,日久之后,连自己的心都会被侵蚀。对方的自尊既被剌伤了,他必不会坦然,敌对的局面因而形成,言语来自灵魂深处,是心灵的波动,像谦虚、同情、温柔、鼓舞和慈蔼等言语,是人类在群体生活中所不可缺的润滑剂。

悉达多对于心灵的光明与黑暗,一路沉思到这里,他发现了迈向真理的三条正道:正见、正思惟和正语,是不是只有这三条路呢?悉达多仍然在思索,他立刻又振起心神专注地寻求其他的正道,就这样,在他思想的底层,突然浮上来另五个意象,那是五个正道:正命、正业、正精进、正念和正定。自昨夜以来的悉达多,问题一来,皆能立刻迎刃而解,不再像从前,一个问题往往要思考竟月而仍不得其解,现在他能改变问题的着眼点,每当问题出现脑际,答案也会如泉水般涌现,以往不明白的事相,如今都能了然于心了,好像五体内的某一个地方,正吊着这样一个智慧的袋囊,智慧从这里源源不断地融入四体,他自己也莫名所以。

他首先想到「正命」和「正业」。正命所指乃正当的赖以为生的工作;正业则是合乎常道的行为。两者都能带领我们过健康而愉快的生活,工作更能增进生活的乐趣,发展个人的人格与才华,是自然的恩赐,所以,能拥有一份正当的工作,要能以虔敬的态度和感激的心情来努力为之。能抱持感激的心,就能发而为报恩的行为,最后就能行布施的大善来利益众人。人世间的和乐与否,系乎人们对于工作的存心与态度如何,以感谢和报恩为轴,并辅以勇气,努力和智慧的,这才是工作真正的全貌,这样的工作,才能为世界带来丰盛和美满,合乎常道的生活,就是具有人生目标和意义的生活,人的生活一如大自然,也有和谐宁谧,人与人之间互相帮助,互补缺失,笑脸相迎,甘苦与共,要达到这种境界,首先每个人要能做好身心的调和工作,要懂得跟自己和平相处,发展自己的特长,改善自己的缺点,自己感到心安理得,行止\xpinyin{了}{liao3}无缺憾的时候,周围的一切也都会跟着美好起来。这一切,还都必须仰仗「严以责己」的态度!

「正精进」,就是努力于将亲子、兄弟、朋友、芳邻等角色扮好,做一个人所该做好的事,人往往要经由大自然和人类之间的密切关系,才能憬然自觉到「人」的问题,凡认为除了自己就没有大自然或其他人物存在的人,是最愚蠢不过的了,外在世界的人与物,能提供认识自己的资料,人的人格在人际关系中显现,人也得以藉助与外界的接触而提升自己的性灵,人在扮演各种角色时,自己的内在受到环境的历练,促成自己独特的人格。所以说,我们周围的种种,都是上天慈悲的恩赐,迈入正道所需的「精进」,是人的特权,也是神明的恩惠,人之所以为人,且不同于禽兽的,就在这可贵的求道精神。

悉达多对「精进」不断穷究时,思想突然触了礁。这个礁石,就是留在迦毗罗卫城的亲人和族人。他为了悟得真理,不惜离乡背井,抛妻别子地来到深山野外修苦行,行禅定,然而他的行为是否离了磨炼自己的灵魂正道呢?他此刻未能在城内克尽人道,自觉未能将人的角色扮好,是不是自误前途了?于是悉达多对宫中的生活,出家的动机,以及现在的自己重新做了一番\xpinyin*{剖}析,如今有了审度事理的量尺了。如果自己的思想和行为不能合于这个尺度,那么虽说已很懂道理,事实上与全然不懂又有何差异呢?今夜这恼人的问题就待来日再想吧!悉达多继续对「正念」做进一步探究。

「念」就是一种愿望,想望或欲望。假如说谁的生活中不存任何愿望,那是万无此理的,人们都是希望日子能一天天地好起来,今天要比昨天更美好,这才具生存的价值,没有明天的人生,就是没有生命的人生。唯有强者,才知道如何好好地活在今天,计划将来,终其一生都在希望中过活,不期待美满生活的人,永远不会拥有美满的人生,但是话又说回来,你的期望中,充满了自我的念头,以为只有你自己才希\xpinyin*{冀}幸福,那么你与他人之间的谐调关系注定要崩溃的,所以「正念」的修习,在于控制无限制伸展的欲望。「知足」是一条控制欲望的途径,同时也是一种生命自觉下的体认。悉达多想到这里,顺便分析了「意念」和「祈祷」的意义。

「意念」和「祈祷」能为我们带来身体活动的能量,思考的行为和思考的能力,皆来自人体内可供创造的力量,我们在睡眠中,活动的能量也需要休憩,同时为了补给能量,人的神识会暂离躯体。所谓神识,一般人称之为灵魂,关于灵魂,有人否定它的存在,但是自古至今,无论史书或名人的笔记中,都曾证实灵魂的存在。灵魂是一种带有个性的意识,平时都在个体处于睡眠状态时才离开躯体,所以当人在熟睡时,对周遭的变迁是毫无知觉的,肉体失去触觉,鼻子失去嗅觉等。待灵魂再度进入肉体,人才能恢复知觉,或讲话、或思考,实在都不是肉体所主使,而是那有了知觉的内在能量所提供的能力。所以,当人行「正念」或祝祷时,自然也不是外在的躯壳所为。

人的意愿,如想跟某妙龄女子白首偕老,想顶天立地,想发展事业,想安享晚年,希望子女成龙成风,无一不是灵魂的活动,只要是人,没有人不是带着这种目的意识过活的。也正因如此,人类有日新月异的文明与文化,才能享受到进步安乐的社会生活。然而这并非表示灵魂一定主宰一切,问题是,灵魂寄居在躯壳内,往往不得不受躯壳中感官上的欲求所支配,而变得很自私,在人群中以自我为中心,任何事都先想到是否于自己有利,将人我的界限划分得很清楚,深怕自己一旦吃亏就无以为生了,有这种心,就不能指望与他人和平相处,也更谈不上济世救人了。所以,人之所以为人的目的,即在于求人与人之间和谐相处,和谐相处当建立在人与人的互惠条件上,诸如相亲相爱,互助合作,分享快乐……,人的目的意识有聚集在这一焦点上的必要,「正念」必须在这样一种谐调和乐的境界中进行,而这时,正念也才可能将自己带到至高至善的境界。

「正念」行之于工作,工作则对社会,对自己本身,以及对家庭都是一种润滑剂,同时不但在生活上能有所保障,也能充实生活的内涵。不管工作是尊贵卑贱,只要对社会是有贡献的,都应忠心尽力地做好它,这才是工作的本质和显现价值的地方。不管今天是别人鞭策自己去做的,还是在自由意志下做的,重要的是一颗好好做事的心,更甚者,对工作中所得的利益和报酬将如何发挥其大用,也是人的心念中非常可贵的一环。有很多人为了满足一己的私欲,不惜愧于职守,把工作的神圣意义都给亵渎了,能使一个人的意念不偏不倚地循正道前进时,「知足」的态度和行事原则是很重要的指针,因为人是贪得无厌的动物,而「知足」的体认是使欲念升华的起点。

「祈祷」是有必要的,悉达多想。祈祷就是感谢的心所外发而成的行为,这种行为是合理生活的基石,人生活在黑暗中,不知道明天,不能测知命运,亦不能预知祸福,中间隔了一道墙,就难以揣度隔壁的人在做些什么,好像行走在漫长的旅途中,前不见店,后不着村的,伶仃孤单。此时,若在途中遇到一两位微笑相迎的路人,你会不由得生起感激之心,在道旁看到绽开得美艳可爱的小花,你也会涌起赞叹的心。同样的,在人生的旅途中,你能健康、快乐而明朗地向前迈进时,你怎能不感怀涕下呢?

感谢的心使你产生向上苍祈祷的心愿。而一般人把「祈祷」局限在很小的范畴内,只在心中有所祈求时,才到寺庙中,虔诚地合掌跪拜,燃香许愿,口中念念的,不外是祈神降祉一类的祷词。当我们时时不忘向天祝祷,而不是只在求助时才想到上苍时,就表示此心常求圆满和谐的生活,并为这种生活尽了最大的努力,这时的祝祷和心愿,一定能使我们如愿以偿的,因为我们抛弃了我执的观念,印证了天人的心,祈祷可说是与天人的对话,许多「奇迹」会发生,都源于有这样一颗与天地合一的心。就人的生活形态来看,一个人的生活领域中,没有神明可让你祝祷,那是无法想象的,自古有许多独裁者企图铲除人民心目中的偶像,事实上,如球之有反弹力,你越压迫,人民的信仰会越深。

向来「祈祷」都被人自己的欲望所操纵了。很多人以为,祈祷有如天官之赐福般能一一实现你的愿望,以为念经可以得什么福报,或祈祷就能得救等等,这些都是以\xpinyin*{讹}传讹的说法,事实上,「祈祷」非如人们所想那样单纯。如果说完成目标的意识是创造活动的泉源,则祈祷就是目的达成后的感恩和报恩的心向,且能使你的心与神明互相交流。「意念」和「祈祷」都能振起我们内在的能量,这一点可曾清楚地了解到?

至于「正定」,悉达多至今也能理出一个脉络了,正定的根本就是自身的反省,而反省是人与光明世界的一个桥梁,想要从嫉妒,恼恨,\xpinyin*{詈}骂等执着的情绪中脱逃,除了反省,别无他途,日复一日地反复自省,才能使自己的身与心达到统一和谐的境地,进而促使自己的心和宇宙大化的心合而为一。当我们把心虚悬着而不做任何自省时,诸类魔王就会趁虚而入,支配你的心,等于把自己的心出卖给魔鬼了。

悉达多将他三十六年来的人生,依据以上的八个尺度做了一番剖析后,决心涤净杂质,让自己真正澄明的思虑浮现出来。悉达多正在想八条正道该如何一一地走顺之际,枝头的\xpinyin*{\xpinyin{啁}{zhou1}啾}声告诉他,天将拂晓了。

他感到有一个小动物走近他。是一只小鹿!它正用自己圆黑的鼻头凑近悉达多的耳边,鼻尖触到了悉达多的耳垂,吐出轻柔的鼻息,一股暖流通过悉达多的全身,悉达多任由这小鹿撒野,自己一动也不动地继续闭着眼睛。一只小雀儿这时也飞来停在他背上,然后又从他背上跳到枝枒上,不一会儿又振翅向空中飞去,动物们这种毫无猜疑而纯真自然的调皮举动,温暖了悉达多的心,并抑制了他一夜未眠的疲倦。

悉达多利用夜间思索的时候较多,尤其是午夜一点到三点。本来这段时间,是花草,树木,动物和人们恢复白天以来的辛劳而酣睡休憩的大好时刻,同时也是大气寂寥平静的时刻,大地就在这时候静静地与天上繁星促膝谈心。悉达多在此时冥想,发现藉由自己往日的知识和体验,一切问题的答案都能源源而来,自己的精神与心灵已达到奇妙的统一状态,杂念不再来纷扰他了。

自昨夜以来,悉达多已将禅定和冥想的时间集中于午夜一点到三点之间,并竭尽全力追逐深藏心底的恶魔,所以当大地苏醒,万物重新活跃之际,八条人生的正道也鲜明地呈现在他面前了,长久以来的疑虑已一一离去,夜间的冥想带给他无比的快乐,无限的智慧涌现心底,他明白,不久的将来,他会到达目的地的。

\section{除去心中的阴霾——童年与出家}\label{sec1.12}

\mt{家庭的亲缘}

悉连多在菩提树下迎接第三天的夜。

今夜,他要从自己呱呱坠地起一直到现在的种种思想、行为和阅历,做一次深入的剖析,于是悉达多端坐闭目,打开心眼,将整个心静定下来。在四野的\xpinyin*{唧}唧虫声中,不时夹杂着野兽的咆哮声,长久以来早已习惯于跟四下的各类声音为伍了,心里无比地安和舒泰,不过这里可比丛林宁静多了,自然也没有城市的烦嚣和\xpinyin*{嘈}杂。在这里,人的心能迅速趋于静定。

首先,他想到一个人出生的问题。昨夜在定中,悉达多领会了「生」的部份事实,每个人都是由一道光的引导而降生到人间,经由双亲的缘而获得肉体,那么双亲是由谁决定的呢?事实上还是自己决定的,我们跟双亲都曾在前一世,或前此的生生世世结下了不解的缘,彼此在心灵上能互通消息,就是这种缘份所形成的力量在冥冥中左右了每个人来生的去向。纵观人世间的生活,我们可以了解到,人与人之间的交往,从陌生到亲密,这期间需历经漫长的时日,彼此的感情在相互交融之下,才能推心置腹地在人生的旅途中互相扶持同甘共苦,也因为如此,关系益形坚固而密切,亲子间的缘份,也是在前生的种种恩恩怨怨的情况下结成的。然而一旦有了血肉之躯,人的心困在五官的欲乐贪婪中而忘失了清明与方正,更无由得知自己来自何处,又将归往何方。

人的内在有一不灭的「本心」,从亘古以来就连绵不绝地延续到久远,这期间受生,舍生,沉浮于生死的川流中,这颗本心,受了躯壳的累,喜怒哀乐于其间,穷通寿夭于其间,因而也就造作了种种的「业」。这些业储存于「本心」中,形成每个人独特的个性和行为的模式,这时个人的个性又成了他自己处世的指引,从而再造许多影响他一生的「业」,如此循环不息。

从这里我们也能明了,为什么做子女的虽来自双亲的躯体,却有时不能与双亲同心。不,应该说孩子离开母亲的怀抱后,就一天天离父母越来越远了,所以说看似关系最密切的父母与子女之间,因为本心与外境的交融,在心灵的世界中,他们彼此在相离最远的地方各自生活着,他们本是由于亲密的前缘而结成亲子的关系,却又因为种种因素,在心理上有着相当的距离,这现象是很出人意外的。

的确,家庭有时反而是纷争最多的地方,因为家庭中的成员朝夕相处,接触频仍,发生\xpinyin*{龃龉}的机会当然多,加之每个个体自有其前生带来的固有习性,以及各成员间前生结下的恩仇,使得家庭成为最错综复杂的结合体。知道上面这个道理,则父子相争,兄弟相斗的情景就不足为奇了。

\mt{童年点滴}

悉达多又想到迦毗罗卫城内的生活,小时候,他有许多专门照顾他的宫女。贵为太子的悉达多,是迦毗罗卫国王位的继承人,凡在他左右的人,当然都是一些卑躬屈膝而唯命是从的人了,净饭王年近五十,才盼到了这么一个\xpinyin*{嫡}传的儿子,对悉达多爱顾备至,自然不在话下,他时常带着悉达多出游,并为悉达多建造了很漂亮的花园与行馆,让他嬉戏于其间。只要他开口,什么好玩的东西都立刻会堆在他面前,这样无拘无束而随心所欲的生活,在悉达多幼小的心灵上播下了傲慢的种子。他没有机会跟宫外的孩子接触,每天绕在身边陪他玩耍的,都是一些宫女,再不然就是身份相近的武士们,他想出宫游乐时,车匿立刻就备妥马或象,在旁恭候着;悉达多心情不好而面露不悦时,车匿就会心慌意乱手足无措,如今车匿己近暮年,正住在宫中的马厩内。

悉达多记得自己六岁时,一位陪侍的宫女因见到他对波阇波提夫人太过撒野,于事情过后,就忍不住对他说:「太子殿下,您母后养育您是很辛苦的,请不要这么蛮横,做个乖孩子吧!」

悉达多一向被骄纵惯了,此时闻言就微红着脸对宫女怒道:「你说什么?你只是一个奴隶,不要管得太多。」

「太子殿下,您要知道,您亲生的母亲在生下您七天后就去世了,您现在的母后不是您亲生的母亲呢。」

悉达多听后虽然非常激动,且顿时心中充满了疑虑,不过因为年纪小,很快就不把这句话放在心上了,虽说如此,悉达多有时独处时,不免会想到:「我真的不是母后所生的吗?那我要去问个清楚。」

但是继之一想,那么温柔和善的母亲,脸上总是\xpinyin*{绽}放着温馨慈爱的笑容,怎么可能不是自己的母亲呢?于是在他小小的心灵中,暂时按捺了任何疑虑,有一天,他偶尔听到隔壁房中父王和一位大臣的谈话。

「……悉达多殿下现在已经俨然一副国王的模样了,虽说没有母亲,王对他的将来是可以放心了。」

「嗯。他能把波阇波提夫人当做亲生的母亲,而夫人也能把他当亲生儿子,这情况的确很教人安慰的。」

「正如王所说的,波阇波提夫人非常温柔贤惠,把悉达多殿下照顾得无微不至,殿下的成就是不可限量的。」

「他成不成器,一定要等到长大以后,才能晓得,现在还言之过早。」

偷听到父王的这些话,悉达多愣在那里,同时心中涌起莫名的不安,他感到全身已经木然麻痹,眼前一片昏黑。

「原来真的是这样啊!」

悉达多心慌意乱地不知如何是好,拔起脚来飞奔至隔壁,跪到净饭王的座椅边,抱着王的膝盖哭道:「父王,我的亲生母亲呢?」

净饭王对悉达多这突如其来的举动和问题震惊得久久说不出话来,他勉强提起精神,轻抚悉达多的头,尽量以一个慈爱的父亲所应有的语调安慰道:「你的母亲不是在吗?」

「父王骗我,她不是我真正的母亲,我的母亲到底在那里?呜……」

「悉达多,你想到哪里去了?谁说现在的母亲不是你的母亲?快把眼泪擦干,打起精神来,不然不像太子了。」净饭王用充满威严的口气说着,就把悉达多的两手拿起来,擦掉他流了一脸的泪水,然后定睛凝视着他。

「父王,请不要再瞒我了,我已经都知道了,拜托父王,请告诉我吧!」说着,悉达多的泪水又涌了出来,他任由泪水纵流满面,只在一边苦苦哀求着。

在一旁一直未开口的大臣,此时心中暗暗着急:「糟糕,事情不妙了!」

他真担心,不知事情会演变成什么样子,正感困惑间,只听净饭王缓缓地叹了口气说:「你的母亲现在正跟星星住在一起,她正在那个充满星星的世界看顾你,她在生下你一个礼拜后就去了,至今已经六年,现在这个母亲不是也一样很疼你吗?她养育你也是很辛苦的,不要再哭了,好不好?不要忘了你是一位太子。」

说完,就把悉达多抱坐在膝上,怜惜地用手来回轻轻抚摸悉达多的头。

「父王,那您是不是我真正的父亲?」

悉达多抬起泪脸,不放心地又问净饭王,他仍抽抽搐搐地,非常伤心。

「当然啦,傻孩子,一点都不错,你是真正的太子,没错。」

净饭王连连点头说,并注意看悉达多的表情,然后摇摇头笑了起来,刚刚还站在身边的那位大臣,不知什么时候竟不见了,可能他不忍心看到悉达多伤心痛苦的样子,这真是一个令人感伤的场面。

自此以后,悉达多对波阇波提夫人的态度改变了。他不再像以前那样毫无顾忌地调皮撒野,甚至对母后像对族中其他的长辈一样,拘谨而多礼,同时,他对整个人生的看法也有了很大的转变,因此,他之所以对人的生死问题如此敏感,与这件事不无关系。继母的存在,以及父王与继母间的关系等等,在他幼小的心灵上,留下了很深刻的印烙,另一方面,悉达多周围的人多少总会因为他是个没有母亲的孩子而加倍地纵容他,问题就出在这里,这样一来,很快就造成了悉达多孤高傲慢的姿态了。

悉达多回忆到这件事时,不禁想到:「那时候要什么就有什么,世上简直没有办不到的事,这种傲慢孤高的心理,原来就是这样产生的啊!」

此刻在菩提树下,一个人清净地坐禅,不断做着自省的工夫,想到自己有时候明知有些事不应该,却偏偏还是故犯了。原来有些坏习性是很早就积下来的,这么一想,他不禁觉得在宫中那段日子,是既值得怀念,也极其可悯的。在他得知生母的真实情况后,他有些无理起来,经常会问父王一些难以回答的问题,或者对身边的侍卫做无理的要求,最后连一向呵护他的波阇波提夫人,也忍不住要斥责他,悉达多有时不高兴,会一两天都不进饭食茶水,或干脆躲进地窖中,久久都不出来。

他在地窖中,常会想:「为什么母亲会死呢?」

有一次在御苑中,他的堂弟哪玛想把抓到的鸟儿拿来给他玩,他竟忍不住呜咽起来,并要求哪玛把它放走。叔叔(哪玛的父亲)见悉达多哭得伤心,就帮着说服哪玛,对哪玛谈了许多因果的道理,最后哪玛被说动了才把鸟儿给放了。不知怎的,他变得很多愁善感,对于他人残酷的作为常于心不忍,这使他每次都提出坚决的异议,且一旦有了某种主张,就非坚持到底不可。这种怪异的行径,常会教人觉得他无理取闹,再说他对波阇波提夫人那种过份温驯的态度,连自己此刻想想都觉得太过份了,那根本就是一种执着的念头在作祟。

悉达多反省到这里,歉疚之情油然而生。他想:如果是自己处在波阇波提夫人的立场,将有什么感受呢?「那我一定会觉得这孩子不知好歹,非得好好教训不可。」他下了这个结论之后,领悟到波阇波提夫人的养育之恩,实不亚于生母摩耶夫人的生育之恩,于是对自己一向所持的幼稚念头感到赧然,悉达多对自己童年的种种往事,仍继续追忆下去。

他这种任性傲慢的心理,后来慢慢推广到其他方面而有增无灭,当他开始跟婆罗门求学时,就一直有种好强争胜的心理,在武艺的修炼中,也总是苦苦地想要比他人技高一等。以往每逢梵天王诞辰日,全国上下都会举行各种活动来热烈庆祝这个佳节,每次都免不了有各项的竞赛,诸如击剑、弓射和相扑等等。在击剑与弓射时,悉达多的表现的确是很卓越的,虽然与赛者中多的日疋身形魁梧之辈,却往往只能望而兴叹,不过在相扑竞赛中,面对庞然的竞争者,悉达多就常感到力不从心,也许因为他有着太子的尊贵地位,每在比赛时,对方都在有意无意间让他得分,最后让他夺魁,在这种每战必胜的情况下,更增长了他的傲慢心。

悉达多的剑术是师事当时的一位高手提布地,因而从提布地这里把握了剑术中「气、剑、体」交融的诀窍,在这方面,迦毗罗卫城内的同侪中无人能出其右,事实上这也是环境造成的,悉达多身为王位继承人,又适逢乱世,为防范敌人的突袭,必须经常提高警觉,所以防身攻敌的武术是必备的,在他十五、六岁时,武术就已经驰名国内外了,当一个人的武功达到某一境界的时候,他必须连带能应付和适应多种多样的情境。我们可以想见,悉达多能以养尊处优的金玉之身,经年过着餐风饮露的苦修生活,与他身怀绝技是不无关系的。

\section{出家后}\label{sec1.13}

\mt{动机}

等悉达多年纪渐长,常常会风闻到宫中的大小诸事。有一次他偶然得知一件消息,这消息足以使他寝食难安,那就是波阇波提夫人蓄意让自己所生的难陀来继承王位。悉达多之所以会感到不安,是因为他想:「如果我不退让,势必会陷父王于窘境,而母后也因此必会纷扰不安,倒不如我现在就隐退,事情或能圆满解决。」

这或许就是促成他出家的主要动机吧,当然,我们在前面提过许多因素,那些因素都曾促使他在心田播下出离的种子,像局势的动荡不安,社会的贫富悬殊,种姓间矛盾的存在等,甚至于围绕在他身边的\xpinyin*{嫔}妃宫女,为了争风吃醋闹得鸡犬不宁,使问题益形复杂。他看着身边纷乘迭起的现象,意识到人生的虚无,对人生早就充满疑惑了,只不过当很切身的王位继承问题出现时,更坚定了他出家的意念,此外,也可以说,他对生母摩耶夫人的思念与爱慕,使他急于想寻求生死的解脱之道。更具体地说,他想探寻一条能与摩耶夫人取得连系的通路。

想出家的念头,随着他的年龄,与日俱增,他对婆罗门中的「四吠陀」和「五明」等学理,都很认真地学习,偶尔遇到外地来的修道人,也会十分留意他们动向。他习「吠陀」之学,为的是增强自己镇定的能力,然而每当他遇到波阇波提夫人,就会因为一种不自然的情绪而很感拘束,甚至尽量避免与她碰面,这给了悉达多极大的不安与困扰,以致「吠陀」之学对他也起不了镇定的效用,他一反过去天真烂漫的个性,变得敏感而内向,使周围的人觉得他越来越难应付,当然,这一切转变都看在净饭王的眼里。

当悉达多年方十七,父王就替他做主娶耶输陀罗进门,虽然就净饭王而言,以为美丽动人的耶输陀罗必定能拴住悉达多的心。而在悉达多这一面,我们前面已说过,美丽的妻子加重了他心中的负担,父亲为他所做的安排,以及强邻压境的政治局面,处处使他有透不过气来的感觉,有时他也试图从寻欢作乐的途径中冲散这种紧张感,有几个女人因而深得他的喜爱,最令他痛苦的是,明知纵欲享乐是腐化的生活,但有时却无可避免地陷进去。美丽的耶输陀罗更增加了他这方面的苦恼,他是十分喜欢这个娴淑而高贵的妻子的,有时在寝宫中静坐,窗外廊间不时飘来女人在园中嬉戏玩乐的笑语,这些嘈杂的声音搅乱了他的心,他不禁为自己竟会爱恋她们而深深自责不已。围绕在他身边的这些女子,都把他捧护得如神明般,只要他一开口,她们立刻四处张罗,以遂他的心愿。他此刻想到,当时若一任这种骄矜任性的个性发展下去,那后果是不堪设想的,悉达多回顾这一段宫中的生活,很惊异地发现自己的想法和行为完全绕在一个我字上,他除了替自己想,或做自己要做的事外,很少眷顾他人的情况。

分析自己那一阵子的心理现象,知道如果用佛法中的「中道之灯」来探照的话,他心中的贪念与妄想是无处遁形的。甚至可以说,连称得上符合正道的行为都是异常的少。分析到这里,悉达多不禁愕然而不知所措。回想当初他对父王及母后的态度,以及同侪间竞争的意念,或爱护动物的行为等,不是太过偏激,就是太强人所难。再说离宫修行,在山中一住六年,也无非是想满足自己悟道证果的欲望。若以大自然的一分子而站在宇宙真理上来论,诚心诚意做一种反省工夫的,才是修行悟道的必经过程。在反省中,把握住人事物三方面的重点来建立一种观念,并寻求合理而有效的思考方法,这样在批判事理时才有准则可循,而在反省之后,既经发现了症结所在,就应期许自己不再重蹈覆辙,悉达多反覆省察自身,想把心中的阴郁一一剥除。

\mt{抛开悔恨的意识}

他不禁又开始思索,究竟出家一途是否正确。辜负了父亲对他的期望,并且又将出生不久的罗睺罗弃置一边。在他所领悟的八年正道中「正精进」一道,即指为求人际关系的协调所应尽的一种努力,尤其针对家庭中亲子、夫妇、昆仲间的人伦关系而言,现在他\xpinyin*{遽}然出家,过着独身生活,就促进家庭的人伦关系这一点来说,他是不够资格了,而选择这一「不够格」道路的人,是他自己。

他偶而会想到耶输陀罗,以及浮现在她脸上的忧伤神情,尤其不忍罗睺罗此刻过着无父的日子,尽管如此,他进一步想到,他的出家,对迦毗罗卫国而言,绝不是一件坏事。父王虽未曾有过立难陀为太子的打算,但是为悉达多出家的情势所趋,难陀势必要做继承人。波阇波提夫人在日渐年老的净饭王面前举足轻重,说话极有份量,她之有权势是理所当然的现象,也因此益发使悉达多感到自己的存在将造成尴尬的局面,也将有着无可避免的纷争,他出家修行,就消极面来说,成全了母后的心愿,解除了父王的困扰,至少王位继承的争端是不存在了,如果他以太子的身份继续留在国内,也难保自己与继母间会没有冲突。

他冷眼旁观到,自难陀出世后,波阇波提夫人的地位益形稳固,而相对的,年迈衰弱的净饭王的权力越来越弱了。悉达多想,自己若是未曾娶妻生子的话,所负责任要少得多,如今他背着「抛妻弃子」的罪名,受着许多人的非议与指责。一个失去了丈夫的妻子,其悲伤孤寂的心情是可想而知的,悉达多对于罗睺罗的爱也是很难割舍的,我们从他为儿子命名为「罗睺罗」(即印语『障碍物』)可见他心情的一斑了。

他把宫中的奢靡、不安和看似快乐的生活拿来与「生老病死」四大问题的解决之道一起放在天平上,突然发现耶输陀罗和罗睺罗的影像在那一头的秤上如炉中的火星般点点消失了,心想,待他得道后,他自会回来安顿耶输陀罗,虽说他俩会有一段苦闷的别离时期,但是当悟道的大喜日子来到时,这种短暂的苦闷又何足挂齿?于是悉达多骤然抛开了一切私人的琐事,为了那远大的目标,做了最后的决定漏夜飞奔出城。

这样一想,悉达多对于自己选择出家一途,丝毫没有悔恨之意,夜幕深沉地垂下来,悉达多面前的薪火暗下来了,他立刻起身加了一些落叶与枯枝,再用竹管对着吹,余灰飞散四处,火势在噼啪声中又旺盛起来,现在菩提树下又被照得通明,顶上的枝叶闪出亮光。

要对三十六年来的思想和行为加以判定其是非善恶,是很困难的,一定要以第三者的客观立场来反观,才能获致中正不倚的结论。偏见执着是毫无意义的,于己无益,所以在严酷的反省之后,内在的愚痴狂妄必一一浮现,而有些事是不能为他人道的,那就像一个\xpinyin*{疮疤},虽然已不复疼痛,平时在衣袖内潜伏着不为人所见,然而疤痕是确实存在的,有时仍不经意地会触及,过失就像这样深藏人们的心底而抹杀不去的,要想去掉陈旧的过失之痕,唯有今后不再故技重施一途,反过来说,即经揭出疮疤,却一味执着,将自己陷入无法自拔的罪恶中,就会失去面对未来的勇气,这无异扼杀了自己逍遥自在的本然之性。

我们对罪恶能警觉「这是罪恶」,是很难能可贵的,起码可以提起我们的罪恶意识而\xpinyin*{遏}阻下一步的罪恶行为。不过问题在于我们太过于介意自己曾犯下的过错,因此生出黯淡的意念,逼使自己过着消沉、自责和堕落的生活。如果结局是如此,则「反省」适足以致人于窘境,这绝不是中庸之道中所期许的现象。因为过去(包括前生前世)的一切体验,就是灵魂历炼的一个过程。在反省之余,能将败坏的残渣抛掉,不再滋扰此心,让灵魂得以在以后的人生行于「中道」的旅途上,这才是反省的最大功德,这个功德是以身心调和与宁静自在的姿态出现的,



由于对自己三十六年来的生活做了很彻底的反省,悉达多的心扉为之开潋,今后决心以自己悟得的道理,努力付诸实践,而且为了继续拓展心智的领域,他将不断督促自己,充实自己,永不休止。

薪火的火势正隆,悉达多此刻的心灵,也正向外延伸、扩展。他拿起一个芒果,剥了皮送进嘴里,甘甜的果汁通过食道而直抵胃部。面前熊熊燃烧着的火,象征永不衰竭的生命,为他的身与心带来无限的温暖。

\section{梦幻的世界}\label{sec1.14}

\mt{实践重于说理}

发现悟道前所需具备的中正原则,可真不是容易的事,这是婆罗门学中的「五明」和「吠陀」之学所未能阐发的。婆罗门把五明和吠陀视为一种生活的工具,有时婆罗门学者,沙门或沙弥尼等,把这些所学掺和了自己的人生体验与智慧,致使一般人无从了解而视之为深奥难解的学问,悉达多自三岁起,因为颖悟过人,很早就跟婆罗门学者学了五明和吠陀,教师的教学堪称有方,能把道理很有系统地灌输给学生,可是这些学问似乎只是一种认知,根本与生活脱节。

悉达多最感奇怪的是,一些满腹经纶的教师,一离开讲堂,就近乎野蛮人般不通情理,过着糜烂腐败的生活。他承认这些老师所说的话十分巧妙动听,常使他不得不打心底里叹服,但是对于如何身体力行,如何为人处世的一些道理从不做讨论,好像那是毫无关连的事,他们只知如何设祭坛,如何向梵天王祈祷,认为这样就是得救之道,又自\xpinyin*{诩}梵天王的使者或代理人,声言只要来信靠他的,都能得神的救助。于是你每天就见他们到处游化,安然地过着一种婆罗门的生活,对于什么是「业」,什么是「因缘」或「因果」的问题从不想做进一步的探讨。

要想过一种静观自得逍遥自在的生活,除了身体力行所悟得的真理常道,再无捷径可寻,这一点,悉达多早有所悟。他深知因果的环节要靠「中道」来打开,而人们由于执着不通所产生的烦恼与痛苦,也需以「中道心」来化解。一切事都得求诸本心,亦即如果你想祈求神佛赐福赐光,首先得除去对神佛的疑惑或对事物的迷惑,他想起自己来到这菩提树的第一夜,恶魔之所以能化为美女模样出现在他面前,正因为他的心尚有余地来接纳这些美女,也就是说,他内心仍有情欲的波纹在浮荡。

坐禅至今,他仍在变换各种姿势,如结跏趺坐、半跏趺坐,或在臀后垫以东西等,希望能寻出一种不易疲倦的姿势,坐禅的目的本在静思反省,由反省才能镇静自己的心,若能采取使自己最感舒服的姿势才是最理想的。有人主张坐禅一定要遵循什么特定的姿势,那显然是舍本逐末,人的体型有别,有的人腿长,有的腿短,有的人身形浑圆,有的人瘦高修长,若一律要求他们摆某一姿势是很不合理的,姿势不合会造成四肢麻痹,呼吸困难等现象,如此一来,坐禅时的心神必被搅乱,那还谈什么静思冥想呢?悉达多一向就对坐禅很熟悉,以往在宫中就常坐禅,但是一向不拘泥姿势。那么他坐禅时通常采什么姿势?他采半跏趺姿势,因为他发现这个姿势能使他安坐一整天而不觉酸痛。

在林间树下坐禅,可想见的常有蚊蝇蚂蛾,甚至毒虫等来骚扰,一不留心就被袭击而皮肤立刻会肿胀起来,情形严重的因而失去性命,这种例子屡见不鲜。悉达多每于禅定前,必涂上一种由除虫草中绞取的汁液,以防不测。但是草液很臭,不习惯的人,闻之则周身不适,因此悉达多在白天游化或沐浴,而将大部份的禅定时间移至夜间,比毒虫更可怕的,就是毒蛇。在他修道的六年间,就曾眼见许多修行者的宝贵生命被它夺去。他本身倒未曾遭遇过此种灾难,但是他仍非常谨慎地防范。他经常在身边准备了装有蚯蚓的竹筒,一旦被蛇侵袭,立即用绳子绑起伤口,以阻\xpinyin*{遏}毒液流遍全身,并在伤口涂上蚯蚓的液体。只消几分钟,伤口上的毒液就就开始分解,因而能免于死难,毒蛇喜出没于砂岩地带,所幸他所处的地方没有砂石,且地质较干燥,杂草也不多,对毒蛇出没的威胁要小得多了。

\mt{矛盾的存在}

悉达多继续反省他二十岁前后的想法和行为。

那时候,他的国家与邻国间经常有纷争,虽然没有大规模的争战与厮杀,但是为了领土问题,双方常会互派间谍挑起小型的游击战,这类战事虽发生于城外,但是城内立刻也会弥漫紧张的气氛,所以城内时时处在备战的状态中。悉达多经常必须到前线巡视战况,只见战场上布满了敌我双方的尸体,一些很活泼的年轻人,此刻只能静静地陈尸荒野,有缺了头的,断了臂的,有枪穿胸膛的,也有半埋地底窒息而死的,甚至有双方互刺不支倒地的,死况凄惨,令人不忍卒睹。想当初这些年轻力壮的武士们,雄赳赳,气昂昂地列队出发时,夹道目送的妇孺们,往往陶醉在一种胜利来归的感激之情中,然而他们何曾想象得到战场上的残酷景象,他们更无从比拟此时整齐华丽的行进队伍和战时腥臊血垢的横陈弃尸,战场上只有死亡在等候他们。

悉达多来往于迦毗罗卫城和战场之间,遇到了战争所造成的矛盾问题,不管是敌,是我,有人死亡是不移的事实,死亡所造成的悲痛、无奈与仇恨,是人类共同的损失。他一直不主张战争,但是当他得知宫中的御厨被敌方派来的间谍暗杀,以及军营中有士兵在睡梦中被奸细刺杀等等消息后,他不得不面对现实来迎战,在城内巡逻时,常会碰到一些武士,他们意态诚纯地向他鞠躬致敬。若他开口问路时,他们会兴高采烈而争先恐后地在前带路,在战场上,也可想见他们那种视死如归的勇气。谁都不愿意死,但是他们深知终日操练,为的是上战场一拼性命,他们的行动受到军纪和律令的支配,毫无自由可言,他们降临世间,到底是为了生?还是为了死?在他们那种紧张待死和放荡求乐的生活中究竟领悟到了怎样的人生?他们是为什么而活的?

悉达多在巡行期间所亲身感受到的生命的艰辛与困苦,使他对于人类生命所存在的矛盾现象,不禁黯然神伤。这种生与死之间所持续着的紧张状态,使士兵们一到了夜晚就放浪形骸起来,常通宵达旦地寻欢作乐,舞女、歌伎、美酒、佳肴,夜夜都把迦毗罗卫城装点得有如彩色的不夜城,酒后的谩骂、吆喝、纵笑、夹杂了女子的娇嗔、调笑,这些没有明天的战士们,把自己淹埋在怒吼与情欲所交织而成的短暂激荡中。当然,担当警卫的人是绝对禁酒的,在职责终了前是滴酒不能沾的,悉达多不会喝酒,所以无从了解一个酒徒的心情,这些人即藉着酒来逃避现实,然酒醒后仍不免要面对紧张的情势,他们是为了一时的逃避,还是为了短暂的快乐,悉达多就不得而知了。

在战争与欢乐交替的生活中,引发了悉达多对出家一途的选择意念,虽然净饭王曾为他建造了顺应四时的行馆,可是他未曾获得心灵的平静,要求心的安住,只一味追求欲乐或逃避现实,是无法达成的,为控制对方,而以暴力和权势的镇压手段,日久之后渐失人心,这是必然的道理,心与心的接触,要靠双方抛弃一切的虚饰,才能产生共鸣。在自我和烦恼的漩涡中寻求心灵的安定与契合,就好像以疾转的风车去追逐猎物,必将一无所获。

理想、现实、欲望、疑虑、战争、和平......无数有待解决的问题剧烈地震撼了悉达多的心。

\mt{梦幻之境}

悉达多越来越沉默了,如果没有什么事,他都尽量避免出现人前,在孤独与静寂中,他沉思默念的时间增多了,而思想亦得以自由发展。

一天下午,他照例在菩提树下冥想。在深沉的境界中,感到自己似乎正为大众宣说佛法,在座的听众,个个热心地倾耳谛听。当他说到半途,豁然了悟听众祈求的是什么,又得到了什么,他们的心中藏着什么念头,他都了了分明。在说法时,他所发射出来的光,竟治愈了老人的\xpinyin*{痼}疾,甚至恢复了他们的青春活力,他接触到人性至美的一面,当他迈开大步向前走的时候,天与地交叠,万物溶合在天地的大圆轮之中,而在这充满正法的圆轮中,自己正自由自在地优游其间,来去自如。

这种境界,不是刻意创造出来的,当悉达多不断憧憬出家的自在生活时,他的思绪就不知不觉地超越了现实的\xpinyin*{藩}篱,而飞腾翱翔于美妙的梦幻之境中。当他回到现实,很惊异竟有这么奇妙的梦,虽然境中的影像消失了,可是真实的亲切感仍留在心底。由于这样反覆寻思,长远以来被自己遗忘的事情,屡次清晰地重现脑际。

梦有两种,一种是睡眠中的梦,一种是幻想中的梦,前者可以赤裸裸地映现自己的影子,而后者能随着自己的思绪任意奔驰,要想使幻想与现实结合,那就是妄念。妄念为人们带来许多苦恼与问题,而悉达多的梦,不同于前两者,悉达多在冥想中所接触的景象,是他将来悟道后所可能实现的理想境界,也就是这一生要靠不断的苦修才能达成的,所以说,他的梦跟睡眠中的梦和幻想中的梦是全然不同的,他的梦具有更积极的意义,但是在这种梦境出现的当时,悉达多并不了解个中的原由,他也莫名所以,每当从梦境中返回现实时,他都十分诧异。

「真是不可思议啊!」他也只能这么想。虽然有着无比的真实感,不过随着时间的流转,那种梦境也消失了。

自从他领悟了「中道」之理后,加上对「八正道」的认知,如今细想起来,当年那个奇怪的梦,不是不可思议,而是本该如此的。这种想法很自然地从心田升起。菩提树下的第三天,第四天已经过去了。经过这么多天的反省,他心中的负荷卸下来了,透体舒畅。心中的负荷不是别的,指的是「执着」。对于生老病死的执着观念,悉达多已一一脱除了。他不再执着生,以为生命一定该有什么形态,或以为该如何做,又不该如何做,如此这般地思前想后,他不再害怕衰老和生病,也不再幼稚的对死怀着恐惧。这一切的执着、害怕与恐惧,都随着人的年龄,日趋扩大,形成难以负荷的重担。如今经由反省、观照,终于重担被卸下来,而能拥有真正愉悦爽朗的心情了。

此心如苍穹,圆融而恢宏,并且仍在继续扩张中,他已具备了勇往直前,所向无敌的浩然气度。

\mt{存疑与求道}

在菩提树下,迎接第五天的夜。

悉达多二十岁以后,已完全不听从父母的意见了,他为自己这种\xpinyin*{迕逆}的态度抱歉,不过出家的意念,却越来越坚定了。净饭王曾几次三番地与近臣或自己的兄弟商量对策。

「怎样才能打消悉达多出家的念头呢?」这个问题一直严重地困扰着净饭王,他真是伤心欲绝。看来阿私陀仙人那个不吉利的预言「太子将出家」马上就要实现了。

耶输陀罗跟悉达多相处的时间比较多,她早对悉达多出家一事有了心理准备。耶输陀罗是悉达多的表妹,跟波阇波提夫人有着姑姪的关系,跟释迦族的关系也算是很近的了,所以悉达多弃她离城后,她并不像其他的嫔妃所表现的那样激动哀伤。当然,在她心中,也深深期盼悉达多有回心转意的一天。

悉达多虽然知道自己给家人带来这么多痛苦,为自己无法一一表示歉意而愧疚不安,但是他坚信出家是迈向悟道的唯一通路,等他悟得了人生的至理,得解脱的将不只他一人,眼看罗睺罗即将长大成人,能代替他在宫中尽点心意,他也觉得心安不少,此时,悉达多满怀信心。他对宗教早有领悟,剩下的,只是如何实践的问题。他毅然决然地,如前面所述的,促使车匿牵出马一起离城。其实在出城前,他早知单身生活有着难以意料的困难与艰辛,为此,他的意志也曾动摇过几次。

现在忆及过去的种种,不禁感慨不已,他要把每件事情的起心动念,彻底地以中正之道来审度,以见出个是非黑白,为求解脱之道,应从「怀疑」出发,对人生不存疑惑的,根本就不会想要寻求答案,自然也不会获得解脱。怀疑能培养出一颗探讨的心,最后真理就能昭然若揭了,通常在探讨答案时,因为没有所谓「中道」的尺度,就容易走错方向,而像这样犯下错误的人,还不乏其人呢!悉达多所悟得的「八正道」,正是疑问和解答之间的桥梁,只是当你执持「中道」的尺度来量度事理时,自己是否能很严密而公正地使用这个尺度将己心控制裕如,也就是说自己对于将心中阴影的去除能达到怎样一个程度,实关系着自己批判外在事物的正确性。如果自己内心还有黑影潜藏的话,生老病死的执着就不会断除,解脱就是执着中超脱出来的一种心的状态。

\section{跟恶魔决斗}\label{sec1.15}

\mt{魔众现形}

五日来的连续反省,是彻底的自我剖析,不容许自己有丝毫宽恕之心,所以此心在反省前与反省后是截然不同的,如如不动,安然静定的心因而产生了。悉达多从冥想中返回现实,正想躺卧下来,突然\xpinyin*{飙}风四起,喧腾之声,掩没了四周的寂静,菩提树被摇撼,树叶\xpinyin*{飒}飒作响,周遭如庞大的生物苏醒般骚动不已,同时有股异样的气味阵阵袭来,不知不觉,「梵天」出现了,正盯视着悉连多。四目相接,悉达多惊讶地凝视着面前的庞然大物。

「迦毗罗卫国的太子,你应该赶快回城了。不管你是多么慈悲、多有智慧,也无法救渡有『私欲』的人。想想看,城里还有你的父王和妻儿,不是还有许多部下在等着伺候你吗?如果你放弃修行,你立刻能成为人间的统治者而过着豪华奢靡的生活。你能过那样的生活,是神所容许的,难道你忘了吗?悉达多,生命如果是轮回的话,现在的原因,就是来世的结果,你若是一国之主,离开人世后,必能再转生为王,如今你选择了苦行,想必来生也不过是过着困苦的修行生活罢了。我看你的生命不过是今生这一世了,如果你答应立刻停止修行,我必助你成为全世界的统治者。」

「梵天王」大声说着,声音如雷灌耳,一副威仪堂皇的模样,悉达多听后很踌躇,对方所讲的轮回、苦行以及我欲执着的种种道理,也都是有理可循的,一些婆罗门修道者也都曾说过的,此刻见眼前这个人也能娓娓道来,就暗自揣想,可能真的是神也未可知。但是令人费解的是,这一位神带来了一股异样的味道,而且是类似兽类的\xpinyin*{羶}臭味。

「请问您是哪一位?」悉达多定睛凝望对方的容貌,准备展开论辩。

「我是梵天王。」对方轰隆之声传过来。

「您刚才提到,生命只有这一世......」

「嗯,是这样。根本没有所谓的来世,我能看到人们所看不到的事物,所以我能肯定这一点。人本来就是为了贪图快乐才有肉体产生的,你难道忘了这件事?」

「如果您肯定只有这一世,请问您是来自何处?如果您来自人世,岂不表示您也只有这一世的肉体了吗?」悉达多至此看破了对方的形迹,因为对方除了会发出异臭外,在论调方面,还充满了矛盾。此外,他那不可一世,盛气凌人的态度,根本和神佛是背道而驰的,反而与恶魔\xpinyin*{採}的是同一步调,这是显见的事实。

「您是什么人?赶快现出原形来吧!」悉达多严词厉色地\xpinyin*{叱}责对方。

自称梵天王的人,经悉达多这一喝叱立刻萎顿下去,这时竟改以哀求的口吻说:「你若得道的话,我们就没有住的地方了。为了我们这一大家子,请你快回迦毗罗卫城吧,我一定助你一臂之力。」

「魔王,我根本无意于名利与权位,这些都不再困惑我了,您在前世中亦曾拥有过肉体,您在黑暗的世界称王,心中必定得不到安乐,因为您必须时时提防您的手下会出卖您,您也不知道什么时候自己的大权会旁落,您这样整日忧心\xpinyin*{忡}忡,为何不设法找寻平安快乐的世界去安住呢?您也有善良的本性,只要把以往的过失改过来,诚心诚意地向神忏悔谢罪,您也会有平安快乐的一天。」

悉达多这样训诫他,可是魔王继续逞他的威怒:「你胡说!我是至高无上的魔王!在这宇宙中谁能比得上我?像你这样的人,我会屈服吗?你这个头脑不清楚的人,还不快离开乌鲁维拉?」

魔王的脑筋,套一句现在的话来说,是一种精神分裂的状态,不针对问题来回答,已到了歇斯底里,语无伦次的地步。

「可怜的魔王啊,您为什么要\xpinyin*{唾}弃明朗快乐的世界呢?您情愿\xpinyin*{蜷}缩在阴暗、寒冷而饥饿的角落,唱着个人的独脚戏,您难道不觉得寂莫吗?您不会感到心寒胆战吗?您既造下了罪孽,就应虔诚地悔过,这样还有重见光明的一天,也能使您的心回复到以往的平静,您所执着的权力,只是徒具形貌而已,您到底掌握了多少人心?请您还是不要固执成见吧!」

魔王丝毫不理会悉达多充满慈悲的劝诫,立刻唤身边的弟子把菩提树包围起来,并渐逼渐近,魔众的脸,色彩不一,形相各异,然都狰狞可怖,这是个可怕的魔鬼集团。悉达多端身静坐,缓缓放出柔和的光,向魔众投射过去。魔王和弟子们,冷不防地被困在金光之中动弹不得,嘴也被什么封住了似地不能发出声音。

「请大家都好好地听我一句话,您们本都有善良的天性,只因您们\xpinyin*{孳}长了自己愤怒、憎恨和自私的心,您们没有爱过人,也没有被人爱过,也就无法发挥慈爱的本性。您们也有自己的子女,必了解双亲对子女的爱心是怎么样的,而神佛的爱与慈悲也是如此。您自甘沦为魔鬼,把大好的内在荒废了,不过神是不会离弃你们的,所以有心弃暗投明的话,为时未晚。您们应鼓起勇气来面对自己,从今以后不再虚伪矫饰,以发挥内在的佛性,现在这一片光,是来自天界,带来了佛的慈悲。请赶快忏悔您们的罪过,表现您们善良的本性吧!」

\mt{正法降魔}

在悉达多上方的一块黑影,此时如寒冰之消解,点点滴滴化成了闪亮的光珠。魔军中,只见一个、两个陆续低下了头,懊恼悔恨的神情表露无遗,风已静止,四周重归于寂静。在寂静中,顽强的魔王也低下头了。黑暗想要掩灭光明,根本是不可能的。魔王的心中也透进了光,他不自觉地双膝跪地,面向悉达多恭敬地合掌致意,四周渐渐放出光亮,魔王与魔军的影子\xpinyin*{倏}尔不见了。

虽说是一个魔王,他也有人类的本然之性,也能感受到慈悲的光,在慈光中,他不但不能措手足,连带地内在的神佛之性也能显发出来,而一向覆盖在本心上的晦暗念头,也在一\xpinyin*{刹}那间被扫除净尽,悉达多与魔王正面起冲突,这还是生平第一遭,在这以前,他未曾听过魔王的名字,更别说是跟魔王谈话了,如今他竟跟魔王对答如流,并能直呼其名,这是很奇异的现象。等诸魔消失后,悉达多环顾四周,想起刚才自己镇定若恒,如如不动的景象,深深领悟到自己已有许多神明守护在身边。

悉达多的内心充满了自信,他知道有许多神明静静地在一旁守护,这是他刚刚与魔王做殊死战时所觉察到的,他知道没有什么事值得害怕的,在心底不禁涌起无穷的信念与安定,围绕着他的,仍是深沉的夜幕。月亮投出青白色的光,照射在乌鲁维拉的大地。然而悉达多却坐在耀眼得如同白昼的金光中。他的心头充满着未曾体验过的喜乐,长留不去,喜悦的泪水滚至膝盖,睁开眼睛,在泪眼模糊中,他接触到蒙蒙的一片,他就这样继续端坐,似来自天外的旋律,优美而庄严,阵阵送入悉达多耳中,好像天界的天女们所奏,缭绕回旋,抑扬顿挫。这大约是梵天界的天众,因为悉达多降魔成功,并扶持了恶境中众生的堕落,特地来祝福他的吧!悉达多静静地聆听这妙不可言的天外之音。

\section{伟大的时刻}\label{sec1.16}

\mt{黎明前}

第七天来临。

他希望今夜仍能保持一颗宁静平和的心,边想就边躺卧下来。然而当他触及昨夜的事,泪水不自禁地又溢出来,无法抑制,他觉得把时间耗在睡眠上是很可惜的,所以决定再来反省他出家六年来的生活,而开始沉入冥想。

静静地,时间在寂静的乌鲁维拉森林中溜过去。从冥想中醒来,悉达多睁开眼睛,只见东方已微露灰白,一夜无事,黑暗在一瞬间又消逝了,当他再度闭上眼睛,正要进入冥想时,突然发现自己一直静坐着的身躯在膨胀,在延伸……最后竟穿过菩提树,而伽耶山此刻就在他的脚下和眼底。悉达多的意识一点一滴地扩张,他的身躯远离了大地,或更进一步地说,是身边的大地也伴随他一同离去了。这不是距离上的远离,实在是一种现实的扩大现象,也就是说,自己的意识远离地面,然而身边的菩提树,邻近的乌鲁维拉和伽耶山跟现实中的感觉全然一样,只是此刻都集中在他的眼前,看得更真切了。

随着意识的扩大,速度越来越快。破晓中的明星在他的脚下,另一个悉达多如一个小米粒般,在遥远的下方端坐着。悉达多如同宇宙,不断向外伸展、扩张……同时,宇宙进入自己的意识中。整个三千大千世界,连同美丽的星星,在悉达多的眼前展开,一切都那么美丽,生命在跃动,他身边的森林、河川、街道、星球,似乎都在造物主伟大的心怀中调运气息。他周围的景致连成一幅灿烂耀眼的连环图画,而自己正在纵横浏览,同时感到自己的肌肤上,遍布了生命的气息。这一幅巨大的连环图轴,不断在悉达多的意识中晃动着。

他终于证悟了宇宙的至理!三十六年来心头所覆盖的乌云,在一瞬间被光明驱散了。悉达多终于得偿宿愿,他已融入大自然的意识中。既与整个宇宙融合为一体,悉达多终于了解到大自然中森罗万象的来龙去脉,人神的存在,人的归处以及灵魂的转生与轮回等真相。

\mt{大地的欢悦}

悉达多从此体证到人的价值,人与大自然同出一源,是与大自然同体呼吸的,人生来就应该在大自然中生存,离开了自然,就不可能有人。人依着自然的理念共生共存,这个道理被他证悟了,而这是在人世间的七情六欲中所无法觉悟到的,一件东西,若我们以「物」的观点来看它,就难免起了喜恶或占有、抛弃的念头,如此一来,我们的心终将不得平静,如果能从「物」的观点离开,辨明何种力量在左右这样东西,它实际的存在是什么,我们一旦能这样超越其上,就能真正认清该物的价值。对任何事物,都能不单凭一己的主观意识来论断,自能正确地探视到事物的本源。

此身能时时受到已证悟至道的诸神诸佛所保护,因而自己也觉悟了真理,这使悉达多感到无比的喜悦,并充满了感激之情,泪水不自禁地\xpinyin*{潸}潸落下。

黄金色的光粒子,不断地撒落在悉达多的身边,粒子在地上发出眩目的光辉,仙乐和妙音不断从天庭乘风而来,这是天人为恭祝悉达多彻悟的一曲喜悦的大合唱,而悉达多也沉浸在美妙的歌声中,心里充满了法喜。本来在心底的一隅还有恶魔纠缠着的,此刻也都化成一道道的光从心中放射出来。悉达多静静享受着「与万化冥合」的殊胜境界。当他沉入了冥想的极致时,就不再意识时间的流逝,时光如矢般飞驰而过。我们所能捕捉到的时间,只是眼前的一瞬间,大自然的轮回未曾须臾中止,悉达多在菩提树下,深深为眼前的一切感动不已,在感激的微颤中,他睁开双眼,由冥想中出来,他从「万物皆备于我」的巍然境界回到现实中五尺多一点的躯殻之中。

抬起头,仰望天空,\xpinyin*{绀}碧色的晴空,万里无云,把伽耶山温暖地拥抱起来,晴空像是弯起了一个大圈圈,这个圈圈不断在扩大,阳光透过菩提树的叶缝,照射在悉达多的脸上、身上,悉达多身边的大地,也都像在为他欢呼,枝叶更茂盛了,花儿更娇艳了,迎着风儿,笑弯了腰。他身后的大树,四周的野草、虫儿,以及停栖在菩提树上的鸟儿,彷彿都正扯着喉咙歌颂生命的奥妙,并且目不转睛地望着他,似乎在期盼什么,此时,悉达多也忍不住地以慈悲的心怀来回应他们。

\mt{永恒的觉醒}

「如果我们把自己的所闻所见告诉他人,会得到什么样的反应?他们会不会了解?或者说,他们会不会相信?」悉达多在法悦之中突然想到这些问题。

「我想,他们不会了解……」他又想。

悉达多从生老病死的苦恼中解脱了,并从生死轮回的「业缘」中超离,体悟到生命的永恒,知道在我们这个会坏灭的肉体中,还有一个不生不灭的「真我」。这个「真我」与宇宙同终始。悉达多随时能清楚地觉知这个「真我」,他已超越了过去、现在及未来的局限,也不再受到「业」的支配与牵引,而能永远脱离生死的漩涡。

目前这个肉体是由于跟父母在过去生中结下了缘而得到的,可说是人生航程中的一条船,由不生不灭的灵魂寄乘着,载浮载沉于生死苦海中,人就以人的形态,动物就以动物的形态不断地生活下去,在生命的旅途上不断进行,人所感受的一切痛苦,都由人自己的心和行为所导致,也就是自己违背了大自然所昭示的中道原则,以致在生存的矛盾中经历了种种的痛苦。说得更具体一点,就是自己舍弃了「正见」、「正思惟」以及「正语」等正道而自甘不合理性的生活,自尝不合理性的苦果。「中道」是一个人之所以为人的最自然的生活路途,它能使你很尊严地成为一个人。悉达多自悟了「中道」之后,常思如何阐述个中道理,只是到底会有多少人能真正参悟生命的实相呢?

悉达多抚着菩提树的根座,慢慢站起身来。他向森林走去,鸟儿从身后紧跟而来,悉达多停下脚步,鸟儿也停下来。当悉达多凝视着远方时,鸟儿就渐渐靠近他,丝毫未有畏惧的神情,在悉达多的脚边安适地啄食地上的饵食。小鸟是菩提树上的「居民」,跟悉达多比邻而居,一直是悉达多很友善的芳邻,常常会停栖在悉达多的肩上或头上,似乎在跟他说什么,而悉达多也会很快活地回应它们,跟它们柔和地说着话。鸟儿们也有着与人一样的丰富感情,每当悉达多对它们有什么特殊表现时,它们高兴得连声音都变了,动作更是迥异于平常。当他意识到鸟儿的这种转变,就回复到原先的态度,说也奇怪,鸟儿们的声音和动作也会立刻回复原样。

在菩提树下,虽仅是短暂数日,但他从未见过鸟儿与同伴间起什么争执,它们和平相处,互不侵犯,每天从早到晚都活泼蹦跳地,快快乐乐,轻松自在,反观人类,像同事间,兄弟间总有牵连不清的恩恩怨怨,未能以平等的态度和乐相处,他不禁慨叹堪称万物之灵的人类,有些地方的表现竟不如一只小小的鸟儿。

\mt{悲喜交集}

他来到尼莲禅河,只见缓缓流动的水,一如往昔地静静流着,当他涉入水中,清凉的水驱散了他昨夜以来未曾合眼的疲倦,他以两手掬水,从头部冲下来,感受到无法言喻的愉快与舒畅,这条河流将流至印度洋而融入大海,大海的水不久就会成云化雨降至地面,再汇集至尼莲禅河中。水,就在天地中重复着这种循环,然而水的本质不变,悉达多就这样浸在水中,感觉自己也是大自然的一部份,河水无言地流着,此刻在水中拦阻了河流的悉达多,彷彿抓住了时间的「流」,而身处时间的「流」中。过去、现在、未来的时间之流,使身处其间的人们无法窥知它,要想觉悟到这一股强大的「流」,舍反省中的「止观」工夫则不可。

悉达多在尼莲禅河中,忍不住地放声哭泣起来,因为自昨夜以来所涌现的感激之情,此刻正如潮水般向悉达多袭来,悉达多在暮霭中,把捡来的柴薪及枯草堆起来准备升火。灰白的烟向着天空冉冉地上升。四周无风,今晚又将是一个宁静的夜。悉达多一面剥芒果的皮,一面不禁吟着难陀婆罗所唱的歌:「琴的弦,恰恰好,妙音响天边……」

因着这样吟唱,他的心神无比地安适。

晚餐本有乞食而得的蔬菜和米饭,不过他决定今晚用野生的果食来充饥,在他的心达到宁静调和的境界以后,即使偶尔心上闪过一丝不安,也能立即回复到初悟道时的心境,一天来临了,一天又过去了。悉达多的心始终如一,没有丝毫的起伏与变化。

\section{与梵天的一席话}\label{sec1.17}

\mt{阿蒙来会}

自悟道后,又过了十三天,换句话说,他到菩提树下修「止观」,已是第二十一天了,他对自己心灵的调和状态充满了自信,希望这种状态能持续不断,他又想,若能带着这样一颗心来过人生,将没有比这更幸福的了。悉达多心情很肃穆地迎接第二十一个夜晚,他决心自今晚起不再进食,直到肉体衰竭也终不后悔,于是他在这样一种心境下进入冥想。

就在此时,眼前陡地大放光明。在黄金色的光团中,梵天赫然巍立其间,这是一位名叫「阿蒙」的梵天,阿蒙比悉达多高,而且身形略瘦,穿着纯白的像是绢织的衣衫,长衫直垂地面,腰间系一绳带,脸上布满了皱纹,乍看像是一位长寿的老人,可是仔细再看,这位长寿的老人又宛如四十多岁的中年人了,悉达多直觉面前站着的人跟自己有如老友般,亲切而热络,或者说是似曾相识般有种奇妙的感觉。总之,浸浴在金光中的这个人,是既庄严又神气。他的两旁各有一人,其中一人名叫库拉利奥,三个人都用很柔和的眼神望着悉达多。像是看穿了悉达多的心,他们的眼中亮着很慈悲也很柔和的光,悉达多眨了一下眼睛,再抬眼定定地望着他们,这三人越看越漂亮,悉达多也产生了一种即将升天的错觉。

名叫「阿蒙」的梵天说话了:「悉达多,你不能死。纵使死了,也仍会再回到人间,你不想留在人间,但是不管你用多快的速度,也是逃不出的。」

他那缓慢的声浪中,充满了威严与慈悲,悉达多自冥想前所持的死的念头,被这一阵声浪冲得一抹痕迹都没有了。

「你悟到了什么?悟道具有什么意义?这些你不会不知道的……」阿蒙的声音,仍然十分严厉。

悉达多不自觉地频频叩首,他虽仍保持静坐的姿势,但是双手平伸出去,上身与头俯贴地面,静待梵天继续说下去。

「你知道了吗?悉达多。」

「我不赞同您的说法,我也曾有心把自己悟得的道理对众生宣说,可是他们并不见得会了解。与其如此,还不如让我就此离开人间。」

「愚痴的人!」

阿蒙喝责着。语气十分严厉,丝毫不容半点妥协。

「你不救渡众生的话,还指望谁来救渡?你得细心地想想,你以一颗慈悲的心,必能在众生中造成镇静的力量,并唤起他们内在的佛性,『法』就是你心中的太阳,你要把阳光投射出去,真理的灯要靠你去点燃,也靠你将灯光传下去。首先你要知道,你我都曾在过去生中结过缘,所以我们终将在同一世界相聚,但是你目前负有某种任务,在任务未达成前是归途无路的,你既已悟道,这个道理不应该不明白的……」

悉达多一面叩首,一面倾耳谛听,他用心耳谛听,阿蒙的语气很严厉,像铁板一样硬,但是悉达多却能从每一句话的余音中,领受到他的慈爱之意,悉达多的胸口好像受到了撞击,他知道自己再没有退缩的余地。在他降生人间前,就曾发过救渡众生的宏愿,如今在未达成心愿时,就想毁弃肉体,是十分不该的。一经阿蒙这位伟大的梵天点破,他那被遗忘的誓愿与决心,此刻在心中澎湃不已。

「我明白了……我会去实践的。」悉达多语重心长地说。

「你能了解就好,我希望你能运用你的智慧来导引众生走向觉悟,我原是你的朋友,名叫阿蒙。我俩早有约定,如果我降生人间,你会协助我,而如果你降生人间,我也会时时护卫你,这个事实,你日后必会明白,我们今后将随护在侧,永保你康泰,使你能四处布教。请相信,我必尽力帮助你。」语毕,阿蒙在金光中不觉莞尔。

悉达多既表明了他的心意,阿蒙的话立刻由严厉的申斥转变为温和的叮咛了。

\mt{颠倒妄想的众生}

人一旦受了躯壳的拖累,就迷失了本然的「真心」,而难以觉悟自己的真正身份,这全因为躯体上的五官所使然,譬如手被扭了,必定感到疼痛,眼睛接触了周遭的诸般现象,就认定都是实实在在的一种存在。所以,就永恒而真实的佛境界而言,一个人在人的世界住久了,往往会被现实的环境所眩惑,以致忘失了自己的本性及神圣的任务,一旦大限到来,才不禁惊道:「糟糕!」

可是后悔已来不及了,生前的善恶因子终将随之转到下一生。已达「如来」境界的人,偶尔也会犯这种错误,菩萨也有所谓的「隔世之迷」,由此可知,我们身处的「欲界」实在是一个大陷阱啊。此外,有的人虽曾发下了救渡众生的誓愿,仍有误入歧途的可能,因为满腔的热血,若不辅以冷静的头脑,往往弄巧成拙,这种例子在许多修道者中屡见不鲜。譬如有的修道者滥用慈悲,使受渡的众生非但未能领受教益,反而增长了骄慢的心理;又如有的传教者为宣扬真理,不知运用权宜的方法,毕竟众人的聪明才智是不等的,然而他们只一味地大肆鼓吹,甚至诋毁外教,经常口沫横飞地唯恐宣说不周,岂料此适足以引人反感。

在我们身处的现象界中,经常有许多欲乐足以困惑人心,有时明知自己该走什么方向,却往往身不由己地走错了方向,故有所谓颠倒众生的说法。也就是说,众生常是黑白不分,是非颠倒的,别人的错容易挑,自己的错则不是茫无所知,就是拒绝承认。阿蒙先前那像铁板一样冷硬的苛责,一变而为充满慈爱的慰词,全因为悉达多能即时从困境中挣脱出来,而把即将暗灭的法灯,再度点燃起来。阿蒙不是别人,正是后来诞生于耶路撒冷宣说博爱之理的救世主耶稣基督,他也禀赋了救渡众生的重任去到西方降生。

悉达多仍不住地向梵天叩首,并深切反省己过,他一直思量自己刚才所起的绝食坐化的肤浅的念头。

名叫库拉利奥的婆罗门说:「悉达多,请勿多礼,把头抬起来吧!回想三十六年来我们一直守护在你身边,却从未跟你攀谈过一句话,现在能敞开心怀地与你交谈,是一件很高兴的事,目前你的心中充满了光明,你要更加精进,我为你能解脱一切的烦恼而感到高兴。悉达多,请抬起头来,我不知如何表达我的喜悦之情,就好像与自己的孩子相遇于天涯一隅般,你可了解我此刻的心情吗?」

库拉利奥高兴得泣不成声了。悉达多这才抬起头。阿蒙和库拉利奥的双眼因哭泣而红肿起来。另一位婆罗门,名为摩西,留有很特别的胡须,体格十分魁梧。摩西也不断眨着眼睛,并不时向悉达多点头示意。

才平静下来的心,很奇怪地,又起了微微的波澜,悉达多的心湖,又荡漾起了就这样死去该有多好的波影,正轻轻晃动着。还好,那只是一瞬间的事,只见阿蒙很温和地劝勉他:「死只是一种暂时的逃避,你仍无法从自己的心田逃出去,因为人的心就是自己的一个小宇宙,无论你逃到那里,你仍会看到自己的心。即使肉体灭亡了,心仍然在那里,你要以智慧、勇气努力地向众生昭示『生』的价值,应力求超越痛苦,绝不可以抱着以死解脱痛苦的念头。」

悉达多被阿蒙的这一席话打动了,因而更坚定起他的心,他想:「从今以后,无论遭遇到怎样的挫折与困难,我都要设法克服。」

他仰视阿蒙的脸,咬紧下唇,像是下定了决心,他对自己先前所起的念头感到羞愧不已。「悉达多,你所体验的一切,正是造物的旨意,今后将有无数众生会模仿你而在人生的迷惑中学习、奋斗,以拓展他们的心灵,一切有违正道的生活,虽然能带来短暂的快乐,但是却也带人通向痛苦的深渊。快乐造成痛苦,然而众生痴迷,不想探究个中的道理,有心的人虽能一时精勤于探究这个道理,往往也会因为悟道的路途遥远而中途向现实妥协。为了追求短暂的快乐,有的人甚至情愿献出自己宝贵的生命,这就是『我执』的心在作祟。这颗心是最难驾驭的,它将永远陷人于苦境。

经由体验,你已了悟了宇宙间的至理,并能为众生带来一把智慧的钥匙,帮助他们开启心扉。这是一个黑暗混浊的世界,你现在既已开悟,就要好好把握住,否则一旦被恶魔侵入,前途就堪虞了,如果你的前面有山,我们会为你铲除;如果有谿谷,我们会为你架桥;如果有河流,我们会为你造船。悉达多,请相信我们,我们必不辞辛劳地协助你,你安心去渡众吧!」阿蒙说着,就举起右手,脸上泛起微笑。

「我也必不辞千辛万苦地传布正法,绝不辜负您们的厚望。」当悉达多至诚地表达了这样的决心之后,三位梵天就放心地静静消失在光团中。至此悉达多的心坚定卓绝,真正达到了如如不动的境界。

梵天消逝后,伽耶山的森林重归于寂静与黑暗,四周的情景犹如大自然开了一个玩笑似地又回复原样。

\section{神游梵天界}\label{sec1.18}

\mt{来到天界}

默然静坐的悉达多,见身旁的火势已微弱,即起身再加了一些薪木,只见火势又熊熊地旺盛起来,野兽的咆哮声划破寂空,乌鲁维拉似乎只有悉达多一人。他想合眼睡一下,夜已很深了,就把一切计划留待明晨再做道理吧!悉达多边想,边躺下来,就在此时,他的身躯奇怪地晃动起来,他再一次体验到「与万化冥合」的美妙境界。身体继续轻轻地动着,不一会儿,另一个自己突破肉体,飞腾空中,脱离了肉体的悉达多,发现自己身处光明宽广的圆顶屋之中,而他正以飞快的速度升至圆顶,很轻易地,他又穿过了圆顶。世界在他眼前开展,视野所及,是一大片的新绿,看来很漂亮的草坪,悉达多正立足于一处平滑的绿丘上。

悉达多眼见自己的血肉之躯,还在菩提树的根上休憩,眼前这一大片漂亮的绿丘,是在人世间无法看到的,绿丘上的斜坡,宽广雄伟,一重一重地连绵到地的尽头。远望伽耶森林的一角,也被着特异的绿色,浓淡有致,草坪也抹上了鲜绿欲滴的颜色,使你不忍践踏其上,那似乎是有着气息的生命。再抬头望向蓝天,太阳闪着黄金色的光,不像人世间的太阳总是放射灼热刺人的光焰。身边这个太阳,此刻放出的光,是那么安详,那么柔和。

悉达多不觉惊呼道:「哦!」

声调中充满了赞叹之情,就这样,他静静不动地沐浴在和煦的阳光下。

当他盯视着太阳时,不禁回想起自己曾在这样的太阳下所渡过的岁月,一幕一幕的景致,如走马灯般在眼前转过,这令人怀念的情景,至今才再邂逅,引发他极复杂的感情,震撼了他的五体,正当此时,在悉达多的周围,有人影晃动,恰似他乡遇故知般觉得这些人亲切而温馨,只见其中一人正手指悉达多呼喊着大家过来,并频频向悉达多做手势。每一个人的脸都容光焕发,肌肤晶莹,十分美丽。不久,悉达多被带引到一处像集会堂的场所,会场上已聚集了好几百千的听众。

座中,有黄种人,有白种人,间或有一些黑种人。听众所着的服装,也跟肤色一样,各异其趣,这个会场超越了时代,也超越了国界,人人穿着自己最好的礼服来迎接悉达多。在人群中,悉达多看到曾在菩提树下跟他相会的阿蒙、库拉利奥、摩西,他们都很喜悦地笑迎悉达多,先回到梵天界的这三个人,还有梵天界的其他天人很热诚地款待悉达多。

阿蒙好像说了一句:「请到这边来!」就把悉达多带领到众人的面前。

虽说是集会的场所,然而这里是野外。一边是新绿的草坪,被修整得十分美丽洁净,草坪依地势微微上斜,数十位菩萨身着各色华服端坐其上,当然,天人们也在座,悉达多端详每一位天人的脸,都有种好生面熟的感觉,与每一位天人在四目相接时,两方都会欣喜若狂地说:「呀,是你啊!」悉达多好似离乡三十多年,如今重返故里,感到分外亲切。人人的眼中满含着久别重逢的兴奋泪水,悉达多站在斜坡的高处,静听阿蒙向大众介绍自己,经过了三十六年的岁月,悉达多开始在梵天界说法。

「宇宙间的万事万物,皆由缘而生,也由缘而灭,人的灵魂从无始以来就存在着,并将继续进化至永远,然而就在进化的途中,人们很容易造下罪孽,其原因何在?人的悲苦命运皆由自己的心田所造,也就是当自己的心田蒙上阴影时,心影以五官六根为缘而产生,人若想要解脱苦恼,除了以合乎『中道』的心来行『八正道』外,别无他途。五官六根对身外的事物产生执着,而『执着』来自不知足的心,若想求得此心的安逸宁静,道先要懂得过知足而了无挂碍的生活,对于人生中的责任与义务能有所觉悟,是一件很重要的事。我见此处结了各色人种,真正是『四海之内皆兄弟』的情景,我们要把这个事实传扬给下界的人,这个使命可说是很重大的……」

悉达多缓慢而清亮的声波,荡过澄澈的大气层,庄严地流布于四座之间,悉达多此时的声音迥异于人间所用的声音,犹如宏亮的钟声,阵阵撞入人心。他的一言一语,化成了流光般的魂体,不断在听众的心中摇曳着,大约一个时辰后,悉达多的说法告一段落。虽然悉达多说完了,可是听众们似未察觉,会场上一时鸦雀无声,接着大地如崩裂般喧声雷动,掌声不绝于耳。每一位与会者都因无比的兴奋,脸上泛起了红潮,全身也因而感动地颤抖不已。

\mt{灵体相离的奇妙境界}

悉达多辞别了众天人,返回在菩提树下静卧的躯体中。这种灵魂(诚心)出窍的事情,就一般人而言,是很不可思议的。他那寄居于肉体的灵,于脱离肉体后,却还能感觉真实地神游另一世界,且历经一段不短的时间……这现象实不可思议。一般人或许都有过一两次身历其境的梦境经验,有的人甚至经常有此感受,这是一种灵体脱离的现象作用。灵魂脱离肉体后所进入的世界,有层次的区分,这完全取决于个个灵魂的人生境界,境界高的,自能神游于高层次的世界,境界低的,只能神游于低层次的世界。在灵魂神游的世界中,也有太阳,并有阳光照射人心。阳光因人心的不同而显出千百种不同的色彩,不明究里的人,只因所见的阳光远较人间的阳光柔美,就赞不绝口地以为那是至美的景象,殊不知阳光色彩的明艳与否实有系于人心的状态,人应继续修炼心性,以期达于更好的境界。

悉达多的灵魂能脱离躯体达一个时辰之久,是他未曾有过的体验,记得往日在宫中的地窖冥想时,曾经验过如梦的幻境,自己在幻境中说法,众生都齐集跟前,而自己在说法,就这样,梦幻与现实间,他往返过好几次,当时的情景跟现在所体验到的情景比较起来,他清楚地感觉到,仍属此次的体验最真实,由于此次的景象太有真实感,使他觉得当初地窖中的冥想所现是接近梦幻的,虽然当时他也曾感到有如现实般真实,这一次,他是在大白天堂而皇之地以光子体(灵魂体)的自己向天众宣说佛法。

悉达多对于自己在冥想中获得的体验,非常感动,他又想到梵天界的种种情景。梵天界中的景致,美得不可以笔墨形容,天女们具有世间少有的高贵气质,仪态轻盈、动作优雅,是地上所谓的美女无法望其脊项的,她们的肌肤晶莹剔透,眼睛像紫色的宝玉般闪烁着动人的光芒,脸上不时漾着可亲可爱的笑意,男女天人的衣裳更是轻柔华美,色彩缤纷。天人之间,更没有「疏离感」,到处充满了自由、和煦与自然的气氛,大家相互包容融洽无间,悉达多更为之讶然的是,连动物和小鸟也都能虔敬地列席旁听他的法语,他跟梵天界中的朋友,其实都曾在一起相互信赖地献身理想、肩负责任。也都能做心灵的交流。

悉达多自回到肉体之舟后,深知自己将来必仍回梵天界,现在总算恍然大悟自己的归途何在,「肉体」是灵魂借以航渡生死苦海的「舟船」。此舟要以「中道」为舵手,否则航行错误,必多走许多冤枉路。

\section{布法之旅}\label{sec1.19}

\mt{转生轮回的真相}

肉体的船,取材自双亲的缘,而这个缘结自人类的本能,也有人把肉体比作灵魂的刀「鞘」,所以说得明达一点,死就是刀出了鞘,或是旅客下了船;而「生」,是灵魂与肉体结合的状态。悉达多就是由于想通了上述的道理,他对自己的过去未来,乃至其他人的过去未来,就都能了如指掌了,自己或别人在前世做过什么事,曾诞生于何处,又在几岁时去世,前世是哪一个国度的人,而且人生是如何过的等等问题,悉达多都能一一找到答案。也可以说,生命的转生轮回是一必然的形态,是可以寻获其脉络的,悉达多因此看到了自己的前世和过去无数世。

我们知道,悉达多出家的动机之一是他对死去的母亲摩耶夫人的思慕,当他探索赐给他肉身的母亲所处的世界,得知母亲正居于菩萨界中,于是欣然地前往母亲的住处,摩耶夫人见到已长大了的悉达多,不禁喜极而泣,并深深地为他祝福。

当一个人受了肉体的牵制,在贪图权位,名利和享乐时因而造下了种种的罪孽,死后必堕至痛苦的炼狱中接受惩罚,忍受煎熬,许多众生都被肉体拖累了而仍不自知,使其灵魂往复于生死道上,受尽许多辛苦,因而心灵的提升受到阻挠。你只要想到就在离你一寸的前方,有着无尽的黑暗在等候,就足够你心颤胆寒,然而仍有许多人明知情况是如此,却口口声声地为自己将步入黑暗找种种的借口,这是多么可悲复可悯的现象啊?很幸运地,悉达多透过了肉体与灵魂,看清了转生轮回的真相,了悟了人生的目的,得以突破身心两方面的障碍,带给我们众生无穷的希望。这个人世的真相是比什么都尊贵的,若不将之昭示众生的话,诚如梵天所言,「悟道」就失去了意义。任何凶残的力量都无法抵敌慈悲的力量,悉达多深知这个道理。很多众生,在现象界中,染了一身的罪恶,待灵魂脱离躯体时,始感叹万般将不去,唯有业随身的确切性,然已后悔莫及了,等到一旦又投胎进了现象界,这颗悔悟而明理的心又被肉体牵着莫辨西东。

慈悲心既尊贵,又伟大,实有慑服众生的力量。

在人世间,我们看到了光与影的相对存在,事实上,人世间充满了对比与矛盾。就因为世间有恶事,我们才知道善的可贵,如果人间只有光而没有影,就连光是什么,或我们处在光中这样的一个事实都不知道了,自然也无法了解什么是影。唯有在接受了光与影的两面,不去顽固地强行划分它们,我们才能享受到真正的安逸,地上的净土,也必须透过人们这样的包容态度才得以形成,反之,人不参透这个道理,无论经过多少时间,也不会有长足的进展,在生活与人际关系方面。悉达多对于悟道的重大意义,至此很深切地体会到,且刻骨铭心。

\mt{怀念恩师}

伽耶山上的二十一天,改变了悉达多的一生。过去三十六年间的经验和人生,仅在短短的二十一天里,以数倍大的幅度翻转过来。不安和困惑的人生至此告终,他战胜了群魔,超越了生死,掌握了一个随时可采取的智慧之囊,抱一颗镇定恒常的大无畏之心,悉达多即将坚毅超然地迈向前去。东方微露曙光,菩提树上的「居民」开始引吭高歌。

那熊熊燃烧的红火轮,不一会儿已转到上空,里面含藏了梵天界的黄金色,有着大自然慈爱与包容的心怀,悉达多反观现实,不禁想到:「我所悟的道,究竟该由谁来接续下去?」

这六年间,他结识了不少人,但他没有把握那些人还会记得他,其实许多人都对他印象深刻。在当时,迦毗罗卫国的王子出家修行的消息,人人争传,所以悉达多在修行林中的存在与出现,都是修行者们很瞩目的事情,另有伽那教的教主拉迦、伽那王子也差不多是那时期出家的,不过,王子出家究竟是很希罕的,在悉达多眼中最先映现的,就是阿罗蓝仙人。

悉达多于出家之初,曾在他门下达三个月之久,很钦佩他的为人,因为他对阿罗蓝仙人有一份特殊的情感,连同阿罗蓝仙人所在的王舍城郊外那明媚的风光,在日后经常很鲜明地浮现他的脑际,阿罗蓝仙人不但是一位博学之士,且心胸宽朗,门下贤德的弟子将近三百人,无论是思想学识或人格,都给了门生很大的影响。当时,悉达多慕名来到他门下,两个月后的一天,阿罗蓝仙人对他说:「我有意将衣钵传给你,希望你能体验我所传的道,我会把自己的经验和知识悉数传给你。」

可是悉达多后来还是离开了他,因为悉达多认为,在阿罗蓝仙人的身旁必不能悟道,原因之一,是他会有依赖心,他之所以被选为衣钵的传人,是由于他颖悟过人,虽有这个条件,但是如果他不能彻底地悟道,就不能达到他出家的目的。

现在他终于悟道了,要想把悟得的境界传扬出去时,他首先想到阿罗蓝仙人。于是悉达多收摄心神,在定中,他清晰地看到王舍城的街道,而他曾修行的森林区也赫然呈现眼底。

「阿罗蓝仙人不知在哪里。」悉达多在阿罗蓝仙人的修行处来回察看了好几次,都不见他的影子。阿罗蓝仙人在六年前就已经一百二十岁了,一百二十岁的高龄在当时是很少见的,这多半因为他生活有节制,为人很和煦,有以致之。如今阿罗蓝仙人下落不明,悉达多纳闷不已。

「奇怪,他到底到哪里去了?」正这样想时,梵天的声音在耳边响起。

\chapter{第二章\ 五位阿罗汉}\label{ch2}
「释迦牟尼佛陀,你如今已证得无上至理而成为佛陀,你若想宣扬佛理,可先去传给憍陈如等五人,……」

五位比丘向佛陀行了皈依大礼,成了佛陀的第一批入室弟子。

\newpage
\section{观自在力}\label{sec2.1}
\mt{成为佛陀}

「阿罗蓝仙人在一个礼拜前逝世,已经来到天界。他的弟子们都已各自去修行,所以你不必前去找他了。」声音出自阿蒙。

悉达多的心里突然起了一阵孤寂的寒意。

「释迦牟尼佛陀,你如今已证得无上至理而成为佛陀,你若想宣扬佛理,可先去传给憍陈如等五人,想当初他们皆曾与你同甘共苦。这五人现在在鹿野苑一地修行,你可以到那里去……」待阿蒙说完,佛陀就开始闭目静观,看到五个人所修行的地方,以及各处一方正在修禅定的五个人。

「噢,他们在修行。」佛陀见五个人仍在精进求道,就决心照阿蒙所言,去跟五人见面。悉达多经由阿蒙而被称为「佛陀」,「佛陀」乃梵语音译而来,意即「觉者」,对宇宙至理既能自觉又能觉人的人,具有观自在力。观自在力就是一种能透视过去、现在、未来三世的超能力,不唯对于一个人的命运,乃至全人类的未来都能如影片之放映,历历在目。

当一个人修炼至菩萨的果位时,必定具有观自在的超能力,并在众生所处的现象界中,能如实看透其中隐含的真相。这种能力也有高低的区别,视修行者修行的工夫而定,即连菩萨,也有因阶段的不同而能力互异的,功力深者,视界或理解力都会随之开展,而事相的真实面就益形明显,诸如原因与结果,真实与虚假等都逃不过他的眼睛。佛陀的「观自在力」(或可说「神通」)则超越了菩萨而达于终点,已契入了大自然的意识中,具有至高的地位。

\mt{宗教同源}

耶稣被称为救世主,他同样秉持了救渡众生的意旨降生人间,他也能随时随地行使观自在的能力,并以之领导众生。耶稣对于人生的至理,并不长篇大论地多予宣说,我们只要翻开圣经就能了解这一点。他以人们最容易接受的方式来宣扬博爱的精神,\xpinyin*{汲}汲于使众生快速把握住爱人的要意,实因为当时人人险恶,沉溺于欲乐,正是魔王大行其道之时,而人人都几近禽兽,天良即将泯灭。耶稣的职责,在于宣扬爱,只取人生至理的一小部分先规范人心。至于佛法和慈悲的大道理,则由释迦牟尼在印度宣扬,像他们这种济渡众生的职责,绝不因时代的变迁而稍存怠心。

佛法就狭义的来说,是佛所说的道理,然我们都知道佛宣说的道理非创自己意,而是阐述宇宙既有的真理,故而佛法就广义而言,就是宇宙的正法。既是正法,其难深奥妙,不言而喻,故传教者须灵活运用三种权宜的方式来使众生信服,此三种方式就是「文证」、「理证」和「现证」,最好能三者兼施。「文证」即指以文字或语言将宇宙人生的至理详尽帮助的一种传道方式;「理证」就是运用科学的探讨方式来将佛理公诸众生;「现证」则是舍文词叙述和科学\xpinyin*{剖}析,而以实际的工夫将佛法显扬出来。

人类在三千两百年之间,先后出现释迦牟尼、摩西和耶稣。此三人分别根据所处时地的实际情况,来实践他们救渡众生的宏愿。释迦牟尼宣说佛法,留下了许多金玉良言,也留下许多文章典籍,让世人从文字中证悟其中的道理;而耶稣,则担负起救助病人的职责,替许多病人治愈了奇难杂病或沉\xpinyin*{疴痼}疾,因而接引了许多众生;再说到犹太教教主摩西,在带领以色列人离开埃及时,更是留下了累累的奇迹,使后人钦慕瞻仰。这三位教主,并非在传教的方式上各有偏巧,事实上佛陀亦曾赢得医王的美称,只因每人所遇的众生根基不同,而在做法上必须有所权衡。

佛法的宣说是释迦牟尼的职责,又因他确已体悟了圆融的佛理,所以阿蒙称他「佛陀」。除阿蒙外,另两位梵天也曾说:「下界的众生现在已陷入了迷惘与罪恶的漩涡,求出无期,也不知道人生的目的和使命。你的职责就在于如何让人类领悟他们本身所具有的价值,并设法解脱一切的痛苦。」

想到这里,佛陀从菩提树下的根座起身,准备自悟道的所在伽耶山出发了。说到准备,佛陀实不需多少时间准备,因为他身边的用具很少,全部财产就只是一个乞食的钵,一个鹿皮制的\xpinyin*{漉}水囊,还有一个储存饮水的竹筒,此外,尚留有些许乞来的食物。所以出发前,要不了五分钟就一切停当,又在离去前的最后一分钟,他把已熄的火堆收拾了一下,之后,佛陀又对他住了二十几天的菩提树,望了最后一眼,这才迈开大步往尼莲禅河走去。他在尼莲禅河中洗去身上的污垢,并把沾满灰尘的僧衣也洗涤一净。

由于烈焰的照射,佛陀潮湿的身子很快就干了,为了等僧衣干透,他在树荫下小坐纳凉。正当他屈膝而坐,凝视河流之际,一个念头浮上心来。他想:「如果陆地上没有像尼莲禅河这样的河流,陆上的生物将会如何?草木和动物又将如何维持生命?将河流比之于人体的器官,则它恰似人体内的血管组织,我们不能想象没有血液流动的人体是什么样子,河流或血管,都是生物生存所不可或缺的条件。看来自然和人体,都为了要维持生命,而造成了自身必要的生存环境。」

在此之前,尼莲禅河一直被视为供给饮水和洗涤污秽的地方,如今他看到尼莲禅河所显示的大自然微妙的一面,尼莲禅河告诉他,大地早已很精密,并很有计划地发挥其全能性。想及此,佛陀的心田重又溢现出混杂着惊异的感激之情,晒干后的僧衣,穿起来格外地舒爽轻快。

\mt{启程前}

他终于要出发了,虽然涤净后的身心十分爽朗,他却有一点紧张的感觉,因为,求道和传道,是两种\xpinyin*{迥}然不同的处境,他人对于自己跟梵天的对话,跟魔王的交战乃至制服魔王,以及对「宇宙即我」的体验等,究竟能了解多少?讲给他们听,他们会相信吗?这些思绪困扰着他,但他随即了解到自己之所以会如此不安,仍因自己被外在的现象束缚了,因此当他从这种现象中脱开,那安然若定的心情又回来了。当问题来临时,我们要学会甩开拘泥和执念,让自己依循着正道走去,一切都能迎刃而解。

佛陀驻足寻思,想把各种体验再回味一次,回想自己自降生人间,已过了三十六个岁月,也将春夏秋冬四季的更替,体验了三十六次,随着大地,在这三十六次的轮回中渡过,此期间,他接触到自然的「法」的姿态,可是反过来说,自然亦常出之以严厉的姿态,不容许地上的人,任意以他们的「知」和「意」来干涉的。如果我们将「知」发展为「智慧」,将「意」扩展为「大我」,而将心胸推至宇宙般恢宏的境界,大自然随时会揭开严峻的面纱来跟我们亲切攀谈。

大自然表现的是一颗「心」,要教的,也是我们的「心」。春夏秋冬之能规律地轮替以孕育万物,就因为在大自然深处有所谓的「心的热源」,发挥了它的力量。此力量来自无限的慈悲和仁爱的意识,如果我们能接触的话,大自然就是自己伟大的老师,知心的朋友,同时就是自己本身。

众生想要深入自然的心底,首先自己需先具备一颗自然纯朴而率直的童心,同时,心底没有悲哀的阴影,能把欢乐分享他人,这样的一颗心是极其重要的,甚者,能心怀慈悲,布施他人而不求回报,更是迎接自然之道。其实,大自然一直都敞开了心扉在等待我们前去,迟迟不前的,不是自然,而是人自己。「自然」似乎是不可理喻的,当他降雨、干旱而致成灾时,就让人接触了他那近似冷酷的一面,然而有时候,他又好像一位不拘小节的浪荡者,当我们真正接触到自然的心灵时,像这类晴雨不定的诸种因果现象,就能清清楚楚地摆在面前了。至于大自然的脉理,早有古圣先贤以独到的慧眼与见识,将之阐扬论述了,但是人类的心胸不够宽广,眼界不够远大,对于宇宙至理的认知常是偏颇不全的,像瞎子摸象,各据其理,如印度自古即有的吠陀,五明之学,就整个宇宙而言,其所述的道理偏狭而有限,有时甚至是出自人类自己杜撰,因此,能够正确评度事理的人就如风毛麟角了。

人自呱呱坠地起,就背负了所谓「业」的苦恼来度他的一生,经过历练后的佛陀,此心不再骄\xpinyin*{矜},不再\xpinyin{阿}{e1}\xpinyin*{谀},既不焦虑,也没有不安,更丢开了一切的执念。然而这样的一位人间的智者,在数周前,曾一度\xpinyin*{濒}临绝境,与死只隔着薄薄的一层纸。如今,在看清了三世的因果之后,对于自己能随时如囊中探物般运用内在的智慧,已不感觉奇怪了。更感安慰的是,自己虽然\xpinyin*{孑}然一身,然而身后却有无数梵天随时协助,如果问他问题,他能毫不迟疑地给予答覆;如果他人有何困难,他也能一目了然,并给予帮助。能有这样的神通力,并持之济渡众生,是他感到最幸福的事,悟道后所得到的充实感,是一般人无法想象得到的,一想及有梵天从旁协助自己的事实,就会感动得热泪盈眶。

「正法」正是他锲而不舍的目标,也是契合真理的心的秩序,在内心确立起一切井然有序的意象,必须使用「八正道」的尺度。合乎正道的思想和行为,能够改善「业」所带来的命运和偏颇的个性。

人在初生之时,都具有浑圆、纯真和憨厚的心,可是随着年龄的增长,被累世带来的「业」牵引,由「业」中又繁衍无数的「业」,最后形成个人的性格。

「悉达多,你不救渡众生的话,还指望谁来救渡呢?你应点燃人间的法灯,否则,人类的末日必定到来。」阿蒙所说的话,又句句敲击他的心,所以悉达多秉持了上天所赐示的正法,将以之接引众生,使他们也同获觉悟,这个职责,他永不会忘记!

\section{向鹿野苑进发}\label{sec2.2}

\mt{两位游化僧}

尼莲禅河的水面,因阳光的照射而金光闪现,那细碎的金光,就像夜空的星星,闪烁跳动,像一个个顽皮的小生命。他带着怀念之情向伽耶山告别,往日熟悉的风景、小鸟和小鹿等,都得暂时说再见了。虽是短短的时间,一旦要离开这个使他悟道的地方,顿时生起难舍之情。

佛陀开始向憍陈如等所在的鹿野苑进发,沿着尼莲禅河轻快地走去,走着,走着,他突然想道:「如梵天所说的以及我定中所见的,五个人真的还在修行吗?」

此时,心中响起一个声音:「你现在从伽耶山,走到拉加库利的街道,再从北方向东北方进行,通过山路,到达玛哈拉国的边境,再由恒河西上,进入\xpinyin*{喀}西国的首都柏拉那西,再找以西巴那达这个地方就是了,你要相信我的话,并体会我话中的意思。」

「会的,我必会照您的意思去做的。」

以旁观者立场来看,佛陀这样好像是自问自答,梵天的声音来自佛陀的心中,佛陀也是只听其音,不见其形。现在他具有能与梵天互通消息的率直的心,并且觉悟到,自己所追求的道路,就近在咫尺,也可以说,就在自己的心中。他放慢脚步,在山间\xpinyin*{踽}踽独行,想到过去和现在,自己已前后判若两人了。回想初到伽耶山时,万念俱灰,感受到身心的枯槁状态,而同样一个人,在不到一个月的短短时辰,于离开伽耶山时,心中亮起无限的光,他的心就是青天与白日,无着无碍,带着这样一颗心,佛陀再度踏着轻松愉悦的步子向前走去。

拉加库利的街道就在眼前了。当佛陀正在树荫下乘凉时,有两位僧侣也来到身边,坐下后,对他说:「天气很不错啊,你从哪里来?」

口气十分热络。

「我从伽耶山来。我在那儿修行。」佛陀回答后,就端详对方。两位僧侣差不多四十上下,一个是长脸,一个是圆脸,跟他说话的那位,就是长脸人,嘴巴稍稍突起,是一副爱说话的样子,看他结实有力的体格,就知道未曾认真修过苦行。如果他的头发稍加整理,简直就是商人的模样,一点也不像修行的人。佛陀很想为二人宣说正道,但是想想又忍了下来,此时长脸人又说:「我是婆罗门种,为了求道,修习过很严格的苦行。如今到处游化,前途好像很遥远,出家已十几年,到过许多国家。现在想到迦亚达耶的修道场去。」

佛陀闻言,心想:「那不正是乌鲁维拉喀萨巴仙人的所在地吗?」

摩竭陀国的频婆娑罗王曾提过这位仙人,且非常推崇他,所以若有缘的话,佛陀很想去见见他,这么一想,他对即将前往该地的这两位僧人,不禁产生了亲切感,于是他问:「您们想去跟乌鲁维拉喀萨巴仙人学道吗?」

「不是的,我们只是慕名去求教他一些问题,我名叫优波迦。有许多道理,我都急于想知道,不知您的师父是哪一位?」

优波迦说完,仰视佛陀的脸。

「我并没有拜谁为师,我敢说,我就是自己的老师,纵观世间的一切苦乐,都由自己的心田所造,只修肉体的苦行是无法驱除心中烦恼的。我因为能去掉执着的念头,所以证悟了无上的正道,我在人生的战场上,历经了千辛万苦,总算是打了一场胜仗,但未曾拜过什么老师。」佛陀以十分自信自足的口气说着,但是他感觉自己词不达意,此话并不尽如他的意。

坐在优波迦身旁的圆脸人,听了佛陀的这一番话后,目瞪口呆地歪过身子,上下打量着佛陀,半晌,才喃喃自语道:「噢,可能是有这样的人,嗯,可能有这样的人……」说着,就催优波迦上路,两人匆匆向佛陀道别而去。

「说话」这件事看似很简单,其实很难的。自己心中有什么想法,若不斟酌地说出来,会带给对方困扰。而且说话时若不站在对方的立场,给对方思考的机会,这也绝不是正确而有效的说话方式,因此,佛陀不断反省自己刚刚的言行。

「嗯,有时还应该带几分修行者的心情来说话的。」

\mt{进入众生的心}

佛陀一心只想着如何把握机会把正法传给他人,以致滔滔然地把自己悟道时的美妙心境宣说出来,不料事与愿违,因为论调太高远,反而使对方关上了耳朵。佛陀心想:「自己有时不免还留有王子的自负习气,或者说自己由于悟道而产生了自满心,所以才无法把握对方的心而把正法传出去。」因为如此,佛陀警惕自己,知道今后凡是与人说话,一定要将他人的心比作自己的心,多为对方设想,不再自以为是地灌输自己的想法。

「当稻穗成熟时,就低下了它的头。」

大自然界一句话也不说,虽然是什么都不说,可是大自然以它自身的形象实实在在地启示了一个道理。虽说自己悟道了,却不可以轻视他人,人要有谦虚的心,谦虚并不表示卑下,而是由于一个人的内涵越深,越具有如宇宙般无限宽大与包容的心所形成的一种外在的形象,谦虚的心,就是让万生万物继续生存下去的一颗心。

「宇宙即我——」就是这样一颗心,一直是释迦牟尼\textperiodcentered{}悉达多的心。

为什么自己一旦背负了这个血肉之躯,就那么容易被细微末节所牵制呢?问题是,自己至今还只记挂着自己的事情,还只是站在自己的立场来看事物。从今以后,应该进入人人的心中,才能再谈到济渡众生这件事,就是看到\xpinyin*{襁褓}中的小生命,也不能心存轻蔑,你怎能料到这个小生命,将来或可能成为领袖,也或可能成为万人引领的救世主呢?你又怎能想象眼前的星星之火,于一瞬间就能将整座森林化为焦土呢?

「不可着急,放下焦躁的心,我不妨漫步遨游,享受其中的乐趣。」

佛陀想着,就安下心来。

\section{怪异的景象——修道场上的妖气}\label{sec2.3}

\mt{人身上的灵气}

佛陀继续向前走去,一路上或采\xpinyin*{撷}水果,或托钵乞食,有时野宿,有时穴居,其间也遇有许多往来于道的僧侣。他在那些修行僧的身上,看到自己往日的影子,心中涌现复杂的情绪。

砂岩上,在烈日\xpinyin*{炙}烤下的苦行僧,正苦闷地枯坐着,坐禅本是为了悟道,然而这些人像是以肉体向大自然挑战,这种严酷的苦行,虽很令人钦佩,然而若仅是恪守着苦炼的信条而无法抛开自我的私念,还是不能悟道的。因为这种肉体的折磨,反而使自己成为肉体烦恼的俘虏,佛陀走近这位砂岩上的苦行僧,想对他说什么,可是随即想到,这些已陷入极端的苦修们,若向他们宣说「中道」,他们是不会听信的,这么一想,佛陀就默默地离开了。

进入拉加库利的区域,触目所及,仍与六年前的情景一样。肉体的苦行仍很盛行,僧侣们聚在一起修习苦行,每个人都为了想要悟道而使出了浑身解数,非达痛苦的极致不罢休。佛陀看到有些苦行场上弥漫着不可言状的邪气。红的、黑的、灰的颜色围绕着这些修行的人,有如水气般蒸蒸上升,没有一个人是带着金色、绿色或紫色光气的,一般人单凭肉眼是无法看到这种身体的灵气,必须透过第三者的「天眼」(即「心眼」)才可看到。一个人身上所发出的灵气,常因其心理的状态而显现不同的色彩,如果一个人被嫉妒,欲望,不满等情绪骚扰的话,身边就被红、黑、灰色重重包围;相反地,常怀慈悲、爱心而能清心寡欲的人,身上就会散发金色、绿色或紫色等等柔和而美丽的光彩,所以,一个人的内心状态,在一个具有天眼通的人面前,是无法遁形的。

灵气的色彩,常与一个人将要去的世界的色彩相辉映。如果身现地狱或畜生的灵光,就很容易引进那个世界的灵,而在个人的精神上或肉体上呈现异相。例如怪癖、失常、病痛、灾变等,原因之一,是由于前述两个世界的灵不断作\xpinyin*{祟}所致,像色彩柔美而给人安逸感的绿色、金色等,就是那光明世界的色彩。同样的,那个世界的灵(或称为「天使」)经常会引导带有这种光的人,使他过着宁\xpinyin*{谧}的生活。

修炼肉体苦行的僧侣背后,常有妖魔鬼怪的灵进进出出的,僧侣们那种不甘后人而强忍痛苦的心,已给了妖魔们可趁之机,他们不知除去心中的块垒,以为只有这样才能备有神通力。神通,本是为实践正法而需要具备的方便法门,这不是目的,而是一种手段。然而大部分的修行人都本末倒置,一心一意追求超能力的人为数不少,以致他们的心就一直处于不平的状态中,他们的修行场,以肉眼观之,场上的人表现出来的非人行为,给人一种怪异的感觉,以天眼观之,就可以清楚看到很多畜生和鬼怪在他们的面前,正极尽煽动之能事。修道场上的妖气,就是来自于人眼无法透视的鬼魅,佛陀曾遇到其中的一个人,很想告诉他这种肉体的苦行,只是一条悟道的迂回之路,正想开口,过去的经验告诉他,对一个不带耳朵的人,无论你说多少道理,都是枉然,再说,一个福份太薄的人,对于正理的领略,须假以时日,对于求道的人而言,也常有这种情形。

虽然有些人很早就已接触正法,并将之视为知识来理解,但是不起而实践的话,也难有悟道的一天,「正道」犹如具有信和行两个轮子的车,能载人抵达目的地。人在求道的途中,难免会遭遇难题,然而所谓的难题,对身体力行的人而言,往往变成易被解决的形式,当求道者每渡一个难关,他的理解就能更深入,而信仰的心也愈益坚定,于是信和行的幅度愈益增大,最后能平稳地抵达悟境。

\mt{伟大的使命}

佛陀一路通过修道场,边走边想:「有朝一日必重游此地,一定要为他们宣说正法。」

出了北门,来到那烂陀,在这里有一位善辩的沙门,名萨\xpinyin*{羌},有许多人都在与他辩论时败下阵来。佛陀早有耳闻,故见了面,总是尽量避开议论,好在也没有什么机会引起唇抢舌战的。再往北走,越过一条河,就是\xpinyin*{毗}舍离的市区。毗舍离的街景如昔,是个很令他怀念的地方,从前常来此地游化,他觉得能像此刻这样无牵无挂地一面乞食,一面游化,实在是很快乐的。想当初自己在这条街上乞食时,心中存着一个「悟道」的目的,可是现在,他已达到目的,如今是再回过头来接引他人,他有着自得其乐的闲情。

「这样对不对?我是不是在逃避什么?到底是不是……」这一连串的问题,掠过佛陀的心田。

社会,采取一种分业(分工合作)的形态成立,不容许有闲散,百姓要种果蔬和五\xpinyin*{榖},工商人士靠交易而生活,每个人都固守了自己的岗位,努力精进地工作着,这就是社会生活的基本法则。如果在社会各阶层间出现了自私自利之徒,则整个社会将失去平衡,各种不合理的现象亦将陆续出现,医生治疗病人,法律家维护人的权益,这些工作中,还含藏了基本法则在内的。

佛陀现在身处何种地位?为了要匡正人们的生活与行为,使人免于陷入地狱的呻吟,他正担负着很重要的职务,他要把社会的生活,引到叫做「中道」的轨道上,而对于人,他也有责任把他们由邪道带领到「正道」上,点燃他们心中的法灯。为此,他在游化时,绝不能心慌意乱,坏了传教的目的。

八正道的遵循,以及为渡越生死苦海所做的努力,其目的并非像人类所创造的物质文明以及由物质文明衍出的权位、名誉和财利等,其目的在于建立适合于人类生存的理想社会,人们处在调和融洽的环境中,过着合乎中道的生活,也就是一般人所谓的极乐世界。人生的价值,是透过一个人的工作和环境来显现的,须赖他是否能以一颗接近宇宙本体的心,来谋人类的福\xpinyin*{祉},使灵魂净化,工作,只是达成灵魂净化的一种手段和过程罢了,不可以错置了工作的目的,甚至沉溺其间而不能自拔。悉达多担负起救助众生,启迪众生的重任,为了达成此一任务,他要时时保持一颗清纯圆润的心,不使自己陷于迷惑,这是悉达多最感重要的事。

佛陀经过这一番分析后,知道自己的心理状态,以及自己所要走的路,绝不是一种逃避。他很自满自足,对于上天所赐的使命,重新涌现出自信、勇气和敬意。佛陀继续往前走去,沿途有山川草木作伴,他逍遥自在。最后到达帕拉那西。

\section{最早的弟子}\label{sec2.4}

\mt{帕拉那西街上的异人}

帕拉那西是一个商业很发达的都市,人口约十万。当然,以今天的眼光来看,它只不过是个很小的都市,但是在当时,这却是极为难得的繁华都市哩。土造的房屋密集地聚在一起,出了街市,就可见宽广的草原向外延伸而去,远接森林地带,土制矮屋,配以窄小的窗子,故屋内终年到头是幽暗如夜,不过这样一栋房子,在炎热和灰砂的防范上,要比木造房子强多了。马路两旁,商店连绵,人来人往,常挤得水泄不通,店中陈列的日用品,琳琅满目,吸引了行人的注意力,尤其是绢织品的买卖,更为踊跃。佛陀往昔在宫中所着的衣饰,大部分来自这儿,在这里,常有上乘的婆罗门来往于道,只消看他们的装扮和走路的姿态,就知道他们必具有高贵的身份,当他们出现在拥挤的街道上,人们立刻让出一条通道来,像婆罗门、武士等身份的人,在街上经常会受到这种特殊的礼遇,婆罗门是一种特殊的阶层,人们是无法不另眼相看的。

佛陀在街上,只默默地走着,在旁观者的眼里,他只不过是一个衣衫\xpinyin*{褴褛}的乞丐罢了。那一身衣服,由于汗垢、水渍,已显得污秽不堪,又因为经常在旷野里和衣而卧,也免不了常在烈日下\xpinyin*{曝}晒,衣质已很脆弱,裂缝和破洞,比比皆是,甚至肩胛处的裂口,一眼就能望见他的肌肤。头发和胡须,由于竟月未剃,已敷满\xpinyin*{颔颊}。这位到底是沙门,还是乞丐,真叫路人费解,与乞丐大不相同的是,他的目光炯炯有神,气宇高雅,步履稳定沉着,当他从你身边经过先看到的是这一身破烂以为是乞丐,而不经意地别过眼时,随即又有异样的感觉闪过心头,使人忍不住要回头再看一眼。此时他那飘逸超然的后影,令人不禁肃然起敬,疑是来自天外的仙人,佛陀现在已经全然不复有优越、阿谀和自卑的容色,专心依循八正道来看,来想,来说,而彻底地实践,在时间之流中,他让一分一秒都带着他的合于中道的思想和行为。

他的目的地鹿野苑,在帕拉那西的郊外,跟毗舍离郊外的苦行林不一样的,是鹿野苑的修道场中,所有种族出身的修行者,都为了求道而聚集一处,他们采行各种方式,然不离以肉体的苦行为中心的修行,在当时,这种肉体上的修炼,完全是一种自力的求道行为,一般人为避开连年争战所带来的不安,故一心只想求得神通,以获\xpinyin*{庇}身之处,可以说这现象是当时的时代背景所造成,此外,跟民风淳朴也有莫大的关系。当时的人由于资质纯厚,感情较丰富,故也较能领悟生命的本质。不像现代人,迷失在物质文明之中,若与其谈神论道,往往嗤之以鼻,虽然经常会身陷矛盾与痛苦中,却耻于面对它们,只凭借短暂的欲乐来稍解于一时,现代人之不易近道,于此可见一斑了。

佛陀对苛酷的苦行虽时有微词,但是这种竭力求道的态度,仍是十分可取的,曾几何时,现在佛教的修行真义已被搁置一边,竟为了迎合时人的心理而成为必须藉助「他力」的一种信仰,更甚者,佛教的寺宇成为众人观光的地方,而平时又只是为死者诵经拜忏,做些形式上的普渡仪规。

人为什么要劳动,该做些什么,什么才是有意义的事情,诸如此类的问题,已很少有人肯认真地去思考了。大家都似乎甘为物质的奴隶,日复一日,形同没有知觉的梦游患者,只在街头巷尾游来转去的,人竟如此,简直是到了穷途末路了。人一旦忘失本心,将有怎样的结局?那只有一条路,就是堕入无底的深渊,永无出期。

\mt{皈依}

鹿野苑的气候很舒爽,阳光和煦暖和大地,此地既无密林,又无泥沼,野兽很少出没,修行者更无须避居洞穴大岩或大树下。许多僧侣就散落在草原或平地,安心地修着禅定,往日在宫中,曾由商人或武士处得悉此地的情形,如今身临其境,发现此地是个比传闻中还要引人入胜的地方。憍陈如等修行的地方就在附近了。他们现在在想什么,以及如何修行,佛陀都一清二楚。他在此地过了两个晚上,仍继续反省和冥想,并于深沉的境界中,看到五人的模样和姿态。现在经由梵天的指导后,佛陀能随时调理心性,察看他所想知道的情景。

一条美丽的清流,正静悄悄地横越鹿野苑的修道场,山坡上的绿色,夹杂着鲜艳的五颜六色,倒映河面,并随着水波上下荡漾。憍陈如等就坐在离岸不远的林子里。佛陀驻足望了他们好一会儿,这才走近前去。憍陈如见到佛陀,稍现惊慌的容色。沉吟了一下,就回头对跋提不知说些什么。不久,佛陀已来到他们身后。

「好久不见了。我随你们身后而来,今天终于见面了。」佛陀拍抚憍陈如,亲切地打着招呼。憍陈如当然听到佛陀说了什么,其他四人则充耳不闻,也不搭理,他们既不想知道佛陀此来的目的,对他的出现也不表欢迎。

「憍陈如,你有没有听到我的话?你看着我啊,我已经悟道了,我已证得了『观自在力』。你们的修行太过偏颇了,本来是为了消除烦恼才修苦行的,可是修到后来,你们反而由于肉体的『执着』观念,而自添了一层烦恼,迷失了自我。……」

佛陀一口气说到这里,可是憍陈如一句话也不听,只固执成见地说:「你已不是我们的师父,也不是王子,而我们是修行的人,是沙门,跟你这种自甘堕落的人说话是很不值得的,你还是到别处修行去吧!」

「憍陈如,请看看我的脸再说话吧,我不是从前的悉达多了。我知道你刚刚跟跋提和阿舍婆誓说:『悉达多来了,一个人修行害怕了,所以来找我们作伴。我们不必理他,他既不是师父,也不是王子,再说,我们也不是武士了,别去管他的事。』跋提,憍陈如是不是这样说的?」

阿舍婆誓和跋提惊讶不已,忙不迭地回答着,且不约而同仆倒在佛陀的脚边,然后,两人抬起头来仰视佛陀的脸,憍陈如为之语塞,只能默默地垂着头。

「我现在已达到观自在天,你们心中想什么,以及如何来到鹿野苑,我一一都知道。你们在伽耶山说,要到别的地方去修『观自在力』,现在你们的心中仍有解不开的结,因为你们如果已悟道,必会知道我内心在想什么,怎么样,憍陈如,你的脸显出不安的容色,那是由于你的心还未找到安逸,为什么?」

佛陀的一字一句都刺进憍陈如的心坎。憍陈如很想辩驳,可是立刻感到自己的心意像被看穿了似地尴尬不已,一个字也说不出来。同时,他从佛陀充满威严的声调中,感觉出他不再是从前的悉达多。他突有所感,自己求道六年多,至今才真正碰到了由天而降的开导宗师,一向倔强的憍陈如,渐渐拾回了久已忘失的率直的心。

「悉达多殿下,我真的无话可说了,我刚刚几乎要堕落了,请原谅我的无礼。」说到这里,憍陈如偷偷抬眼望着佛陀。

「噢,你不必道歉。就因为你们坚持要离开我,才使我有机会重新反省自己。回想六年前,与你们共同修习肉体上的苦行,如果长此以往,我必因营养失调而死去,还谈什么悟道呢?那天我听到村姑的歌,才领悟到毫不放松的修行是不正确的,如果能把偏激的行为和思想拨至中庸的道路上,这才是正法,也正是开悟的途径。如今,我能看到你们的内心,也能看透你们的过去和未来,我真的已经达到观自在的境界,憍陈如,你应该把心搁在正道上去真正地生活,当然,你对我的事多少还存有疑问,要记住,诚实的心比什么都重要。」佛陀谆谆地开示憍陈如,跋提等在一旁,也深觉自己过去的错误频频点头,可是憍陈如虽觉得佛陀的话很有道理,仍是一团疑雾。

「悉达多殿下,我实在不明白,在短短的一个月前,你还只是一个平凡的修行人,为何有这么大的变化呢?这几十天,对我来说是一片空白,现在的您,真的是变了一个人似的,我看到您,也能感到很宁静。像这样宁静平稳的心境,还是生平头一遭的体验,请您立刻把所悟得的道理教给我吧!请答应我吧……」憍陈如说到这里,就向后退两三步,只见他魁梧的身形仆了下去,不一会儿,又站起来,在佛陀的周围绕了三匝。又忍不住伏在佛陀的脚边,表现了无限的虔敬之意,其他四人见状,也依着憍陈如刚刚的举动,皈依了怫陀。

就这样,佛陀收了五个弟子,正如梵天所说,佛陀一路来到五个人修行的地方,并将他们收为弟子。

「不要再犹豫了,你只管放心去吧!」梵天的话,又浮现在佛陀的心上。

六年,看似很长,却又很短,虽说跟五个人同甘共苦,却免不了是城中生活的延续,五个人总视他如太子般护卫,如今,他们以师徒的关系,紧密地联系在一起,互相成为心灵上的朋友。他们之间,有一线坚韧的「心」是永远不会断的,佛陀这样想。

\section{启蒙与光明}\label{sec2.5}

\mt{观自在力人人可得}

一个具有观自在神通力的人,对于他人心中的隐秘,能一清二楚地窥得,不,可以说是清晰地呈现在他眼前。观自在力,是人在其表面意识与潜在意识达于交融的状态下所产生的一种超能力,非但如此,它还是在灵魂转生的过程中培养出来的内在能力。一般拥有这种能力的人,被称之为「观自在菩萨」,而佛陀在这方面的能力更是达于极点,所以才被尊称为「佛陀」。「佛陀」乃古印度语,除了译为「觉者」外,其涵义是很深远的,所以保留其音,而韩国、日本也依中国所译的音「佛陀」而沿用至今。说真的,要想证悟「佛心」,绝不是一生一世可达成的,需要渐次修炼,所以有许多人都是先证阿罗汉的果位,亦即先了脱自己的生死,再谈到进一步的修炼。一个人在修炼之初,一定得先靠自己的力量把握自己的心,而功力的深浅,则有赖所下工夫的大小。等达到某一程度之后,自有神明守护左右,此外也有神明从旁导引,不使误入歧途,待时机成熟,「观自在神通力」自会不求而有了。

人生在世,为的是实现正道,要想过调和的人生,自己可以创造调和的环境以达成此一目的,「观自在力」并非来自外在,而是伴随着调和的心境而产生,我们不可舍本逐末,把修行的目的给混\xpinyin*{淆}了,憍陈如等人虽说是武士出身,但是跟着佛陀出家,秉着一颗坚毅的道心,此刻又倾注全力地谛听佛陀的法语,这就是佛理观之,他们早已具有准备悟道的根基了。关于人能修得天眼天耳等神通的一般道理,他们五人过去曾在婆罗门的经典中读过,也深深理解到个中的来龙去脉,只是憍陈如,因个性倔强,自认自己的修行方式没有错,故对于佛陀能以不同的方式证道而深感不解。

「悉达多殿下,您当初破坏修行的戒律,喝下\xpinyin*{腥羶}的牛奶,这等于是放弃了修行,现在竟能达到悟境,反而我们这些一直守着戒律的人怎么得不到这种能力呢?」憍陈如边叩头,边急切地问道。

佛陀早已看穿了他的心思,同时他知道这也是其他四人心中的疑问。

「我很了解你的想法,因为我也曾有过这种疑问,相信不久你就会明白了。不过,你想立刻消除疑问的话,首先,你得承认我是佛陀,要相信我已彻悟了真理。」佛陀安详地告诉他。

「哦……是……佛陀。」憍陈如很小声地像对自己说着,庞然的躯体,也好像一下子萎缩在一边,另四个人,也随之仆倒在地,似乎把心中所有的东西都抛出来了。

夕阳染红了西边的天,连带把嫣红的云朵投映于河面,使清流顿成一条随风波动的彩缎。这一切都告诉你,黄昏来临了。

\mt{法音流布}

在夕照中,佛陀用很缓慢的语调继续说:「你们的生活太过偏颇,只须放眼望望这四周的树木花草,你们就能了解,一棵树,如果枝干粗的话,它的根必也特别宽大,枝由根而生,小枝又依大枝生长,绿色的叶因而繁茂了,整棵树既雄伟又壮丽,相反地,若是枝叶重过根干所能负荷的,根干则不支而断倒。我们看,当根和干,干和枝,枝和\xpinyin*{枒}调配适度时,才能在风雨中依然挺立。这些枝叶与根干,很明显地告诉我们一个中庸的道理,它们显现了一颗『中道』的心,人的道理又何尝例外,心就是人的『根干』。事实上,任何事物都具有其『根』,若全部忘失的话,则那叫做五官的『枝叶』,就会不安地摇晃起来,带给你无穷的烦恼。你,也就别想拥有安逸而正当的人生了。」

五个人全部低着头,使尽了全身的注意力,来听受佛陀开示。

「我在年幼时,曾一度走错了路子,过着养尊处优的生活,并滥用权力,为所欲为,欲望虽能满足于一时,可是心灵无法平静,反而是生活越富裕,心中的疑窦也越大。你们看,宫中的贵族,和宫外的奴隶,过的是怎么样一种天壤般悬殊的生活,同样是人,为什么有的人就注定是如此不幸?太阳的光,超越了阶级,平等地普照大地,只有人的社会才有许多不平等的现象,这是为什么?

我在未得知生母亡故的真相前,一直在继母的照顾下成长,那时真是无忧无虑,要什么有什么,逍遥而自在,可是一旦知道自己还有一位亲生母亲,心里因无尽的思念而深感不安,又因看到国与国间的连年争战,除了大肆的破坏之外,就是许多无辜者的死亡,此外,每天还得忍受随时被刺杀的威胁,所以,虽然父王为我准备了适应四时的行馆,我的心却没有可安住的地方。就人身而言,是不可能在这种优越的环境中觉悟真理的,在离城的六年间,我不需对衣、食、住等问题伤脑筋,也不再意识到敌人的存在,过着相当安稳的生活,不料,本是为了消除烦恼才修习的肉体苦行,反而在自己的心上生出对肉体的执着,这一来,妨碍了我的悟道,后来我领悟到,粗陋的生活和严酷的苦行,都不是悟道的必要条件。」

佛陀的声波,在五个人的心灵深处回荡,有的闭着眼睛,有的低垂着头,没有人显露出辩驳的意图。他们跟着佛陀的话语也进入了往日的情景,并回顾六年来修行的种种,虽然思绪各异,然忏悔之情则一。佛陀停下来,询问憍陈如:「憍陈如,你以前不是常听歌女们唱歌吗?
你可曾注意她们弹奏的琴弦?如果弦调得太紧,会怎么样?又如果调得太松会有什么样的声音?你们可曾注意?」

「哦,对……」

对于佛陀突如其来的问话,憍陈如有些措手不及,他略微思索了一下,回答道:「太紧的话,就会断掉,太松的话,就不好听。」

「正是如此,人生也像琴弦一样,旋得太紧,难免要断掉,所以严格的苦行,反而会替人带来新的烦恼,走错了这一步路,不但是肉体,即连心灵也会固执起来。我们要知道,肉体只不过是人生的一个渡船,真正的舵手是这颗『心』,是永远不变不坏的,它才是你的本来面目,一旦忘失,只好任由肉体摆布,如失舵的渡船,飘泊不定,永远达不到彼岸。

我们还可以说,如果你遗忘了真正的自我,则憎恨、嫉妒、诽谤、嗔怒、欲望等情愫就会趁虚而入,占据你整个心田,你就只好永远注定被它们牵着走,而离真正的自己越来越远,很难再回头了。如果明白这个道理,你就会发现,所谓痛苦,都是人自己的心和行为造出来的,由于有生的事实,而后会病,会老化,乃至有死亡,这是人人必经的过程,说穿了,人生就是一个生死的苦海。如果人想从这苦海中超脱,就必须彻底地由自己的心和行为下工夫,取之于中庸之道,离开过与不及的两个极端,这是不容忽视的要件。

肉体上的五官与外界接触后,自然会在心上产生各种形象,其间因果与善恶的判断,本应仍由『心』来担任,取代了『真心』(或说『本心』)的地位,形成每个人不同的人生。譬如说,有许多人,只要是对自己有利的事,说谎也很不在乎,与人交往,从不出以真心,为了保护自己,把自己伪装起来,而整日为着如何伪装而煞费周折,烦扰了自己的心。但是午夜梦回时分,他们免不了会扪心自问,如此伪装,所为何来?一个人的骗术纵使能瞒天过海,却瞒不了自己的良心,这是任何人都无法否定的事实。由于这个事实,我可以断言,人的内在有一颗心与光明正大的本体同源,人在初降人世时,心本是圆润而丰硕的,可是后天的环境与思想骚扰了它,你不见有许多人,由于受到某种伤害而生起憎恨的心,这颗憎恨的心必将遮蔽善良的光芒,而将自己陷入益发悲惨的境地。

人在漫长的人生旅程中,由于旅程上的荆棘与坎坷,使自己原本光明正大的心笼罩在不幸的阴影中,以致失去了光明正大的安全感,从而辗转反覆于痛苦的漩涡中,我为了追求一条康庄的大道,才决心出家,并跟各位在幽暗阴森的山林间苦修六年,这期间,我心中充满疑惑,并暗自摸索了一阵子,苦于无法接近解脱的门径。在我接受一位少女布施的那一刻起,我领悟了中道的真义,然而却触怒了你们,事实上,那只是导火线,也许我们六年来朝夕相处的日子中,已有许多误会和不满的种因在你们心中了,再加上你们也因悟道不成而满怀忧郁,所以,见我不忌腥羶,就立刻产生厌弃的心理,一个个不由分说地离我而去。

你们可知,当时我注视你们沿着尼莲禅河渐去渐远的背影,我内心感到非常寂寞,你们不知道我当时已领略到一种中庸的道理,因为当我体力不支时,我就无法集中心神来修道,更遑论肉体毁灭时的情况了。你们走后,我下定决心重新反省自身,于是在菩提树下,将自己三十六年来一切经历,痛切地予以\xpinyin*{剖}析和检讨。当然,我在反省时,是用了合于『中道』的尺度,尽量以第三者客观的态度来评断事理,完全抛掉了私我的念头。很侥幸地,我终于能够不再执着生与死,乃至任何困扰我心的现象,当我具有这样一种无牵无挂的心怀时,我证悟到人生无上的真理,我看到生老病死带来的一切痛苦,人若想出离,惟有时时用一把尺度来规范自己的行为与生活,否则稍不留心而误入歧途,就无可避免地被痛苦缠绕了。」

如流水般宣泄不断的法音,又像划过苍穹的流光,发出永恒的光辉,因为那中间,有着无尽的慈爱。五个人的心扉敞开了,那一道道的光,就如甘霖之于沙粒,毫无声息地\xpinyin*{翳}入他们的心底,他们在佛陀的法语中彻悟前非,踏出了新生的第一步,周围暗下来了,远处的修行僧们劈柴生火的声音,隐约传来。

「各位请抬起头来,天色已晚了,我们也捡些木头来生火吧,等生好火,我再继续跟你们谈。」

五个人这才从忘我的境界中回到现实。憍陈如、阿舍婆誓、跋提等人的面庞上,充满了眼水和灰尘,显得很狼狈,每一个人都满怀感激和喜悦,由于长时间的蹲\xpinyin*{踞},两腿麻\xpinyin*{痹}而不自知。

「唔,好的,佛陀,我们这就去生火。」

「咦,对啊,天都暗了,竟然没发现,得赶快动手了。」

阿舍婆誓说着,忙着站起身来,因为不曾留意自己蹲得过久而发麻的腿,以致一个\xpinyin*{踉跄},又\xpinyin*{跌跪}下去,他稍事停顿,再支撑着起来,不以为意地对大家说:「我去捡柴!」说完,就一拐一拐地快速跑至河边,并要跋提把自己捡得的柴火搬至佛陀处。憍陈如为了要火种,也快步向邻近的修道场走去。

\mt{慈悲的灵光}

他们选了一处更平坦的地方,然后围着火堆坐了下来,五个人全神贯注地看着佛陀,渴望他立即开启金口。

他们中间的薪火,照亮了每一个人的心,他们感到未曾有过的温暖,现在,每一个人脸上,荡样着生命的光彩,满布希望的火花,直到昨天,他们还在疑惑的道路上徘徊,像是迷途的羔羊,身陷昏迷状态中,可是现在,他们\xpinyin*{幡}然蜕变,自己都惊异不已。现在温暖他们的心田的薪火,不啻是上天赐予的神火,每个人的脸上,都泛起兴奋的红晕,那不是安详与喜悦的象征吗?佛陀环顾五个人,眼中频送柔和的慈波。他看到五个人的身后,正微现光晕,而五个人的身形,被一道光勾勒出来,他们充满了光明。

「佛陀,您背后好像有柔和的黄光环绕着,跟薪火是不一样的,您说我的眼睛是不是花了?」阿舍婆誓对这不可思议的现象感到很惊讶,说完,他又转头望向其他四人。

「阿舍婆誓,你也能看到了吗?这种光,因人而异。如果一个人能依循正道而思想,而行动,就能体悟到慈悲、知足、柔和、宽大等善行的真谛,尤其能因此而断绝一切妄念,产生调和的心境。因为心灵上不再有阴影,所以必能被慈光所包围,你们注意看哪,你们周围都已起了这种慈和的灵光。」不等佛陀说完,阿舍婆誓已忍不住别过头窥了一下跋提的身后,想证实佛陀所说的话。

的确,有类似黄金的光圈,正泛出柔和的亮光闪亮着,围绕在四个同伴的头顶。阿舍婆誓不只一次地用手揉揉眼睛,不相信地眨了好几下。顷刻间,他们都明白是怎么一回事,眼见奇迹在自己的身边出现,由于无法抑制那泉涌般的喜悦,而忍不住掩面放声大哭起来。

\mt{缘生的道理}

一个修行得道的人,头顶后会闪现黄金色的光,那是因为他的心,已能与另一光明的世界相交感应,反映出那个世界的光,我们也可说,这个光,是那个世界回应的感激之光,关于这一点,佛陀的心底,了了分明。当一个人开悟的时候,会经常显现出梵天的神态,脸色红润,充满了慈悲的胸怀,佛陀突然有种错觉,觉得这里不是米迦达亚的修道场,更像是梵天所在的国度。

四周是黑暗而寂静的,森林、河川、大地的一切,都像是沉入了无边的梦境,唯有薪火,像是抵抗黑暗的侵袭,仍发出噼噼啪啪的声音继续燃烧着。

「比丘们,你们要定下心来,继续听我说。」

佛陀扫视了四周之后,安详地说:「你们因着与双亲所结的缘而来到人间。你们的环境,也因着你们前世的造作行为而有贫富的差异,可是你们因为受躯壳所累而不明究理,一般人在人世间生活得越久,越习惯于人世间的生活,而不想去了解世态的真相,我们既跟自己的父母结了不解之缘,却因为汲汲于名利的追求,而遗忘了与双亲之间的感情,甚至连报恩的心都不复存在了。

由于受到欲望的驱使,让自己一生都过着不平不满的生活,而忘了对自己既得的恩惠心存感激,固然有了地位、名利等使人钦羡的境遇,但是这些都不是能存诸久远的,我们的心田,含藏了形形色色的种子,一旦机缘成熟、各种苦乐的果报,有待自己品尝,虽然每个人都是赤裸裸地来,又赤手空拳地去,然而这个事实不容忽视。有人生长在贫穷的环境中,最后连心灵也贫穷了,反过来看,生活在富裕之家的子弟,因为眩惑于眼前的享乐,而忘掉感激之心,以致沉沦堕落……有许多人都在自己所处的环境中,成为欲望的奴隶,忘掉满足的喜悦,替自己造下痛苦的情境,要说是悲哀,还不如说,人类是多么愚蠢。

比丘们,你们首先应该觉悟这类事实,并设法从这种苦闷中脱开,自己的心,其实是被自己束缚起来了,若想从束缚中挣脱,应学会过正确的生活,此外别无更好的途径,注意,要摒去一切歪曲的念头和行为。我们来看米迦达亚的自然美景,山河草木之间的关系,是那么调和,它们顺着大自然的生存法则来营生。在动物的世界中,我们总有一种『弱肉强食』的印象,但是你仔细去观察,大自狮虎小至蚊蝇,只要是不饥饿,它们绝不会去袭击其他生物的,它们甚至于一面营自己的生活,一面还提供他物生存所需。我们还可发现一个事实,如果草食动物无限制增加的话,草木等来不及供应,必至枯萎,然而,没有草食动物的存在,草木也不会迅速繁殖。它们就在自然的法则下,互生互长,既使自然欣欣向荣,自己又从中获受利益,自然界的这种互依互存的生存法则,使得自然界恒常地处在调和宁\xpinyin*{谧}之中。

有许多识见短浅的人,将动物界的弱肉强食现象误以为是生存的法则,甚至有人还将此观念奉为人生的信条,由此一信条,人生的问题接二连三地产生了,他们把自己放置于与他人敌对的立场,互争短长,自相残杀,我们且看,老虎再残暴,有没有残害自己的同类?这一点不可不深思。我所要昭示的『法』,就是强调大自然的万生万物本自互相依存的道理,人类更应互相帮助,达到至高的融合境界,使全人类拥有一个安定的生存环境。

为了圆满幸福的生活,我们要切记两件事。首先,我们必须\xpinyin*{摒}除个人的私念,接纳周遭的一切,以拓展一己的胸襟,其次,我们对社会大众要常思有所贡献,为大众服务,的确不是轻而易举的事,但是这仍然关系到一个人的观念问题,你不要认定那是苦差事,而应该抱着必成的信念去实行。如果事情还未尝试,就先却步,当然永远没有达成的一天。所以,先要训练自己能面对一切困难,具备苦干实干的精神。『为大众做事』的恢宏气度,是可以培养而得的,如何培养?要在日常生活中,确立良好的生活态度与习惯。

下面有八条正确的途径,可以改善我们的生活。

1.正见:对世间实相的正确认知。

2.正语:本真诚的心说有益于人的话。

所谓正语,是指凡事合于不偏不倚的中庸之道。不但对对方所说的话语做客观的评论,甚或对自己的看法,也能以第三者的立场,坦诚地予以剖析。人往往在利己主义下伤害到他人,又伤害到自己,可悲的是,他们不知害人害己的因果,却一味对既成的事实愤\xpinyin*{懑}不平。

3.正思惟:对世间实相做正确的思惟。

4.正念:专心致志地使意念集中于正确的方向。

经过五官,在我们心中产生的种种现象,一定要能在心上加以思考和判断。这类思考,是不能有所偏差的,因为思想是行为的原动力,也是创造的泉源。如果在心中总是打着歪曲的念头,且一直念念不忘时,不调和的情景必在你身边形成,久而久之,必波及大众,而你,又从众人那儿得到回响。

动念头的时候,绝不能只想到自己的利益,也绝不能有些许伤害他人的思惟。对人对己,经常在心中想着圆满的事情,抱持一颗中庸的心,不要让愤怒、咒骂、嫉妒、怨恼、恶毒等情愫在心中\xpinyin*{醖酿},要对自己所拥有的一切,怀抱满足和喜悦的心,那么你的心中,必充满光明和宁谧。

5.正业:正当而有益他人的行为。

我们亲自选择的工作,也是上天所赋予的天职。透过工作,我们能学习人生,能充实我们的人生体验。不要忘了,要注意一切可资学习的环境,工作环境是其中之一,职业是大众互存互依的一种行为模式,我们要重视自己和他人的工作,因为每个人都在他自己的岗位上尽他的一份力量,同时,我们既得一份工作,也要视之为神圣地全力以赴。此外,既有这大好机会来谋互依互存的生活,我们要有感激的心,同时更要感激自己能健康地生活与工作。感激的心会进展为报恩的心,会为我们的人生带来丰硕的果实,一位农夫,每天辛勤地埋首于田间,但是他因此保住了自己的家眷,所谓『收获』,就是把多余的东西,布施他人,这就是报恩的心所导出的报恩的行为。对他人的不幸视若无睹,甚或幸灾乐祸,整天只沉溺于如何利己的思惟中,这种人无异于自播苦种而将自食其苦果了。

6.正命:合理的经济生活。

7.正精进:努力精勤地修学正道。

8.正定:以正智入于周\xpinyin*{徧}清净之禅定。

人可以离群独居吗?人可以不走人的道路吗?不断冥想和反省的生活是必需的,我们可由此清除心中的阴影。我把自己三十六年来的生活,所想的和所做的,都一一地加以反省,以致渐渐把心垢涤除了,我之所以能够从一切执迷中脱离,并得到安逸的心境,就是靠的这个工夫。你们也应从现在开始,把自己过去的一切行为和思想,毫不保留地一一反省、忏悔,并督促自己不要再犯同样的错误。

反省是一条茫茫人生的航路,为了近道,为了修行,是慈悲而伟大的造物主赋予人的一种能力。你们看,动物就没有这种反省的能力,所以动物永远在迷茫与\xpinyin*{懵懂}之中。……」佛陀的法语,如不尽的泉水,不断涌现出来,晶莹清澈,汇成一道流光,被五个人收揽在胸中。

\mt{忏悔的眼泪}

五个人身后的光轮扩大了,他们立刻依循了正道的标准,发掘了以往修行上的过失,此外,他们终于领悟到,为了要得到心安,除了累积正确的思想和行为之外,别无捷径。

「佛陀,您一定要原谅我们在伽耶山的行为。……」憍陈如说着就哭出声来,这时,他感到佛陀慈悲的光辉射进他的心中,憍陈如高大的身体,因感动而前后摇摆,他的\xpinyin*{恸}哭,惊醒了四周酣睡中的花草,也震撼了大地,每一个人,都投身在佛陀金口亲宣的法语之中。

「憍陈如,你对自己的过错,有着这么痛彻心扉的忏悔,是很值得嘉许的,千万别忘了这颗率直的心。」佛陀说着,就举起右手,表示了赞许,同时赋予他慈悲的光辉。

憍陈如的忏悔之情,激起了其他四人的共鸣。伽耶山\xpinyin*{麓}的过失,不只是憍陈如一个人造成的,他们都曾对佛陀出以\xpinyin*{忤逆}与蛮横的态度。如果当初他们之中有任何一人能看出一杯牛奶所具有的意义,以及了解悉达多放弃苦行的心意,他们也不会那么轻易就听信了憍陈如。这整件事,他们也要负绝大部份的责任,所以,憍陈如的忏悔,同样是他们四人的心声,阿舍婆誓、跋提、十力迦叶和摩男俱利,都在佛陀面前低下了头。

憍陈如对佛陀充满慈悲的话,不知如何回应,只是不住地战\xpinyin*{栗},久久不能自己。半晌,他才平静下来,谦\xpinyin*{卑}地说:「佛陀,我誓死不忘您的教诲,我要以中庸之心为心,继续精进,请您不弃地继续领导我。」

阿舍婆誓也站起身来说:「佛陀,我从现在开始也要努力于修习八正道,还要痛切地反省自身。」说完,就悄然离开佛陀,消失在黑暗的林中,等阿舍婆誓离去后,跋提,迦叶和摩男俱利,也各自从佛陀的\xpinyin*{膝}下离去,去找一处理想的静思场所。只有憍陈如,仍留原位,跟佛陀一起进入冥想。……

\section{涤除心垢,迈向阿罗汉之道}\label{sec2.6}

\mt{无想的大前提}

冥想并不单单指一种达到「无想」境界的修行方式,如果心上的尘埃不扫除,只一味求「无想」的工夫,会招来与心垢相感应的结果。比方说,一个人怀抱求名求利的心来拜神念佛,很快就会招致欲望的心魔,阻碍了近道的机会。想祈求神佛的\xpinyin*{庇}佑或想求如神佛般的神通,这本是人类很自然的心理倾向,但这是舍本逐末的行为,是会误了自己的前途的。

一个人心念不纯,却贸然进入「无想」之境,往往就在这一瞬间同时引来了魔王、阿修罗和畜生等的灵。当然,所谓「无想」,并不止于在冥想的状态中才有的,诸如工作或念佛时,到了忘我的境界,也是一种「无想」的境界,只是有久暂之别罢了。「无想」的境界有许多层次,也各有涵义,然而不出一个原则,那就是必须靠不断的反省,以累积下来的工夫,使自己的人生境界趋于明朗,把握了这一原则,「无想」的心境则不难求得。

憍陈如是迦毗罗卫国的一名武士,有着卓越超群的武术,正因如此,他时常藐视他人,对他人出以轻蔑的态度。虽然他也能对上位的人表现谦恭的态度,但是对下面的人,就难免倨傲不逊,颐指气使的,而那样的憍陈如,此刻只身留在佛陀身边,持续不断地修习冥想,他的反省极其认真,以致他充满法喜的面庞上,又不时纵走两三行忏悔的泪水。佛陀看在眼里,知道憍陈如是彻底悔改了,其他四位弟子,一如憍陈如,也不遗余力地把自己的过去(自出生的时刻起),生长的过程,个性的形成,以及说过的一字一句,都毫无矫饰地剖析分明,并悉数向上天忏悔,惕励自己不再重蹈覆辙。

\mt{光明普现}

这是第二天的黄昏,他们容光焕发,面\xpinyin*{颊}上的红光较前一天更明朗了。同时,身后的泛光,也渐形扩大,并闪出亮光,因为心头那些日积月累的尘垢,已被洗刷了一部份,而留有余地容纳光明的缘故。泛光的亮度,与一个人心灵的调和程度成正比。所谓心灵的调和,是人们对于佛性的自觉所产生的一种心理状态,因着这种状态发而成行为,使自己的周遭环境也因而调和了。调和的环境与调和的心灵相辉映,人在其间,就能安然悟道。调和来自反省的努力和谦虚的居心。一块\xpinyin*{璞}玉,虽然有高尚美好的资质,但是没有外在的切磋,其光泽永无显发的一天。我们的思想和行为,也要靠正道的洗炼,才得以发出光芒,这光芒(身后的泛光)则因人而异。

五位弟子围绕在佛陀的身边,互相讨论昨夜禅定的心得。憍陈如先开口:「佛陀,由于您的引导,我终于能丢弃那居高傲慢的态度,寻回了本来的自己,我真有说不出的感激。」一向恃才傲物的憍陈如,自成为佛陀的弟子以来,最先觉悟到谦虚的可贵。

佛陀说:「憍陈如,你现在也能明了,狂傲自大的态度,受到伤害的终究还是自己,替自己带来痛苦的环境,你之所以会误入歧途,是因为你有一颗不明事理的愚痴的心,你若经常能替他人着想,宽谅他人的行为,你自己的心,也会包容你自己,而使自己脱离愚痴。若对任何人、事、物产生执爱的心理,则愚痴、愤恨、嫉妒等过失就会接踵而至,最后你必忘掉满足,成为欲望的牺牲者,这一点,一定要觉悟才好。你们在伽耶山所造下的遗憾,却成为我反省和悟道的开端。由这件事,我们也要知道,所谓『求道』,就是不放过任何能使我们反省规过的机会,无论是多么细小的问题,都可能成为我们醒悟人生的启示,时时持着一颗求道的心,是我要你务必铭记在心的。」

佛陀的这番话,完全是他自己的体验,因此,字字珠玑,打动了每一个人的心。这时,憍陈如百感交集,不知是感激,还是歉疚,他极力想忍住,但是泪水仍大颗地滴下来。

\mt{人生之路荆棘多}

今夜,一样是万里无云的星空,偶尔,你会看到划过天际的流星所抛下的光影。佛陀继续对他们说:

「人类,无论贤与不肖,多少都是盲目的,也很容易犯错。如果我们认清这个事实而多替他人着想,自己的心灵也会因而被拓展。可是大部份的人不去想这个道理,只顾汲汲于维护自己的权益,最后不但是把别人,甚至把自己都遗忘了,不调和的诸现象,于此产生。所以,你们从这个角度来看悲与苦的成因,要以中正的基准来探讨,并从而反省,这是很重要的态度。

有了过失,要率直地承认,不要替自己找理由,既经承认,就要悔过,避免重犯同一种过失。这样的生活,能滋润你的心,能使你渐离苦恼,要正心,要实践正道,还要摒弃有明天的念头,一切改善,都从此时此刻开始。

人生是无常的,什么时候会死难临头,是人类平凡的心所无法预知的。今日事,今日毕是很重要的观念与态度,要经常以此整顿自己的心灵,同时,对自己能好好地活了一天而心存感激,也是很重要的生活态度。

禅定是根据正道,将自己的思想和行为做深切的反省,以驱除心中的阴影,心中只要有阴影存在,光明和安详就会被蔽塞。有阴霾的心,在冥冥中是跟黑暗的世界相通的,所以带着这样的心来修禅定的话,不\xpinyin*{啻}是邀约魔王、阿修罗等恶灵共餐饮,一旦被他们支配,就注定要失去自由,不要在心中造成任何执迷,执迷会使自己离开自觉之道,经常保持安详宁静的心境,不再胶着于外境的变迁,以此心进入禅定,必能晋达高妙的境界。若想获受禅定三昧的法悦,应从今天开始,努力依循八正道来生活,对八正道,不要等闲视之。」

这些清凉的法音,真的像源源不断的泉水,流入他们的心田。佛陀停下来,端详五个人的脸,跋提、阿舍婆誓、憍陈如、迦叶和摩男俱利,对佛陀所说的每一个字,都深深印进脑海,就在这一两天,能有着与过去六年来\xpinyin*{迥}然不同的心境,这是很不可思议的现象,他们知道这全是因为过去错误的修行方式,浪费了他们许多宝贵的光阴,如今,为求得更深入的真理,每个人都面现急切的神情。

摩男俱利开口了:「佛陀,您看像我这样的人能不能悟道?我自听了您的开示,才醒悟到过去的生活充满了黑暗,几乎可说没有一件事是对的。我经常只为自己着想,并且沉\xpinyin*{溺}于欲望中而不能自拔,对地位低下的人,嗤之以鼻,甚至不把他们当人看。我对自己很宽厚,对别人很刻薄……。到后来,我连自己都厌恶了,这情况一定有改善的方法,请佛陀慈悲开导吧!」

其他的人,也与摩男俱利一样,有着相同的困扰,他们齐声要求佛陀给他们开示。

「摩男俱利,你起码已能看到自己的缺点,一般没有悟道的人,要想找出自己的缺点是很困难的,而且也难保不会再犯同样的错,能对自己的罪行产生悔改的心,是很难能可贵的,圣人也是由不断的努力,才得以净化自己的行为而达到圣境,所以,他们更能善体众生的心意,而以慈悲的心怀救渡众生,这样的人,心中充满了光明,才能平等对待大众。一个人的人格,不是由贫富来决定的,有的人虽然生长在贫穷之中,但是心灵很富裕,能包容万事万物.,反之,生活在富贵之家而心灵贫贱的,也大有人在,你现在已拥有一颗谦虚的心,要好好守护着。」佛陀对摩男俱利的率直,很表赞许,摩男俱利感动得掩面而泣,他感受到佛陀无边的慈悲。

\mt{自己的路}

求道者的心是很美妙的,如儿童般清纯爽朗,毫无虚饰,要领悟人类的价值,显发真理的光辉,这都需要靠人自己的努力、勇气和智慧,并非假借外力就能一蹴可几的,没有人能例外。大家都平等地\xpinyin*{禀}赋了如来的智慧德相,其分别只在于人心中的欲念或杂念的多寡,以及外在努力精进的程度罢了。还可以说,人的智能,有一种需要亲自尝试和领悟,方得开展的结构,神佛慈悲的光,好比月光,能照亮旅人的路,却不能替人行走,只有旅人自己的心,才掌握了流连歧途和踏上正道的关键。

他们五人是在佛陀的指导下,找到了正途,从而向反省之道进展。第五天,性情刚烈的憍陈如,首先开启了心扉。他亲自证悟了生死轮回的真相,他对佛陀告白,说他自己在前世就已是个修道的人。因此,他觉醒了,在人生的旅途上,他找到了目标,终于达到阿罗汉的境界,他成为第一位阿罗汉。第六天,阿舍婆誓也得道了,接着跋提也达到悟境,甚至能流利地说出自己未曾学过的古印度语,他的喜悦,如一股暖流,直接流入佛陀的心田。迦叶和摩男俱利,也在第七天先后悟道了。他们同样也列居阿罗汉之果位。这五位比丘,只花了一周的时间,就在反省与实践的过程中,开启了心眼,对于做佛陀的弟子,也真正是当之无愧了。

\mt{幸与不幸之间}

回顾在迦毗罗卫国的那段日子,五个人经常在佛陀身边,并且随侍在侧巡行战场,他们的武功卓越非凡。当初佛陀离城,净饭王要派武士前去护卫时,最先想到的就是他们五人。净饭王本是一番好意,为助佛陀的修行,却不料这五人的到来,反而延迟了佛陀悟道的时间。

能悟道的第一阶段,在于能「了解自己」。要了解自己,必须有独处的时间,以备反省与沉思,所以朋友或亲人的出现,变成一种障碍,使自己失去了反省自我的机会,再说,有时候不免要顾及团体行动所必需的一致性,譬如自己想静坐沉思时,有人提出什么意见和问题,就得热烈讨论,把自己原本的计划给破坏了,或有人生病时,也必不眠不休地守护一旁,或至远处采\xpinyin*{撷}药草,经常得深入野兽出没的山区。同时,五个人因为年轻气盛,往往依小卖小地起争论不说,甚至有时围在薪火边,就开始高谈阔论起来,而话题很容易就转到了托钵的状况和女人的种种。诸如那一家在布施时特别大方,又那一家有一对美丽的姐妹花,经常会准时在屋檐下等他们来……。托钵乞食是出家人一天中的大事,但是在乞食时所接触到的凡情常会点燃他们内在的情焰。

由于那一口牛乳的纠纷,五个人毅然离他而去,当时他的身心感受到极度的不安与痛苦,对人生起了极大的怀疑,这才开始痛下决心做最彻底的反省,也因而有机会过他想象中的真正的出家生活,如果不是此一机缘,永无悟道的一天也说不定。这整件事,看出了幸与不幸之间的微妙,五个人\xpinyin*{唾}弃他,对他而言,是很不幸的局面,然而在不幸的背后,他却能证悟了人生大道。现在这五位当年的武士,一跃而登上阿罗汉的崇高境界,在佛陀的宝座下,成为最早的一批弟子,在佛教史上,占有一席极重要的地位。想来六年的苦行也未见得白费,他们离开当时的悉达多,竟成了他们悟道的转折点,人生的境遇是如此奥妙而不可测啊!

\mt{另一旅程的起点}

佛陀眼见憍陈如等在他面前开启了心眼,对于今后的传道生涯,充满了无比的信心,他的内心同时涌现出无穷的希望。

五位阿罗汉特为佛陀建造了一间小屋,以避雨露和风沙。每到中午,他们就列队至城中乞食,过着如小鸟般无牵无挂的生活,从不预存第二天的食物,入晚时,就继续反\xpinyin{省}{xing3}和修禅定,精进地调理自己的心性。他们和其他的修行僧之间,没有任何交往,婆罗门和一些修行僧虽看到佛陀所居的小房子,也曾耳闻师徒六人的与众不同的生活方式,不过,并没有多大的兴致来听佛陀的道理。这大约是因为佛陀所居的小屋和六人一组的小教团,跟其他的教团比起来,太过于寒\xpinyin*{伧}的缘故吧!人类的愚痴,于此也可见一斑。一般人往往被五官的感受所左右,也被外在形相所眩惑,一件外观美丽的东西,其内容也立刻被肯定。当时的婆罗门,是知识阶级,他们一向自视很高,认为自己才是最优秀的,而一般人也都认定这种现象,对婆罗门而言,其他的教,都是异端或外道,不会再有更好的道理了。婆罗门还有一样作风,那就是谁附和他们,他们就跟谁相处,对于真正有价值的事或物,与他们无关的,就毫不眷顾,五位阿罗汉曾企图说佛陀所昭示的道理,并力劝他们来皈依佛陀,可是他们没有一点反应。

说法渡众,将是另一条遍布荆棘的路。	



\chapter{第三章\ 佛陀和弟子们}\label{ch3}
来皈依佛陀的人越来越多,佛陀特别重视弟子们的律己行为,作为佛陀的弟子,应具有自度度人弘法利生的大悲愿,有恒常的服务热忱,及尽义务的责任。

\newpage
\section{耶萨的苦恼}\label{sec3.1}

佛陀自来到鹿野苑后,此刻已迈向第三个满月。月亮挂在夜空中,丰润圆满,呈现出一幅美丽的图画,它与地面如此靠近,似乎只要你一招手,它就会悄悄地走过来,给人的感觉是那么地亲切。

白昼的太阳,光辉灿烂,赐予万物无比的生命力,但是夜晚的月亮,却像是包藏了无限的生命,那么神祕,令人兴起故国之思,家园之情。

仰望着这一轮满月,佛陀深深体会到生命的奥妙与神奇,不禁伤感了起来。固然说一个人多愁善感,会变得虚幻不实,但是如果此心坚硬如石,凡事无动于衷的话,又凭借什么来触碰宇宙的至性之理呢?浮游于天际的这一轮满月,表现出人心的本来面目,佛陀怔怔的望得出神,不觉时间已悄然从身边流过。

天亮了,整个河岸笼罩在浓浓的晨雾中,佛陀踏着柔软而生机蓬勃的牧草,就像在乌鲁维拉森林中一样,享受着晨间散步的乐趣。

佛陀散步时,其实是在和大自然作交谈,这是很好的时机,他能置身于草木间,深刻地思索做人的道理,藉草木的清新,涤尽心中的尘思俗虑,早晨是一天的开始,是一天中最重要的时刻。

今早的雾特别浓,岸边的新绿与水面的倒影,似乎无精打采地在晨雾中喘气,这情景正象征着人们善良的心被一层不谐调的黑雾遮盖了。雾很深,佛陀只能看到五、六步之远的地方,他感到前方有个人蹲在岸边,正缩成一团,待佛陀走近前去,发现是个年轻人,两眼正瞪视着水面,由他的衣着看来,显然是富豪出身。这个人,心事重重,好像有什么事正深深地困扰着他,而他也正试图从困扰中挣脱出来。

佛陀看在眼里,已明白是怎么一回事,并知道即将发生的事,果真,他看到年轻人的鞋子正并列在身旁,白色衣衫的一角,由于朝露和污泥,湿漉漉的,还有点点斑驳。年轻人正双手抱头,好一副绝望无助的神情啊!

佛陀在他身后轻轻地呼唤:「年轻人,请好好听我说。不可以为了逃避一时的痛苦,就想到死。你就是死了,你的苦恼还是存在着,你若真想死,就等苦恼解脱后再寻死吧!我可以教你解脱苦恼的方法。」

听了这一番话,年轻人转过头来。他皮肤白,脸孔细长,有着书生的模样。他先是默默地注视着佛陀,然后连忙站起身来,整整衣裳,面对佛陀跪下来:「梵天哪!我好痛苦。我不知如何是好,请救救我吧丨」

他必定是在河畔逗留了一个晚上,他的脸上布满了苦闷与疲倦的痕迹,但是依他的服饰及容貌,一望而知是个出身好人家的子弟。此时,他的眼中浮漾着泪光。

他已苦闷了一个晚上,当他决定以死了脱此生时,就误把佛陀当成来接引他的梵天了。

佛陀一手按抚着年轻人的背部,年轻人跪在他膝下,肩膀不住地颤抖着。

「我不是梵天,我是佛陀。我看你很烦恼的样子,有心助你脱离苦境,你出身良好,本来过着悠闲自在的生活,只因向外追逐物欲,成为欲念的俘虏,产生种种烦恼,你的心一面向外驰逐,一面又想超脱烦恼,那是办不到的。如果想从烦恼中解脱出来,一定要先除去惹你烦恼的原因。」

佛陀的话,句句打入年轻人的心。年轻人不禁抬起头望着佛陀的脸,说:

「我名叫耶萨,家住帕拉那西的郊外。我是独生子,父亲是俱梨迦长者。我本来过着逍遥自在的生活,但是现在因为我心爱的一个女人背叛了我,而使我烦恼不安。那女人是家中歌舞班中的舞娘,知道我深深地爱着她,但是昨天晚上,我撞见她和乐师干着见不得人的勾当,我真不敢相信我的眼睛,觉得自己好像在悬崖边被人推了一把似的,整个人往下坠。我愤恨难耐,夺门而出,不想再回去了,真想跳到这河里,一了百了。在我正在向神祈祷时,您就出现了,我以为您就是梵天神哪!您的声音充满了慈悲,稳定了我摇曳不定的心。请救救我吧!告诉我如何才能得救!」

耶萨诚恳地企求着,就像在母亲膝下哀求着的孩子。

「耶萨,爱心是可贵的,但是迷恋的心会使你丧失正确的判断力,人们在欲求不满中丧失自己,创造痛苦。

你因为看到舞娘那不可告人的一幕而在心中产生愤恨,使自己痛苦不堪,由此你可知人心的变化莫测了吧!那全是由于个人沉湎于外在,自甘于情欲的俘虏,陷身于情欲的泥沼中,迷恋貌美的女子,乃人情之常。你忘了这样的女子,必有许多其他的追逐者,而你却想独占她,自然会产生许多痛苦。所以许多人,为了贪图一时的快乐,无意中就已播下许多的苦种。

你现在应设法使自己的心回复到认识那舞娘之前的状态,当时你既没有痛苦,也没有悲哀,是如此地心平气和。你想,再漂亮的花,也有凋谢枯萎的一天,同理,人到了年华老大时,鸡皮鹤发,弯腰驼背,有何美丽可言?

眼前映现的一切,都是无常的,如果谁能够实践我所教的正道,谁就能远离人生的痛苦,生存于永恒的喜悦中,要知道,一个人眼睛所看,耳朵所听,身体所感受的一切,都是无常而不实的。在正确了解五官所带给我们的种种烦恼之后,就不应再受五官的支配,应懂得利用五官做自己丰润灵魂的资具,肉体不能长存,五官无法永保,要领悟支配肉体的心,才能做永远的自己,如此就不会再相信自己肉眼所看的是真实的了。

即使舞娘真如你所说的那么美艳动人,但只要她是眼睛探索得到的对象,她终究有消失于这地表的一天,又如再宝贵的物品,也没有人能带着进入另一世界。如果人们知道能带到另一世界的,只有自已身心所体验过的一切,就必定会努力充实内在,并发掘那真正平安自在的真我。」

「佛陀啊!我知道了。听了您这一番话,我真像是大梦初醒,才深深了解到自己有多愚蠢!我恳求您收留我吧!让我做您的弟子,我也想象您一样,能获得永久的安宁。」

耶萨深深领略到佛陀慈爱的胸怀,他那本来为情困扰而致黯淡的眼神,此刻再度燃放出光彩,那两行懊恼的眼泪一变而为喜悦眼泪。

阳光透过晨雾撒落下来,正如耶萨的心境,一道慈光正透进他的心窝。晨雾乘着气流,在河上流动,鱼儿在水中,隐约可见,它们忽左忽右,充满了生命的活力。

佛陀牵着耶萨的手,带他进入庵内,在庵内,五个弟子们正在做晨间的冥想。

「憍陈如,我介绍这个人跟你们认识,他名叫耶萨,住在帕拉那西郊外,希望你们能交个朋友。」

憍陈如等齐声应答着。耶萨像是变了一个人似地,刚才在河边的狼狈相已消失得无影无踪,他以充满自信的语调向他们寒暄道:「我名叫耶萨,以后要麻烦各位了。」

耶萨的举止温文,加上所使用的帕拉那西语言,用词讲究,音调悦耳,他的出现,跟憍陈如等五人正好成一强烈的对比,于是,佛陀的第六个弟子诞生了。

正当佛陀与六个弟子讨论有关修行和游化的事情时,忽听得庵外人声沸腾。佛陀注意地听了一下,站起身来,走到庵外,只见村人们正聚集在岸边,好像在找什么,有一个商\xpinyin{贾}{gu3}模样的人走近佛陀:

「请问您从昨晚到今天早上的这段时间内,有没有见过一个年轻人?知道的话,请立刻通知我们:」

佛陀问道:「是什么样的年轻人?哪里人?」

那人回答说:「是我们主人俱梨迦长者的宝贝儿子,我们的主人和夫人一夜没睡地到处找他呢!」

说话的男子,眼眶发红,也像是一夜未曾阖过眼,他虽然已疲惫不堪,但仍显露出诚恳的态度,这使得佛陀深受感动,于是佛陀说:「我去问问我的弟子们,您请等一下。」说罢转身进入庵中。

「耶萨,你的家人来找你了。我看你还是回家去吧!」

「我已经死过一次,是师父您救了我的命。我已决心改头换面,重新做人,我不想回家,请允许我留在此地吧!」说着,耶萨双手合十,面对佛陀及其他五人,一动也不动,表现了坚定的决心。

佛陀想了一下,令耶萨留在庵内,带着五个弟子来到岸边。

佛陀对刚才那位男子说:「我的弟子们说,没有见过那样的人……」

「刚刚有人在下游处发现少主人的一双鞋子,他大概投水自尽了。那些小伙子现在还在打捞遗体了,真可怜哪!」说着,对方淌下了眼泪,就好像自己的孩子掉进水里一样,伤心无比。

佛陀看了很不忍,心中有些困惑不定,自己为了拯救一个痛苦中的年轻人,却不得不暂且撒这个谎,于是只能在一旁默默无语,五个弟子似乎也有同感。

「打扰您们了!」村人们说着,就沿着水流的方向走去。

佛陀不禁想到耶萨的未来,这件事迟早要让耶萨的父母知道,并让他们了解才好。虽然事关耶萨本人的意志,但他毕竟为人之子,且是独子,出家求道,对他的双亲而言,无疑是一项很大的打击。

「唉!这件事真难处理。」佛陀苦苦思索着。

这时蹲在角落里的耶萨看到佛陀为难的样子,说道:「佛陀,我绝不回家。其他的问题,我自会处理,只求您把我当弟子来开导,我不想再经历上一次的烦恼了。」说着,面现坚决的辞色。

佛陀因为考虑到耶萨的家庭背景,故而想劝耶萨先回去安慰双亲,让双亲了解自己出家的意愿与决心,可是耶萨误会了佛陀的意思,执意地不肯改变初衷。

当天下午,发生了一件事。

耶萨的父母在村人的引导下,来到鞋子被发现的地方,直哭得肝肠俱裂,村人们也在一旁陪着流泪,每个人都强烈地感受到一种世事无常的无奈与凄凉,同时感觉生命的渺茫与不可捉摸,昨天还那么生气蓬勃的一个年轻小伙子,今天却消失得无影无踪,耶萨的躯体就这样被河流吞噬,跟着河流,永不回头了。不管你如何声嘶力竭,耶萨那令人怀念的笑容是永远不再出现了。俱梨迦夫妇和村人们,如今也只好默默地祈祷上苍,让耶萨在九泉之下,能安心的瞑目。

佛陀在远处眺望这边的情景,眼\xpinyin*{眸}并不断地向他们传送慈悲之光。

当此之时,一个身着破旧僧衣的年轻人正朝着俱梨迦长者与村人聚集的方向大步走去。刚剃过的头,泛着青灰色,谁也无法想象这就是一向养尊处优惯了的耶萨。

耶萨来到双亲身边,笔直地坐下来。望着面向河流恭敬合掌的父母,他开口道:「爸爸,妈妈,您们看,我不是活得好好的吗?请不要担心,我获得了新生,已经变成一修行人了。让您们为我担心,我真抱歉!」接着,又面对村人说:

「让各位乡亲费神了!」说完,深深地一鞠躬。

俱梨迦长者震惊得一时说不出话来,耶萨的母亲则迫不及待地说:

「耶萨……耶萨,你真的是那么痛苦吗?你有什么痛苦,快说给我听,你想要什么,还怕得不到吗?快说呀,老实地说。」

她紧握儿子的手,深怕他又从眼前消失了,此时究竟是高兴呢,还是悲哀?她也分辨不清。

做父亲的,见到儿子平安无事,不觉喜形于色,他目不转睛地看着耶萨,见他这一身出家修行的打扮,只当是耶萨一时的冲动罢了。

「爸爸,我自离家出走以来,让您如此费心费力,心中很过意不去。您和妈一向那么疼我,我永远不会忘记您们的养育之恩。我已决心投靠我的救命恩人,他本是释迦族迦毗罗卫国的王子,名释迦牟尼,现在己是一位觉者,我要在他门下过完我这一辈子,并帮助可怜的人。爸,希望您能原谅我,我想尽力去拯救其他的可怜人,请您们成全我吧丨」

耶萨认真地提出了这个问题,不由得令俱梨迦长者紧张起来:

「你这话可当真?你是独生子啊!家里的香火要靠你延续啊!出家修行谈何容易?你不要再待在这里了,快跟我们回去吧!」俱梨迦长者面色发青,全身不住地颤抖着。

耶萨不为所动,语气坚决地说:「不管您怎么说,我是不会回去的了,我要出家修行。」他的语气虽坚硬,容貌却十分柔和,坚定的信念,更令人不解,旁边的人,面对父子俩的争执,谁也插不上嘴。耶萨的母亲则因无法抑制感情的冲动,双手捂住嘴,神情十分凄楚。

「既然你的意志那么坚决,我也没有办法,你就让我见见你的师父吧丨.」

俱梨迦长者很清楚耶萨的个性,知道自己是无法说服他了。他很快惊觉到,那位人称佛陀的师父,不知是怎样的人,竟使他的儿子立下了出家修行的坚决意志,他想一见对方的庐山真面目,以求证一下。

耶萨依言站起身来,将父母带至简陋的庵里,屋子堪称简陋,实在只能遮风蔽雨。那位令耶萨心折的人物,就住在这里。俱梨迦长者不住地想,能令耶萨折服的人,一定是个了不起的人。

庵的格局很小,人再多就要挤不下了。佛陀正坐在草\xpinyin*{蓆}上,经过耶萨的介绍,耶萨的父母很敬谨地向佛陀问安。

佛陀对耶萨双亲的来意,了如指掌,于是在俱梨迦长者提出质问之前,就\xpinyin*{谆}谆地将人事物间的因缘、父母子女间的亲情、心理与生理之间的关系,以及人生的意义与使命等道理\xpinyin*{剖析},并强调一个人想要求得解脱,一定要过一种正确的生活,否则没有终南捷径。

小屋渐渐暗下来,佛陀说了将近五、六个小时的话,俱梨迦长者夫妇在内心产生了共鸣,终于明了,那正是耶萨所以皈依佛陀的原因,于是俱梨迦长者\xpinyin*{偕}其夫人,也一同皈依了佛陀,分别成为佛陀的第一个在家修行的男弟子与女弟子,亦即优婆塞与优婆夷。他们很高兴地答应儿子留在庵中过出家修行的生活,并恳求佛陀多加照拂。

俱梨迦长者夫妇自此以后,也负起了宣扬佛法的责任,而当佛陀及其弟子传道时,他们也常做经济上的援手。儿子既然留在庵中,做母亲的不免天天要来探望,并不忘带些吃食。俱梨迦长者看在眼里,很不以为然地责备她说:「你那样做,会妨碍耶萨的修行。」而,俱梨迦长者本人亦何尝不想去看看儿子,听到佛陀的晓喻,只好强自忍住了那股冲动。

耶萨在家时,学过吠陀及五明之学,且颇有心得,故而很快就能明了佛陀的教义。佛陀对耶萨的资质禀赋,也甚觉满意,果然耶萨在出家的第十天,就达到了阿罗汉的境界。

耶萨因为能彻悟前非,重新整\xpinyin*{饬}自己的心灵,远离世俗的执着,所以耶萨的心,呈现难得的美质,只要一旦打开心扉,悟道的时机很快就会到来,由于一种使命感,以及觉悟后的\xpinyin*{炽}烈情怀,耶萨求道的资质并不亚于其他五人,如今他已称得上是一位气度恢弘的修行者了。

\section{传道之旅}\label{sec3.2}

耶萨和其他五人相处得极为融洽,好像他们已是十几年的知交。

佛陀和六个弟子讨论日后教化活动的有关事宜:「我们最近要到摩竭陀国的王舍城去一趟。耶萨可以在帕拉那西一带推广教义。憍陈如,你们可以至其他各地游化。大家当利用这一个雨季,好好琢磨自己的心,时时把我的教义铭记在心,以备来日领导众人。」

憍陈如代表其他四人说:「我们一定竭尽心力地去做。」

耶萨一直就在想着如何把他悟得的道理说给往日的朋友们听,同时也很向往王舍城,于是欣然地向佛陀提出保证。

雨季的尖峰期一过,游化的时节接着来临,这时,来皈依佛陀的已有八十人左右。自从耶萨弃俗出家以来,人们对佛陀有了新的认识,这是信徒增多的原因之一,另一方面,佛陀传教的热情,也激起许多人的共鸣。

现在佛陀与弟子准备到各处布教了,六个弟子遵照佛陀的指示,领着新进的弟子们,一同离开鹿野苑。

如果有心推广传教活动,皈依的人自然会与日俱增。佛陀一向谨守「严以律己,宽以待人」的原则,宽容之量源自一颗爱心,它能温暖地包容身边的一切。也因为如此,佛陀特重弟子的律己行为,他怕有的皈依者只重外在的表现,只图获得某种神通来标新立异,未能真正依循正道而行。于是佛陀标出了出家修行的条件与规范,例如有人前来拜师,必须先修习一个月的佛法,待此人真正心定了,才准正式入门,凡是心浮气躁或身心不平衡的人,都不是入门弟子的适当人选。

一个佛陀的弟子,应该具有自度度人,宏法利生的大悲愿,有恒常的服务热忱,并有尽义务的责任心。有的人是一时之间有那样的热忱,但个性柔弱,无法持续到底,终难达成佛门的严格要求。此外,他还需具备超人的勇气,能刻苦耐劳,并有克己与自制的工夫,还要能宽容不如自己的人。

佛陀经过周密的考虑之后,规定有意出家的人,起码应在山中反省七天,将自己的过去做一个总检讨,并反覆审察自己出家的心愿是否纯正,待发掘出丰润坚实的内涵之后,亦即当自己身后开始闪现光晕后,就是可以正式起誓入门的时候。向佛起誓道:「皈依佛,皈依法,皈依僧。」这就好比在茫茫大海中的船,有了可靠的舵,不致于到处晃荡,而无法到达目的地。这是起码的誓言,以表明自己的心向。

据说耶稣在传教时,只要是跟随他的人,他一律欢迎,但对跟随他的人,管教非常严格。有一次弟子们乘坐的一条船在外海遭遇风浪,眼见船就要翻了,弟子们对耶稣的信心开始动摇。不久耶稣出现了,责备他们说:「原来你们的信心是如此经不起考验。」弟子们这才知道自己的信心是多么薄弱,甚至有时候,耶稣会直指弟子们的心事,说:「你现在正在想到底是为什么?」同时很严厉地追究下去。

佛陀则不然,佛陀最注重入门时的考验,然而一旦入了门的弟子,佛陀都能以宽容的态度对待,即使知道他们有犯错的念头,也不会一一加以指责,更不会出之以严厉的态度来考验他们,多半任由弟子们自己去寻求解决之道,使弟子们透过平时的为人处事,自己去领受个中道理,毕竟悟道是个人自己的责任,须视个人的努力如何而定。

这三皈依的起誓仪式是佛陀在准备到王舍城前所拟出的,在此之前皈依的人,则未曾行过这种仪式。因为日后说教的机会增多,皈依者也势必增多,要想使一群人结合成一个目标远大、荣誉感强烈的教团,一定的仪式与程序是必需的。

到王舍城去,佛陀本是有一个目的的,那就是去教化当地的拜火教,但这件事一直深藏在佛陀心底,未曾告诉任何人。

佛陀一行人在帕拉那西街上游行,并直向东方沿恒河下游而去。

雨季刚过,新绿铺满大地,阳光十分耀眼,所幸不是热燥的干旱天,否则这样的游行,真要举步维艰了。此时空气清新,大地被雨水浸润得很舒软,身上的阳光也很柔和,这一个为数不少的僧团在结队通过大街时,很引人注目,许多路人甚至停下来看他们一一过去。

帕拉那西一地,一边是工商业中心,另一边是婆罗门聚集之地。其中一派摩诃婆罗门,拥有许多弟子,他们为了修行或主持祭典之事,过着非常忙碌的生活。耶萨认识其中一位名巴巴里的修行人,巴巴里过去曾往访过耶萨的父亲,从他那儿获知许多有关释迦牟尼佛的事,耶萨并不知情,就登门拜访巴巴里,想向他宣扬佛陀的教义,而在五、六年后,巴巴里的弟子们在一次到摩竭陀国留学的机会里,修习到佛陀的教义。

佛陀一行,由耶萨做向导,选了一条帕拉那西的山路来游化,这条山路既可避过正午的炎热,也是一条通往巴达利盖马的捷径。

在中途,一行人选了一处阴凉的地方,或休息,或禅定。当佛陀闭目养神之际,耳边突然响起一群人的喧腾声,有一人来到佛陀面前并问道:「修行人,请问您有没有看到一个下贱的女人,我确实看到她往这边来的,请告诉我,她往哪个方向跑了?」

佛陀仍闭着眼睛,回答道:「你们以为这样追赶一个女人是很威武的事吗?你们的心思,我摸得一清二楚。你们把这个女子带到山上来玩,却不料那个女子跟一般女子不同,她会偷东西,她偷了贵重的东西之后就逃之夭夭。你们如果找到那个可怜的女人,心就会安了吗?你们好好想一想吧!」

那些年轻人,因为心事一一被道破,顿时生起了惭愧心,叭哒一声地跪在佛陀面前。

不管你是谁,一旦心事被人看穿,是最难为情的,当你在一个很了解自己的人面前,是无论如何也傲慢不起来的,一般人都急于炫耀自己的优点,唯恐自己的缺点藏之不及,可说十个人中就有十个人是不愿坦白表露心迹的。

这群年轻人都曾习过婆罗门教典,他们深知修行到某一程度后,就能具备这种看透对方心思的能力,因此,他们认定这一位修行者已经是位得道的仙人了,他们为自己对一个女子紧追不舍的行为,也深深感到不齿。他们都来自富裕的家庭,正值欢颂青春的年龄,养尊处优惯了,所以当听到佛陀那一番发人深省的言语后,就有手足无措的窘迫感,在他们之中,自然有一些天资较聪颖的,既经人点醒,也就兴起悔悟心,愿意当下就皈依佛陀。

这件事很快地传到帕拉那西的一群修行者耳里,那些修行僧多属自我意识强烈之辈,对佛陀强大的感化力感到害怕。

几年之后,当耶萨返乡,在帕拉那西传布佛陀的教义时,由于乡人对佛陀的种种不凡的言行早有风闻,都渴望一睹佛陀的真面目,所以带着虔诚的心意来听耶萨说法的人很多。

耶萨是一个美男子,说起话来也很动听,在皈依佛法的女信众间,耶萨成了一个受憧憬的目标,所以耶萨在布教时,常故意在脸上涂抹污泥,以避免女信众们的遐思异想。

他身为佛陀的第六名弟子,常保持一份警觉的心怀,也不忘精勤于受持戒律,他从不使心意向外奔驰,因为心意一转向外界,就很容易被情欲、物欲所操纵,接着也会被权位、名誉等欲望所摆布。

一个人即使时时恪遵克己的工夫,但是只要稍稍受到外物的诱惑而动了心,就会像高山滚物般,其堕落,是加速行进的,若想再恢复到本来的状态,已有积习难改之势,如果下定决心要痛改前非,就得花费数倍于前的努力。

耶萨深知其中的道理,所以他远离一切可能引诱他的环境,经常收心自律,保持稳定的心境。异性们就是想尽办法来骚扰他,他仍能不为所动。

佛陀继续带领一群新近皈依的年轻人,以及耶萨的亲友们,往巴达利盖马前进。佛陀如今不复有初至鹿野苑会合五比丘时的不安心境了,他充满了勇气与信心。

在抵达目的地后,佛陀一行人马上在街市或村落外展开传教活动,路过该处的修行僧们多半会好奇地驻足谛听,于是日复一日,加入他们的人,越来越多了。

当到达拿兰陀村时,是他们离开帕拉那西后两个月左右的时候。

白天,阳光炙热。雨季过后,就没有再下过雨,路面滚烫,好像已被太阳灼焦了一般,但是一行人仍不断向前走去。如今交通工具发达,几乎是想到那里,就能立刻抵达那里,但是在当时,一般人都只能徒步,尤其修行者,非得靠体力与脚力不可,除了走路,没有更好的办法到达目的地。当他们自山路直下进入闹市时,已经是黄昏时刻。

佛陀是个怀旧的人,这里曾是他离开迦毗罗卫国后的第一个修行场,此处的景物依旧,满山深绿,每一株树的位置,都和七年前一样,他记得很清楚。那在暗中摸索的时期所待过的洞穴,也仍完整地存在着,到处散落着未用完的薪木,站在洞穴中,他觉得从前的自己又回来了。七年的岁月,在迎接时,总像是遥不可期,但如今回顾起来,又倏尔已过,恍如一场梦幻。

想当年,自己曾望着那一堆微微燃烧着的薪火,纳闷不解,不知何时才能悟道,而焦躁不安的随手拾起木柴用力摔向岩石,随即传来木柴碰击岩石后「嗒!」的一声,被火烧脆了的一端因而断落了,又跳回自己的脚边,掉落脚边的这一块焦炭,仍是那么毫无表情地瞪视着他。此刻,佛陀怀着七年前的那种心情再度俯视地上烧剩的柴薪,这些柴薪在佛陀的脚边似乎也以同样的心情看着佛陀,佛陀若有所思的轻轻拾起一块,凝视了好一会儿,又轻轻放在脚边。

当时,佛陀不断祈求神祇,希望开悟,但丝毫不明白自己到底在祈求什么,只把神祇看成一个离自己很远,会在空中飞逝的莫名的东西,而领悟的境界,也是如此。然而自从自己领悟了「宇宙即我」的大道理之后,他才恍然大悟,神祇就在自己本身的心地和行为之中,也极其自然地存在于自己的每日生活之中。

出家后的六年岁月中,他走的是一条迂回曲折的道路,由于这一条遥远的路,使他有个机会接触到大自然。

佛陀循着过去所走过的路,沿途的自然景致,丝毫未变,存在其间的动物、植物与矿物,还是那样互相依靠着,维持着一种稳定的和谐,自然教导了一套和谐的道理,强调了守中道的心,会带来幸福。佛陀自开悟后,那属于自然奥秘的中道的神理,在他心中不断放射出灿烂的光芒。

佛陀在这旧日的修行场中,和弟子们商量今后布教活动的种种事宜,六位阿罗汉,遵照佛陀所拟定的日程表和作息表,或游化,或沉思冥想,继续过着确立自我的修行生活,佛陀则拟只身前往乌鲁维拉森林,那是为了要接引另一批想皈依的人。

\section{治病}\label{sec3.3}

佛陀将众人召集于洞穴前,宣布道:

「同道们,我最近要到乌鲁维拉一趟,要离开你们一段时候了,这中间若有婆罗门的修行僧来找你们论战时,千万要把持住自己,不要身陷论战之中。

如果激怒了对方,对方必会用言词来攻击你们,此时,你们必会在自己的心中生起歪曲的念头,想要折服对方,要想折服他人,是谈何容易呢?必定会替自己惹来许多烦恼而苦不堪言。所以要切记,我们绝不可主动去挑起对方的怒意。不管对方说什么,都不要忘记忍辱的心,当然,也不能为了忍辱,就在心里留下芥蒂,这样更不好,因为芥蒂会引发心中的怒火,形成痛苦的原因,所以,遇有任何情况,都不可在心中散播有害的种子,应记住,只把有用于自己修行的事留在心里,藉它们来修养心性。

凡是大树,绝不会逆风而立,因为它顺从大自然,所以才能够稳立不坠,中道之心,就好比一裸大树,不会与大自然作对,以得自于大自然的坦率之心,来实行感谢和报恩的行为。

依据中道的原则,以体谅的心情来互指对方的缺失,互相帮助,向对方伸出援手时,万不可期望对方有所报答。期望获得报答的心,就是一切欲望的根源。反过来说,受人恩惠的人,则不可忘恩负义,应该具体地以行动表达出内心的感激之意。

太阳和大自然对我们施恩时,就从来不要求回报的,所以,我们应竭力为众生服务,以表达我们的感谢,将服务变为一种报恩的行为。我们身为众生的先知先觉,要设法使他们从痛苦与悲哀中超脱出来,这就是我们修行者所应秉持的任务。

我现在就要到乌鲁维拉去了。我不在的时候,希望你们继续修行,净化自己的心灵,矢志做一个尽善尽美的人。」

八十多个人,听了佛陀这一番严明的训词,都非常感动,矢志不负佛陀的期望。

憍陈如不放心佛陀一个人出游。他说:「佛陀,请让我跟您一起去吧!一路上也好有个照应。您就答应我吧丨」

「憍陈如,请不要操心。你身为佛门的先进,负有重要任务,务必好好指导新入门的师弟们。不久的将来,和我在过去世中一道传教的人们会来王舍城的。你暂且负起先进的责任,勉力向上,一面修养自身,一面引导众人。」佛陀心意已决,憍陈如只好作罢。

于是,佛陀准备上路了。弟子们齐集于王舍城的南门恭送大驾。

佛陀停下来,感慨万千。就在五、六个月前,他还只是一个籍籍无名的行僧,没有人来接送他,如今被尊为佛陀,受到众人的仰赖与崇敬,他感到责任重大,当然不会因责任重大而感到犹豫的,他的心一如晴空,明朗而宽阔。

佛陀一度是迦毗罗卫国的王子,兵将翕从,要什么有什么,但是那时候的悉达多王子,经常陷在焦虑与烦恼中。抵御外侮和固守城堡,原不是最难承当的责任,他忧苦的是缠结于其中的许多矛盾的现象,使他过的是如牢狱一般的不自在的生活,待他好不容易挣脱了那样的桎梏之后,又同样在暗中摸索好多年。如今,他成了一个彻底醒悟的人,再也不畏惧什么,也不会为周遭的一切动辄不安了。

武士和王子,弟子和佛陀,是两种不同关系的组合,一样的成员,两样的称谓,这不就是相对的观念虚无不实的写照吗?佛陀今天的心境,早已超越了相对的一切,他将弟子们视为自己的分身,甚至就是自己本身,所以他能一视同仁,丝毫不为他们多担不必要的心思。

佛陀的衣着简陋,犹如一个奴隶,但是他有着奴隶无法企及的自由,那不但是行动上的自由,同时也是心灵上的自由,一般富裕的商贾和有权势的官员,看起来是既富且贵,实质上,他们都极易成为金钱与权势的奴隶,过的是极度不自由的生活。

再从佛陀的衣着来看,他走到哪里都可以自在,不虞盗贼的抢劫,因为他身上没有贵重物品,盗贼根本不屑一顾。这种无忧无虑,轻松自在的生活,只消过一天,就会教人永难忘怀。

佛陀与弟子们分离之后,越过高山,渡过河川,又经过无数村落,终于在五、六天后到达目的地。这一次,佛陀过着那种独来独往的游化生活,他有充裕的时间来思惟、考虑说教的方式和今后的动向等等,佛陀打算在乌鲁维拉多待一个时期。

他靠着昔日坐禅所栖息过的大树,回想几个月前的修行生活,不住地环顾伽耶山周围的一切,他举目四下搜寻,不见一度是他好朋友的小动物们,算来已有九个月的光景,也许它们长大后,各自到别处营生去了。不知小鸟们一向怎么过着日子?想到这里,佛陀抬头望着树上的鸟巢,适巧一堆白白的鸟粪掉落下来,一部份打在他脸上,佛陀像遇见了故友般,高兴得笑出声来。

从前修行时遗留下来的薪火堆,依然在原处,薪木还原封不动地积存在大树后的洼地上,于是他决定在太阳未落前,先做好入晚的准备。袋中有芒果、苹果和米蔬等,都是沿途热心的人家布施的,足够维持四、五天,竹筒中已盛满滤过的水。当佛陀把一切整理妥当,就到尼莲禅河沐浴。洗净满布灰尘的身躯,感觉全身舒畅,心中了无挂碍。

待佛陀沐身上岸,这才发觉以前经常露面的小鹿没有出现。想当初,那只小鹿不时来到身边,跟他要吃的,当他睡觉时,小鹿儿来到身边,会径自躺下来,等佛陀睡醒,真是一个可爱的小动物离开它九个月了,不知是被猎走了呢?还是被土狼吃掉了?佛陀不禁有此着急,并觉得很寂寞,他学着鹿的叫声,试了好几次,还是不见小鹿的影子,佛陀久久不能释怀。翌日,佛陀又到附近的林中寻找,他想,小鹿只要在树丛后探个脸来,他也就放心了,但是,始终无法找到它。

动物的生命,是朝不保夕的,几乎每一时刻都是它们生死乖隔的分歧点,不知何时、何地,会被比自己强壮的对方袭击,即使在这样的环境里,它们还是很重视生存的每一时每一刻,不忘为保存种族而努力,当做母亲的产下幼儿之后,就时时保护它们,不让它们受到外敌的侵袭,一直到它们能自立更生为止。弱肉强食,在它们的世界里,是极自然的法则,它们也都无可奈何,但只要活着一天,它们就会好好地度过一天,那种珍惜生命的行为,连人都自叹弗如。小鹿被猎食了吗?或者还活在某一个地方?佛陀为小鹿祈福,盼望它能有所增长。自佛陀来到乌鲁维拉,又过了四天。由于粮食告罄,他准备到邻近的村落去托钵乞食了。在村子里,凡认识佛陀的人都热络地说:「好久不见了,您到那里去了?这么久没看到您,以为您舍我们而去了。欢迎欢迎,请到屋里坐坐!」说着牵起佛陀的手,拉他进屋。

这就是人情的连系。虽然佛陀对他们来说,只是一个托钵的修行僧,却因为佛陀有一长段时间未露面,就替佛陀担了不少心思,佛陀跟他们,可说算不上有多深的缘份,只因见过几次面,彼此有着说不出的亲近感罢了,佛陀替那只小鹿担忧,不也就是这一份亲近感的缘故吗?人与人、人与动物、与植物、与自然,一切的一切,所交织成的心心相印的景况,就源于在这世上同享生命的一切,在其心灵深处都孕育着极自然的感情啊!对村人们充满深情厚意的言语,佛陀不禁从心底发出会心的微笑。

当他行经某一村人的住处,照样被请进屋内,在内室中躺着一位老太太,佛陀伸手在老太太的额上轻轻按抚,栖身于老太太体中的病魔,很快地便逃遁无踪了,自腰部以下,一直闹着所谓神经痛,也突然消失了,老太太可以坐起身来了,老太太淌下感激的眼泪,双手合掌恭敬礼拜佛陀。老太太的家人看到这奇异的景象,对佛陀兴起了异样的感受,感激佛陀的仁心仁术。

不久,许多村人都亲自将米饭和蔬菜送至佛陀的坐禅处,表达了他们至诚的谢意。

「悉达多尊者,明后天起,在我们这儿要举行一个大祭典,请您也一道来参加吧!」

一说话的人并不知道现在的悉达多和过去的悉达多,虽然外貌不变可是实质上已大不相同了。佛陀笑容可掏地回答道:「好的,我会尽可能来参加。」

他此来的目的,即想在祭典那一天会见优楼频罗迦叶尊者。

村人继续说道:「届时各方的修行僧都会前来聚会,我们每年也都去参加的,迦叶尊者经常来我家劝募祭典的费用,如果您有意,我愿为您俩介绍认识。」

「谢谢您!以后打搅您的地方一定很多。」

「哪里的话?您帮了我们那么大的忙,该道谢的是我们。假如有什么事我们可效劳的,请不要客气尽管告诉我们。」

缘份是很不可思议的,常常是一缘牵一缘。所谓牵连缘份的丝线,虽然在现象界中不能经由五官来察知,但是以实在界的眼光来看,该见面的,自然会见面,有缘千里来相会,有缘的人常觉得这个世界很小,好像总会碰到一处。佛陀很感慨地想着,凡是有缘份的人,就像是棋盘中的棋子,冥冥中被下棋的人安排着,不知在何处,总会见上一面的。

村人离去后,佛陀仰望天空思忖着,不知明天的天气如何,如果明天是雨天,则后天的祭典将要顺延,希望明天是个晴朗的好天气,他打算明晚在优楼频罗迦叶尊者的住处住一宿,后天再参观祭典。

关于迦叶尊者的事,他早在七年前,即在天舍城时,听频婆娑罗国王谈起过。听国王的口气,可知国王非常推崇这位尊者。他一直就想着有机会必一睹尊者的丰采,多年来的心愿即将实现,佛陀不禁雀跃起来。

他在心里不断盘算着当如何与迦叶尊者将话引到正题来,又应先提那件事,并如何把话题展开等等,他虽不敢确定迦叶会接受他的看法,但仍充满了信心。

佛陀于静坐时,不断地思前想后,最后他发现,凡事只有临机应变一途,不必一定要固持某一方法,这样想了之后,他很快就沉入睡乡。

一如往常,在小鸟初啭啼声时,佛陀就睁开眼睛,深深地吸入一口新鲜空气,然后整理行装,趁着凉爽的时刻,前往祭典所在地,当他来到山麓,已见迦叶的弟子们,正忙碌地往山上搬运祭祀用品。

佛陀沿着溪谷爬上去,爬到半山腰,放眼一望,脚底尽是美不胜收的景致。悠悠流着的尼莲禅河,正迎着朝阳闪闪发光。大自然的景象,是何等的鲜艳夺目。

佛陀停下脚来,目不转睛地沉醉在眼前的胜景之中,当他欣赏大自然的美景时,造物的伟大、慈悲的心怀,就给了他一种深刻而真切的感受,有了造物的慈悲,我们才能体会出生命的喜悦。

「我要向众人宣说这慈悲的真义,慈悲就是法。……」

接近山顶时,山路开始陡峻起来,四处有大岩石,阻挡了佛陀的去路,佛陀在路边捡起一根树枝,将之权当拐杖,一步步地攀登而上。僧衣已满布泥垢,加上淋漓的汗水,使佛陀看来狼狈不堪。但佛陀并不因此减慢脚步,他专心一意地往上爬。

到达山顶后,已见许多来自各地的修行僧侣早他一步聚在那里了,有人已经入定;有人正倒竖着以试自己的耐力;更有的正拿着已燃起火的小树枝,不断烧灼自己的手臂。各色各样的把式,使得这个集合场所,更像是杂耍竞技的游乐场所。

这类修行方式,如果是为了锻炼身体,自然很好,但如果是为了悟道,就一无可取了,佛陀很想提醒他们,但对于冥顽不灵的人,显然起不了什么作用,于是佛陀默默地经过他们,未曾多加理会。

山顶的地势很平坦,祭坛大约是设在平坦地区的中央处,在靠近祭坛摆设处,有一年轻的修行人,正与另一年纪老大的修行人搭讪。他说:「祭火是为了赶走疾病和恶魔而设的,这里的祭典,听说很灵验,我一直在想,一定要好好地观摩,以便日后我也能主持这样的祭典,来帮助别人赶走病魔和恶魔。」

老僧人对这番话并未起共鸣,只不起劲地漫应着:「真的吗?此话当真?.」

那年轻人也许就是在此地修行的一个弟子,而那位老僧人看起来像是婆罗门的修行人,他好像也是以观摩的心情来参加这个盛会的。

佛陀选了一处看得见尼莲禅河的地方坐下来,目不暇接地鸟瞰美丽的大自然。由高处向下望,视界最广,也最能一目了然,判断事理,若也能如此由超然处着眼,必能增加它的正确度。心之所在,也应据于高处,以便时时能明确指正人们难以觉知的错误。而为了把自己的心性和行为提升至高层次的境界,就要以法做为心的根据地,然后以之克服心中的一切恶欲念,不然没有更好的方法。

佛陀在伽耶山顶得到了一个心得,那就是,凡事由高处看,才能有广阔的视野,任何细节都能尽收眼底,同时,才能使自己心胸扩展。

\section{拜火}\label{sec3.4}

佛陀自来到伽耶山顶,为了打听借宿的事,找到一个个子精瘦的弟子,问道:「对不起!出家人,我由王舍城来,想要参观明天的祭典,请问能不能在贵处借住一晚?」

那男子仔细打量了佛陀之后,回答说:「请稍等一会儿。」说完,快步走向庵中。

到处都有忙碌地工作着的修行僧,当然都是迦叶尊者的弟子们,搬桌子的人、准备供品的人、放置供品的人、劈柴起薪火的人等等。他们在佛陀的身边穿梭不停,佛陀避开这群忙碌的人,站在一旁,为避免妨碍他们的工作,他静静地看着他们工作。

他把视线移向庵里,只见那个瘦小的男子正面向这里,指手划脚地对优楼频罗迦叶尊者说着什么,佛陀的视线偶然和迦叶尊者相触,但他很自然地把视线移开,静候该男子过来。不久,一个男子在他眼前出现,不是原先那个矮小的身形,反之,却是一个肩膀宽阔,目光锐利的大汉。

「是你要住宿吗?」他的口气显现轻蔑的意味,并上下不住地打量着佛陀。然后又接着说:「我们现在正忙得不可开交,哪有时间关照你住宿的问题。你的来路不明,又不像是信徒,我们早已经把信徒和修行僧的住处分配好了,已经没有空房了,你就随处去找个地方露宿好了。」

不待大汉提起,佛陀知道自己不是这一派的信徒,在这里只被当成一个过路僧罢了,此外,他的僧衣污秽不堪,给人的印象极为恶劣,他们的确不愿意为一个来历不明的过路僧多费心思,虽然同是修行者,但他们却不能了解佛陀的心理。

佛陀的个性宁静沉着,且心怀慈悲谦和,只要有眼光的人,一眼就可看出佛陀并非常人,可是这个大汉,单凭外在的服饰,就否定了他的一切,心中并不住地嘟哝道:「好一个穷酸和尚。」

佛陀心平气和地再次恳求道:「我知道你们很为难,但不知道有没有补救的办法?」

那名大汉很不情愿地替他到庵中探问,可是带回来的答案并无两样,于是佛陀下定决心亲自到庵中去见优楼频罗迦叶。但迦叶已入里间,佛陀等了很久,一直未见他再露面。

设于广场中央的祭坛渐渐摆设好了,祭坛前堆满了燃烧用的柴薪,祭坛与营火之门,没有教主的座位。优楼频罗迦叶会在这里施法吗?

对这一教派的人士而言,火象征着神,物质的形成的确不能缺少火,火之为一种能源,是不变的事实。在当时的印度,以火为媒介的信仰,相当普遍,人们拜火,究其原因,不外乎是认为火乃活动万物维系生命的一盏明灯,人很容易受到物质的摆弄。火生热,发光,起风,使物质变化,于是人们不知不觉间,就产生了一种火是神祇的化身的错觉。迦叶的信仰,就以祭火为主干,认为火是神,能赶走恶魔,由于在他主持的祭典中,确曾发生过几桩奇迹,以致吸引了许多的信众。

以下就几桩奇迹的实录,来分析个中的道理。

当教主指出对方行为的一两点问题时,对方都能立刻痛悔前非,此时,被点醒的人,心地变得纯净而澄澈,一个人会生病或遭遇灾难,大部份是因为他的心上为一阴霾蒙蔽的缘故,如果阴霾得以扫除,病症就会在瞬息间痊愈。这时他本人及四周的人,都会认为是奇迹出现而惊服于教主的神通广大,于是拜师入教,成为虔诚的信徒。这一个例子,与其说是教主的神通治好他的病,不如说是他本人的心替自己治好了病,但一般人是不会有这种自觉的,而教主也不见得能深明个中真相,往往自以为自己真有治病的神祕力量。

另一种情况是,教主的身后常有动物灵或魔王,往往还是一些神通力较大的灵。

大部份病人都有恶灵附身,这些附身的恶灵,灵力没有教主身后的灵大。但这一事实,教主和病人都茫然不知,教主在修行时,虽然这些灵也会显现出神明的样子来教导他一些道理,但教主本身是莫名所以的,加上身后的神(?)似乎也总是预知了许多事情,所以教主也就不疑有他地寄以无限的信心。譬如病人出现时,教主在迷茫中由背后的灵来透视病人身上的恶灵,如果病人身上的恶灵比较弱,他就会发出命令,要对方暂时离开病人,于是该病人在恶灵一离身时,就豁然痊愈了。在恶灵的世界中,一切皆重力量,力弱者必定屈服于力强者,正如动物的世界,由于这一种情况,不时有新兴宗教成立而大行其道。

疾病和灾难的种因还是在自己,病的痊愈与否,也要靠自己,虽说肉体上的疾病,若借助外力,可以加速痊愈,但还是取决于病人自己的态度与心性。灭除附身灵而在群众间造成奇迹,本无可厚非,但重要的是,能不能在人们的心上点燃神理的法灯,使他们不再遭遇病难,这一点,魔王和动物灵是无能为力的,他们既无法点燃法灯,他们也就无法真正拯救病人。

至于某一宗教是否有一贯的原则,是否属于正教,那就要看该教主以及他的信众们的生活态度如何?性格如何?有没有过着浮华奢靡的生活……。

一位真正的天使,是不会作威作福、心急气躁或浮华不实的。他也会显现奇迹,教诲恶灵,也会向人们传播真理,教人们如何离苦得乐,从这角度看,天使和恶灵自有不同处。

优楼频罗迦叶于众教主间,名气最大,灵力和人格都能震慑四邻,但是拜火的行为是错误的。

佛陀本想亲自与迦叶尊者谈论自己住宿的问题,然而迦叶尊者一直未再出现。

闪亮着红光的太阳已升至中天,不久就要西沉了,太阳西落后,天气开始转冷,山顶各处有人生起薪火,有些修行者就围在火边聊天,有的人自吹自擂,有的人极力赞扬迦叶尊者的功绩,又有的人将一路上的见闻提出来当话题,真好像有说不完的话。

佛陀在庵边静坐,见迦叶尊者偕同一个男子出来,也就张开眼睛,抬头望了望他。

白天在庵外远远望过他,但在近处望他时,发现他体格相当魁梧,还有着一副较为温和的容貌。

于是佛陀立刻起身,唤住了迦叶,说:「我来自王舍城,名叫释迦牟尼悉达多,请您准我借住一宿吧!」说着点头为礼,等待对方的答话。

优楼频罗迦叶的表情瞬间转变了。迦毗罗卫国的王子——释迦牟尼悉达多,他早有耳闻,也听频婆娑罗王谈起过,同时村人们也不时会谈到他悟道后的种种事迹,所以在心理上,不自觉地与他处于敌对的立场了。

他暗忖:「这面对面的相遇,倒还是第一次,可真是来了一个不速之客。」他的情绪有些激动,为处理此事而感困惑。

他想了一会儿说:「请您稍待一会儿,我们商量好再通知您。」说完,反身入庵。

佛陀对迦叶的心事了如指掌,心想这下子可有事要发生了,优楼频罗迦叶立刻召集了手下的大弟子们,共商安顿佛陀的事。结果不出佛陀所料,有人来回话说:「我们已没有可住的地方,不过储藏祭器的洞穴此刻空着,就请您委屈一下吧!」

他们这样的安排是有用意的,佛陀早已看出了端倪,由于是头一次的经验,他不敢掉以轻心。

那回话的人带佛陀到洞穴入口处,就径自回去了。佛陀想打开洞穴的门,以便察看里面的情形,但由于一片漆黑,看不出所以然来。于是又到广场,要了一点柴火,在洞口生起薪火。

佛陀这时再去打开洞门,探头进去,立刻逼来一股动物的腥臭味,有条大蛇住在里面,庵中的人既知这个事实,却带他前来,如果他刚才不疑有他地走进去,等于是给这条大蛇奉上一顿丰盛的晚餐。然而此刻,他又不能在外面露宿,只好拿起燃烧着的柴薪,再次探望里面的情形,一条大约有十七、八公分粗的大蛇盘桓于洞穴一隅的枯木上,正抬起头,作势准备随时扑击过来。

佛陀保持宁静的心神,向大蛇表明心意,表示自己毫无敌意。然后将火把留在洞外,一面开始挖掘地面,以便自己容得下身,由于地面是软土,所以挖起来并不费力,这样挖着,不过是想使自己便于躺下。当佛陀这样忙着挖掘时,大蛇在一旁只静静地看着,毫无攻击的迹象,佛陀又在挖好的洞中铺上草,然后睡进去,把脸转向洞外。

大蛇在佛陀的四周发出令人作呕的味道,在那难以忍受的气息中,佛陀依然是一觉睡到鸡啼时分。

大蛇依然没有任何移动的迹象,当佛陀睡醒时,大蛇只不过垂着一个头,注视着佛陀,而且没有昨晚那种作势攻击的样子了。

脚边的火早已熄了,爬出洞外,看到泛着鱼肚白的天空,美丽的星星即将自天幕姗姗退隐,晨雾正随着气流不断向四处扩散,连绵的山峯也正浮游于云海中。

烟岚缥渺,山容泰然,风光明媚,眼前简直就是一幅图画,一旦置身于这似梦还真的良辰美景中,实在看不出虚幻与现实的分野在那里。自己疑似处在超现实的世界中,但一回头,又看到了身后现实的一切,虽然现象界和实在界被划分得清清楚楚,但两者真正的差异还是在一念间。心气飘浮的人,看不出山川的秀丽,自然的和谐,心思稳定,才能了解左右现实的可能性,领悟现象界与实在界之间的关联。

佛陀此刻站在洞穴之前,看到了处在虚空界中的自己,他在洞穴前抓起一把土,回到广场,广场上早已聚集了带着供品前来的村民们。

祭坛早已于昨晚搭设完成,坛上有着各式供品,有水果、蔬菜和菜肴。五、六个迦叶尊者的弟子正在祭坛四周铺设地毯。已有许多修行僧,跪在祭坛前虔诚地祈祷着,佛陀则到与洞穴反方向的沼泽处,找到一块人迹罕至的幽静场所,开始进入冥想。

随着心气的调和,佛陀的意识渐渐扩张,使得伽耶山和自己的肉体越来越小,终至变成一颗米粒那样大小。这不是伽耶山或自己变小了,而是另一个自己与大宇宙融合而增大。

我们在佛像看到的佛陀身后的光轮,实际就是另一个真我的实态,那是和心灵的调和度成正比而出现的。诸佛菩萨所显现的姿态,就是在昭示人人所具有的一种与宇宙同一的宽广的意识,佛陀一面将肉体的自己和意识上的自己两者之间的差距拉大,一面重新确认缠绕在肉体上的执着心有多蠢,而被它捉弄的人类,又是多么渺小。

人的命运如何,决定者全在自己,善恶的选择,也单凭各人的自由意志。如果珍惜自己的生命,就应该选择善道来依附,这是人人都明白的道理。可是一般人都有好逸恶劳的通病,只注重肉体的感受,于是在肉体的支配下,变成一个微不足道的小人,一个人的命运如何,取决就在于此。

山顶上的人声越来越嘈杂,人们已越聚越多了,眼看着祭典马上就要开始。山路上,人人争道而行,忙着赶赴盛会。

\section{迦叶尊者投佛门}\label{sec3.5}

佛陀自冥想中醒转,做了一次深长的呼吸,然后走向山顶,一路上,男男女女,扶老携幼,到处是人,山顶的广场上,祭坛前的柴堆已点燃,火舌正向四面八方乱窜。祭坛的四周,由迦叶尊者的弟子们先围绕了一圈,在外侧,村人们皆已席地而坐,正向火神虔诚地祈祷,修行僧们各结螺发,都曾受过严酷的苦行,他们的眼睛射出如魔王般咄咄逼人的光芒,投向熊熊燃烧着的火焰。

这类祭典,在婆罗门教的教典上早有记载,为遵照教义维护圣火,故而应举行仪式来祭拜火神阿克尼。大部份的信徒来自摩竭陀国以及其东邻的安迦国,迦叶尊者自\xpinyin*{诩}为一个彻底开悟的人,前来投靠他的人都极依赖他,他们的修炼方式极严酷,教人不敢相信竟有这种行为。譬如于严冬季节,到喜马拉雅山麓的冰水中浸泡几小时,或者靠近熊熊燃烧着的烈火边,直到肌肤发红为止,可以说,他们简直是敢死的一群修行僧,他们就靠着这种令人咋舌的修炼方式,赢得了信徒们的敬畏和崇拜。

祭典进行不久,众人的祈祷与颂赞声,响彻整座伽耶山,朗朗的祝祷词在山野中回响,把人们的心引入另一幻梦般的世界。虽然节奏非常单调,但一些对声音和旋律有敏锐感受的人,听了那样的祝祷词,各种杂念会立刻停止,整个人终被那音律所支配。

商人们在交通要道处或不妨碍祭典的地方摆设摊位,摊位很简陋,只是利用破蔴袋或草蓆置于地上,再于其上放置土产和衣物等。

鹿皮制作的漉水囊做得很精致,但价钱不斐,佛陀在欣赏摊版上的一个漉水囊时,忽然想起一件往事,那时,五个人还没有离开他,有一次,阿舍婆誓不小心喝了不洁的水而发高烧,加上闹肚子,着实痛苦了好几天。佛陀四出寻找药草,然后煮给他吃,以现在的说法,阿舍婆誓得的是一种赤痢。

如果阿舍婆誓喝的是地底涌出的泉水,或是山涧的清水,就不致生这样的病,因为他喝的是小溪和大河中的水,那里的水充满不明的微生物,若不用漉水囊过滤,身体衰弱时饮用了,很容易感染疾病,幸而佛陀从没有因为喝了那样的水而闹病。

玩蛇的人,吹起乐曲,一只眼镜蛇正跟着旋律婆娑起舞。

在当时,玩蛇的风气很盛行。玩蛇人能很技巧地运用一根细细的棍子和竹做的笛子,就能掌握住眼镜蛇的动向。眼镜蛇很善解人意地在玩蛇人的指挥下,前后左右地扭摆身躯,它是那么可爱,你无法想象它是一条会伤害人的毒蛇。玩蛇人对蛇特有的感情,抓住了面目狰狞的毒蛇的心,而使得他能任意摆布眼前的怪物,所以爱的力量是多么大,不分人或动物,都会在它面前屈服。

人们似乎特别不喜欢爬虫类,即使是一条小蛇,见了就不舒服,甚至感到恐怖,这是因为人类过去经常遭受爬虫类的袭击,对和爬虫类作战的恐怖情形心有余悸。爬虫类在地球上的历史早于人类,尤其是蛇,大约在五亿年以前就出现在地球上了。

最初,它们的体型并不大,但后来渐渐增大,在爬虫类的顶峰时代,有过更庞大的,甚至能将大于人类两三倍的动物吞入肚中,自人类出现以后,对蛇的袭击防不胜防,备感困扰。别的爬虫类多半群居于一定的场所,迁徙时,也都是成群结队,人类只需多加小心,就能避过锋头。对蛇则不然,蛇随时随地都有袭击人的可能,凡是有生物存在的地方,就一定有蛇,人类对蛇的印象就这样代代传下来,如今人们只要一看到蛇,过去的印象立刻显现,而产生一种莫名的恐惧感。

蛇同样来自大意识,它们并不如人们所想象的那么好斗,它们也是为了在生存竞争的原则下求取进化。

蛇不易饲养,不过如果从小饲养的话,蛇性是很驯良的。关于这一点,蛇可能较其他爬虫类易于饲养,可以说,它们的坚强生命力和转生轮回的经验,形成它们那种有耐力的体质。

蛇固然来自大意识,自古就被视为罪恶的象征,它们大都喜欢栖息于潮湿黑暗处,在狙击猎物时,不发一点声响,待靠近时突然扑向前去,然后以躯体将对方缠绕起来,直至对方被绞死为止。那种不置对方于死地就不甘休的固执行为,是别的动物所不及的。

蛇一点也不开朗,它的残酷与执着就是罪恶的基本形态,一个人在染上恶习后,就跟蛇一样,这种说法,也足以帮助蛇的性格了。

总之,人类在过去有过与蛇搏斗的历史,故使得人类总想躲避它。

山上已响起了大合唱,但前来膜拜火神的村人,仍然络绎于道,有许多村人于膜拜后就四处徜徉,或流连于祭坛四周,或光顾道旁的摊贩,呈现了一片热闹的景象。

佛陀也跟着人潮在各处盘桓浏览,有的人拿鸡鸭来换衣服,有的人以芒果换鸡蛋,也有人以米换头巾,讨价还价你来我往的,好不热闹。

当时的社会,盛行以物易物的交易方式。然而各物品间的价值很难订定,在交易时,双方往往为了评定物价而起冲突,就在唇枪舌剑中,人人如临战场。当交易进行到高潮时,喧哗之声扰乱了祭坛上的庄严气氛,最后唱诵的人不得不停下来。于是祭典也就在这情况下自然而然地告一段落,整个场所立刻成为一个嘈杂喧嚷的市场。

村人们各属何种阶级,一眼就可看出来,这可以从他们头上裹着的头巾的质地以及身上的装饰来区分。首陀罗(奴隶)则不裹头巾,就是心中想要什么东西,也不能随时驻足购取,必须寸步不离主人——刹帝利(贵族)或吠啥(工商界富贾)的身边,对他们而言,这是当然的一种生活形态,他们倒也不会随便抱怨,不过有时对过分的不平现象也会产生愤懑和反抗的心理,若真的起而行动时,被主人杀死了,也只得自认倒霉,这些身为主人和奴隶的,一齐出现在山顶广场上。

佛陀在人群中看到各阶级间的不平等现象,心中隐隐作痛。自己出家修行的目的之一就在破除这种现象,如今放眼所及,尽是这矛盾现象,教他如何不沉痛?

优楼频罗迦叶在祭曲结束的当儿,命一个弟子去探看洞穴的情形,他心想释迦牟尼很可能已被大蛇吞下肚了。这一条大蛇是他饲养的,平常他都喂以猪或鸟,但咋天的情况稍有不同,释迦牟尼昨夜进去时,大蛇已有好几天未进食了,所以迦叶有理由相信释迦牟尼已进了大蛇的肚子。

弟子跑步回来道:「洞里留有睡过的痕迹,大门并掩得好好的,那个修行人不见了,大蛇在洞里好好地盘着,看来那修行人好像被蛇吞下肚了。」

迦叶听后,微微点了下头,说道:「那太可怜了!」然后问其他的弟子:「有谁看到那位修行僧?」

待众人回答没有时,迦叶如释重负般,也就不再说什么。

佛陀杂在人群间将迦叶尊者这儿的情形全部看在眼里。本想祭典既已结束,应该向尊者致个礼,只是对方似乎并不欢迎自己。如果此时出现在他面前,他必会因感意外而采取敌对的态度,不如先返回森林,待明朝再做道理。

翌晨,佛陀再次来到伽耶山上,迦叶尊者与弟子们正做完早课在闲聊,佛陀为向尊者表达昨天的谢意,就向人群走去。

当迦叶尊者一眼看到佛陀,吓得魂不附体,脸色顿时大变,他僵直地坐在那儿,一句话也说不出来。

「昨夜承蒙您让我在洞中过夜,我睡得很好,真谢谢您。」当佛陀这么称谢时,四周的弟子面面相觑,手足无措起来。个个屏住气息,想看他们的师父如何应付。

迦叶本来是目噔口呆的,但他到底是经历过大场面的,很快就恢复镇定地说:「欢迎,欢迎,请到这边坐。昨晚真委屈了您,因为我们实在太忙了,还望您不介意。」就这样说着不着边际的客套话。

佛陀并不看四周的人,只定定的凝视着迦叶尊者,然后徐徐地开口道:「您好像自以为是一位阿罗汉,但其实您并不是。一位达到阿罗汉境地的人,无论对方是怎样的人,也绝不会使对方遭遇不幸的,您很怕我,也不欢迎我参加祭典,理由之一是你怕输给我,阿罗汉是不会计较胜负,也不会有恐惧心理的,在众人面前伪装出慈善的容貌,在背地里却多方设计陷害他人的算是阿罗汉吗?请回答。」

这时众人为顾全迦叶尊者的颜面,都悄悄地一一离去了,迦叶对佛陀的话无言以对,佛陀的话句句刺进他的心窝,点燃了他心中反省的灯,那盏灯的光晕逐渐向外扩散开来。释迦牟尼佛的传闻他早有耳闻,但他这一生中却从未亲自接触过比这一番话更真切而受用的东西。

迦叶尊者站直了身子说:「您说得一点也不错。请您宽恕我一次吧!我实在没有什么好辩解的。」迦叶说着,一点尊者的气势都没有了,变成了一个谦卑的修行者,他并垂下头,双膝跪地,俯身至地,鼓起最大的勇气对佛陀说:「请收我为弟子吧!虽然我的年纪大了点,还请您不嫌弃。」他紧紧拉着站在眼前的佛陀,老泪纵横。

佛陀温和地指正他以往的行为与思想,告诉他人性的尊严与伟大的道理。他一一领教之后,对于自己内心受到的感召很感吃惊。佛陀说:「您的心情我很了解,但您的弟子们今后怎么安顿?您要多考虑考虑!」

听了佛陀的这一提醒,迦叶尊者说:「我能碰到真正的佛陀,真是三生有幸。我会告诉弟子们我现在的心情,看他们将做何打算。」

迦叶擦了一下眼泪后站起来,走到弟子们聚集的地方,那边,弟子们正也好奇地望向这边,现在他已卸下心理的重担,脚步轻快,面色从容,头部四周现出光轮。

迦叶召集了他的五百个弟子后,宣布道:「各位皈依弟子们,我自今天起不再信仰阿克尼神,而改皈依释迦牟尼佛,做释迦牟尼佛的弟子。我今天才觉悟到,佛陀的教义才是我多年来梦寐所求的,所以我下定决心要皈依佛陀。再说一遍,我今后将投身佛陀的门下,各位如果同意我的看法,就跟着我来吧!」

他的语气肯定,吐辞诚恳,显出了分外的活力,弟子间立刻起了一阵骚动,大家相互传递意见。他们一直是如此仰赖优楼频罗迦叶,又崇仰着火神阿克尼的威力,孰料一夜之间,他们的教主改变了态度,真教他们不知何去何从?虽然他们的神是阿克尼,但因为是无形的,还是靠教主做传达的媒介,故教主在他们的心目中无疑是很重要的,所以喧哗的情形并未持续多久,弟子们很快就平静下来,因为他们也想通了,既然他们所信赖的教主追寻到了佛陀的教义,那么身为弟子的他们自然也应该皈依佛陀才是,再说,就是另求解脱之道的话,怕也难找到比这更好的道路了。

想当初迦叶尊者的教团,连频婆娑罗国王都极力推崇的,如今这位一度受到拥戴的教主既要皈依佛陀,则这位佛陀,必然是一位非比寻常的人物。佛陀的教义究竟如何,以后可以慢慢学习,为今之计,还是依从师父的做法,也来皈依佛陀吧!

就这样,弟子们争相表示了跟从的热诚之意,全部愿意皈依佛陀。

优楼频罗迦叶带领弟子大踏步来到佛陀面前,并跪下来说:「佛陀,您都听见了吧!他们全都要皈依,不知您愿不愿意全都收为弟子?	我恳求您!」迦叶说完,带着满脸期盼的神情等待佛陀回答。

佛陀点点头,首度宣布了做为佛门弟子应具备的条件。他说:「既然各位诚心向道,我就为各位说法吧!我们要以八正道做尺度,来一一反省个人过去种种的思想与行为,先忏除自己以往的错误,然后行皈依佛、法、僧三宝的仪式。因此,请迦叶尊者起带头作用,为期七天,在山野间确实地修行之后再说吧!我现在要回到王舍城,希望八天后能在该地与各位见面。」

\section{三宝}\label{sec3.6}

优楼频罗迦叶尊者深深地一鞠躬,说:「我们会遵照指示那样做,谢谢您答应收留我们。」接着,就指挥弟子们拆除原有的祭坛,并将洞里的大蛇放回山中,恢复其自由身,一切准备妥当,他又陪护佛陀回到乌鲁维拉森林,于下山的途中,迦叶不住地向佛陀表达他的谢意。

对于火神的祭典,可说人人都怀有疑问,教主的事暂且不谈。有许多人遵行严酷的苦行,不但无法达到阿罗汉的境地,甚至有许多弟子受到病魔的侵袭之后就一蹶不振了;也有人修行得越严酷,心地越褊狭,像是完全变了一个人,这些问题,在以前都被认为是个人修行不够到家,虽然人人心中存有疑窦,但并不真正把它们当问题看,也就不了了之。再加上对阿克尼的祭祀,自有其历史性,人们不易起怀疑心。

如果一个信仰,伴随了形式和经过计划的进行方式,就容易流于形式而内容空泛,那就是说,该信仰很难直抵人心,调和性灵。故婆罗门在漫长的历史文明中,成为学问的形式。原因之一是,知识和智慧被混为一谈。此外,人往往受环境的支配,习惯于受眼前现象的影响有以致之,谁不想从事轻松的工作而排斥困难的工作?谁不想寻求富裕的生活以避免贫穷?这本是人之常情,但是当这种常情不加审察而任其发展时,就丧失了属于人类本有的佛性。

优楼频罗迦叶所带领的信仰,一来,火神本身已成问题,并不是一个信仰的对象。二来,其信仰已成形式化,一切只着重在对火神的祭祀上。许多迦叶的弟子们从这种形式化的信仰上一变而皈依佛陀后,对于佛陀指导的修行方式有些不得要领,如果他们心中的成见无法除去,就无法领悟心中的真性是什么,同时也就无法真正悟道。佛陀要他们抛弃对阿克尼火神的信仰,对他们来说,是一项很大的考验,几乎可以说是完全改变了他们原来的生活秩序,但这件事有引导他们接近真理的可能,则佛陀也愿意他们如此做,无疑显现了佛的慈悲。反过来说,优楼频罗迦叶的行动,也需要很大的勇气,一般人都唯恐在自己的弟子面前有失面子,就是佛陀说得再有道理,也会僵持到底,但是他却很轻易地抛弃了自己的一切来追随佛陀。

「一位达到阿罗汉境地的人,无论对方是怎样的人,是绝不会使对方遭遇不幸的。」

「阿罗汉是不会计较胜负,也不会有恐惧心理的。」

佛陀的这两句话深深地打动了迦叶的心。

他确曾自诩为一个阿罗汉,使别人也误认他是一个悟了道的人,而把弟子们带领到今天,然而在佛陀面前,他就像一个初生的婴儿一般,虽然他曾是一教的教主,此刻在他内心深处,他已承认自己只不过仍是个修行者。当他有了这一种坦荡荡的心情之后,他才听得进佛陀所说的每句话,在决心皈依佛陀之后,他又成为一个修行者。

他鼓起勇气,摆脱了教主的这一束缚,重获自由,并和几个弟子护送佛陀到王舍城,其余的弟子则进入林中,开始为期七天的反省生活,以期去除造成心中阴霾的因子,也就是说,要问问自己的心,是不是要皈依佛、法、僧。一星期过后,有了心得的人才可入门。皈依的心坚定与否,可视身后的光轮而定,反省是上苍的恩赐,也是人类拯救自己的唯一途径,无论何人,只要能够适时地知错改错,向上苍谢罪并立誓不再犯错,他的身后就会放射出光轮。这就是皈依佛、法、僧的必要过程,不如此,则无法领悟人性的尊严,就无以为佛门的弟子,要常自问:要不要皈依佛?要不要皈依法?要不要皈依僧?亦即信不信佛?信不信正法?有没有修炼身心而做一个僧人的自觉心?

佛陀的境界,是任何人求之而不可得的,但是要知道,那也是佛陀历经了无以计数的修炼过程而得来的,要入门当佛陀的弟子,必先从信仰佛陀开始。没有信仰,就产生不了信心,信心是一切的起步。就是在日常生活中,也无一不需要信心。我们之所以会制造某种货物,是因为相信能把它卖出去,人与人之间的交往也是如此,尤其在家庭中,因为能互相信赖,才能够和谐地相处在一起,如果赖以为生的「信」崩溃了,人就没有办法再活下去,同理,如果无法再相信自己,那也只好选择死路一条。

心的世界,一律受到「信」的支配,尤其在我们身处的这个世界上,人人对自己的命运都难以把握,自然将「信」看得极为重要,相信佛陀,等于相信支配万物的大意识,也就是相信自己的正确的心,佛陀的智慧是那么伟大而具有慑服力,所以佛陀的门生首先要做的是,诚心皈依佛陀。

所谓「法」,就是佛陀所说的正法。万物皆依循着正法在宇宙间生灭与消长,这正法源自于大宇宙的造物的意识(或称神),万物要自己领悟这一点则相当困难。虽然人类在口头上或许能了解,但在亲身体验上,唯有佛陀是认识得最真切的了。所以,我们若能相信佛陀的佛法,并遵从他的法,就是成为佛门弟子应具备的第二个条件。

所谓皈依僧,就是修道者本身能否在日常生活中实行合乎中道的八正道\footnote{即「正见、正语、正念、正思惟、正命、正业、正精进、正定」。}。八正道是自律之道,为了觉悟到自己的神性,就必须持著名之为中道的规矩来匡正自己的行为,领会内在调和的心性,如果不从心性上自觉到此事的重要性,只在表面上守着条文与形式,那么就失去了做一个佛门弟子的意义了。

所以,出家修行不是一件简单的事,谨守严格的戒律,目的在破除过去累积下来的恶习,恶习是业障之所以形成的要素,而业障又是悟道的大障碍。所以当我们下定决心破除习气时,的确是需要勇气和加倍的努力。此外还要有耐力,才可期以扭转以前的习性。如果吝惜于一丝一毫的努力,是无法从恶习中解脱出来的。

由此看来,身为一个佛门弟子,第一当有自觉心,其次要严守戒律。

\section{僧团的形成}\label{sec3.7}

优楼频罗迦叶有两个弟弟,一个叫那提迦叶,另一个叫伽耶迦叶。在伽耶山谷下居住的那提迦叶,发现他哥哥优楼频罗迦叶所珍藏的祭典用品浮沉于河川之上,不禁大吃一惊,以为哥哥遭遇了变故,于是他漏夜赶到弟弟伽耶迦叶的道场,偕他一同又赶至伽耶山顶。

他们在那儿看不到一个人,连祭坛都不复存在,两人惊惶不已,那提迦叶不安地想道:「哥哥一向那么善良,可能已经被山贼杀死了吧?」

他们忙遣自己的弟子们到附近去查看,假如说山贼来袭过,应该有流血的痕迹,哥哥的住处一定很狼狈才是,可是到处都干干净净,好像整理过,于是二人又赶快下山,向山下的村人打听,这才知道,他们的哥哥跟着一位年轻的修行者到王舍城去了。

伽耶问那提道:「哥哥有没有跟你说过他要到王舍城?」

「会不会因为祭典结束而接受优楼频罗王的招待?不过,通常到王舍城都是跟我们一齐去的,这一次不知有什么急事……」伽耶对那提的问话,也同样想不出所以然来。

两人最不解的是,为何哥哥要把珍藏着的祭典用品抛弃在河里呢?那提担心的是哥哥是否发生了什么变故,赶忙和五、六个弟子快马加鞭地上路了。如今唯一能解开疑团的方法就是快到王舍城去,亲眼看看是怎么一回事,一到达王舍城的大街上,就到处打听,都不得要领,最后来到郊外,在山野间奔波,终于找到了。他们的哥哥和弟子们,此刻都已剃了发,正在闭目冥想。

他们苦苦寻找的哥哥,正好端端地坐在那里!

两人哑然失笑,但却惊讶不已。伽耶想:「我们那位伟大的哥哥怎么千里迢迢地跑到这里来冥想?这个人就是优楼频罗迦叶吗?」

那提发现伽耶在发呆,于是二人把弟子们留在原地,径自跑到好像是兄长的那个人身边。「哥哥!哥哥!是我们......」

迦叶微怔了一下,随后睁开眼,看到两个弟弟,好似久别重逢般地说:「噢,伽耶吗?啊!那提也来了。欢迎,欢迎,快坐下来。......」

伽耶迫不及待地问道..「哥哥改变信仰了吗?为什么事先不和我们商量?而且竟把祭祀用品抛在河里,这是怎么一回事?请把事情的前因后果说个明白好吗?」

迦叶点点头,于是一五一十地把如何遇见佛陀,如何受到佛陀的感化等事说出来:

「这是我这做哥哥的不对。不该不跟你们打一声招呼就到这里来的,但当时我不得不那样做。本来我自以为已经开悟,却不知已铸下大错。我的心中充满骄傲、欲望、愤怒、嫉妒,就是虔诚地礼拜阿克尼火神,也无法将它们去除。虽然我祈求的是安稳的心,但是我的心永远不可能安稳,因为我是那么自大,那么虚伪。我很真切地认清了这一事实。正因为我遇见了一位真正伟大的智者,我才懂得正视自己的心。那天,在伽耶山上举行拜火祭典时,佛陀来了。他一一指出我所想和所做的事,并指正我错误的地方。我如今到了这一把年纪,才遇到这一位真正的佛陀,想来也是侥幸得很。佛陀已经答应收我做弟子,我现在正在做反省工夫。」

两人听了哥哥这一番话,更感惊讶了。

那么气宇轩昂的哥哥,如今像换了一个人似地,如此纯真坦率。在两人的眼里,过去的那个哥哥已经离他们好远好远了;眼前这个哥哥所表现的那种宁静坦然,紧紧扣住了两个人的心弦。

「哥哥,佛陀的教义是怎样呢?」

那提暂且把疑虑置诸脑后,脑中浮现出感化了哥哥的佛陀的影像,他此刻已到了忘我的境界,只想一探究竟。

优楼频罗迦叶静静地闭上眼,开始约略地叙述给他们听:「我们是与双亲结缘而来到人间,又因互相结缘而成兄弟,任何事情的产生都要靠缘份,苦乐也自有因果,妄想必会招致痛苦,同理,为所欲为也会替自己带来烦恼,一切事情都有一个缘生的道理。我们首重于不造恶因,你们会发现,苦乐的感受都是我们自己在身心两方面制造出来的。我们纵使尽力祭祀阿克尼火神,祈求他驱除病魔,但如果自己的心和行为脱离了中道,同样会制造出痛苦,火神是没有办法的,我已悟出只有过正确的生活,才能与上天感应,而恶魔不敢近身的一个道理,佛陀的教义,就在教我们以正道为心灵的尺度,每天过着正确的生活,如此方可近道。」

两人同时点点头,似乎心领神会了。

伽耶在倾听哥哥说话时,心中就早已打定主意要抛弃原来的信仰了,他在来王舍城之前,对哥哥的境遇以及佛陀的种种,脑筋还只如一张白纸,毫无印象,在以前,他认为拥戴伟大的哥哥和推广拜火教,是他赴汤蹈火,在所不辞的使命,如今,哥哥的一番话,把他坚定的决心击得粉碎不说,又在他心中把佛陀的影像越塑越大。

「哥哥请您让我们见见佛陀吧!虽然从哥哥那儿我已知道佛陀是怎样的人,但是我想亲自瞻仰一下,如果我能见到这位使哥哥转变的智者,相信我也会和哥哥一样地皈依他的。」

伽耶说着,眼中闪烁着希望的光芒。

「你这样说,我很高兴,我会去请求佛陀,带你们去见他,佛陀一定肯见你们的。」做哥哥的这一次露出了极开心的笑脸,把他们带到佛陀的住处。

佛陀对修行者们的说法告一段落,正在休息。当三个人来到佛陀面前,佛陀已知三人来意,就像跟三个人继续谈话似地,他说:

「正如你们哥哥所说的,除了以正道为心灵的尺度,每天过着正确的生活之外,没有其他的捷径可循。你们以往所崇拜的那些很不实际,真理只有一途,那就是实践正道。你们俩虽然也和哥哥一样做了明快的决定,但还是应该先考虑到弟子们的去路,好好再去商量吧!」

佛陀说完话,与三人会心地笑了一下,因为要继续去向众人说法,就点点头离席而去。

「哥哥,他果然是一位佛陀,他还知道哥哥对我们说了什么,我心中正好有皈依他的打算,也被他料准了,今后如果我不努力矫正自己的心,怕是很难开悟的了。谢谢您的指引!」伽耶感激地执起哥哥的手,兄弟三人情绪都十分激动。

迦叶说:「真是太好了,我们从现在开始一切都从头做起!我们当从身心两方面好好去体会佛陀的教义,五天以后,我的弟子们都会陆续来到这里,他们会带着一颗彻底忏悔的善心来皈依佛陀,你们现在也快回去,和你们的弟子商量商量,是否皈依佛陀,也好做个决定。」

那提和伽耶二人便带着弟子各自回去,并将双方的弟子齐集于一堂,向大众宣布优楼频罗迦叶目前的情况,最后并道明了解散的决意:「想跟随我们兄弟俩的人,我们非常欢迎……」

跟迦叶尊者所遭遇的情形一样,群众间立刻起了一阵骚动。其中一位年长的弟子站出来说:「阿克尼火神才是绝对神圣的,万物都是由火的燃烧形成,怎么可以随便抛弃圣火呢?我至死绝不背叛阿克尼!」

也有人采较温和的态度:「那位佛陀既能使优楼频罗迦叶尊者跟随到今天,可见不是一个简单的人物,他到底如何,我们可以去听听他说的道理之后再评断。」

当晚,众人的意向大致是如此,这些弟子们见他们的领导人竟然流露出童稚般的神情,睁着明亮的眼睛,声称要抛弃阿克尼,他们也就不期然地对未来充满了憧憬,也愿意试试冥想式的修行。于是,以兄弟俩为首的一群修行人,大约有九百七十人,意见总算趋于一致了。为了表明心迹,个个都削了发,向佛陀所在的王舍城,浩浩荡荡地出发。

这件事很快传布到各地,沿途所经,吸引了正在林中修行的游化僧,他们也想尽快一睹佛陀的丰采而加入了这一行列,摩竭陀国王也听到这个消息。

数年前,国王虽曾见过释迦牟尼悉达多一面,也曾料想到他将来必能成为一位伟大的智者,但岂料他竟使自己信服的迦叶尊者也皈依了他,成了一位拥有广大群众的领导者,这就使他非常不安了。

灵鹫山由于这一群僧团的出现,顿时热闹起来,自佛陀带弟子来到这里,人数一天天在增加之中,而增加的速度相当快,来皈依佛陀的,往往是以团、群为单位。这现象真是前所未有,如果依这无以核计的增加率来看,再过十年,佛陀的教团无疑地将成为一个民族大集团。

人是宇宙大意识的后裔,想要皈依大意识,这是最自然不过的事,再说,佛理若能如此向人类散播传扬,总是一件可喜的事,然而以当时世俗的现象,以及人心的自私面来看,一个为数庞大的团体终要遭忌的。当时可说是战乱之世,人的生命财产没有保障,国与国之间难保和平,阶级与阶级之间亦时起冲突,人们常被逼迫迷失于心物的选择之中,因此人世间明显地分成了两种观感的人,一是沉迷物质的人,一是忧世敬神的人。后者则多隐匿山林,潜心苦修,致力于超脱现世的苦恼,以求来生的安乐,就当时的时代背景而言,皈依佛陀即意味着出家修行,亦即抛弃世俗的一切,担负起济世渡众的职责。入山后,山中有水果可以果腹,食的方面,没有太大的困扰.,衣着方面,由于气候温和,一件也就绰绰有余了。并非人人能过出家生活,出家而不能守清规,反而有损佛门清名。同时,佛道本在阐扬中道之理,强调心物的调和,在使众人了解心与物正确的循环之法而过着正确的生活,既不役于物,自然也用不着苛待自己,所以佛陀并不主张无限制地增加出家人的数目。

佛陀还有一套为入世在家的人宣说的佛理,期望在家身体力行之,而少数的出家人,则应尽心尽力地既谋自救之道,又思将佛理传扬出去,以济渡世俗的芸芸众生。再说,传道的对象,不应当只限于一个民族,一个地区,正法应使之不断地流转,不断向需要的地方流布,这同时就是互助合作的真义所在,当我们帮助他人时,不应该限定对方的身分、国籍、人种等等。

如此想了之后,佛陀决定灌输皈依者一些新的观念,其中一项课题是,不要一味劝人出家,可以向耶萨的父母看齐,做一个在家实践佛理的信徒。

二千五百多年前的印度,虽处于兵荒马乱中,但已有了高度的文明。

印度的艺术自公元前二千五百年到一千五百年左右在印度河畔极为发达,其后在阿萨加王时代,其领土到处建造了需要高度建筑技术的佛塔等,至今尚留存着。

在文学方面,于公元前二千年移至印度西北部的阿利安人所作的称颂大自然的抒情诗歌,即其一例,印度的文学特重音韵与节奏。

这样一个历史悠久,文化优秀的国家,如今成为地球上的落后地区之一,徒然拥有五亿人口,人民连生活的意欲都丧失了,固然这与她长期处在殖民地的境遇中有关,但也可以说,她虽产生了一位伟大的智者,却未能善用这位智者倡导的真理,只将它们形式化,而错误的信仰养成了他们无精打采的生活习惯。

在地球上,我们可约略地分为东西两大文明,西方在所谓的合理主义下渐渐发展成物质文明,然而东方的精神文明却在错误的信仰下一直向后退,形成了东方似乎不如西方的局面。眼前似乎占优势的西方合理主义,难道就是万无一失了吗?由于机器无限制地代替人力而衍生许多问题之后,已在人类的生存问题上投射了些许阴影,已有许多专家学者提出了令人忧虑的醒世诤言,也就是说,欲望和利益的追求,其结果是,不可知的混乱或无可挽回的死亡局面正在等着我们,眼看着就要找不到一处适于人类居住的地方了。无论是不伴随精神力的机器文明,或者不伴随机器文明的无精打采的精神文明,都无以拯救人类。

那么什么东西才能拯救人类呢?那是能使人领悟出心源的正法,为达成人类的和谐,怀抱「四海之内皆兄弟也」的胸襟,本着平等互爱的原则,与人互助合作,过着合乎正法的生活,人人能如此,这世界才有希望。这不是什么新的理论,这是亘古不移的真理,只看我们有没有发自肺腑将它实践出来罢了。虽然有心皈依的徒众日益增多,佛陀并不改变心的状态,仍然惕励自心,毫不怠忽。

带着肉身的人,只要境遇如意,迎合自己的人多了,就容易陷于骄矜自满之中,他的谦虚求道的心将不知去向,一心只想出人头地,尤其在宗教信仰方面,这种例子特别多。加上不可知的神秘的力量,使旁观者产生莫名的疑惧,恶魔也就趁这个机会,一步步侵占那个人的心,时日一长,思想起了很大的改变,行动也变得越来越怪诞,内心时时刻刻在变迁,在受干扰,心性不是滞留不前,就是向后退转,离开悟的路途是越来越远了。

人的心性是如此善变,佛陀因为悟出此理,所以在对众人说法时,经常强调这一点,要修行的人注意这一个修行阶段,同时他本人也深自引以为戒。

灵鹫山被皈依佛陀的人群盘踞着,热闹异常,佛陀坐在最高处。

不久法会开始了,佛陀那宏亮、从容而温和的声音,直震林野而去。

各位同修们:

你们的眼睛闪闪发光,就像在燃烧,以这样一对眼睛来看东西,是看不清楚的,因为你们的心被欲望之火蒙蔽了,这欲望之火是不知足的心所引燃,只要不知足的心存在,你就无法正确判断事理,也将永远得不到心安理得的心境。

你们的耳朵在燃烧,以燃烧着的耳朵来听声音,会听得正确吗?自以为是的骄傲心,造成那样一双耳朵,使你永远听不到别人的声音,如果你不能抛弃心中的高傲,你永远得不到心灵上的平静。

你们的嘴巴在燃烧,口沬横飞,唾液四溅,不能体贴对方的心情,只想和对方议论,争论会刺伤彼此间的感情,产生对立,最后会发展到逞强斗狠,还有何慈悲心可言?

如果所言充满慈爱,就能给对方生存上的鼓励,我们透过语言,正确地把意思传达给对方,就不致引发对方不正常的感情,亦不致引起不必要的争端。

如果舌根燃烧,就会耽溺于美食佳肴中,也间接引燃贪婪之心,如果火势继续下去,而不想办法熄灭,则整个身子都要在欲火中被化为灰烬了。

这时,你已是一切欲望的俘虏,只得听任使唤而无能为力,本性已被淹没了。

像这样五官会使心燃烧,而心则更加受到五官的操纵,痛苦因此而产生,就算祈求火神给你来世的幸福,如果失去清明端正的本心,是无法从痛苦中解脱出来的。生老病死带给我们的痛苦,无法由外在的苦行来获得解脱,这样的苦行,不安和疑虑仍然经常盘踞在你们的心中,由于问题无法解决,长期累积下来,心性就不再柔和了。

为了熄灭心中燃烧着的火,我们当依循八正道,过一种正确的生活。当然,重要的是,先要根除引起燃烧的原因,如果不根除,任何一处余烬,都有复燃的可能,我们的心,要修养到宽广而圆润,有时余烬的烟仍会遮住心,但是心中的慈光也仍会在烟中透现出来。我们要不断以八正道做尺度,从反省中消除余烬和烟火。

渐渐地,人心整个被慈爱充满了,在慈爱的光中,我们才能从痛苦中解脱,光明的世界就能呈现在我们眼前,这时我们才恍然大悟被五官左右的肉体是多么虚幻而无常,并获悉做人的真正道理在哪里,至此才得以领悟出心安理得的境界。

光明的世界又叫实在界,那实在界就是你们将来应该归返的安乐境地,也是真正的世界。肉体实无常,它只是人生航路中借以渡航的一叶扁舟罢了,如今,有一条船,跟你们所有的肉体船同存,虽然你们无法想象,它却是朝向实在界航行的一条船啊!」

佛陀的声音,没有半点妥协,但充满了慈悲,佛陀说法,犹如乘风破浪,法音随着光波的移动,震憾了整个灵鹫山,整座森林都被佛陀的光波所笼罩,显得光辉灿烂,听众的座间,弥漫着天界的灵气。

迦叶三兄弟倾耳谛听佛陀的法语,直佩服得五体投地。

佛陀继续说下去:

「  你们在实在界和现象界不断地轮回着。要知道你们的肉体为了适应环境,并视缘份及约定而出现在这世上。

从乌鲁维拉森林到王舍城来,有的人是走路来,有的人是骑马来,有的人骑象来,都各有不同。不管你们是如何来到这里,来的人本身,应该没有改变。

在这人生中所拥有的肉体,就如同你们来时所用的交通工具,是适合人生航路的一种交通工具罢了,不要对交通工具执着,生老病死等苦恼,都由于人们对这个有交通工具作用的肉体起了执着而引起的。

八正道就是解脱之道,迦叶兄弟三人能抛弃几十年的信仰而成为我的弟子,他们实在很了不起,是有勇气的修行者,以勇气、智慧和努力来矫正缺失的人,才称得上是消灭了心火的修行者。

谁能破除陋习而勉励向道的,谁就能得神佛慈悲的加被。」

说法终了,每一个人都生气蓬勃,法会虽已结束,但没有一个人站起来,他们把佛陀的话深深印入脑海,并带着佛陀的话进入冥想。

聚集在王舍城的佛弟子,如今多达一千七百人,已形成僧团,这么多人聚集在一起,不得不具体组织起来了。

佛陀的存在早已传遍整个印度,很多人争先恐后地由各地赶来,想亲自瞻礼佛陀。

\section{出家必备的条件}\label{sec3.8}

自迦叶三兄弟皈依释迦牟尼佛以来,给摩竭陀国的修行者以及其他地区的人们很特别的感受。

如前述,优楼频罗迦叶的名气很响亮,不只是摩竭陀国,及其四邻各国也都仰慕他的盛名,而几乎所有的修行者都以观摩其拜火祭典为难得的盛事。

有许多人由于去观摩祭典而当下就成了入门弟子的,即使没有入门,也以潜心习其教义为荣。所以一年几度的拜火大典,各方的修行者,纷至\xpinyin{沓}{ta4}来,都将他尊为仙圣,因此频婆娑罗王也特别礼遇他,城内一有祭典仪式,一定礼聘迦叶尊者前来主持,不料那迦叶兄弟三人竟抛弃了先前的信仰,去皈依一个年轻的憎人,那僧人也不过是以小国王子的身分出家罢了,这真是一件不可思议的事情。

然而随着时间的流逝,释迦牟尼给人的冲击越来越大了,他到底是怎样的一个人物呢?他在瞬息间能看透你的心事,还能知道你的过去、现在和未来,他会温和而适切地指导你人生的方向和过正确生活的方法,你有病痛,他会治病。在他周围不断出现的奇迹,使人们叹为观止。

释迦牟尼佛最特别的地方,在于他能说出正确的道理来引导人过正确的生活。同时,他以平易近人的说法来说深奥的道理,使得不仅有高深学识的婆罗门知道他的法,就是文盲也了解他话中的意思,而且他能使接触他的人按捺不住由心田涌现出来的感动之情。

迦叶三兄弟都是年过一百岁的人瑞了,优楼频罗迦叶甚至已一百五十岁,他们的师父佛陀才三十六岁。

老实说,三十几岁的人和一百多岁的人,若由人生经验来推论,说他们是具有小孩与大人的差异,也已很勉强了,因为这之间的年龄差距太大了,又如果说迦叶三兄弟已老朽到隐居于一\xpinyin*{隅},则做佛陀的弟子也不算什么了,然而他们以人瑞的高龄竟有着壮年的精神与干劲,而独据一方,主持着崇高的信仰活动,正因如此,迦叶三兄弟和佛陀之间所构成的师徒关系,就不得不教人称奇了。

不能以常识推理的地方,才能见出佛陀的伟大,所以迦叶尊者的皈依,打动了许多人。佛陀在灵鹫山说法的当儿,一百多岁的兄弟三人既谦虚又认真地听讲的模样,足够唤醒在场众人的心。

当佛陀说到对一切执着就是痛苦的来源时,优楼频罗迦叶开始滴下大颗大颗的眼泪,满脸是悲喜交集的感激的表情。

法会结束后,他在心中高喊:「佛陀,在我心中燃烧着的欲火已被熄灭,如今喜悦正涌现在我的心头,这是多么令人高兴的事啊!佛陀,谢谢您渡了我,虽说我活到这一大把年纪,却一直被五官摆弄,今天总算让我享受到心中真正的喜乐。谢谢您!谢谢您丨」

这时的优楼频罗迦叶一点也不像是一个一百五十岁的长者,神情憨直而愉快,可能是洗刷了心垢的喜悦,由衷地从体内汩汩而出的缘故吧!

佛陀得知优楼频罗迦叶的心声后说:「我说优楼频罗,您已真的从一切执着中得到解脱,珍视您现在的心地和行为吧!不可以重蹈覆辙,应该重视每一刻精进的累积。」

「是的,佛陀,我要一面不偏不倚地反省自身,一面精进地过好每一天的生活。……」优楼频罗迦叶向佛陀叩首,并将双手置于顶部,恭敬地回答道,说到后来,由于感激的情绪,已泣不成声,其他的修行者也跟着流下了眼泪。

年迈的迦叶反覆思索着,唯有从小我中解脱出来,将人人都视如来自大意识之母的同胞兄弟,不分彼此,没有你我,则内在智慧的宝库,才会为你开启门扉。

佛陀走到迦叶身边,以充满慈爱的眼光望着他,并轻轻拍抚他的背,安慰他,佛陀此时充满了光明。

在迦叶的肩头,突现一滴如真珠粒大的水珠,金光闪闪,那是佛陀对于充满了光明的迦叶所显现的喜悦之情,那一滴水恰是佛陀所降的甘露法水。

经过佛陀的开示,人们才知道人生的目的是什么,而由于能够心平气和,一些奇难杂症也都能不药而愈了,如此,出家的人也愈益增加。

佛陀的教义否定阶级制度,主张人性平等,上自婆罗门,下至老百姓,出家者的身分和阶级不一。有些做妻子的抱怨她们的丈夫不顾家计,遁入空门。那是憍陈如有一次于晨间游化时遇见的事。

「那个名叫释迦牟尼的修行人不是一个好东西,他夺去了我的丈夫!」

「叫他还我丈夫来!」

「求你带我到我丈夫那儿去!」

就这样演出了一场闹剧。有的女人狂吼怒号,有的女人跪在憍陈如的脚边,声泪俱下地苦苦哀求。此情此景,憍陈如那有乞食游化的余地?

于是憍陈如快步来向佛陀报告,佛陀告诉他说:「憍陈如,无论是什么謡言或责难,等时间来解决吧!不要忘记忍辱的道理。这些人抛弃了社会上正当的工作,未负起抚养妻儿的责任,本也是不合正法的。那样的人,你可以劝导他们回到妻儿身边,劝他们在家里实践正道也是一样的。」说完,佛陀进入林间,开始沉入冥想。

憍陈如依言将新进者召集起来,把佛陀的意思转达给他们,劝他们不要抱着逃避的心理来出家,但是没有一个人肯回去。

不久,佛陀来到众人聚集之处,开始宣布皈依的要旨:「今后想入门做弟子的人,起码要具备以下几点:首先,以八正道为尺度,在山林间反省七天,反省你出生以来的一切言行与思想,以期个人能扫除心中的阴霾,使心田重现光明,此外,还要问自己:是否皈依佛?是否皈依法?是否皈依僧?如此彻底下定决心后,方得正式入门。如果经过七天的静思反省,还无法显现光明的,最好还是不要出家。」

大家听了这一番严格的规定后,一时感到非常紧张。

佛陀来传布正法时,本来就因对象不同而有不同的传布方法,根基不达精深奥妙之理的人,他只宣说人道的正法,亦即教以正确生活的道理,使其认知人生的目的,过着中正和平的生活,若强使他过着简约的修行生活,日久会替僧团带来弊害。所以对于不适宜出家的人,佛陀都尽力使其能自觉做人的目的,并在家好好实践正法,如果人们是为了逃避战乱或现实中其他的痛苦而想到出家的话,也就是说并非为了在这世上建立理想的佛国净土,则这样的出家修行显然毫无意义。

佛陀早就预见了这一点,所以他强调了入门前的严格程序。

\section{婆罗门僧来寻衅}\label{sec3.9}

佛陀本来有心接纳一切想出家的人做弟子的,但事实上不那么简单,早在憍陈如遭遇那件事之前,在来归的群众间,就发生过有人抱怨生活简陋的事了。

抱怨生活的人,在经过开示后,明白悟道的重要性,认清物质不可恃的道理后,多半能回复平静。然而那些妇人,当她们失去了生活的依凭,自然就丧失了理智,做丈夫的突然之间离家出走,甚且遥遥而无归期,她们就难免六神无主起来。

许多临时出家的人,经过佛陀的提示后,回到家里,勉力从事正业,并实践正法,成了在家弟子。

有一天,发生了一件事。

那天佛陀一个人进入洞穴里,想要反省一下连日来身心各方面的状况,看看自己在生活中,或在辅导众人时,是不是犯了什么错误?别的弟子,亦三五成群地聚在一起,互相检讨修行的得失。就在这时候,有一个情绪激动的婆罗门修行者,一边大步走来,一边高声喊叫着:「把释迦牟尼叫出来,我是巴拉得凡加,那个会花言巧语拐骗我徒弟的释迦牟尼在那里?……」

跋提刚巧在附近,就快步走到怒不可\xpinyin*{遏}的婆罗门僧人面前说:「这位出家人,请不要动怒,有话好商量啊!您的弟子会来到这里,必定是他自己领悟了其中的道理。」

跋提是往日迦毗罗卫国的武士出身,是最早的五个阿罗汉之一,此刻他正沉着地应付这婆罗门僧。但那一位修行者仍是涨红了脸,气急败坏地咆哮道:

「你就是释迦牟尼吗?」

「不,我是他的弟子跋提。」

「和你谈又有什么用,除了释迦牟尼外,没有人会听懂我讲的话,我是摩诃婆罗门\footnote{知识道行超越常人的人。}。哪能和你们这群小伙子谈话,再说,你这非婆罗门种的小子,怎么可以随便称呼我出家人!不怕冒渎神明?好一个叫化子的集团!」

尽管对方是如此恶形恶状,跋提依然保持平和的态度,内心一些也没有受到影响。相反地,婆罗门僧丝毫未察觉自己正在演独脚戏,自说自唱呢!

「我是释迦牟尼,听说您是摩诃婆罗门。」

跋提忽听身后响起了佛陀的声音,很感意外地说:「真抱歉!佛陀。」说着,合掌跪伏在佛陀的身旁。

他本来不想惊动佛陀,想靠自己的能耐把这一婆罗门僧打发走,不料事与愿违,仍劳佛陀的大驾前来解围,心中很感歉疚与不安。

其他的弟子深怕佛陀发生事故,也很快地从四面聚拢来,对那位修行者,作出围攻之势。

「你们都到那边去,这一位大德不像是会动粗的人,你们都各自好好去用功吧!」

就像教训可爱而调皮的孩子一般,佛陀那严厉但充满慈祥的声音在弟子们的顶端响起。

婆罗门僧的眉间,青筋浮现,他瞪视着佛陀,心里想,这家伙大概就是人称的佛陀了,如此说来,佛陀跟普通人一样,看不出是指挥上千人的领导者,而且那样子,很轻松自在,这不由得使他产生了疑惑,这个自称已开悟而又被别人尊称为佛陀的人,态度是那么和悦,气质那么高尚,难怪俗语有言「百闻不如一见」。他在决定来这儿时,心中非常怨恨佛陀,恨他拐走了自己的弟子,他一直都很自傲,身为摩诃婆罗门,不但具有崇高的社会地位,且智识也是超人一等的。

虽然弟子们离开自己而朝佛陀这儿跑来,但佛陀并不属婆罗门种,佛陀固然是释迦族的王子,究竟是较婆罗门低一级的出身。于是他想他的弟子们所以会皈依佛陀,必定是佛陀动用了武力或什么方法来使他们就范的。

如今面对面地相会了,佛陀完全没有他心目中的那一副凶蛮的容态,相反地,为人诚恳,而他在不知不觉间也被吸引了,很自然地从心底涌出亲密的感情,难怪别人都那么心悦诚服地尊称他佛陀。

「怎么样,我们就坐在这里谈谈吧!」佛陀说着,就在草丛间坐下来,并招呼对方也坐下来。

婆罗门僧心想,此地虽属佛陀的地盘,也不能输给他,于是他又回复到先前的心情,使劲地瞪视着佛陀,佛陀则轻轻地接受了他这一强烈的视线,静待他坐下。

他大模大样地与佛陀相对而坐。刚坐定,他就开口道:

「你曾大言不惭地说你是个开悟了的人,是佛陀。但你又不属婆罗门种,说话的口气倒不小,你根本就是一个骗子,你就是用那些甜言蜜语拐走了我的弟子吗?我看过的骗人勾当不少,但像你这种招摇撞骗的歹徒倒还不多,最近有许多非婆罗门种的低级人都自称是神的使者,释迦牟尼,想必你也是其中的一个,今天我要让你露出狐狸尾巴,不然,你把我的弟子们放出来……怎样?」

释迦牟尼默默地承受了他的一切责难。假如在这节骨眼上,你跟他顶嘴,无异火上加油,他的情绪非但不能平稳下来,甚且如大火之燃烧,会熊熊地蔓延开来。

人的感情是很微妙的,当你激怒了对方,对方必将加倍的怨恼怒掷过来,一切当留待时间来解决。一个人怒火中烧时,眼睛和嘴巴的周围,就会布满了红焰,由鼻孔中呼出的气,也都有着灼手的焰流。

从人体发散出来的光轮,常因一个人心境的不同而不同。诸如从色彩方面来说,有红、蓝、淡红、灰黑等。我们可以透过一个人的后光轮来了解其心理状态,而配合了这一心理状态来进行谈话,就能收到意想不到的效果,说法时,想要契合听众的根基(素质),使自己的法易于被接受,也须藉重这种对后光轮的透视。被灰色光包围而心境黯淡的人,你就是说得多么天花乱坠,他还是不会与你起共鸣的。

如果遇到这情形,最好以开朗的话题来做开场白,如此对方的心境会渐趋明朗,眼前这个婆罗门僧,佛陀只有「等待」一途,让时间来平息对方的怒火。那就是说,他要说什么,就让他说什么,让他把郁积在心头的苦闷宣泄出来,等到他的心情转好时,理智就会开始起作用,他才会发现自己可笑的地方,而心中有了接纳他人意见的余地。

婆罗门僧不住地耸肩振臂,大声咆哮,他说:「释迦牟尼,你自称自己是开悟的人,是佛陀,不知道有什么证据。你大声回答我,让你的弟子们也都听见。……怎样?释迦牟尼……,你没办法答腔吧!你这个非婆罗门种的家伙,甚至什么是修行都分不出来。你只不过是个豪族武将的出身,还是回去好好当一名武将吧!你不是婆罗门种,你没有经历过婆罗门种各阶段的修行\footnote{婆罗门种有少年、中年及老年三阶段的修行。},我想你会说你已经经过沙罗门的修行\footnote{即老年时至各处游化的一种修行方式。},但是前面的阶段没有经过,什么都别提。……快,快回答,回答啊……」

对方似乎越来越愤怒,几至无法克制自己的程度,再这样下去,眼看要动武了。远处的弟子们都惊惶失措,焦急万分,有五、六个人已作势一跃而上,以防万一。

这时,佛陀静静地开口了:「婆罗门,我的道友,请先冷静一下吧!您的心不平静的话,您就没有办法听进我的话。」

「什么话,我有一对灵敏的耳朵,要讲就快一点讲。哼,说什么自己是已开悟的人,根本就是一派胡言!」他看佛陀既不分辩,也无不悦,以为是默认了他的话,心中不由产生一种胜利后的优越感。接着又说:

「你到底把我的弟子藏到哪里去了?你们这群非婆罗门种的人,是不可以祭祀神明的,你们这批不够格的骗子,到底懂什么?快说你把我的弟子藏到哪里去了?他们听你的话会有什么用?快,快交出我的弟子来!」男子说着,下巴微微翘起,正沉浸于美妙的优越感之中。

佛陀用慈和的眼光望了望对方,对方的情绪已稍微好转,愦怒之火将熄,此刻正陶醉在自己高贵的身分尊荣感中。佛陀就把握了这一时机,缓缓开腔道:「我的道友,你的家中可能有很多亲友和修行者前往拜访吧?」

男子听话题忽然转到自己身上,很吃惊地看着佛陀。

「您都会以山珍海味来招待他们吧?」

婆罗门一听,不由得脸上又现出得意之色。

「身为摩诃婆罗门,家中当然是车水马龙,贵客盈门啦!我家的传统,向来都是以山珍海味来招待贵宾,想多积一点德嘛!」

「假如客人们不吃你的菜,那些菜应该是谁的?」

婆罗门想了一会儿,说:「那当然还是我家的嘛!」

「如此说来,您刚刚招待我的那一顿,我不吃的话,就还是你的啰?」

男子闻言,不由脸色大变。

他想不到对方不开口则已,一开口就让自己落入圈套,对方那仅有的一言,竟使他词穷语遁,无所措手足。

佛陀不再开口,婆罗门亦无言地窘坐一旁,他实在待不下去了,只得默默站起,垂头丧气地往来路走去。

\section{十字架上的爱}\label{sec3.10}

佛陀将正在远处焦虑发愁的弟子们召集起来,说:

「修行者们,注意听着:

一个真正的修行者,不能因为他人言语行动的影响而在自己的言行上引起不调和的现象,那样做,等于中了对方的毒,一旦中毒,你就免不了会悲哀和痛苦。

你们大概还不十分暸解我教条中的旨趣。今后若再有其他的婆罗门种来找你们理论,你们要学会不去理睬。不要当面「吞食」对方的感情,不可以忘记忍辱之心,不管你受了什么样的委屈,都要设法忍下来,既忍之后,不要再心存任何芥蒂。八正道的要义,在于要大家身处任何情况中,都不忘记心上的八条戒尺,随时匡正自己的行为,这就是唯一提升自己,拓展自己心灵,且获取安心立命之道的方法。」

修行者们听了佛陀的话,重新体认到八正道在日常生活中的可贵性。

刚才有许多弟子看到那位婆罗门僧穷凶极恶的样子,都曾为了怕有事情发生而内心惶惑不已,其中甚至有五、六个人为了保护佛陀,已摆好了围击的架势,当此之时,这五、六个人已经「吞食」了婆罗门僧的怒意了,他们忘了八正道,产生了与那男子一拚死活的心。说起来,身为佛陀的弟子,想要维护佛陀的安全而出此下策,原无可厚非,但也不能因此就使自己丧失理智,踰越了八正道的法的规范,因为保护佛陀的心和情绪激昂后的心,已经不在同一层次之上了。

情绪一激昂,心境就失去平静,愤怒和悲哀不但使自己无法看清四周确实的状况,还会在自己的内心种下易于愤怒和悲哀的种子,将来再度面临类似情境时,就会身不由己地被两者操纵而迷失自己。

人的行动最容易为感情所左右,虽然没有感情的人就算不上是个人,然而一任感情莽撞而不作他想者,其下场必定是很可悲的。

对事物感动、感激的感情,和对事物愤怒、悲哀的感情本属同一领域中,但其来源有所不同。使心脏亢然怦跳的愤怒,是生自感情领域的局部的表层,而感动、感激之情,则属于感情领域中较深沉的部位,即由智性和理性所连系而成的部位。

我们的心灵分为表面意识和潜在意识两种,具有本能、感情、智性、理性和意志五种作用,这五种作用在到达潜意识的某一部份时,会融合为一体,所以说,五者属于同一领域,这就好比探出海面的各个岛屿,其在海面下某一深度的地方,必是连成一体的,而岛的观念在这里就消失了。

如果人的行动受感情左右的成分大,我们就应设法使我们的感情能发自意识的深沉领域,那里有以和谐为基础的五种作用之源,这样的感情,与天赐的自然之理相契合,能使拥有它的人获致福德智慧。

造化运用属于循环秩序的法,起风降雨并滋润大地,法是极富科学意味的,所以人类应遵循自然所启示的法理,正确的思想,正确地生活着,才能稳立于世上。

假如一个人任由表面意识上所呈局部存在的感情作用所左右时,轻易地就动了肝火,如此必替自己招来许多无谓的烦恼与痛苦,也就是说,由于法的循环之理,愤怨会得到愤怨的果报,悲哀会得到悲哀的果报。因此,不要忘了怀抱慈悲,经常以爱心对待他人,造一个心灵丰盈的自己。

敬师、护师,是表现师生爱的行为,如果自己生活上的喜悦以及心理上的安定都得自恩师所赐,为了报答这个恩典而常思为恩师效劳致命,本是人情之常。

敬师的心与孝亲的心是毫无差别的,护师是感恩图报的行为表现,但是如果因而怀恨为害恩师的对方,爱的行为就变质了。

正法昭示的爱,不是有着爱恨成份的爱,只有合于上天所赐的慈悲心中发出的爱,才是真爱,进化赐予大地有生命者慈悲的恩惠,它创造各类环境,俾使生物生存,也使我们享受喜悦,爱就是有生命切实享受造化所赐慈悲的环境,并遵行其旨意的行为,奉献、祭祀及牺牲,都可说是爱的行为的表现。

耶稣知道犹太将要背叛他,他也知道如何躲避这一劫难,以免被逮捕,只要不被逮捕,他就能避开十字架上的厄运,但耶稣没有那样做。

为什么?因为他要证实他的「爱」。他亲自献出自己的肉体以遭受十字架上的痛苦,因此实践了他所昭示的爱。

这一个世界是在动物、植物和矿物三者互相为提供生存的环境,而牺牲自己的此一情况下成立的,人类的餐桌上有鱼肉,有蔬菜,动植物牺牲了生命,提供人类所需的养分,于此可见一斑。我们应当以感谢的念头来食用它们,为着建立一个人间的乐园而加倍努力才是。

人与人之间也是如此,有人做衣服,有人制鞋子,有人种稻,有人建屋,人人分工合作,使众人能各自透过自己的工作岗位来奉献己身,使他人生存,自己也得以生存。换句话说,这个世界是经由每个人奉献了自身的爱而得以成立的,假如没有奉献者,世上的任何组织都不能延续到第二天。

耶稣被钉在十字架上,向人类表现了他的爱;不论是当时的犹太人,还是今天的文明人,大家都吝惜于奉献一己的爱,只兢兢于自身的保存,甘为欲望的奴隶而泯灭内在的天良。人类藉着他物(动、植、矿物)的奉献而生存着,却忽略了如何奉献己身来报偿它们,镇日过着随心所欲的日子,则和谐的生活基础自然有塌陷的时候了,人类是为了实现佛国净土(即理想世界)的理想而生存于世的,同时也为了这一目的而接受动、植、矿物等的奉献。

如果活在自然的慈悲和爱中的人类,竟忘了本身的使命,只知耽溺于自我保存的欲望之流中,将来也只有沦亡一途了,我们不要忘记承继造物的慈悲心,做为散播爱种的人,我们要时时警觉到人类的本然。

人们对佛陀的责难及中伤,随着僧团的扩大而增加,像前次那样的事件,不止一次地发生,但随着时间的流转,在人们真正领会到佛陀教义的真髓后,佛陀的盛名传播得更远了。

许多弟子都能遵守佛陀的教义,刻苦自励,矫正了自己许多的缺失。由憍陈如开始,其他四个阿罗汉以及耶萨等先进们,将佛陀所教的正道,根据他们自己的体验,传授给晚进的弟子们,他们照常在天一破晓时,就持钵下山,对施予饭食的人,致上诚挚的谢意,过着少欲知足,知恩报恩的生活。白天,他们进入林中做观想的工夫;入夜,则围坐营火边,回想一天以来的种种言行,各自反省检讨,或复习每天的日课。

僧伽的组织在其中自然孕育而成,以佛陀为中心,很有秩序地发展着。

\section{奉献竹林精舍}\label{sec3.11}

佛陀的名声,一天一天地向外远播。

前面已说过,摩竭陀国的频婆娑罗王,早已风闻佛陀的种种,只因冗务缠身,一直未能找出一个适当的会面机会。

回想初遇释迦牟尼悉达多之时,曾约定了再会之期,亦即约好悉达多修成佛果时,两人再会面谈道,故而他再也按捺不住想见佛陀的愿望,立刻派出武将,到佛陀所在的灵鹫山,请佛陀至王舍城一行。

佛陀见来使很郑重地提出了邀请之意,同时也很怀念那一段见面之缘,于是欣然带着一群弟子随来使下山。

走出森林,早有许多贵族和婆罗门等候在道旁,他们见佛陀一行人到来,就上前迎接,然后在前面开路,引导一行人一路到王宫。

在王舍城外,频婆娑罗王已由卫队护卫着,在轿中迎候佛陀。佛陀一行人抵达时,国王面现欢欣之色,很快地由轿中走出,与佛陀寒暄。

身边的武将、婆罗门以及佛陀的弟子们,都蹲跪下来,静静地看着这感人的一幕。

佛陀对脸稍长但笑容可掬的国王说:「国王陛下,好久不见。陛下政治清明,盛名四播,请接受我深挚的祝福。」

「很高兴您已悟道,还请您多教化!」国王兴奋得脸孔泛出红光,虽然佛陀和频婆娑罗王只见过一次面,此刻重逢,犹如多年不见的故旧好友一般,两人都抑制不住内心的欢悦之情,

佛陀伸出手紧紧地握着国王的手,怀念旧时的情景。

年老的迦叶三兄弟,正坐在草地上,国王发现他们,立刻像往日礼遇他们时一样,很诚挚的跟他们寒暄。他对优楼频罗迦叶说:「欢迎您来访,看您精神抖擞,我真高兴!听说您已成为佛陀的弟子,我更是由衷地钦佩,我本来还不相信,想您曾是一方的教主,怎会轻易改变立场,后来才知道您这样做是免得您的弟子们继续走错路,您会这样想,真教人佩服!请告诉我,您现在的心境如何?」

优楼频罗轻轻颔首地答道:「佛陀的确是一位伟大的领导者,我这一大把年纪的人,得以遇见真正的佛陀,还被收为弟子,实在很高兴,现在我的心境很平和,什么执着都没有。」说着,就面对佛陀说:「谢谢您。」

然后又转向国王说:「我真是一个幸运的人丨」

频婆娑罗王听了这一番话,益加觉得佛陀的伟大,国王身后的贵族及武将们,还有婆罗门看到此情此景,深深明白外间对佛陀盛德的传说,果然句句是实。

优楼频罗迦叶接着又说:

「我以前信仰阿克尼火神,对于这件事,我要向信任过我和关照过我的人,深深致歉。陛下,我很感激您,您到目前为止,还大力地支持我,虽然火是万物创造的根源,但我去信仰它,这一件事根本上是错误的,一个人的心被火焰笼罩时,就有陷入生死苦境的顾虑了,我发现所有的愤怒、讥谤、嫉妒、不知足等情绪,都来自欲望之火的燃烧。

我本为了求得和平的生活而信仰阿克尼神,但如果我不能体会火的神旨,反而会因他而燃起烦恼之火,凡有生命的,都会为了自身的五官而迷惘,阿克尼神会在五官的火上加油,增加了人们对生老病死的执着。

我所以能抛弃阿克尼神而解脱了生老病死带来的苦,都是因为我能够领会佛陀的教义,并努力去实践。我如今证悟了自己的本性,我的心中充满佛陀的慈悲,就好像在黑暗中有了烛照,我能依佛陀的法知悉人性的真实面。

我真心感谢佛陀,当然,我的这一心情,此刻我的弟子们也都共同享有着,皈依佛陀而得到心安的那种喜悦,相信每个人都感受到的,我就是这样抛弃了从前的信仰而皈依了佛陀。」优楼频罗迦叶强抑住即将夺眶而出的眼泪,佛陀一边听迦叶的话,一边点着头。

频婆娑罗王咀嚼着迦叶尊者说出来的每一句话,几乎等不及听迦叶尊者说完话,就急切地向佛陀求教道:

「释迦牟尼佛,请讲佛法给我听,并请用我听得懂的方式讲我听。」

旁边的贵族及婆罗门,看到国王那么虔诚,都如梦初醒般,他立刻抛弃了固有的成见,准备接受佛陀的教诲,他们都急于一探佛法的面貌,他们等着听佛陀亲自宣说的佛法。

国王的心情也是一样,他也急着想听佛陀说的法,因为到目前为止,国王从未听过也未见过谁会对一位活生生的人合掌顶礼的,佛陀的弟子们对佛陀如此,优楼频罗迦叶尊者对佛陀也如此。

频婆娑罗王是一个大国的国君,他不只是一国的国君罢了,他文武双全,深得民心。他能处处为人民着想,国家一旦发生灾变或战争而人民陷入困境时,他必实时赈灾济贫,他期许自己所治理的国家,能国泰民安,有这样的领导者,难怪摩竭陀国上下一心,一朝有事,人民皆争赴沙场,为国捐躯。

频婆娑罗王之所以有此仁心仁政,实由于他有坚定的信仰心,国王对宗教的关心,对修行者的礼遇,向为人所称道,对婆罗门自然非常重视,但对他们流于形式的宗教组织经常心存怀疑。

国王对于能祈神保佑之类的宗教信仰者非常的关切,只要他佩服的,从来不计较对方是何宗何派,如果他从臣下及商贾处得知某某修行僧的优异事迹时,立刻会礼聘入城,甘受教诲,并丰厚地给予物质上的供应,因此频婆娑罗王的身边,除婆罗门外,像以往的迦叶尊者一样,不知有多少修行者来来往往。

六年前,当国王听说释迦牟尼悉达多已投身修行者的行列时,立刻生起虔诚的心,想见释迦牟尼一面。他曾一度想把释迦牟尼留在宫中,一同谈玄说理,共探真理的奥秘,但那个梦想在释迦牟尼的意志下幻灭了,国王见释迦牟尼气宇非凡,气魄昂然,既为探寻真理而拒绝了自己的好意,就在心中默祷,祝他早日悟证道果。

往日的王子,如今已成为佛陀,不但拥有广大的信众,且正传播佛理,六年前的他和眼前的他,虽同属一人,但如今仰望他时,真感到他有着难以侵犯的伟大气质。

佛陀的身材虽然不魁梧,但却能使与他并立的人自觉渺小而卑微,有时你会觉得佛陀的身躯不断在膨胀,终至达于云天。

国王就这样注视了佛陀一会儿,然后将视线移至地面,并静静地阖上眼。

佛陀抬头环视了一下四周,就开始说话了。

「陛下,我们的肉体是人生航路中的一条小船,我们的心,功能比如船夫,如果没有这样一种心,是既不会烦恼,也不会有苦乐经验的,有了这个心,才产生各种思想及欲望,每一个人透过自己的心,在出生的环境中,受生活习惯、教育及思想等的作用,往往会误入歧途,替自己带来痛苦。

肉体本是航渡人生苦海的一条小舟,而操纵它的各个人的心,才是永久不灭的,我们五官可感受的一切现象,都不过是使心思丰盛的媒介罢了,但我们却被那些现象迷了心窍,丧失了真正的自己,成为人生的苦因,我们必须透过良缘,造成善因,实践正道,才能获得善果。换句话说,我们要走的是一条能使自己心平气和的道路。

要断绝苦因,一定得设法了断生死,解脱之道,首先在于能正确听取他人所说的话,一个凡事以自我为中心的人,是无法静听他人说话的,且经常会误解他人的意思。同样一句话,在他听来,会不禁火冒三丈,一切的情绪如怨恨、嫉妒、轻蔑等都由此而生,将自己卷入苦恼的漩涡中;其次,要学会说正确的话,一个喜欢逞口舌之利的人,最后受伤害的还是自己,因为对方那受到伤害的心,是不会容许你怡然自得的。如果一句话是经过自己的嘴而被误解,那等于在自己的心中种下了苦种,要想使自己的话正确,就要能体会推己及人的道理。

此心向外奔驰,就会为我们带来欲望,带来痛苦,若能转心于内,就能把握事理,不致自寻烦恼,匡正心灵的戒尺能助我们解脱痛苦,促进人世间的和谐。」

佛陀说法时,如水流的渲泄,未曾间断,而且铿锵有力,听众们都沉醉于其间。

很多老修行们甚至看到淡黄色的光轮正不断由佛陀身后放出,不由得合起掌来,国王此时也看到那圈光轮。一看到这景象,他的双手不期然而然地就合拢起来,刚刚他还不解为何佛陀的弟子们会对佛陀合掌,此刻自己也有了这一体验,才知道是怎么一回事。

国王很感激地向佛陀称谢道:「如此宝贵的道理,真谢谢您,您真是一言九鼎,使我深深体会到做一个仁君的重要性,我将认清自己的立场,完成艰巨的任务,还望您以后不吝赐教。」

国王诚挚的托付,给了佛弟子们极大的鼓舞,于是佛陀的教义,犹如法灯,在王舍城内点燃了。

数日之后,频婆娑罗王派出一位使者,往访灵鹫山上的修行场。

「我由摩竭陀国来,名叫迦兰陀,是一个商人。前几天跟着国王听过佛陀说法,因而得到许多启示。国王深切盼望佛陀永远留在摩竭陀国,本人也得有机会表示心意,所以想捐一处精舍供佛陀留驻,今天特为此事而来,不知能不能见到佛陀?」

憍陈如从来人的衣着及谈吐中直觉到是一位殷实的富商,随即憍陈如把迦兰陀带到佛陀的修行处。

迦兰陀一见到佛陀,就热泪盈眶地不住称谢,谢谢佛陀给他的启示。然后转达了频婆娑罗王的心愿,同时恳求佛陀收他为在家信徒,他对佛陀说:「我有一块地在王舍城北门东北方的迦兰陀村一带,那一带,茂林修竹,很适合做佛陀的道场,不知尊意如何?地点一决定,我就会马上按照佛陀的指示去做准备,请您接受的我供养吧!」

佛陀看迦兰陀的心意很诚挚,于是很高兴地接受下来,并说:「你既诚心捐献,我就不客气地接受了。谢谢你!迦兰陀村的竹林地带,是个很好的地方,附近又有一个很大的修行场,我曾游化过几次,很怀念那个地方。」

\section{布施的功德}\label{sec3.12}

迦兰陀见自己的任务能顺利达成,心中很快慰,他本来还忧心忡忡的,不知佛陀肯不肯接受国王及自己的要求,如今一切如愿以偿,他高兴地说:「谢谢佛陀!国王一定也很高兴。最近我就想供应您及您的弟子们一些粮食,并请教您有关精舍的建筑,好让木工们动工,在动工前还望佛陀能拨冗光临,我将会感到非常荣幸的。」迦兰陀边说边鞠躬,并高兴地笑着。

从修行到悟道的这一段路程是非常艰辛的,经常挣扎于死亡的边线,然而等到开悟后,大自然的一切架构与功能,以及人性苦乐业缘都已昭然若揭,了如指掌。

对佛陀最具意义的一点是,经过这一艰辛的悟道历程,他已能与上天各界互通消息,所见所闻非三十六年俗世生活中所见闻的总合所可比。

这一次迦兰陀的乐捐,佛陀早已听梵天们谈过:「等您开始布法后,必有许多人聚集您身边,且有人会布施布法所需的一切。」如今这一段话灵验了,弟子人数已多达一千多人,的确需要一处聚会所。

身为佛弟子的人都要有所警惕,那就是一旦误解了佛陀的教义,或行为稍有偏差,都会被人指责的,认为不够格当佛陀的弟子,所以既出家了,就不应各自为政,单独营生,否则如何传布正法。

佛陀基于这一点的认识,深知有一处精舍的必要性,佛陀并深切地感觉到,出家修行的人,应过着相同方式的生活。

此外,来皈依佛门的人,多来自社会各阶层,有婆罗门,有商人,也有武士及奴隶,如此一来,更需有一共同可资遵循的法度。

具有奴隶身分的人,由于一向过惯了屈膝卑下的生活,自卑感都极重,佛陀要设法替他们消除此一心理障碍。

人生而平等,差别只在于各个人的理智与意志力的不同,故形成了不同的际遇,于是际遇的差别就在人们心上播下了自卑与优越的种子,以致形成人与人之间的对立,人们因而丧失自己,忘了自己的本来面目,人们既然同来自宇宙的大意识,应该没有优劣之分,而能力的优劣则属于每个人努力的结果。

所有的人,在造化之前是一律平等的,太阳毫无私心地把它的光普照大地,就已经表现了那平等无别的爱意。平等,是建立人间天堂的一个基础,只是在人间,人们不免有着男女、老幼等的分别,各人在自己生长的环境中,因着知识、经验等,而各有特征,所以人人都有自己的立足范围,这可以由人体中的「小宇宙」来帮助。

人体之所以能够做精密的机动式运作,是由于人体的各部器官及其特有的功能相互完善地搭配,并发挥作用,任何一项器官发生问题,就会影响其他器官的运作。

这世上的每一个人也都具有相同的存在意义,因此,平等的观念是大自然昭示给人类的生活方式,不遵照这一指示而产生优劣的阶级差别,人类的社会只会日趋混乱。

当时的印度社会,阶级意识相当浓厚,一切规定都极为严格,就是今天,仍留有遗风。

奴隶永远是奴隶,无法过着独立自主的生活,一切得听主人的使唤,且注定辛苦地过着操劳的生活。

当时的宗教团体,大大小小的,已不下数万,却没有一个团体肯接受奴隶的皈依,奴隶被认为是无药可救的一个阶层,所以奴隶们,在当时来说,只是一些具有人形的动物罢了,佛陀挺身打破了这一种不平等的桎梏。

虽然奴隶要过卑贱的生活,但他内在的性灵则与他人是一般无二的,他有权利改善本身的命运,所以只要有到处流浪的奴隶,佛陀必差人带来,为他们宣说法理,渡他们出家,因为佛陀认为若放任他们不管,他们终究还是会被人捉去做苦工,甚或被山贼杀死。

由于佛陀的教团中没有阶级种姓的差别,很引时人的注目,这种现象的产生是极其自然的,因为佛陀早在出家前就对当时的社会制度极感不满了,如今自然会尽一切力量来改善现况。然由于人们根深蒂固的观念与习惯,在教团中,一时之间有些人还是不免对出身问题有着强烈的分别心,并不能彻底体会佛陀的用心。

佛陀在定中知道那些人易犯这种毛病时,就会立刻将之唤来,谆谆诲以人性平等的道理,直说得对方流下懊悔的眼泪,被感化的人很快会以平等心回到自己原来的工作岗位,渐渐地,教团中由于阶级差别所引发的问题,也就一天一天地少了,佛陀为此,经常感谢上天慈悲的加被。

迦兰陀回到王舍城,马上向频婆娑罗王报告了这一个好消息。

迦兰陀为了竹林精舍的兴建,经常与迦叶尊者、憍陈如或耶萨等磋商,竹林精舍很快就动工了。

不久之后,精舍落成了。

由于迦兰陀的财力足够雇用许多木工,因此精舍的完成较预定的时间早,且构筑得十分壮丽堂皇。

国王将佛陀及其弟子们请来,并引导他们至各处观看。

「佛陀,您就替这一座精舍起一个名字吧!」国王露出欣喜的微笑说着。

在茂林修竹中兴建起来的精舍,可说是最适合修行者修炼身心的地方了,佛陀听国王如此要求,就开口道:「我想就叫它竹林精舍吧!我由衷感谢陛下您,还有迦兰陀施主。

像这一类的布施,能除去一个人心中的贪念,一般人都认为金银财宝会替自己带来幸福,或认为声望与名位就是步向幸福的一条捷径,事实上这些东西都是令人增长贪念的媒介,不会替您带来真正的幸福。

身外的事物都是无常不实的,身外的一切事相,都是乍看实在,等事过境迁,又都归于虚无,欲望在经过对无常事相的追逐之后,便在人的心上刻下痛苦与悲哀的痕迹,为了自觉到这一点而行的布施,其功德是很大的,能够使自己的心灵更丰盛,这一次您二位的乐捐行为,将会点燃众生心上的法灯。

一般人在面对强权的命令与使唤时,多半会无可奈何地屈服于一时,然而利心却随时在待机报复,所以我们往往看到这情形,即一旦时机成熟,失势的一方立刻会遭受到众人的抨击与唾弃。认清这个道理,我们一方面不可以强权来压众,另一方面,在遭遇任何横逆与侮辱时,都要隐忍下来,且不可在心中生起任何不满与怨恨的念头。

行正道,首重一颗善心的增长。来自「善我」的心是生不出怨恨的,怨恨来自「假我」,是被欲望所支配的一种不满的心理状态,如能日复一日地实践正道,怨恨心自会消失,而完成一个能忍辱负重的完美的自我。

一个人如果能恒常过着忍辱知足的生活,远离嗔恨与怨恼,必有开悟的一天。

没有财力的人,如果看到国王与迦兰陀施主的布施,而能心生随喜,也有同样的功德。」频婆娑罗王和迦兰陀边听边不住地点着头,并在心中反覆咀嚼佛陀的话。

佛陀布教的目标之一,即在于对贫苦众生的救济,当时的人对布施祈福的观念很浓厚,都认为贫困而无法布施的人,就永远没有得福的希望。这种想法,都是婆罗门的司祭们灌输给大众的,他们也就坐享人们的布施,摆出一副不可一世的神圣姿态。因此佛陀特别强调,就是无钱行布施的人,只要对他人的布施生起随喜的心,也同样是一种布施,他把布教的重点置于人性的平等与尊严上。

频婆娑罗王和迦兰陀再次领受到佛法的精深博大,并不断反省他们自身有没有陶醉在自己目前的地位、声望及财富之中。

佛陀自接受了国王的款待后,就与弟子们做着一切移居竹林精舍的准备。

这以前,佛陀与一千多个弟子住在森林及洞穴中,过着如乞丐一般的生活,今后他们将有一个像样的场所供他们避风雨,修道业。

每一个人都风尘仆仆,那脸庞和僧衣,由于经久的风吹日晒,都泛出黑油油的光,无论怎么看,这一批人都与他们即将住进去的竹林精舍极不相衬,这真是竹林精舍的一大奇观。

弟子们都以精舍为中心,或到乌鲁维拉,或到王舍城,甚至远到帕拉那西,羌巴及毗舍离等地去游化了。他们布法时,或一人独行,或几人一组,多半花好几天的时间,不时还要回到精舍接受佛陀的指示。

随着游化地区的扩大,弟子们的数目增加,于是竹林精舍增建房舍来容纳他们,然而日久之后,还是容纳不下,就有像迦兰陀一样的施主出现,捐赠土地,增建精舍。

佛陀经常要听取弟子们游化时的心得报告,有的人成绩很好,有的人经常遭遇逆境,但佛陀重视的并不是游化的成果,而是把重点放在弟子们游化时的心得上,他希望弟子们藉着游化来增长见闻,帮助自己成长,所以他把弟子们视为自己的分身一般,时时给予鼓励与支持。

由于有了固定的修行场所,弟子们能安心地四处游化,脸色开始红润,有许多人甚至自身后射出金黄色的光轮,足见他们都在努力实践正道。

佛陀本身也经常以严厉的尺度来批判自己,他不容许自己有些许的错失,因为小的过错会日渐扩大而形成扰乱身心的原因。

一年一度的雨季又将来临了。

弟子们聚集于竹林精舍内,向佛陀报告各地布法的情况,并请示将来布法的方针与做法。首先提到粮食缺乏的问题,由于有了精舍的存在,接着又想到金钱的储存问题,意见各殊,众说纷纭的。

佛陀见状,立刻制止说:

「存一点粮食在雨季中食用就可以了,雨季前大家要勤快地去化缘,不需要存得太多。」这时大家才猛然回想起往日在野地的生活情形,重行体会什么才是少欲知足的生活。

雨季期间,粮食一定较往常要少,到民舍去乞食时,民舍中的布施也在减少中,因为布施的人们同样要为雨季的缺粮期做准备,如果到山中去寻觅果食,往往由于山崩或水患,也是非常危险的事,从前在野地里群居时,弟子们的衣食问题都是各自解决的,如今一千多人共处于一个屋子里,要同桌用餐,需要的粮食数量自然就多了。虽然迦兰陀经常会布施,但佛陀认为不能养成依赖他人的习惯,所以他原则上还是要求大家维持从前乞食的生活方式,如今碰到雨季,众人就有了囤积粮食的种种意见。

佛陀自搬入佛陀后,还是要求自己过着如森林中所过的日子,从不储积不需要的东西。这种生活方式,是来自一种对本身佛性的自觉,只获取生活所需,多余的即布施他人,能使人经常沉浸于一种知足的快乐中。

现在的社会,已步入物质文明高度发展的时期,照理说,人们已达成了幸福快乐的愿望,生活越来越舒适,也越来越方便,许多事情都有机器来代劳,要什么就有什么,要怎样就能怎样,然而事实证明,人们心灵深处并不因此而感到满足或心安理得,那是为什么?因为我们内在的欲望在作祟。若想过心安理得的生活,必须把持住自己,不为外界的种种需求所迷惑,否则没有更好的办法。

知足的生活,是指人们在知悉人生的目的,并了解自己身负的任务而继承上苍的意志后所过的一种生活,所以人们越是知足,生活态度也愈积极,内心也涌现无比的爱及奉献一己的热诚。

要在文明的现代具备这样一种精神,又当如何呢?现在我们所处的社会,充斥了无数以自我为中心的人物,人与人之间已经丧失了那种可贵的共识,对社会人群的责任心与爱心皆已日渐不知去向,社会上现在已形成一种风气,那就是人与人相处时,都各自设法维护自己本身的利益,稍有道德意识的人,则在不违反社会规范的原则下从事自己的营生,商人尽力赚取利润,薪水阶级者则领薪水办事。在这样的心情下,商人多余的利润,薪水阶级的休闲时间,都将运用于何处呢?他们会纵情于欲望的满足,还是献身于社会的福利事业?

一个少欲知足的人,一定会将他的爱奉献给社会大众,到目前为止,有许多宗教信仰者,已迷失在善与爱的观念中,忽略了现实生活中的实践问题,只一味战战兢兢于抑制自我的满足与欲望。

欲望不是光靠抑制就能消弭的,如果我们体会了人生的目的何在,所有不善的欲望都会自然而然地消失,代之而起的将是神佛的广大的爱,这种爱将成为个人生活的原动力,充分地被发挥出来。当然,若想达到此一境界,个人必须肯于反省自身,且需具备斩除心田中的恶苗的勇气与努力,而斩除恶苗的一切努力都不应勉强,应每天精勤地督促自己不达目的不罢休,直至达到从心所欲而不矩的境地。

只要是心里仍留有「正在努力」的感觉,这种努力就不是真的,要能将努力溶合于生活中,每日在内心中充满着一日即一生的喜悦。

弟子们都能遵从佛陀的开导,如佛陀所说,多储存粮食是没有必要的,如有余粮,还应设法布施给贫穷的人。

其次他们又讨论到游化的问题。

当弟子们游化在外时,往往会和婆罗门种的修行者打照面,不免时有论争。

佛陀说:

「各位同修们,在摩竭陀国、喀萨拉国等地,多的是一些熟谙吠陀及五明之学的婆罗门僧,他们的智识相当丰富。而你们呢,自接受佛法以来,出家修行的资历是很浅的,所以当你们遭逢婆罗门僧或外道的挑战,就很容易丧失正确判断事理的思维,而在心中播下愤怨或争执的种子。

以后若再遇到故意挑衅而又说不出大道理的人,就不要理睬他,不要忘记忍辱的心,不要离开正道,继续过着扪心无愧的生活。

不要太容易受到他人言行的影响,否则心生迷惑,给自己惹来困恼。依据佛法,建立起每天的生活,日久必有开悟的一天。

伟大的智慧都是由不断累积而得来的,在不久的将来,佛陀在过去世中结过缘的弟子们,都将再度来做我的弟子,你们身为他们的前辈,应竭尽心力地发展自我,努力过着忍辱全功的生活,积极地从事济渡众生的工作。」

被灯火照耀着的佛陀的脸孔,充满了无比的信心。弟子们见佛陀日复一日地不断充实自己,深深体会到以佛陀为师的一种幸福感。

佛陀的话很简明,谁听了都能立刻明了,然话中的寓义极深,包含了宇宙中的至理,一个人在遇到与自己对立的任何情况时,都极容易丧失自我,使信念发生动摇。

偏差的思想能毒害人心,对宗教的偏执想法也会造成人心的褊狭,当一个人将目的与手段混淆时,不幸即接踵而至。

思想是一切行事能量的源头,思想稍有偏执,其行为所带来的后果,即不堪设想。

\section{迦毗罗卫国的期盼}\label{sec3.13}

释迦牟尼在达到「宇宙即我」的境地而成为佛陀后,匆匆过了一年,佛陀的心境,可以说就像在泥沼中盛开出美丽的莲花一般。

他抛弃了城堡、王位、双亲、释迦族人、甚至于连娇妻爱子都不要;毅然决然地出家修道。如今佛陀的心无着无碍,连那污点斑斑的僧衣都能显现出他那安详宁静心态。

质地样式细柔华贵的王袍,看起来是很漂亮的,并象征着权势,但掩蔽于其下的胴体与心灵,却是那么虚假与焦虑,没有机会一窥生命真实,只能庸庸碌碌地终其一生。

六年来的苦行,称得上是一场接着一场的苦战,但如今所获得的安详的心境,真是笔墨所无法形容出来的,那是一种充满法喜的宁谧境界。

人体充满了污秽,诸如眼屎、鼻垢、齿垢及耳垢等,没有一处是干净的地方。丑陋而污秽的身体,如果藏于其间的心灵能够觉醒并发出光辉时,也就和污泥中的莲花一样,能绽放令人瞠目结舌的鲜花嫩瓣。那就是说,在心灵中,能够开凿出贯串天地智慧之泉,就会有那样灿烂的成果,释迦牟尼由于抛弃了王位,而使他心中的莲花得以绽放,他终于抵达了佛陀的境界。

他的妻子耶输陀罗,还有儿子罗睺罗,可能会不断地怨恨释迦牟尼的无情无义,但是时间一到,得知佛法的真谛之后,也就能了解他为何要远离迦毗罗卫国了。

释迦牟尼成为佛陀,且拥有庞大的弟子群到处布法,这消息也很快地传到东北方的迦毗罗卫国。全国上下正盛传着这件事。

罗睺罗如今已经七岁了,却从未见过自己的父亲,因为在他一出世时,他父亲就离城而去了,罗睺罗平时只能从母亲、祖父母和宫女等人的口中获取一些父亲形象的概念。

耶输陀罗深知身为自己丈夫的释迦牟尼,已经是回忆中的人物了,即使是出现在眼前,也不会再留守在自己身边了。耶输陀罗的唯一希望,就是看着她和释迦牟尼所生的罗睺罗长成一个雄赳赳气昂昂的男子汉大夫,接掌迦毗罗卫国,以安慰自己受过创伤的心。

她所以能够把丈夫暂抛诸脑后,是因为罗睺罗成长得十分顺利。

阿私陀仙人的侄子卡加那在摩竭陀国游化时,听到佛陀的弟子们谈起佛陀的日常生活,于是把这消息带回迦毗罗卫国:「大王,我在摩竭陀国听到太子布教的种种传闻,他替人治病,劝人向善,连大财主都皈依他,而且频婆娑罗王也经常听他说法。」

老迈的净饭王听到这消息,不觉精神一振地说:「什么?连频婆娑罗王也听太子说法吗?」

他一边问,一边想起自己的儿子成长的情形,惊喜交集,虽然满脸笑意,感从中来,心中却波涛汹涌,起伏不定。

「大王,在王舍城北边的山坡地,有一处竹林精舍,听说是大财主迦兰陀捐建的,太子现在就和弟子们住在那里。」

「噢,是这样吗?那为什么憍陈如和阿舍婆誓等人没有把消息带回来?他们如今怎样了?你有没有听到有关他们的消息?」

国王说着,迫不及待地想一睹成长后的爱子,并听听他说的法。

阿私陀仙人在三十七年前就曾预言这件事,他说释迦牟尼悉达多太子长大成人后,不是各邦拥戴的大盟主,就是开悟说道的救世主。

净饭王虽然认为悉达多能成为一个悟道者是一件光荣的事,却衷心希望他能继承王位而把释迦族的威名传布全印度。

如今这一希望似乎已全部断灭,显然悉达多已成为一位悟道者,连频婆娑罗王都要听他说法。

净饭王如此一想,就对卡加那说:「您的叔叔是一位伟大的修行者,当他在悉达多太子一出世时,就预言他会成为一个悟道说法的人,当初他出家时,我真觉得天昏地暗,一切都那么无望。阿私陀仙人是一位很有远见的修行者,希望您也能成为一位了不起的婆罗门。」

卡加那听净饭王如此推崇自己的叔叔,很感骄傲,同时想起幼年时,叔叔传给他各种修行法门,他也不禁对叔叔怀念不已。他很感伤地说:「我也很感谢我的叔叔,托了他的福,我今天能直接和大王谈话。」

净饭王仍沉湎在回忆中,说:「……当时阿私陀仙人抱着太子,眼泪不停地流着,本王当时很奇怪他为什么要流眼泪,他回答说,要等太子长大成人还要一段时间,但他活不了那么久。他说他虽有幸遇到太子,却不幸没有多余的生命来看到太子将来的成就,故而很感遗憾,说完并老泪纵横地把长满白须的脸贴近太子。本王当时实在不知道他为什么要哭,如今看来,您叔叔的话一一应验了?」

国王此时也已泪水盈眶。卡加那仰慕的说:「大王,我也想皈依佛陀的正法。什么时候请国王引荐,让佛陀收我为弟子吧!在此之前,我要好好磨练自己。」

待卡加那离去后,国王唤来波阇波提夫人,跟她商量请太子回迦毗罗卫国的事情,虽说是商量,净饭王抑制不住兴奋的情绪,不断地自问自答着。

自佛陀出家后,国王突然衰老了许多,早已没有七年前的精神与体力了。

佛陀的堂弟难陀,早已长大成人,允文允武。王位继承人,非难陀莫属,但在净饭王的内心深处,极希望罗睺罗能继难陀之后来接替王位。

波阇波提夫人说道:「大王,我很希望太子能回来,罗睺罗已经七岁了,一定也很想见见他的父亲,难陀也长大成人了。赶快派人劝他回来吧!耶输陀罗,你以为呢?」

耶输陀罗也在一旁,这时见波阇波提夫人问她意见,就说:「请为罗睺罗的心愿着想,就派人劝太子回吧!」

虽然这么说,耶输陀罗很了解自己丈夫的个性,如果以邀他回宫说法的理由去请他,他或许会回来,如果是要他回来恢复旧时的王宫生活,已是不可能的事了。

「悉达多已成为一个开悟的人,因此他一定能了解我们的心情,知道我们盼望的是什么!」

净饭王像是看到悉达多回到迦毗罗卫国一般,正憧憬着父子团圆,相依为命过日子的情景。

当事情差不多讨论出结果时,净饭王和平常不一样了,显得神采奕奕,净饭王当下召来三位大臣,磋商派谁前去游说悉达多。

净饭王说道:「如今悉达多太子已成为佛陀,并领导着许多弟子,就住在摩竭陀国境内的竹林精舍。我很想邀他回宫,请各位多费点神想想,派谁去请他最合适。」

年纪最大的须克鲁达那说:「就由我来担当这一个重任吧!」

净饭王说:「你已是七十多岁的人了,怎么能到那么远的地方去?不错,您的心是还年轻,但我怕你的身体吃不消。」

年纪较轻的阿姆利德达说:「我想悉达多殿下不一定会听我们的,还是派车匿去一趟吧!他是一直照顾殿下的人,能跟殿下轻松的交谈。」

另一名叫都鲁达那的臣子建议道:「宇巴这人如何?他年轻,脚下功夫好,走得快,车匿虽是适当人选,但七年前的事给的他刺激太大。」

净饭王终于探纳了最后这个建议,宇巴很快被唤至国王跟前,他虽然不高,但外表潇洒,动作敏捷。他来到国王跟前,跪伏在地,等候国王派令。

「宇巴,你马上到摩竭陀国王的王舍城去一趟,那里悉达太子正领着大群弟子在布法,你就去劝太子来迦毗罗卫,该没有什么困难吧?」

宇巴深知这是一椿艰难的任务,在此之前,车匿已屡次遭遇失败,他每次都被太子拒绝而无法达成任务。现在这个重担居然落在自己肩上,宇巴实在没有把握,心中不断思索着,不知该怎么办。

「你明天一早就出发,这件事就交给你了,好好去吧!宇巴。」

「是,大王。」

宇巴曾经到过太子的修行场去送过衣食及日用品,头一次是由车匿带路,第二、第三次是独自一人去,但都被太子赶出来。这一次,他又要奉国王的旨意去劝太子到迦毗罗卫来,不知情况将如何,他并不抱太大的希望。

他言不由衷地对净饭王说道:「大王,我曾到太子那儿送过衣食,所以路我很熟,虽然以前都没达成任务,不过我相信这一次太子必会接受王的旨意。」

这是一番违心之论。他是被净饭王的威仪所震慑,说出了不合心意的话。

净饭王似乎很满意,这使宇巴更加不忍,离开大殿,宇巴长长地叹了一口气。

宇巴花了好几天的工夫才到达目的地,当时没有旅馆之类的住处,只能借住于一般的民舍。

身分高的人,大多事先连络当地的富买,商借住处,宇巴在王舍城中打听佛陀的住处,立刻有人指点,足见佛陀在一般人心目中的地位了。

「佛陀,迦毗罗卫那里来了一位使者。」

跋提是迦毗罗卫国的武将出身,自然认识宇巴,他一见宇巴,立刻兴起了故国之思,两人叙旧一番之后,他带宇巴来见佛陀。但通报时,故意不说出来者是谁。

「什么事?请他到这里来坐吧!」

佛陀也假装不知地回答道,其实他早就知道迦毗罗卫国会派人,然后佛陀接着问:「是宇巴吗?」

这时正要返身离去的跋提大吃一惊,当他边走边回顾佛陀时,不小心又撞到大柱上。他红着脸问道:「佛陀,您已经知道了吗?」

不久又恍然大悟道:「佛陀,真对不起您!」说完话立刻出去了。

宇巴被带到佛陀面前,恭敬地行了叩拜的大礼。

佛陀开口道:「宇巴,您一路辛苦了,所幸平安无恙,我一直很担心你旅途上的安全。我知道你为何而来,请你回去禀报父王一声,这一次我不能跟你回去,等我这儿的事都安顿好之后,自会找一个日子回去看望他老人家的,听说家中一切安好,我很高兴。宇巴,你近来可好?」

宇巴什么都不用说,因为佛陀已替他把话说完。

他神情恍惚地抬头望了望佛陀,一时之间,不知该如何启齿。

「殿下说得一点也不错。我真不懂,我是为何而来,我只知国王陛下、耶输陀罗王妃,还有罗睺罗殿下等都在引领翘盼您早日回去。殿下请暂且抛开一切,跟在下回宫一趟吧!」

佛陀很了解宇巴的心情,但他一想到弟子们的事,责任攸关,他不能骤然离去,撇下不管。

教团成立甫一年,弟子的人数在逐日增加中,每天都有新人加入。他要向这些人宣说教义,让他们在自己的心中牢牢地筑起遏阻邪恶的墙,并维持心灵的秩序。

佛陀在木片上写下自己的心意,交宇巴带给净饭王,以免宇巴未达成任务而受责。宇巴淌下了感激的泪,谢谢佛陀周到的设想。

今天的悉达多殿下,已与往昔在山林间从事严格修行的悉达多殿下大不相同了,不但气色好看,且再也没有拒人于千里之外的冷峻态度了。

佛陀具有如泰山般的包容力,正温暖地包揽着宇巴,目的虽然没有达成,但宇巴感到十分轻松。当净饭王指派任务时,他知道任务艰难,心中难免颓丧,料想自己是第二个车匿。但现在离开佛陀身边时,如释重负,心头有着未曾有过的爽朗感,将抵国门时,他突然想到焦急等待着的净饭王,突然心头又有了阵阵的压迫感。

宇巴好不容易地向国王一五一十地报告经过情形,然后交出佛陀托他带回的木片。

国王看完后,对宇巴说:「宇巴,辛苦你了,幸而一路平安,下去好好休息吧......」

国王有些垂头丧气的样子,等宇巴退下后,他马上进入寝宫,他躺在椅子上,对人世间错综复杂的因因果果,有着异样的感受。

这一次又失败了。

在这七年中,国王可说无时无刻不在想着佛陀的种种,他不明白为何自己的爱子就不替自己想想。当然,要他回城,是件不容易的事,如今他以出家之身,有许多人要他照应和指导,他当然不能说回宫就回宫了,但至少也该露一次面让做父亲的看一看哪!

宫中的人在听过宇巴的报告之后,也都觉得悉达多太子离他们越来越远了,于是再次受到寂寞感的侵袭。

净饭王不住地想,悉达多是我所生,但却不能为我所拥有,他深深地感受到一种从未有过的无奈。

想着想着,净饭王泪如泉涌,他无意去揩拭,就任由眼泪往下宣泄,透过书房的窗槛,他望向远处黑漆漆的森林。

国王可说是不虞匮乏,应有尽有。华衣、美食、楼阁、城堡、军队……他可以在自己的国度内支配一切,但尊为国王的他,有一件他不能支配的事,那就是他爱儿的事。

悉达多的心,国王奈何不得!

净饭王直到年华老大的今天,才感受到人性的不可思议处,也才明了人心的难以捉摸。耶输陀罗以其女人的直觉,早就知道丈夫不会回来,故而也看开了。她立下了决心,要把罗睺罗看守好,绝不让他从她身边离去,她要亲自教育他,并小心地看护他,谨防他对婆罗门学或其他的哲学发生兴趣。

在内宫,每当罗睺罗问到父亲,耶输陀罗总回答说:

「你父亲事情忙,分不开身,所以还不能回来看你。有妈妈在你身边,你还是好好用功吧!」

耶输陀罗倒不希望罗睺罗太接近他的父亲,她很怕罗睺罗探问太多有关父亲的事情,说不定哪一天,罗睺罗也会离他而去。

「丈夫已离我而去,如果罗睺罗也抛下我,我还有什么生趣?绝不让罗睺罗离开我,他是我活在世上的唯一安慰了。如果这世上有神明的话,他绝不可能一点慈悲都没有地伤害一个女人的心,一个女人只想好好守护家庭,和丈夫孩子过着和乐安稳的日子,她就只有这么一点小小的愿望,如果连这点小愿望都盼不到,那这世上还有什么慈悲可言?」耶输陀罗哄罗堠罗入睡后,自己踱到窗边,仰望满天的星星,心中不住地这么想着。

不管耶输陀罗的心情有多郁闷,夜空仍美丽地闪烁出点点星光。

偶尔,有颗流星划空而过,就像悉达多从自己身边离去一般,在远远的天边留下一道白色痕迹后,就倏忽不见了。

「神啊,请成全我这小小的愿望吧!不要再让我遭到任何不幸。」耶输陀罗望了望罗睺罗的睡脸,对着月亮诚恳地祝祷着。

又大又圆的月亮,在耶输陀罗的眼泪中,一缕烟似地溃散了。

\section{机缘}\label{sec3.14}

佛陀迎接他三十八岁的这一个新年。在房中,他进入了禅定三昧的境地。他要精勤地净化自己的心灵,使自己持续地处于心灵的调和状态中。

去年一年,由优楼频罗迦叶尊者开始,陆续有许多人来皈依佛陀,为了竹林精舍的建设以及弟子们的教育等等,佛陀一直过着很忙碌的生活,看来今年仍会像往年一样忙碌,他要求自己即使在忙碌中,也要保有一颗宁谧的心。

进入三昧境界后,他常与另一世界的灵相交流,这时,在过去世的生活中,那些思想与行为造成自己的业,就都了然分明地呈现眼底,他尤其看出在开悟前的生活,如何被过去世的业所拖累,以致延误了开悟的时期,于是他重新反省他的过去。

他在过去世中曾经身为国王,经常向下发号施令,在他印象中特别鲜明的例子是他具有强烈的自我,情欲的苦恼。他什么事都假手他人,对自己过于宽厚……,诸如此类,佛陀现在已能充分地了解,自己今生所想、所做的事,都与过去世有密切的关连。

心灵的世界是很灵敏的,变化之大,是无限制的,因此,在对弟子们或在家弟子们说法时,佛陀要求自己丝毫不得偏差。

新年过后,他似乎比去年更忙碌了,有一次,在禅定状态中,他听到一个声音:「佛陀,最近会有三个前生的弟子来访,这些弟子在接触佛法后,不久都会成为领导者。」

佛陀自开悟后,经常能收到这一类的讯息,不过他在听到声音时,仍会先衡量声中的内容是否合于正道,然后再决定是否接受,因为有的魔王也会模仿诸佛的声音与姿态来乱真。

在人间修行圆满而从生死轮回中超脱出来,这在真实界来说,是极其当然的事,然而要在物质发达的人间开悟并传布真理,却是一件相当困难的事,因此,诸佛及天众,经常会来佛陀面前,表现他们赞叹与拥戴的,心情。

众生之所以可怜,乃是因为他们无法预测自己的命运,而困在五官的烦恼中,若能打断烦恼的锁链,谁都能看清自己的未来及真面目。我们为了排除烦恼,必须点燃心上的法灯,然后在正道上全力以赴,肉体只暂时为我们所用,是永远带不走的,佛陀的任务是,除去众生心中对无常现象的迷惘,并在犹如泥沼般的污浊世界投注慈悲之光。

由冥想中出来的佛陀,再一次体证了生命的真实,对于自己在人间所过的生涯,感到无上的喜悦。

由于频婆娑罗王的通令,迦兰陀捐建竹林精舍的事,很快地从摩竭陀国传遍全了全印度,如今连三岁小孩也知道有佛陀这样一个人物了。

佛陀的令名远播后,皈依的人更多了,无论是年老或年轻的修行者,为了寻求佛道,皆千里迢迢地来请佛陀收容他们。然而同时,也出现了嫉妒他、诽谤的他人。

当光线强烈时,物体投射出来的阴影也格外显明。游化于各地的弟子们,经常会听到有关佛陀的闲言闲语,当然的有弟子就跟着起了疑惑,不过大部分的弟子都能谨记「忍辱」两字而自勉自励,如果是深切了解佛理且身体力行的人,是不会被不实的传闻所迷惑的。

佛陀经常这样说:「一个人的真价值,唯有在最困难的境遇或最安逸而可以为所欲为的环境中,才得以显现。」

阿舍婆誓就经常玩味佛陀的这句话。

这句话显现出人们处于两种极端的状态时,会有怎样的表现,而言外之意,则在阐明「中道」的重要性。

阿舍婆誓在王舍城郊外游化时,总会或多或少听到不愉快的事,但他从不理会,他深信合于正道的生活,日久就能获得人们的肯定,他的心因此从未动摇过。

今天像往常一样,阿舍婆誓一听到鸡鸣,就起身打点,到外地游化去了。当他到达王舍城的大街上时,天还没有亮,星星仍在空中闪灿。农家做早饭的炊烟,在破晓的夜空里嬝嬝上升。

阿舍婆誓来到他经常往访的农家,家主人正站在屋檐下,一见到阿舍婆誓就说:「师父,今天恐怕没有东西可供养您了。她正喊肚子疼,我真不知怎么办才好!」

五十多岁,身体瘦弱的老农,正困惑地抬头望着个子高挺的阿舍婆誓。

「是吗?真不巧,我的药草没带在身上……」

阿舍婆誓很替老农着急。

不一会儿,他突然想起佛陀的一件事。

佛陀每当听到有弟子腹痛或腰痛时,都用双手按住患部,因而解除了病人的痛苦。

阿舍婆誓起初不明就里,但他透过心眼,发现佛陀的双手充满光明,当手上的光传至患部后,病人感到一阵温暖之后,病就不药而愈了。

阿舍婆誓想:「我不妨也试试吧!说不定会有用。」

老农急切地说:「师父,您能不能替他看看是什么毛病?我真着急,一夜都没睡。」阿舍婆誓随老农之后,进入屋内。

室内阴暗,地面上正散放着各种农具。

房间内透出微弱的灯光,他藉此稍稍看清了室内的情形。另一端传来一个女人的呻吟声,她似乎正被什么绞住似地非常痛苦。

阿舍婆誓进入房内,见躺在床上的女人,全身淌着冷汗,双眼发红。他来到女人身后,左手按住腹部后方,右手轻轻按住胃部,然后慢慢闭上双目,心中默念佛陀,盼望光明能传到女人身上。

老农在一旁,目不转晴地注视着他妻子的脸。

不久,呻吟声消失,女人沉入睡乡。

奇迹发生了!

老农见此情景,眼眶充满了泪水,向阿舍婆誓合掌道:「师父,真谢谢您哪!您真是佛陀了不起的弟子,不知道像我这种下贱的农夫,能不能听佛陀教的道理?」

「太阳不但照耀婆罗门种,也照耀一切的生物,佛陀的法就像太阳一样,对谁都一视同仁,最重要的是,能依据佛理,每天过着正确的生活。」

「是这样吗?我们也可以听佛陀说法吗?可以不必祭拜神明吗?」

当时的印度,拜神的风气很盛,人们惯常摆设祭坛,向神合掌或祈祷。

自有人类以来,这是必然的现象,人们既然无法掌握自己明天的生命,自然会希冀另一种力量来保佑自己,同时使自己在危难中有所依恃。所以不设祭坛,不向神明祈祷,就于心不安。

而祈祷后,并没有祈祷后的踏实感觉,只希望一切由神明来负责。

祈祷神明已变成一种习惯。习惯实在可怕!人们不知道习惯与执着是串联着的,人们终其一生在习惯的枷锁中过着苦闷的日子。

阿舍婆誓对老农说:「真正想解脱的,就要从心地和行为两方面去实践正道,否则即使祭拜再多的神明,也是无济于事的。使自己的每一步都踏在正道上,才是脱离苦海的根本办法。」

说完,阿舍婆誓向老农告辞而去。

那女子的胃痛来自风寒,自经过阿舍婆誓的医治,她又能精力充沛地干着活。

从此这对夫妇也成为佛陀的在家弟子,开始了他们的新生活。

太阳已渐高升,人们又开始络绎于途,阿舍婆誓的心情非常轻快,他既卸下了被称为执着的重担,走路时的步伐自然无牵无挂。

王舍城的修行者,本来可说全属婆罗门教的信徒,修苦行的人们,都在山林中修行,偶尔也会选择某一民家的屋檐处栖身。

身为摩诃婆罗门(大婆罗门)的删闍耶,在弟子群中,有一个来自拿兰陀的修行者,名字叫优波室沙(即后来的舍利弗)。

他曾学过婆罗门教的吠陀及五明之学,是一个理论家,虽然他博览了婆罗门的教典,并参究了其他宗教的教义,心中仍不免疑云重重,他为了寻求真正的解脱之道,继续到各处去游化。

他已有两百六十个弟子,都是以婆罗门的教义为主,将他自己研得的理论来传授给他们,但他心田深处,总有疑问存在,变成心中日积的块垒。

优波室沙有一个朋友,名拘律陀(即后来的目犍莲),聪明才学也不同凡响,二人互相期许,谁若侥幸得遇真正的明师,必定提携另一人,共受法益。

拘律陀也约有一百六十个弟子,和优波室沙一样,也曾师法删闍耶,但心中总有不满意的地方,因听说真正的觉者已经出现,他有意去寻访。

帕拉那西郊外的大婆罗门巴拔里,以及喀萨拉国的阿西搭拔等人,预言佛陀的降世,在修行僧群中,成为很热门的话题。随着这预言的流传,有许多修行者竟自称自己就是那位降世的

佛陀,优波室沙及拘律陀曾先后造访那些人,结果都大失所望。

所谓真正的觉者,是能看透过去、未来及现在三世,能配合众生的根基来说法,其过去世的弟子也会一同来到这世上帮助传扬佛法。法是不会自相矛盾的,其中有一贯的道理,优波室沙所要找寻的,就是这样一种普遍的真理。

在婆罗门的经典中曾载有大转轮王佛陀应具有的相({\kaishu 译注:佛陀具有三十二相八十随形好,法相庄严圆满}),优波室沙实在不相信那些自称佛陀的人就是佛陀。

真正的觉者应有特殊的气质,并非只是外貌庄严而已。当然内在与表常互有牵连与影响,不过外貌有时可以伪装于一时,是容易混人耳目的,所以,谁才是真正的觉者,一时之间是难以判断的,像删闍耶,优波室沙和拘律陀也是在长期与他共处的情况下,发现有不足的地方,才决定另访明师的。

阿舍婆誓乞食完毕,正要返回竹林精舍,一路上他不断想起自己治病的经历。

优波室沙背靠树木而坐,静静观察这位迎面而来的年轻修行僧,只见这位修行僧仪态端庄,面色红润。和普通的修行僧不一样,他的周遭散发出从容不迫的气息。

「这个人不是泛泛之辈!」优波室沙不自觉地告诉自己。

阿舍婆誓走过去了,优波室沙想要唤住他,但不知如何启口,于是只好跟着他,心想一有机会就要和这年轻人交谈,但不知对方会如何。

走着,走着,优波室沙突然闪过一个念头:说不定这位修行者就是佛陀吧!他非常心折于阿舍婆誓的威仪。

阿舍婆誓早已到达阿罗汉的境地,目前正在力除心上的执着,向着心胸更宽阔的菩萨境界迈进。这时他觉察出身后有人跟着,他以为是有人要找他挑战,为了避免无谓的争执,他改变路径,转到往村落的路上,但是身后的男子仍紧跟不舍,阿舍婆誓只得故意若无其事地在一株大树下坐下来,当他开始吃乞化而来的稀饭时,优波室沙立刻铺上一块草织的垫子,毕恭毕敬地对看起来比自己还年轻的阿舍婆誓说:「修行人,请您就在这垫子上用餐吧!我是婆罗门种,名叫优波室沙。」

阿舍婆誓观察优波室沙的样子、态度,并听了他说话的口气,知道不是来向他挑战的修行者。但见他对自己如此恭敬,认为他必是认错人了,于是回答道:「谢谢您的好意,我坐那里都无所谓,您也请坐,我何德何能,竟让您这位年长的修行人来让坐呢?」

「不,千万不要这么说,请先坐在垫子上再说吧!我叫优波室沙,是一个婆罗门修行僧,到目前为止,我也见过不少的修行人了,但是从没有见过像您这样心灵安和、举止稳健的出家人。请您教我怎样修行才会达到像你所有的那种心境,请受我一礼,您愿意收我为弟子吗?」阿舍婆誓这才坐到草垫上,对优波室沙说:「修行人,请抬起您的头。」

优波室沙合掌于顶,恭恭敬敬地向阿舍婆誓顶了一礼,然后说:「谢谢您,请边吃边教我吧!」

说着,抬头望着看似只有三十几岁的阿舍婆誓。

「我叫阿舍婆誓,目前正在跟迦毗罗卫国王子出身的释迦牟尼佛学道。自王子出家以来,我都跟在他身边做护卫,至今也有八年多了,我也曾跟着他修过苦行,不过若要我教你佛法,恐怕我能力还不够。」

优波室沙见阿舍婆誓如此谦虚,心中大为感动,故而急切地说:「您一定能教我的,您既然那么谦虚,可见您的老师一定是一位伟大的领导者。」

停了一会儿,优波室沙又说:「请您教我佛法吧!我恳求您,即使是短短的一刻钟也好!」

优波室沙从阿舍婆誓这儿知道有一位更了不起的修行者之后,心中感到未曾有过的快慰,阿舍婆誓见对方如此诚恳,就暗自思忖该如何将佛理以对方较能了解的方式说出来。静思了一刻之后,阿舍婆誓开口了:

「我们的眼睛、耳朵和舌头等可接触到的东西,都是无常的,都会因时过境迁而改变形相。如果对这些无常的东西起了贪着之念,那心中的痛苦是可想而知。

执着使我们的生存变成一种痛苦,也将使我们抑制不住面临老境的悲哀,我们虽然想永远健康青春,但我们仍然免不了会生病,会老,会死。

这世上充满了欲求不满所造成的痛苦,也充满了生离死别、难割难舍的苦。

这一切的痛苦,都有它一定的因果关系。」

优波室沙听到这里,睁大了眼睛,眼中散发出光芒。

\section{皈依}\label{sec3.15}

阿舍婆誓心中念着佛陀,要说的话自然浮现心头,他就一句一句的拾起,对优波室沙说:「佛陀所昭示的,就是为消除种种痛苦,我们应设法找出痛苦的根源,然后以正确的行为和生活方式来铲除它。

放眼目前的各派宗教,无一不在教人膜拜神明,希冀藉助神明的力量来消除痛苦并获得幸福,又有许多修行者实行严酷的苦行,不断在暗中摸索,他们之中究竟有多少人了解心灵与行为的正确准则?

祈神或虐待肉体,都称得上是一种正道吗?佛陀说,整个宇宙的法,是人为的智识所无法了解或改变的,大自然的统摄之理,实在是人类自私的自我所无法领略的,为此我们当拓展心胸与见识,才能与大自然相契合,才能从各种烦恼中求得解脱,要拓展我们的心胸与见识,就应该过一种合于正道的生活。

万物的存在,都有一定的关系,没有一样东西可以单独存在的,草木必须依赖水、太阳、沃土等而生长,我们的肉体也因大自然的生物而得以保存,任何人只顾做自我保存的努力,发展自私的欲望,就无法与人维系和谐的生活。所以佛陀说,诸法之道亦即中庸之道,是不偏不倚的,佛陀由此教给我们八条正道,并要我们以此做为心行的规尺,时时反省自己在生活上的种种,以净化自身。」

优波室沙听了阿舍婆誓的话,心中的疑点一一冰释。

他衷心感到喜悦,因为他已找到了真正的皈依处。

「阿舍婆誓大德,真谢谢您,我的疑问总算解开了,真不知如何感谢您才好,请领我去见佛陀吧!我有一个朋友名拘律陀,也是删闍耶的弟子,我们曾约好,谁若先见到真正佛陀,一定要通知另一个人,请您务必答应我!我马上就去拘律陀那里,把这好消息告诉他,如果明天能见到佛陀,那该多好啊!拜托!拜托!」

优波室沙说着,又向阿舍婆誓合掌顶礼。

他兴奋异常,在他的心田,充满了无可比拟的喜悦,他长久以来的心愿即将实现了,他即将遇见真正的师尊,真恨不得此刻就随阿舍婆誓前去晋见。

阿舍婆誓与优波室沙约好了下次会面的时间地点,优波室沙的面庞为一圈淡黄而美丽的光轮笼罩着,阿舍婆誓很感欣慰,显然优波室沙的心上已点燃了法灯。

两人分手后,优波室沙一路向拘律陀的住处走去,轻风徐来,树叶在风中摇曳,好像一波波的浪纹,也在为优波室沙欢呼。优波室沙的脚步轻盈,耳畔传来树叶沙沙的美妙旋律。

他赶向拿兰陀的修行场,他强抑住兴奋的心情,尽可能放慢了脚步,那种想冲到好友面前报告好消息的冲动,时时撞击他。

抵达修行场后,他很快就找到了拘律陀,拘律陀正在跟弟子们谈婆罗门教的种种。

拘律陀和优波室沙一样,体格魁梧而结实,优波室沙的容貌比较温厚,然拘律陀则经常眼透炯炯之光,给人精明干练之感。

一个弟子靠近拘律陀,告以优波室沙的来意。

拘律陀点头示意,将话告一段落,然后来到优波室沙身边。

「好久不见了,近来好吗?」

拘律陀说着,露出纯白的牙齿开心地笑了。

「很好,你看来也很健朗,刚刚听你讲的那段话,觉得你很有说服力。」

「你真是过奖了,什么事都重经验,若说演讲,我那能跟你比!」

二人相视而笑。

「今天我来的目的是,拘律陀我们可能有机会见到真正的圣人了。我在王舍城郊外遇到一位名叫阿舍婆誓的修行人,他真了不起,我已跟他约好时间去见他的师父,他师父是一个圣人,你也一道来吧?」

说到这里,优波室沙呈现严肃的神情,跟平常不大一样了。

拘律陀默默地注视着优波室沙的眼睛,心想这一次优波室沙和他可能会遇到真正的佛陀了。

「优波室沙,那真是太好了!既然他是你千辛万苦找到的师尊,又替你解开了心中的结,那他必定是真正开悟的人,你一定得带我一起去!」

他二人情同手足,在修道的历程中,常同甘共苦,在心灵方面有着非常难得的默契。

两人曾经一道寻访明师,但至今未曾过真正使两人内心洋溢喜悦与光明的明师,他俩从前每一打听出某地有某圣者时,就会立刻相偕往访,可叹的是每次都乘兴而去,败兴而返。这一次就不同了。拘律陀从优波室沙的眼神中看到了他的心,自己也不期然地起了同一个念头,想早一点见这位佛陀。

二人当天夜里,就悄悄地离开了修行场。

夜空为乌云遮去了大半面,这是悄然行动的最佳时机,但夜暗星不明,赶路就像在黑暗中瞎摸乱闯一样。

他们过了村落,进入山林,走到崎岖不平的山坡路时,不免步履踌躇起来。于是二人互相呼应,勉力而行,目的地是王舍城外的竹林精舍,暗中摸索着赶夜路,就好像佛陀已在他们心中点上了法灯,他们一点也不以为苦。

佛陀到底是何许人物?觉者的境界又是如何?他们想到过去遇见过的教团领袖们,以及他们的长相,然后再想一种不属于他们的类型,而这一类型具有伟大的包容力,如此想着,他们的步履越来越轻快。

天际露出曙光,东方现出蓝白之色,浑圆的穹苍,正慈爱地俯视着他俩。

「拘律陀,我们休息一会儿吧!应好好休息一下,以便见佛陀时,能有更充沛的精力。」「不错,我们是应该休息一下,我想中午就可以到达目的地了。」

两人找了一处洞穴躺下来,但优波室沙太兴奋了,睡意全无。

他们来到阿舍婆誓指定的地方,虽然两人一夜未曾合眼,此刻仍神清气爽,面色怡然,毫无倦容。

阿舍婆誓如约前来接两人,并说:「佛陀欢迎您两位到精舍。」

优波室沙闻言,不觉高兴地说:「阿舍婆誓大德,真打搅您了丨」

于是介绍了身边的拘律陀,拘律陀看了阿舍婆誓的举止风度后,深觉一个做弟子的都能有如此非凡的仪表,他的师父更不必说了,必是真正的佛陀无疑。

他恭敬地对阿舍婆誓说:「请多多指教!」

阿舍婆誓未耽搁时间,很快就带他们来到佛陀的处所。

两人早就耳闻竹林精舍得自一位富贾的捐献,如今见精舍的架构虽然朴实无华,但气势宏伟,都不由吃了一惊。

这里既能容纳一千六百多人,其规模是可想而知的,还有一点不同的是,这里未设任何祭坛,也全然没有祭祀神明的神秘气氛,倒像是一所为修行大众所提供的聚会所。

精舍恰如其名,四周为竹林密密地围绕着,幽僻寂静,充满了宁谧的气氛,适合灵性的修养,两人恍如来到另一个世界。

进出精舍的修行者,虽然衣着简陋,但是都态度和悦,给人亲切的感觉。凡遇到阿舍婆誓的修行者,都恭敬地向他合掌行礼。

两人在见到佛陀时,突然有些胆怯,等行礼问讯后,等佛陀开口,一直注视着两人的佛陀,这时微微一笑开腔了。

「你是优波室沙,你是拘律陀,是吗?欢迎你们到这里来。我看得出你们曾多方探访明师,是热心的求道者,你们辛苦地走了一夜,不觉得累吗?」

佛陀穿着朴素的僧衣,态度开朗,但隐约透着威严,两人都像被什么定住似地,一时答不上腔。

「佛陀,请您收我俩做弟子吧!」

优波室沙也未得拘律陀的同意,就代做主张了。

他在见到佛陀的一瞬间,心里十分激动,就像是与离别多时的恩师重逢一般,他一面感觉自己气平心安,另一方面,又必须集中意志来压抑从心底涌起的感激之情。

还有什么需要说的呢?不只是优波室沙有这感觉,拘律陀也有同感,拘律陀从一开始就身不由己地,无法克制从体内不断涌出的感激,因而脉搏跳动,身体摇晃。一听到优波室沙说出心中的愿望,也立刻接口道:「佛陀,好久不见了啊!」

说着,一个堂堂七尺的男子汉,竟嚎啕大哭起来,这一种情景,一般人是无法想象的。对一个初次见面的人,竟会感动得流下泪来,一个大男人由于无限的怀念与思慕,情不自禁地大哭起来,这就物质世界而言,是很不可思议的。

为什么会有这一种情况呢?佛陀已经是一个大彻大悟的人,透过肉身,他可以不断放出光芒,于是在过去世中与他缘份深厚的人就经由此光,打开了心扉,潜意识之下的意念带因而产生作用,将记忆达于表面意识,超越了现世的感受,回复到对过去世的记忆。

一般人只有现世的记忆而已,然而一旦心扉开启,过去世的事就好像昨天才发生过,一个人经由对过去世的体认而有了激动的心情。

其他人是无法看透这情形的,只有身为开悟者的佛陀,能透视过去、现在和未来三世,因而能明了这一情形,同时,心扉开启而对过去世有所觉醒的人,才会体会这一种感激之情。二人在佛陀面前哭着。佛陀也流着泪。

「你们所尝过的辛劳,都不是白费的,从今以后,只要你们肯依正道修行,自能将你们过去的智识转变成智慧,最后能达阿罗汉的境地。好好修行吧!你们也可以跟你们的弟子商量去留的问题。」

佛陀说完,又好好地安慰了二人。

优波室沙和拘律陀见佛陀情理兼顾,很喜悦地说:「我们一定依您的吩咐去做,我们马上回去把事情安顿好。」

虽这么说着,二人实在连一刻也不想离开佛陀一步,二人在正式被收为弟子后,又回到拿兰陀修行场。

优波室沙向父母亲报告了他皈依佛陀的经过情形,这消息大大使做父母的震惊,但由于优波室沙的详尽说明,以及所表现的坚决态度,他们终于了解儿子将有更好的去处。做母亲的满怀希望地对儿子说:

「我也很替你高兴,你终于找到一位真正的智者来做你的老师了,希望你一面努力锻炼身心,一面祈祷自己早日也成为一个优秀的修行者,你应该到删闍耶师父那儿去辞行,感谢他对你的教诲。」

优波室沙的母亲是一位很出众的女子,她平常很重礼节,优波室沙依言来到删闍耶师父处,说出了自己的辞别之意:

「师父,您对我的照顾和恩情,我将永志不忘。由于我对人生充满了各种疑问,如今我已找到一位能解答我疑问的人,人称佛陀,我将率弟子前往王舍城了。」

删闍耶并不能体佰优波室沙的心情,仍以逐出师门的方法来处置他,虽然删闍耶也听过佛陀的事,当他知道自己的高足以及更多的弟子都将前去归投他时,他大为震怒,他立刻把优波室沙的兄长唤来,要依婆罗门的教条来严厉制裁优波室沙。

当时的婆罗门,分门别派,各据一方以加强自己一派的势力,并不断自我炫耀着,都认为自己才是正统正传,不承认别派的存在价值,删闍耶所率领的婆罗门,也是如此,从不因别人的优点而反省自己的不是。

删闍耶头脑好,又极有口才,是一位十足的学者,可惜心地狭窄,并不适宜当一个宗教家,初听他说法的人,都会被他话中的道理所折服,但日久之后,都觉得他不过是个理论家罢了,于是有不少人离他而去,删闍耶深以此为忧,故在收弟子之初,都先要他们缴誓约书,以防信徒离散。

删闍耶为人偏激,好勇斗狠,像他这样精力充沛的人,本应是一个很好的领导者,但如前述,由于风度不够,信徒虽然熙来攘往的,却很少固定下来。

优波室沙和拘律陀就在这样的师座下学习婆罗门,由于师父的教理和实际生活脱节,早就有离此他往的心愿了。

优波室沙也早就料到若皈依佛陀,他和师父之间必有一场纷争,果不出所料,删闍耶的表现非常激烈,根本不像一个宗教家。

如果盲目地将自己的信仰视为至高无上的信仰而起了执着心,则最后连自己的行动都无法控制了,有时连一般人都说不出口的侮蔑之词,竟会发自堂堂的宗教领袖之口,不可思议!

这样一来,使信徒们何去何从呢?这一位宗教领袖已不如一个贩夫走卒了。

每一种信仰,总不免代表一种思想、一种主义,且在长久的岁月中,有了固定的程序,一旦有不合程序处,信仰者就自然产生排斥的心理,这一心理趋势,在教理越有矛盾冲突时,就越明显。因为信仰者若承认了这一矛盾的存在,则信仰会自根本崩溃,故而只有不去承认,并力除外来的非难,此外没有更好的办法,尤其在当时,信仰是生活上的基石,因而也很容易就跟政治连结在一起,所以信仰与政治就如车子的两个轮子,一旦滚动起来,就支配了各阶层的人心,如果两种集团的领导者无法摆脱欲望与执着时,就有许多人要受害了。

这种例子,自有人类的历史,就已屡见不鲜了,如删闍耶的态度,可见错误的信仰会导致多么可怕的后果,当时的印度,由于各宗各派的宗教树立,整个社会形同战乱的局面,弱肉强食的血腥事件时有所闻,每一个地方的领导者都在为了贯彻自我的主张而与他人对立着。

佛陀因此不断告诫弟子,执持一己的主张,是不合正道的,即使别人无法了解正法而诽谤正法,也要谨守「忍辱」二字,不可为了好强争胜而乱了自己的方寸,佛陀在修行时体悟过这一点,即当心灵被愤怒之情支配时,与天界的交通会顿时停止,一直等到愤怒的振动停止后,才能再有相互交通的可能。

如果我们经由五官在心中筑造了烦恼的巢,想再除掉这个巢就很困难了,就是为了不再筑造新巢,「忍辱」二字是非常重要的。

优波室沙情愿忍受删闍耶的责难,不作任何抗辩,接着,优波室沙和拘律陀仍各自带着自己的弟子,向竹林精舍赶来,佛陀迎接二人时,把竹林精舍里所有的弟子集合起来,对他们说:

「现在要介绍一下刚刚归依佛门的沙罗门们,这一位名叫优波室沙,坐在他后面的是他的弟子,优波室沙将会向你们布法,成为一个很好的领导者。在优波室沙左边的修行人,名叫拘律陀,是优波室沙的朋友,他们自幼就是修行伙伴,也是同道,他和优波室沙一样,将是很好的领导者,坐在他身后的,是他的弟子。

他们从今天起已皈依了佛门,各位也请好好地依据佛法来修行吧!」精舍中的弟子们听沸陀这么说,立刻哄然起了一片议论之声。

因为他们听佛陀说这两个新进的沙罗门,将要指导他们这些先进的修行人,心中起了不平之意,无论在那一种团体中,一向都是前辈指导晚辈的!当然也不是说前辈就要一辈子指导晚辈,我们要尊敬前辈,而前辈是否成为领导者终其一生来尽义务,这是另外一回事,当我们看一个灵魂在过去世中的经历,有时晚辈反倒能负起责任起带头作用,这种情况不管在那一种团体,也都可能发生的。

由于当时的社会早有严密的种姓制度,使人们对身分阶级有着牢不可破的固定观念,即使在佛陀的僧团中,仍不免有人受着阶级制度的影响而有先进、后进的差别意识。尽管佛陀一再地强调平等的观念,弟子们那种旧日的习气仍难根除。

弟子们骚动了一阵,许是在心上有过了反省,不久也就重归寂静,竹林精舍又恢复了往常的宁静,然而优波室沙和拘律陀听了佛陀那一番出人意表的介绍词,心情反而紧张起来,不知所措。

\section{\xpinyin*{盂}兰盆会和供养}\label{sec3.16}

优波室沙和拘律陀在事后,与自己带来的弟子们互勉,希望自今以后,一切从头开始修行。

二人在精舍中,无论是清扫工作或洗衣工作,往日都有弟子代劳的,如今都亲自动手。自皈依佛门后,大家都是佛陀的弟子,早就没有往日的师徒关系,二人要带头打破这层关系,虽然许多粗重的杂事做起来很辛苦,他们都强忍,努力适应这种生活。

每当佛陀说法时,他俩倾耳谛听,唯恐听漏一个字,因而往往进入了忘我的境界,若心领神会了一个道理,就立刻起而实行,但是二人越是认真,周围的先进们就越感到不满,那些先进们虽然在佛陀面前仍表现得很平静,但心中无法摆脱阶级的意识,对二人不怀好意。

这一段时期,耶萨已回到故乡,在帕拉那西的郊外,向许多婆罗门、吠舍(工商界人士)和首陀罗(奴隶)宣说佛陀的正法。他为了聚集更多的皈依者努力着。皈依者增多时,他将请佛陀前来。一到晚上,他就和俱梨迦长者一起商量并计划这件事。

耶萨在故乡很受欢迎,尤其女孩子们,更是趋之若\xpinyin*{骛},想诱他还俗的女孩子可真不少。耶萨很有男子气概,脸庞具有挺秀的轮廓,加上是地方上望族的独子,这些条件都是女孩子梦寐以求的,尤其因为佛陀之法主张平等,并将性别及阶级等的差别意识减少到最低限度,所以无论是谁,都有机会接近耶萨,耶萨自然和女孩子见面交谈的机会也就多了。

耶萨出家的动机起自一位女性,因此他对女人的看法就与众不同,有了高度的警觉性,不管女孩子如何引诱他,他绝不让她们跨越雷池一步,为了防范女孩子的非非之想,有时就故意在自己脸上涂些黑泥,穿着破旧的僧衣。

耶萨因此发现,虽然自己已达到阿罗汉的境地,但要时时拨正自己的心念以配合正道,也是相当困难的。他曾数度陷入苦恼之中。虽然烦恼之火早已由他心田熄灭,但是心头只要一有空隙,似乎就另有火种来填塞,人心难以捉摸,阴晴不定,要时时善加守护。

耶萨不断以中道之理规诫自己,在心理与行为上,精勤地下工夫,当他无法挥掉心上的暗影时,就念念坐镇在竹林精舍中的佛陀,以增加自己的信念。

优楼频罗迦叶则早已抛掉了一切的执着,他在自己一百多岁的高龄上,得护受佛法的无边利益,对于这一点,他很觉心安而悟出了至上的静定之理。他在竹林精舍的郊外,默默沉入冥想,进入了三昧禅定之境,他的两个弟弟也和他一样,在修行道上互勉互进。

竹林精舍中的僧团,本来一直由佛陀最早的五个弟子——憍陈如、阿舍婆誓、跋提、迦叶和摩男俱利——在照顾一切,并负指导之责。如今闯进了优波室沙和目犍莲两人,大家始终无法接受,于是有关两人的闲话就时有所闻。憍陈如很感困扰,去问佛陀:「佛陀,自从优波室沙和拘佛陀来了之后,到处都有不平之声,请问该如处理?……」

佛陀也早就觉察出教团中的一股不协调之气,于是指示下午说法时,要全体人员参加。

憍陈如离开佛陀的房间,立刻知会了摩男倶利等人,传下了一道通令。

修行者们都很好奇,不知将有什么重大事情宣布,一一怀着心思来到集会的广场。

大家聚在一起时,不免要各谈自己的修行经验,渐渐就分成许多小组,谈得十分激烈。

每个人的皮肤都晒得黝黑发亮,显出健康的样子。但中间也有不少人为了想早一点得阿罗汉果,显出了焦虑不安的神情,他们不知做反省的工夫,由着自己走入岔路,时常感到痛苦不堪。虽然他们也曾接受前辈和同修的指导,并设法矫正自己的错误,但人数日益增多后,指导者的人手有限,在修行时不免就产生了过与不及的现象,其中多的是思想言行偏激的人。

佛陀来到众人面前,广场上立刻鸦雀无声,佛陀环顾了一下四周,缓缓地开口道:「人应该累积转生轮回时的各项体验,知道我们人之所以会出现在此时此地,是因为有着永恒的生命,经过实在的意识和物质的现象转生而来的。人们在心灵被抹煞的现象界中,各自处在自己业力选择的环境中,尝试着灵魂的新的体验,人们虽然盲目地过着一生,但由于自身累世的体验,都渴望在富饶的心灵和和谐的社会中生活,佛陀就是来指引众生一条可以达到此一目的的路,并为众生带来光明,佛陀是带着这个心愿降生人间的。

修行者们,你们之中,有人带着狭窄的心胸很早就成为佛陀的弟子,但也有人在转生的过程中,已拥有既宽广又丰盛的心灵,至今才成为佛陀的弟子。

他们今生今世虽然比别的弟子晚一点来到佛陀身边,但是他们在过去世中早已结过佛缘并修习过道法,所以你们不应以来到佛陀身边的先后时间,来推断那个人的性灵。优波室沙和拘律陀两人,在佛陀的前世中,早已修到了菩萨的果位。

修行者们,在今生有缘学佛的人,如果能每天精进地修行,不断反省自身,不久自己就会悟出这个道理。你们应该尽力去回忆过去转世的体验,就可拥有取之不尽的智慧宝库,宝库的门扉到底由谁来开启?无疑地是你们自己,宝库的门扉是在你们实践了正法之后自会开启的,你们心灵王国的统治者不是别人,而是你们自己。

为了达到般若波罗蜜多(最崇高而深沉的智慧)的境界,必须切断对一切事相的执着,而以到达快乐宁静的彼岸为目标。修行者们,做一个佛弟子,不可以拘泥于这一世的境遇而心怀不平,心中有不满的情结,正表示个人的欲望正在孳长。

欲望会在心中造成邪恶,遇有机会,就会毒害他人,这样也就替自己播下了苦恼的种子,佛弟子的心胸不可那般狭窄,在今世有缘先遇佛陀的弟子们,对晚辈应有谦虚的态度,从而确立起健全的自我。」

佛陀的话如钟之撞击,一声声地回荡在四座间,那些被不满情绪包围着、不知不觉脱离佛道的人,也因此发现了自己的渺小与可耻。

「各位修行者,佛陀经由过去六佛的转生,来为众生点燃法灯,并济渡他们出离苦海,各位是在过去的佛世间,由于任何一种缘份而成为佛陀现世的弟子。优波室沙和拘律陀早已在过去的佛世间体验过缘生的道理,在今生,他们曾四处寻访明师,最后来到这里。因此无论是优波室沙,还是拘律陀,他们五体发出的光芒,就如同从喜马拉雅山顶滚下的雪块一般,晶莹剔透,各位如果也能依佛法生活,自己就能看到那道光芒。」

佛陀的话在此告一段落,阿舍婆誓站起来发问:「佛陀,我第一次遇见优波室沙时,看到他头部的四周有一圈淡黄色的光轮,但如今这光轮已包围他整个身子,请问佛陀,其差别何在?」

佛陀注视着阿舍婆誓,笑着说:「阿舍婆誓,你问得好,我们看,当雨季之时,摩竭陀的天空都被乌云密盖着而看不见太阳,当雨季过后,偶尔也仍会有此现象。

一个人如果能依佛法来矫正自己的思想与行为,当他脱离执着时,心头的暗昙除去了,再经反省和禅定的工夫,由身中投射出来的光自有不同。

太阳的光本是平等普照的,只是云层偶尔遮住了它,神佛的光亦复如是。当人们的心灵达到某一程度的统一和谐,就能现出光轮,光轮的差别在心灵的和谐程度上。

优波室沙和拘律陀因为自接触佛法后,即努力精进地依佛法修行,所以早已将心中的暗昙消除了,他们不久将成为阿罗汉,若更进一步地修行,就能达到菩萨的境地。」

佛陀的法,经过阿舍婆誓的发问,给了在场听众一番更真实的感受,佛陀的姿态,在阴暗的竹林精舍广场,渐渐浮现在痛彻前非的弟子眼底。

他的身体除了有一圈光轮之外,还放射出好几道强烈的光芒,向四方照射着,就像黎明时即将从地平线上跃出的太阳光,神奇美妙,光彩夺目。弟子群中,有的人是头一次看到佛陀的光轮,置身其间,都惊讶不已,敬畏、感动、喜悦等情感,从心底涌现出来。

经佛陀的开示,优波室沙和拘律陀更深切地体认到缘生的道理,也是坚定了他们的信念。自佛陀说法后,又过了七天。

拘律陀谨守着佛陀的正法,将他从出生到现在的所想、所做的事情,一一地加以反省与检讨,俾扫除心头残留的暗昙。反省得越深入,眼前越明亮,终于他发现自己置身于前所未见的旖旎美景之中。他的心境很安稳,身体没有丝毫怪异的感觉,美景中的绿色,非常绮丽,他以往所游历过的任何一处名胜,都不曾有过这种绿。

拘律陀置身于这一片天地中,对一切事物的看法都幡然改变了。他非但感觉到生存的喜悦,且对身边所发生的一切,都产生了感激涕零的心情,同时,他从不同于以往的角度接触到佛陀伟大的一面。

拘律陀对优波室沙透露了自己的这一心境,他说:「优波室沙,我从昨天开始,心中充满了佛陀所说过的那种光明,感到无限的和平,体会了三眛的境界,就好像佛陀全身射出的光轮给人的感觉,除伟大之外,不知如何形容,优波室沙,你的身体也射出美妙的光轮,而且在你右边还站着一位像佛陀一般的人,正微微笑着……」

拘律陀用手揩拭沿面淌下的泪水,为了确定自己刚刚所说的话,他似乎不相信地用手再揉眼睛,然后凝视身边充满光明的人。

「你已经能够看到了吗?真是我佛慈悲,我心中还有阴影,所以还没有到达那个境地,大概是法的实践还不够吧!」优波室沙一边高兴地握着拘律陀的手,祝贺他有了长足的进步,一边不忘反省自己。

五、六天之后。拘律陀回想起已故的母亲。他很想知道母亲亡故后,在那一个世界过着什么样的生活。他因此闭目心念母亲,不一会儿,母亲的影像现在他眼前。

「咦?......	那不是母亲吗?啊,面容多么憔悴瘦削,正寂寞地往这里看,她似乎想要什么?」

拘律陀简直不敢相信自己的眼睛,他镇定地注视着站在暗处的女子,他的母亲显然想要喝水。

于是拘律陀将一钵水递给母亲,说也奇怪,盛满水的钵一端到母亲的嘴边,立刻就化成一团火,拘律陀尝试了好几次,但总是被火隔住了。

拘律陀无法解开这一谜团,于是来到佛陀这里,报告自己的经历,并问佛陀是什么原因。佛陀回答说:「拘律陀,恭喜你了,你的心眼已开,你非常精进。  你的母亲虽然生前是婆罗门出身,且经常受信徒的供养,但是她既无服务人群的想法和作为,又极自私而爱慕虚荣,所以你母亲目前已入了饿鬼道。」说着,佛陀静静地注视着拘律陀的眼睛。

拘律陀闻言大惊,目瞪口呆,不知所措,等他从惊愕中甫定,又急忙问佛陀:「如何才能解脱我母亲的痛苦呢?母亲给我的养育之恩,我要设法报答,我要把母亲从火焰地狱中救出来。为了救母亲我什么痛苦都能承担,佛陀,请问我该如何做?」

拘律陀的母亲,个性倔强,在死时心中充满了嗔恨心,所以死的时候面容有如阿修罗,这件事,做儿子的心里明白,所以他更加怜悯自己的母亲,如今自己已成为佛陀的弟子,深明母子亲缘的道理,心想若能搭救,他一定赴汤蹈火,在所不辞。

佛陀看出拘律陀的心思,就对他说:「拘律陀,人各有命,也各有业缘,你应当明了这个道理。你母亲在现象界中造了罪孽,自然要在另一世界受满痛苦的果报,等苦报受满才有希望再转至较好的境界。

难得你有一片人子的孝心,愿意替你的母亲担过,这种心必能感动,使她规过向善。然而业报的力量是很大的,要想消除,须付出加倍的努力、勇气与决心,你若有心解脱你母亲的痛苦,单靠你一个人的力量是不够的,你可求助于其他的修行大德,集合他们的道行和神通力,或可解除你母亲的痛苦。

七月十五日,是结夏安居的圆满日({\kaishu 译注:在印度,四月十五日至七月十五日,由于天气燥热,毒虫众多,外出易误伤生命,故修行僧皆安住精舍,致力参禅修学,又名「坐夏」。})其中必有许多得道的僧众,你可敬备香花水果等诚心供养他们,诚心求他们助你一臂之力。

此外,你为了报答母亲的养育之恩,在有形的供养之外,还应替你母亲多行布施,将你的力量奉献出来服务大众,更重要的是,以你的神通力,让她知道自己坠入饿鬼道的因缘。要让她知道,当心灵忘失正道而只知沉溺于五官的知觉中,并随着烦恼过日子,终究会住在适合自己住的地方。

各种遭遇的前因后果,都来自本身的造作。所以留在阳世的人,若能好好实践正法,能使堕入恶道的亲友的灵,易于了解正法的真谛。拘律陀,你务必要好好依佛法精进哪丨」

拘律陀将佛陀的话铭记在心,知道自己未修行好时,是无法救渡母亲的,他进一步想到,他之所以能见到佛陀,是因为母亲赐给了他这一个肉体,为报答这个恩典,他也要竭尽心力来供养他母亲。

「拘律陀,心情和悦地生活着,保持身体的健康,这才是对亡母最好的供养。也就是说,对佛法的供养,将是堕落恶道的灵最好的精神粮食。」

除了拘律陀外,其他的弟子这时也都已聚来听佛说法了,同时也悟出修养自身是人子的尽孝之道。

自此以后,每年的七月十五日被订为盂兰盆会的日子。({\kaishu 译注:盂兰盆,本为印度语「乌兰婆\xpinyin*{孥}」的转音,意译为「解倒悬」。倒悬者,比喻困苦到了极点,犹如受着倒挂的痛苦。})

今天一般的信众在祭祀时,只是在形式上供一些香花水果于佛坛,或诵诵经,完全忘了佛陀所强调的实践正法的祭祀意义了。

祭祀祖先应具有的真精神,即在于世上的子孙能健康、开朗地过着合乎正道的生活。这意思就是说,意识界中的地狱界充满了苦恼,在那儿的灵往往不堪其苦,就转而求助于在现象界的亲人,人情之常,都是向亲人或好友求援的,很少去找陌生人。又因为他们是在现象界造成自己心灵的黑暗境界,故而很容易又会将心思转至现象界。

地狱属四度空间,虽然属三度空间的现象界看不到四度空间的世界,但四度空间的世界却可以看到三度空间的世界。

地狱的灵常会找机会来依附生前的家,这时,如果他看到在阳间的家人都能过着快乐和谐的生活,反顾自己却生活在黑暗中,就会不期然地反省自身,这样阳间的亲人等于提供了他一次自我教育的机会。反之,如果家庭中的亲人,各怀鬼胎,互相憎恨、不满,把家庭的气氛搅和得十分不协调,地狱的灵感到非常投合,就会安住下来,日久就进入人们的意识中,可缓和一下地狱中所受的痛苦,有地狱灵依附的家庭,由于又增了一个捣乱的灵魂,家庭中不调和的气氛也益加浓厚了。

滞留现象界的地狱灵,不但在生前造下罪孽,如今又搅乱了现象界的人心,他将背负双重的罪恶。于是地狱灵就陷于更难解脱的状况,同时使得现象界继续混乱下去。

两种世界,经常是透过居住其间的人的心灵而互为因果的,此世混乱,彼世也受到骚扰;此世和平,彼世也得享有宁静。所以说,要祭祀祖先,就应该使家庭的气氛开朗,把反省的光芒传送给祖先的灵,又由于家庭的气氛开朗,间接地减少了地狱灵的数目,则地狱灵给予现象界的影响也就相对地减少现象界中的佛国净土(理想国)也就能一步步地实现了。

如果现象界的人终日为了五官六根的烦恼而团团转,即连自身都难保了,还谈什么祭祀祖先,给祖先的灵反省的光芒?既然祭祀祖先时,是希望祖先的灵得到安宁,或使在恶道上的祖先灵得以超升,做子孙的,就应该力使自己过着佛陀所示的正确的生活。

佛弟子们最好能将举行盂兰盆会的这一天,用于为不幸的人们服务,既然平时已因忙于事业或家务,没有足够的时间来贡献己力,不妨就利用这一天来尽尽心意吧!

所以,凡是了解佛法的人,就知道一般人在祭祀祖先时,是多么的不切实际!

\section{舍利弗与目犍连}\label{sec3.17}

另一方面,优波室沙也很严格地下着工夫,为求正道而不懈。

有一天,佛陀把他唤至跟前,说:「优波室沙,你过去所获得的智识,经过实践,似乎已达智慧涌现的时候了。你是否因为心上的暗影已消除,而身心两面都感到无比的喜悦?」

「的确是的,佛陀。我从来没有经验过这样的心情,我的心胸坦荡荡,是佛陀的教义洗净了我的心灵,清除了心中的一切邪念,一想到佛陀,我就非常倾慕,这种心情很奇妙,是不容否认的。佛陀,真谢谢您!」优波室沙说着,眼眶中充满了泪水,他在听佛陀说话时,心里就一直抖颤着,不能自己。

佛陀早已明了优波室沙的心理,很感欣慰的是他已达阿罗汉境地了。

「优波室沙,你本来学的是婆罗门中专重理论的一门学问,但由于你能拓展那狭窄的心胸,依佛法修持而日益改善了,如今你已能轻松地容纳现象界中的一切事相,不必再靠有限的学识,能了解一切道理,这全都是因为你坦率地实践了佛法的缘故。合于正道的心,亦即坦率、广阔而有余裕的心,凡事都有原因与结果,如果每事都注重结果,就算有再丰富的学识,也看不清事理的真相而容易犯错。」

说着,佛陀伸出右手,以右掌心按抚优波室沙的头,一刹那间,他全身震颤不已,一股暖流通遍全身,那是佛陀的灵光所加予他的感受。

接收了佛陀的灵光之后,优波室沙脱口说出前世的古语。古语连绵不绝地说出口,毫无间断,佛陀也回复了过去的意识,与优波室沙亲切地用古语交谈,两人都沉浸在过去世的回忆中。

华严经十地品,即载有这一类的情事,记载了有关灵现象及对过去世的记忆,例如:

「此菩萨念知无量宿命差别,所谓念知一生,念知二生三生四生,乃至十生二十三十,乃至百生、无量百生、无量千生、无量百千生、成劫、坏劫、无量成坏劫。我曾在某处如是名,如是姓,如是种族,如是饮食,如是寿命,如是久住,如是苦乐。我于彼死生于某处,从某处死生于此处,如是形状,如是相貌,如是言音,如是过去无量差别,皆能亿念。此菩萨天眼清净过于人眼……」

转生累积了各世的经验,时而为武士,时而为农夫,又时而为僧侣,如此谋求心灵的成长。

人就是在各种生长过程中,渐渐体会大意识与自身意识的关系,然而有时候会走到岔路上,心灵因而为之黯淡,人稍一不留神,就会走错路,一旦离正道越来越远,就很难再拾回神佛的心。虽然幼儿的心,无拘无束,较接近神佛之心,但在人世间,过于单纯是无法生存的,我们应当一面过自立自主的生活,一面发展神佛之心,不入邪道,这才是合于正法。

「业」由五蕴而来。五蕴指的是色、受、想、行、识。此五者能覆盖住人的本性,故名之为「五蕴」。

「色」指五官能接触到的一切有形物质;「受」是面对现象而承受事物的心的作用;「想」是面对情境而想象事物的心的作用;「行」是面对情境而产生善恶等行为的作用;「识」是面对情境而产生的分别的观念。故说「五蕴」能引人造业,形成业力,左右一个人的命运,人一旦被「业」支配得团团转,就无法体会内在的神佛之性。

优波室沙经过了一番对过去世的回忆,如同邂逅了久别的父母一样,兴奋之极,泪流满面,优波室沙内在的智慧之门因而开启了。诚如佛陀所预言的,优波室沙和拘律陀都已达到阿罗汉的果位。

竹林精舍中的修行者们所穿的僧衣,质地非常粗陋,而且终年不换,由于灰尘与汗垢,已经污秽不堪。这是由于佛陀的习惯如此,所以弟子们也理所当然不以为意,沐浴洗衣,本是僧人自己的事,但是由于生活的重点不在装饰外表,故而僧衣总显得脏而邋遢。

佛陀由于疲劳过度,身体不适,他忽略了自己,一心为了传道,并为解决弟子们的种种问题,此外,他也未注意卫生问题,佛陀认为此时正是他静静观察己身的好时机,于是安然地在房内休养。

佛陀病倒的消息很快传入频婆娑罗王耳里,国王立刻派御医\xpinyin*{耆}婆前去慰问。

耆婆来到佛陀床前,顶完礼问道:「佛陀说的法,能安稳我们的心灵,在我们迷惘的人生大道上呈现光明。国王陛下听说佛陀病倒了,非常焦急,特派我前来代为问候,佛陀的金玉之身,千万得善自珍重哪!」

「谢谢你!劳你老远跑来,真过意不去。可能是吃坏肚子了,我已休息过一阵子了,没有关系的。」佛陀说着,想从床上爬起。

耆婆赶忙趋前制止道:

「请不要起床,好好躺着吧丨」

佛陀的脸色已好转,且精力也已恢复,但是耆婆仍不放心地配好了处方,要佛陀喝下去。数日后,耆婆再度来慰问佛陀,这时佛陀早已康复,正坐在房里。

「佛陀,您实在辛苦,有那么多弟子要烦神,过团体生活,要注意卫生问题,万一有什么传染病,岂不要白白断送人命?我实在不好意思说,佛陀的衣服看起来很脏了,这样对衰弱的身子更有不好的影响,我以医生的立场来说一句话,希望你这儿卫生要多加改善。」

耆婆知道佛陀早已远离执着,过着简约朴实的生活。但是他认为太简约了,就会有马虎的事出现,像环境不够卫生就是。他想从这件事着手,来为佛陀做点事,于是说:「我知道如何保持环境的卫生,请让我试着做一做吧丨」

佛陀点点头,笑着说:「那就难为你了。你身为国王的御医,想要为僧团服务,弟子们将感恩不尽,请你随时指导他们吧!」

耆婆就先和跋提商量工作的步骤,想从厕所先着手。在地面先挖二公尺宽、二公尺深的沟渠,约二十公尺长,然后在其上搭盖遮蓬和围墙。

接着要改善的是僧衣,修行者都不重视衣着,有的人甚至穿着从死尸身上剥下来的衣服,他们那种若无其事的态度,以今天的眼光看来,实不可思议。

佛陀接纳了耆婆的建议,传话下去道:「我们所穿的衣服,虽不求华丽,但应该保持清洁,并经过日光消毒,我们固然不应该在心里恋着华衣美服,但也不可以故意穿着\xpinyin*{褴褛}肮脏的衣服,表示自己在学道。」

佛陀的话传下去之后,立刻有了反应。王舍城中陆续有人来布施僧衣,频婆娑罗王也着人时常送些干净的僧衣。

衣服在当时的印度来说,是很宝贵的。当时并不像今天,有机器大量生产成衣,即连一块布料,都要花人工花时间慢慢织出来。所以有的奴隶,经常是衣不蔽体的,状至狼狈,甚至农民在耕作时,也是袒胸露背的。

印度因为气候炎热,不穿衣服也可以生活,人民大都不注重穿着,可是地位高的人就不一样了,所以从衣着可以看出一个人的地位,修行者们,由于身处于这样的时代背景中,自然也就不太重视衣着,同时,不重衣着也具有摒绝世俗的修行意义,他们更不愿终日的为了衣着来白费心思。

有一次耆婆带来一块迦西出产的名贵料子,那块料子是他替拿兰陀的一位大富豪医病时所得到的馈赠。耆婆心想,这料子实在名贵,送给国王穿用倒还说得过去,若自己穿用,就嫌太奢华了,不合身分,接着他想到佛陀,于是他来到佛陀这里。

「佛陀,请答应我一个请求。」耆婆态度严肃地说着。

佛陀微笑着回答说:「你经常为僧团服务,还没谢谢你呢!你有什么请求,快说吧!」「可能佛陀不会答应,不过我还是要请求。」

「哦?	」

「佛陀是世上最尊贵的人,像您这样的人,衣着竟如此朴素,实在不陪衬,这里有一块名贵的料子,是别人送我的,我穿不合适,心想佛陀穿才最合适了。请您接受我的布施吧!让我也种种福田。」

耆婆说着,立刻把料子拿出来,交到佛陀手中,佛陀高高兴兴地接受了耆婆的供养,而且当场披到身上,耆婆高兴地淌下眼泪。这消息传到王舍城的商贾耳里,从此精舍中的僧衣大大增加了。

就这样,佛陀弟子们在耆婆的好意安排下,经常有干净的僧衣穿。

佛陀所领导的僧团,由于优波室沙和拘律陀的加入,传教的活动更有变化,而僧团的朝气更蓬勃,佛法的流布方式较积极之后,请求入门的人当然更多了。

佛陀将优波室沙和拘律陀唤至跟前,告诉他们说:「近来同姓的人越来越多了,你们可愿意改个名字?」

「佛陀,我一直有这样的念头,如今已像重生了一般,心灵的污垢已除去,也得到了安稳,我想改个名子,未尝不好。」

优波室沙很早就听人家为了辨别而称他「拿兰陀的优波室沙」,觉得这是很麻烦的事,很想改个名字,只因佛陀未曾提及,他也不便开口。

「优波室沙,你的母亲身为摩诃婆罗门的妻子,很懂得慈悲之道,相夫教子,非常贤惠,是王舍城中数一数二的贵妇人,你就依你母亲的名,从今天起叫舍利弗吧!」

「好的。能够和母亲那样慈爱的人起同样的名字,我非常荣幸,我今后不要输给她才好,一定要成为一个优秀的修行者。谢谢您,佛陀。」优波室沙见佛陀如此称许自己的母亲,心中很感骄傲。

拘律陀被佛陀取名为目犍连(或称目连)。

舍利弗自被更名后,有一次利用游化之便,来到拿兰陀看他的母亲,母亲身体健朗,看到儿子,其喜悦之情,自不在话下。

「和从前比起来,你脸色好得多了,修行的生活如何?」

「妈妈,您的气色真好,身体也硬朗,我真高兴。最近佛陀依您的名字给我起了一个法名,叫舍利弗,佛陀他夸赞您老人家呢!」

「优波室沙,你何其有幸,得遇伟大的佛陀,我很感安慰。千万要注意身体的健康,做一个出色的修行者,努力把佛陀的正道传扬出去,特别是传给那些思想守旧的修行人。我听说你已达阿罗汉境地,这真是一个大好的消息,我真以有你这样的儿子为荣,你哥哥目前还在删闍耶那里修行,听他说,他想去找你。」

「哥哥也想离开吗?那太好了,相信哥哥有一天也会明白过来的。佛陀的道理才能帮助我们找到自我,才能济渡众生,这是我的亲身经历。妈妈,原来在过去世中,我跟佛陀就在一起修过道的,知道这个事实后,我和佛陀高兴得拥抱在一起,一同回忆过去那段日子,高兴得流下泪来。妈妈,佛陀真伟大!他那种气度与境界,不去接近他是无法想象的,我现在心境很安和,我现在才知道世上的小小智识,那能和藏在心底的浩翰智慧相比较啊!」

舍利弗抑制不住心中的兴奋,极力想把自己的感受描述给母亲知道,但即使是这样,他还是不能表达他的感受于万一。母亲在一旁静观儿子的一切,见他容光焕发,说不出是那里,但的确与从前大不相同了。

「优波室沙,你真如鱼得水般神气活现的,就像已把千斤重担卸下了一般,我从未看过你这么高兴过,真是太好了!」说着,做母亲的不住端详儿子开朗的笑脸。

突然,她想起了什么,对舍利弗说:「对了,你应当也把佛陀的正法传给你父亲,既然你有了这么了不起的体悟,相信你父亲也会起共鸣的。」

「妈妈,佛陀和其他师父不一样的地方是,他从不用艰深难懂的话来说法,他的话就是小孩子也听得懂,谁都能马上就了解,佛陀说法的特点,就是平易近人,教给我们心灵和行为应有的正确道理。妈妈,您别急,我想您和爸爸都会有机会听到佛法的,并能在每日的生活中领尝永恒的生命及不可思议的缘生的道理。」

舍利弗若不是因为在游化的途中,他真想住下来,以便向父母及兄长介绍佛陀的教义,他想如果全家都能皈依佛门,那该是何等幸福的事啊!

舍利弗回到竹林精舍,修行的伙伴们如今都已改口称他「舍利弗」,这个名字总使他联想起自己的母亲,也给了他某种激励的作用。

他与目犍连二人从此以新的名字、新的面目来游化。「舍利弗」与「目犍连」也因此在经典中被流传下来,成为家喻户晓的人物。





\chapter*{用语诠释}\addcontentsline{toc}{chapter}{\large\CJKfamily{hei}用语诠释}

\textbullet 八正道

\textperiodcentered 正见——对世间实相的正确认知。

\textperiodcentered 正语——本真诚的心说有益于人的话。

\textperiodcentered 正思维——对世间实相做正确的思维。

\textperiodcentered 正念——专心致志地使意念集中于正确的方向。

\textperiodcentered 正业——正当而有益他人的行为。

\textperiodcentered 正命——清净身、口、意三业,顺于正法的生活。

\textperiodcentered 正精进——努力精勤地修学正道。

\textperiodcentered 正定——以正智入于周徧清净之禅定。

\textbullet 三昧禅定——三昧,亦云三摩地,正定之义。大乘义章:「以体寂静,离于邪乱,故曰三昧。」禅为梵语禅那之略。一心审考为禅,息虑凝心谓之定。禅定达三昧时,意识的自己会脱离躯体在意识界神游。三昧亦有阶段,依个人意识和谐程度而达于不同的境界,故意识神游时,自己是否已到达三昧境界的顶点,则很难断定。

\textbullet 不生不灭——人的性灵本是不生不灭的,只因寄居于肉躯中而不得自在。肉体是物质的存在,日久必有坏空的一天,但人们由于被五官六根所蒙蔽,都难以察觉真实不灭的内在,内在不灭的性灵一定要靠过正确的生活才得以显发的,否则只有生生世世轮回于生死之中,故「不生不灭」在于帮助灵魂的永恒性。

\textbullet 五官六根——眼、耳、鼻、舌、身谓之「五官」,再加上「意」,即成为「六根」。六根是诸恶造作的根源,世上之所以有混乱的现象,即由于人们放任六根的享乐,任使欲望孳长而迷失正道的缘故。譬如看到漂亮的东西就想得到手,有好吃的东西就吃得腹胀难忍,只要是夸赞自己的话,就信其真实而将对方视为知己,像这类的情形,就是五官六根的作祟所导致。

\textbullet 五明——

\textperiodcentered 语文学的声明。

\textperiodcentered 工艺学的工巧明。

\textperiodcentered 医药学的医方明。

\textperiodcentered 论理学的因明。

\textperiodcentered 宗教学的内明。

\textbullet 心眼——亦称「第三只眼」。意念澄澈,心地宽阔,心眼自然会开,当然有时畜生灵也会支配人类的意识而令人看到冥界的景象,故心眼是否真的开启,并不能由看到他界事物而定。如果心存愤怒、嫉妒等念而仍能看到他界事物,就要多加注意了,此时大都是畜生灵在作祟,切不可沾沾自喜。

\textbullet 止观——所谓止观,乃停下来察看之意,此词出自天台宗,智顗大师并有摩诃止观和小止观等论者,一直流传至今。止观者,回顾自己的过去而加以剖析,亦即一种反省的工夫,在基督教中,谓「止观」为忏悔和改过。

\textbullet 比丘——梵语,乞士之意。意谓「上从如来乞法以练神,下就俗人乞食以资身,故名乞士。」

\textbullet 四吠陀	梵文「Veda」,是明智的意思。

\textperiodcentered 梨俱吠陀——宗教的赞歌。

\textperiodcentered 沙磨吠陀——祭祀仪式的颂文。

\textperiodcentered 夜柔吠陀——祭祀仪式的歌词。

\textperiodcentered 阿闍婆吠陀——俗世相传的咒术。

\textbullet 如来——佛陀的十个称号之一。如者真如也,乘真如之道从因果之来而成正觉之故,名为如来。大日经疏日:「如诸佛乘如实道来成正觉,今佛亦如是来,故名如来。」如来一词,说法众多,兹略举二说以资参考。

\textbullet 肉体烦恼——烦恼由迷惑产生,由于对肉体的迷惑而产生了金钱、地位、名誉及其他各种执着,而此执着缠缚人心,使人受其支配,失去自由意志,故而时感苦恼。

\textbullet 佛国土——亦即「净土」,是充满幸福安乐的真实境界,也是宗教家们穷生生世世要带领众人达到的境界,追求幸福快乐是人的本能欲求,但人世间由于人类的妄想执着,无法真正达到幸福快乐的境界。金钱、名位、权势等,证明并未能给人任何安宁与幸福。唯有生活于世上的每一人都能灭除心中劣等的欲望,循正确的途径过有节制的生活,才能期于达成世间真正的安和与乐利,也可说自己能如此,则自有一片佛国土在心中。

\textbullet 伽那教——比佛教稍早的教,有灵魂转生之说,认为人如果解脱痛苦的话,就可免掉生死轮迫。此教团以苦行为主,势力相当大。

\textbullet 忍辱——此为菩萨行的六种修行工夫之一。其他五种分别是布施、持戒、精进、禅定及智慧。菩萨修「忍辱」行,在于坚忍心志,不受动摇,以使心性保持静定的状态而有利于做进一步的修行。不知忍辱的人,最易迷失自己,渐离正道而不自知,自然也就无法正确判断事理,更无法臻于悟境了。

\textbullet 我执——认有我身之执念为「我执」,亦日「人执」。俱舍论二十九曰:「由我执力,诸烦恼生,三有轮回,无容解脱。」

\textbullet 沙门——梵语,意即「勤息」,「勤」者勤修戒、定、慧,「息」者息灭贪、瞋、痴。故「沙门」乃致力上两项工夫的修行人也。

\textbullet 沙弥——梵语。男子初出家受十戒者之称,旧译「息恶行慈」。

\textbullet 沙罗门——当时的僧侣,男的称「比丘」,女的称「比丘尼」。这些僧侣们修行的最后阶段是到各地游化,以体悟自我,如此游化着的修行僧被称为「沙罗门」。

\textbullet 彼岸——佛家语。人生有如一个生死苦海,只要渡过了这一个海,就能到达真正快乐而无生死之苦的「彼岸」,故「彼岸」又可称「开悟的境界」。


\textbullet 法喜——闻法或见法而生喜悦之心,或日「法悦」。

\textbullet 法轮——宇宙,是顺着循环之理来轮转的。地球以太阳为中心而在一定的轨道上运转,原子的世界亦复如是,以核为中心,有许多负电子绕于四周,做循环的运作,人的灵魂也不例外,在现世的生命终了时,再回意识界,以待时机再重回现世,由此观之,所有的事物,都以圆环的方式周而复始地运行不已,此之谓「法转」。渡众的教主们,将宇宙的真理流传人世,以点燃人心中的法灯者,谓之「转法轮」。

\textbullet 阿修罗——佛经所说八部众之一。阿修罗,梵语,具权势欲、斗争欲。人而具此特性者,或可谓之人间的阿修罗。

\textbullet 信心——所谓信心,是指相信自己有正确的心的一种信念。此信并非只靠膜拜神明或偶像就可得到。在人的心中都有守护自己和指导自己的守护灵及指导灵,只要保持正当而坦率的心,则指导灵与守护灵自会从旁协助,使这个人过着犯错较少的人生。相反地,若心里起了自我维护及不知足的欲望,自己无法与内在的灵沟通,就要遭遇一连串的不幸了。所以真正有信心,应该是相信自己为善的心,过正当的生活。

\textbullet 信仰——以善心过着正确的生活,才是信仰的真谛所在。现在一般人都把崇拜偶像、诵经等事视为信仰,这是很大的错误,因为人心中都有守护灵和指导灵的存在,人本来就应该相信自己的善性,并自力更生,只有人类能够创造发明,拥有开阔而自由的心灵,动、植物则没有这种能力。造出天堂与地狱的也是人,动、植物也没有这能力。了解了这一点之后,就能明了人类信仰的方向何在,如果只求拜偶像来使自己获救,无异放弃了自己具有的创造能力任人摆布,这样连做一个人都没有资格了。

\textbullet 畜生灵——现世的动物,一如人,也有所谓的「灵」,能影响到人类的动物灵,是蛇、狐狸。蛇和狐狸的生命力强韧,在地球上的生活比人类还久,故对世上的生活及形态,有特殊的心得,蛇很狡猾,狐狸则是动物的本能很强。人的意识,如果倾向动物本能方面,则动物灵易附着其上,蛇和狐狸依附人身时,也能使用人的语言,被附着的人,会引起灵媒现象,会说出神秘的语言,会未卜先知(多半是灾变之类的事),有时很准,然而他们本是动物,故不会道出正法,其说词往往有背正道。有的人自诩是神或教主而因此狂傲自大,滥显神通而驾驭他人,一望而知,此人必已有动物灵附身无疑。

\textbullet 神性——又称「佛性」。做为一个人,应有神佛之子的自觉。一个人不管有没有这种自觉,他的心本具有神性佛性,善良是人本来的面貌,虽然自古以来就有「性恶」、「性善」之说,但是由「谁都无法对自己的心说谎」这一点来看,人是有良知的。纵使一个十恶不赦的大坏蛋,他也有悔不当初的情感表现,可说人的良心经常在与邪恶对抗而守护着我们每一个人,这一事实任谁都无法否认,所以我们应该发展善良的本性,不要把自己陷在痛苦而无以自拔的境地。

\textbullet 般若波罗密多——皆梵语音译而来。「般若」指「最高的智慧」。此智慧由体证真理而来,是最周遍、最彻底的智慧。「波罗密多」,译成中文可有两种意思:一、凡事圆满达成时,印人皆称「波罗密多」。二、任何一件事从开始到完成,中间所经的过程、方法,印人也称「波罗密多」。

由于此词含义广泛,故保留原音。另可解释成:经由最高的智慧以达宁静安和的彼岸。

\textbullet 执着——自我保存的意念,就称为「执着」。一般人在心中经常有的念头,诸如——「这东西是我的」、「我已经做了某某事」等,只要自己好,就顾虑不到别人,如自我嫌弃,自我满足及自我逃避,也都是一种执着。如果执着太强,身心的平衡会崩溃,体弱多病,人际关系失调,甚至也无法专心工作。

\textbullet 婆罗门——以吠陀及五明之学来传世的一种宗教。创自公元前二千数百年,历史悠久。最初婆罗门教极重人心与大自然和谐关系,并以此做为布教的重心,但经过历史的变迁,一切教条都形式化了,即连传教者本身,也只重知识理论的传授,在释迦牟尼的时代,婆罗门已成为势力极庞大的宗教团体,且占据社会上的最高阶层。

\textbullet 循环——这世上的一切事物(包含精神)都在依圆轨运作。地球的自转与公转,造成了画夜与四季的变迁,此外,所有的物质都依质量不灭的原则循环着,人的灵魂也是如此,有了此一自觉后,人们才知道过正确生活的重要性。

\textbullet 智慧——本书所提到的「智慧」,不同于一般人所谓的「世智」,我们亦可称之为「佛智」。「佛智」乃累积过去生中所习的一切而成,并包含了对宇宙至理的觉悟。「佛智」与「世智」具有层次上的区别,「世智」来自现象界的知识,包括以知识为基础的能力,「世智」是现实的、感觉性的、且是相对的。

\textbullet 菩萨——为「菩提萨\xpinyin*{埵}」之简称。意即「觉有情」。菩萨者,既是一已经觉悟了的有情众生,进一步又去使其他的有情众生觉悟,故是一个「自觉觉人」者。

\textbullet 意识——人的意识可分为表面意识、意念带以及潜意识。我们一般人多半都靠表面意识及意念带过生活,意念带像录音机一样,会把一个人每天的意念行为记录下来,潜意识则指的是另一个自己,这一意识若能与表面意识相沟通时,人就能得着完全的解脱。为求解脱而做的修行,在于使表面意识调和并净化意念带,净化是靠反省及正确的行为而导致。

\textbullet 解脱——指一种彻底觉悟的境界。截断了一切的执着,使自己身中的另一个自己和永恒的生命合而为一,并脱离了人世间一切因缘纠结的束缚。一个人在达到此一境界时,不但了解过去、现在和未来的因缘因果,而且对于大宇宙的结构也了如指掌。

\textbullet 过去世——亦即「前世」。佛教讲三世因果,强调人的性灵不灭,不断在生死的轮迫中,前世意味者降生此世前的各生各世。故佛家有言	「欲知前世因,今生受者是,欲知来世果,今生做者是。」

\textbullet 实在界与现象界——实在界,又名「意识界」,「现象界」亦名「物质界」。因「意识界」中的一切不会消失,故称之「实在界」,反之,「现象界」是沧海桑田,时时在变化的世界,并与空气、大地互相循环而有,与「意识界」相较,则「现象界」实不值留恋,没有「意识界」中所具有的安定与永恒性。

\textbullet 摩诃——印度语音译而来,有「伟大」、「巨大」之意。

\textbullet 禅定——禅定约分九个阶段。第一阶段是反省,经反省而除去心中的阴霾,就能进入第二、第三阶段,如果想省略反省的阶段而保无想状态时,很容易为魔所乘。能够和守护灵交谈的第四禅定阶段,其内容也因人而异,切不可以为与灵做过交谈,就是进入了禅定的第四阶段,释迦牟尼佛的禅定已达第九阶段,是普通人难以企及的、与大宇宙合而为一的禅定。

\textbullet 禅定三昧——入三昧禅定之时,另一个自己会脱离肉体而神游实在界。三昧境界有不同的阶段,脱离肉体的另一个自己,因所见的境界不同而三昧的内容亦有不同。实在界又有好多种世界,都与每个人的意识有关,各人依其意识的状态而与最符合自己意识的世界相感应,故虽有人入了三昧禅定,亦很难判断其境界是否已达最高点。

\textbullet 转生轮回——生命、物质都顺循环之法轮回。「转生」乃转入生命形态之谓也。「轮回」指如轮子般能迫转。人的灵魂在这一世脱离躯壳,回到物质转换的世界,依生前的造作而转换,再投生至现世。灵魂是不灭的,依意念而轮回,恶念得恶果,善念得善果。

\iffalse
\textbullet 法轮——宇宙,是顺着循环之理来轮转的,地球以太阳为中心而在一定的轨道上运转;原子的世界亦复如是,以核为中心,有许多负电子绕于四周,做循环的运作。人的灵魂也不例外,在现世的生命终了时,再回意识界,然后再回现世,由此观之,所有的事物,都以圆环的方式周而复始地运行不已,此之谓「法轮」。渡众的教主们,将宇宙的真理流传人世,以点燃人心中的法灯,谓之「转法轮」。

\textbullet 动物灵——现世的动物,一如人,也有所谓的「灵」,能影响到人类的动物灵,是蛇、狐狸。蛇和狐狸的生命力强韧,在地球上的生活比人类还久,故对世上的生活及形态,有特殊的心得,蛇很狡猾,狐狸则是动物的本能很强,人的意识如果倾向动物本能方面,则动物灵易于附着其上。蛇和狐狸依附人身时,也能使用人的语言,被附着的人,会引起灵媒现象,而说出神秘的语言,会未卜先知(多半是灾变之类的事),有时很准,然而他们本是动物,故不会道出正法,其说词往往有背正道。有的人自诩是神或教主而因此狂傲自大,滥显通来驾驭他人,一望而知,此人必已有动物灵附身无疑。

\textbullet 沙门——梵语,意即「勤息」,「勤」者勤修戒、定、慧,「息」者息灭贪、瞋、痴。故「沙门」乃致力上两项工夫的修行人也。

\textbullet 沙弥——梵语。男子初出家受十戒者之称,旧译「息恶行慈」。

\textbullet 四吠陀	梵文「Veda」,是明智的意思。

\textperiodcentered 梨俱吠陀——宗教的赞歌。

\textperiodcentered 沙磨吠陀——祭祀仪式的颂文。

\textperiodcentered 夜柔吠陀——祭祀仪式的歌词。

\textperiodcentered 阿闍婆吠陀——俗世相传的咒术。

\textbullet 五明——①语文学的声明。②工艺学的工巧明。③医药学的医方明。④论理学的因明。⑤宗教学的内明。

\textbullet 智慧——本书所提到的「智慧」,不同于一般人所称的聪明智慧,那只是「世智」,本书所指者亦可称之为「佛智」,「佛智」乃累积过去生中所习的一切而成,并包含了对宇宙至理的觉悟。「佛智」与「世智」具有层次上的区别,「世智」来自现象界的知识,包括以知识为基础的能力,「世智」是现实的,感觉性的,且是相对的。
\fi

\chapter*{后记}\addcontentsline{toc}{chapter}{\large\CJKfamily{hei}后记}

这里有几个问题,其实是不成问题,愿在此处,就个人的所学与了解提出来讨论,望读者诸君不吝指教。

(一)关于称谓

本书对释迦牟尼佛陀的称谓,依三个阶段而有不同。第一个阶段是未出家前,故文中加有「太子」的头衔;第二阶段是出家后,悉达多既已抛去名位,作者即直呼「悉达多」,并示读者以「亲切感」;第三阶段是得道后经梵天所赐称,作者亦随即以「佛陀」尊称之。

(二)对「正精进」的诠释

「正精进」是佛陀所悟得的「八正道」之一。「正精进」在一般佛教书籍中多解释为「努力精勤地修学正道」,惟独在本书中,佛陀还强调了「努力于将亲子、兄弟、朋友、芳邻等角色扮好,做一个人所该做好的事」,并认为「人的人格在人际关系中显现」。我们仔细想一想,什么事情才是最需要人穷毕生精力来努力达成的?「良好的人际关系」无疑是这个问题的答案。再能干的人,没有「人缘」,如孤掌之难鸣,发挥不了作用。

佛陀本是为解决人的问题,才出家求道的,所以他对这一点体会尤为深刻。显然「努力于改善人际关系」的这一个解释,要较前一个解释深刻而具体。而我们更可从作者的笔底看出,日本人对佛教的认识与体会,也较我们国人实际而具体。我们一般人把「佛理」悬得太高而又画地自限,把大好的极切近人生问题的道理束之高阁,这是很不明智的。固然佛理在探讨宇宙及人生的本源时,自有其高妙深邃的地方,但是八万四千法门中,一定有契合自己的法门,能使自己日进又进。

(三)关于「宇宙是神所变现」

文中对佛陀悟道的内容所做的文字帮助,其中第一句就揭\xpinyin*{橥}了「这个宇宙是神所变现的」。这是与佛教一向所持的说法有出入的,故笔者曾就此问题请教过佛教界的大德,今谨综合所论,有以下两个结论:

一、对「神」的认知问题——凡是未能达到「悟境」的人,对于「神」的概念是很模糊的,人因贤愚不同而对神有高低层次的认识与了解。这个「神」字,本身在文字上就包含了多层的意思,如我国人有各种各类的神,乃至土地公也是一种神的存在。在一神教中,此神乃「耶和华」,乃「阿拉」,是万物的创造者。在佛教中,同时有「佛」、「菩萨」和「神」的存在。除了「佛」有无上的真知真觉外(「菩萨」修到第五十一地后,可登佛位),「菩萨」与「神」都有因功力之大小而分出等位的。「神」则有时仅表示一种具有神通的存在。我们可以这样说,「佛」的真知真觉,使他自己体悟到宇宙的本体,简言之,「悟道者」承认宇宙中有一创造万物的本体。此本体,在大乘佛教的「唯识」论派中,名之为「神识」。有言:「万法唯心造,万相唯识变」,所以文中的「宇宙是神所变现」的说法,与佛理丝毫无相悖之处。本来为了渡众之便,佛教中早有「本体不变」「名相各异」的说法,我们似也不必在此一「神」字的「名相」上起执着了。
二、
二、「上帝」与「阿赖耶识」——一神教中的「上帝」虽高居造物主的宝座,但是教徒们也强调「上帝在你的心中」,「上帝与你同在」,有了这个信念,人能勇往直前。再看佛教言及的「神识」(即「阿赖耶识」),是每个人与生俱来的一种不生不灭的「存在」,本是寂然清净的,只因人们妄想执着,种下了许多的「业因」,以致变现出各种不同的人生。如果众生能努力修炼,去除妄想杂念,俾能「转识成智」,最后能证得宇宙的本体,终致与万化冥合。而释迦牟尼佛来到人间,为的就是要我们觉悟这一点。由此观之,我们或可了解,佛教中清净不染的「神识」,不是与一神教中的「上帝」有了共通的存在意义吗?

基于以上两点,译者在原文特于「神」字下加一「识」字,以全其意,也可免去不必要的争论。

(四)三教教主会于一堂

想来这是一切宗教书籍中所未曾有过的场面。

当释迦牟尼悟道后,苦于自己所悟得的道无法传扬出去,原因是众生根基浅显,必不相信他所悟的境界,与其受辱而无功,不如隐翳而全道,以此了脱一己的生死。当此时,有三位梵天出现面前,其一就是后来降生以色列的耶稣,他提醒释迦牟尼曾立过的「普渡众生」的宏愿,并帮助济渡的重要。另一位梵天是后来带领犹太人离开埃及的摩西。这一幕至少帮助了三件事:

一、「佛」为便利渡众,常会因众生不同的根基而示现各种不同的身分来接引众生。

二、宇宙至理的传扬,需要各种不同的手段。如「文证」、「理证」和「现证」的权巧运用(详细帮助请见本书文字),因而有佛、耶、犹太等教在传扬时所使用的不同方式。(详细帮助请见书中文字)

三、「三教教主会于一堂」的景象,给人一个启示,那就是世界各宗教的诞生,皆由于其教主本着济渡众生的心愿而来的。事实上因怀有这种心愿而下生人间的圣者不乏其人,只因其威德力和影响力不若佛陀等,故未能成立宗教团体。同理,在我们的周围,不也常可看到一些古道热肠,具有人溺己溺胸怀之人吗?

(五)凡人的形貌

本书中的释迦牟尼佛在悟道前后,纯以人的姿态出现。作者毫不保留地指出佛陀在出家前,悟道前所显露过的人的一般习性,藉由菩提树下的反省,赤裸裸地剖析出来,摆在读者的面前。诸如傲慢,争胜时贪欲,不满现状等,都是他成佛前的形象,即便在成佛后,在把佛理传扬的当儿,也常因发生许多困扰,而继续在「反省」中获取教训,所以,在本书中,佛陀更平易近人了。也可以说,本书还告诉我们,「悟道」,仍有境界的高低,若要更上一层楼,是必须继续精进、谦虚、忍辱、行善,以及靠不断的反省才能达成的。人,贵在能时常反省自身,这不是很平常的道理吗?谁又能像佛陀一样,做最彻底最不留余地的反省呢?

本书必然还隐伏着一些特异的观点,今限于篇幅,不便多所探论。好在佛教的特色之一,是容许教徒依个人的理解与体悟来阐述经义的。只是本书有许多论点比较反常,唯其如此,本书才有可看的地方。不管如何,我们应有雅量容纳一些异见,而真理,常在异端中显现其光辉。










\end{document}