%!TEX TS-program = xelatex
%! TEX encoding = UTF-8 Unicode

%========================================全文布局
\documentclass[twoside,openany]{book}
\usepackage[screen,paperheight=14.4cm,paperwidth=10.8cm,
left=2mm,right=2mm,top=2mm,bottom=5mm]{geometry}

\usepackage[]{microtype}
\usepackage{graphicx}
\usepackage{amssymb,amsmath}
\usepackage{booktabs}
\usepackage{titletoc}
\usepackage{titlesec}
\usepackage{tikz}
\usepackage{enumerate}
\usepackage{wallpaper}
\usepackage{indentfirst}
%========================================设置字体
\usepackage{ctex}
%\usepackage[CJKnumber]{xeCJK}
\usepackage{xpinyin}
\setCJKmainfont[BoldFont={Adobe Heiti Std R}]{Hiragino Sans GB W3}
\setCJKfamilyfont{kai}{Adobe Kaiti Std R}
\setCJKfamilyfont{hei}{Adobe Heiti Std R}
\setCJKfamilyfont{fsong}{Adobe Fangsong Std R}

\newcommand{\kai}[1]{{\CJKfamily{kai}#1}}
\newcommand{\hei}[1]{{\CJKfamily{hei}#1}}
\newcommand{\fsong}[1]{{\CJKfamily{fsong}#1}}

\renewcommand\contentsname{目~录~}
\renewcommand\listfigurename{图~列~表~}
\renewcommand\listtablename{表~目~录~}
\usepackage{romannum}
%========================================章节样式
\titlecontents{chapter}
[0em]
{}
{\large\CJKfamily{hei}{}}
{}{\dotfill\contentspage}%用点填充
%
\titlecontents{section}
[4em]
{}
{\thecontentslabel\quad}
{}{\titlerule*{.}\contentspage}

\titleformat{\chapter}[display]
	{\CJKfamily{fsong}\large\centering}
	{\titlerule[1pt]%
	 \filleft%
	}
	{-7ex}
	{\Huge
	 \filright}
	[{\titlerule[1pt]}]

%========================================设置目录
\usepackage[setpagesize=false,
            linkcolor=black,
            colorlinks, %注释掉此项则交叉引用为彩色边框(将colorlinks和pdfborder同时注释掉)
            pdfborder=001   %注释掉此项则交叉引用为彩色边框
            ]{hyperref}

\setlength{\parindent}{2em} %首行缩进
\linespread{1.2}              %行距
\setlength{\parskip}{15pt}    %段距

%========================================页眉页脚
\usepackage{fancyhdr}
\pagestyle{fancy}
\fancyhf{}
\fancyfoot{}
\fancyfoot[LE,RO]{\scriptsize\thepage}
\setlength{\footskip}{6pt}
%========================================标题作者
\title{生死的觉醒(下)}
\author{高桥信次\ 著\\ 慰宣\ 译}
\date{}
\newcommand{\mt}[1]{\textbullet \textbf{#1}}
%========================================正文
\begin{document}
\TileSquareWallPaper{1}{TGTamber}%背景图片
\pagenumbering{arabic}
\maketitle
\tableofcontents
%\newpage

\zihao{4}
\noindent
%\chapter*{封面语}\addcontentsline{toc}{chapter}{\large\CJKfamily{hei}封面语}


\setcounter{chapter}{3}
\chapter{第四章\ \xpinyin*{丕葩利}\textperiodcentered \xpinyin*{桠那}出家}\label{ch4}

竹林精舍东北边的婆罗门城市,有个富家子弟丕葩利\textperiodcentered 桠那的年轻人,对所学以婆罗门经典产生了许多疑问。

\newpage
\section{富家子丕葩利\textperiodcentered 桠那}\label{sec4.1}

弟子们由于佛陀的正法与感化而步上正道,他们的心亮着永不熄灭的安泰的法灯。
尤其是舍利弗和目犍连二人,每当听到佛陀的纶音,都会喜不自胜地泪水泉涌不止。

对这两人而言,如果今世不能遇到佛陀,他们就像是迷途的羔羊,无法听到来自灵魂深处的声音,也无法做佛陀的左右手,参与佛陀普渡众生的工作。
自从跟佛陀相遇后,一切的苦恼、迷惘都一扫而空,而得以无止境地享受那有始以来就禀赋了的生命的喜悦。

目犍连的心眼已经打开,他看到佛陀的过去世,以及过去世与现世的渊源,也能领会到佛陀说法的内涵与过去世所说的完全契合,因而深深体悟到转生轮回的奥秘。

一个人能由苦恼转为喜悦,并从迷惑中走出来并建立起坚定不移的信念,这都是受佛陀正法之赐。

灵魂不生不灭,死亡纯属不可能;灵魂生生不息于转生轮回中,肉体的生与死,不过是短暂的幻象。

微妙的心灵被肉体层层包围着,并受着喜怒哀乐的摆布。这四样东西伤害了灵魂,使人们的生活失去协调。只有在了解佛法之后,心量才能日益扩大。一颗心守住八正道,并精进地实践之,终有成道的一天。

舍利弗的智慧和目犍连的神通,皆为佛弟子之冠,这使许多弟子由羡生妒。加之二人精进不懈的求道态度,更感动了先他们而来的同道们。
他们的真挚与坚定,绝非出自偶然,诚如佛陀所说,是跟过去世有关连的,过去的种种实决定了今世的一切,使他们今世有着异于常人的理解力。

佛陀此刻在竹林精舍的林间,沉入冥想。

在达到禅定三昧之时,有一个声音自心田响起:

「不久会有一个修行人前来求道,他会成为你的弟子,他不久可达到阿罗汉的境界,将佛法传扬出去。」

佛陀结束禅定后,一直回味着刚才的经历,但对于他将来的事情,丝毫没有困扰与迷惑。日常生活中的琐事,如教团的管理、传道的方式等等虽不免多所思虑,但他并不为这些琐事所束缚,也不会自寻烦恼。
这一切都因为凡事都维系在「缘生」的道理上,能觉悟此理,就能顺利地掌握一切。

佛陀时常能在心田听到来自大宇宙的真理之声,与其说是听到,不如说这声音自己来自心灵深处。

说得更具体一点,那是从佛陀的潜意识中涌现出来的。

佛陀的潜意识已扩展到宇宙间,且在超越了局限的空间之上,他的生命伴随着各类的形式而存在着。

人的表面意识,若其波动达到合于正道的神的意识状态时,生命里的各种声音就能够出现而被自己听见。

表面意识的波动越细微,心灵深处的卫量则越正确,当此时,可以超越时间与空间,不论何时何地,任何问题都能迎刃而解。

所谓「般若波罗密多」(译注:梵文音译。即指「到达内在于自己的伟大智慧(潜意识)—佛智。」就是指这种能掌握存于意识中的声音的能力。这种能力可以把内在的声音视为一种内在的语言,有时可变成一个念头浮现心际。

有许多人误解「波罗密多」之语是经由肉耳听得或领悟。

用肉耳听得的灵语是和自己潜意识中的灵语绝不同的,那只可以说可能是动物灵或地狱灵等恶灵的私语。为什么呢?因为潜意识源于自己本身,所以可由内心听到。那些与人不同的动物灵、地狱灵是自身以外的东西,故他们的言语都是由人的身外传来。

来自体外的「指示」语是很不利于人的,如果一个人不能实时察觉而听信之,或对此言语产生兴趣,那么这个人的一部份意识就会被控制,毒害而成为双重或三重人格的人,最后就变成一个废人。

当一部份意识被外灵控制后,这个人便会将听到的话误为「波罗密多」之语。

一个人本身生命中的灵绝不会做混淆视听的指示,因为这个人若遭遇不幸,他内在的灵也会同样遭痛苦。

一个人如果感情冲动或者自大狂妄之时,就应该立刻收敛自己,努力反省。

即使没有前述两种现象,但心中隐约有着骄傲的念头时,也可能有着被地狱灵或动物灵依附的情形。

潜伏在心底的恶劣念头经常是不易被人察觉的,这种恶劣念头,即称之为心魔。心魔是当一个人以欲念为出发点来寻求一种超灵的力量时产生的。这时心魔往往成为该人的第二种性格而促使其依之言行。

一个人成为心魔的俘虏,心和身就无法再依自己的意志来活动,命运的轮也就大大地改变了它行驶的方向。试想这是多么危险的事。

佛陀始终怀着感谢的心情,经常自我反省,为的是救渡更多在错误的观念下苦恼挣扎的众生。

佛陀深怕自己的话有任何错误而影响了全人类的和谐。

佛陀再度进入冥想。

四周的树林,也像帮助佛陀冥想一般,寂静无声。

偶尔会有野鸟拍翅的声响,显示出时间的移转,深沉的宁静,给人一种错觉,好似周遭的一切都已完全静止。

佛陀的意识,逐渐溶入大自然中。

不一会儿,淡淡的,柔和的金黄色光芒,笼罩住佛陀的全身。然后再向四方扩散、扩散,最后和大自然融合为一。

在此同时,竹林精舍的东北边,一个婆罗门种的城市,有个名叫丕葩利\textperiodcentered 桠那的年轻人。

他的父亲摩诃康毗拉是个拥有无数佃农的大富农。

桠那是独生子,在双亲呵护备至的照顾下顺利地成长,过着幸福的生活。他的所学以婆罗门经典为主,他兴趣盎然地跟着修行人学习。

随着年龄的增长,他对经典产生了许多疑问。

其一就是有关神的惩罚问题。

他问教导自己的修行者提出这类问题,所得到的答案都是千篇一律。至多不过是照着经典再解释一遍,没有一点使他信服的创见。

婆罗门行者对他提出的问题很感棘手,最后都相继离他而去。当然,他也不会去参加一年数度的婆罗门祭祀。

他的双亲睹此情景,就向神明祈祷,求神开导桠那,使桠那不再整日闷闷不乐,并为桠那的大不敬行为而道歉。

桠那对双亲盲目的信仰态度很感疑惑。一天,在双亲为他祈祷之后,他忍不住问道:

「爸爸是在向神祈求我不幸吗?您们屡次都去祭拜……」

摩诃康毗拉对桠那的话感到十分惊讶。他说:

「你在说什么?就是因为你不相信婆罗门教神,我们才去向神明祷告,希望你不致因此受罚。怎么会去祈求神来给你不幸呢?别再说傻话了,我从你祖父母那儿听过许多谤神遭殃的故事,很多人都因为毁谤神,而遭到粉身碎骨的命运,我不希望你有这样的命运。」

做父亲的一面凝视桠那,一面谆谆地教诲。

桠那回答说:

「是呀,您们去向神祈祷,希望我幸福。可见这个神会惩罚没有做坏事的人。这样我更加不信他了,我绝不相信会随便惩罚人的神。」

康毗拉对此一说词,更为吃惊了。

康毗拉和桠那一样,一直都对自己信奉的神有所怀疑,但是他即使怀疑,要抛弃这个信仰,还是需要勇气的。毕竟信仰并不是一两天形成的,这是代代相承,早已根植于生活中了,如果抛弃它,生活的基石就会崩陷,生存也就失去了依恃。

其实在当时,已有许多人摒弃了信仰。他们组党结派地到处打家劫舍。不信神的人就靠这最便捷的方法来生存。

信仰超越了阶级,能规范人们的行为,所以一个不信神的人多半不会受人信赖。

无神论者通常为社会所不容。

他们无法组织家庭,只是恣意地为所欲为,想要吃的喝的,或想要女人,就成群结队地到城里去掠夺。

城里或乡下的居民们对他们都心怀恐惧。为了自卫,也有不少商人和农民会组织起来对抗他们。但是在当时严密的种姓制度下,一般居民是不能有武装组织的,所以居民们都生活在不安定之中,不知何时会有灾祸临头。

他们必须依靠神,跟神在一起,才能稍感安心。

盲目的信仰就这样深深盘踞在人们的心中。

另一方面,无神论者悲惨横死的下场到处得见。

有人在抢劫途中,不慎从马背上摔下,头胪破裂。

加之他们向来不注重卫生,肆酒纵欲,损伤了身体,因此也很容易被传染病袭击,往往死亡无数。

有人一旦看到这种情景,就传言他们是遭到神的惩罚了。

桠那的父亲倒并不那么迷信,只是他认为即使婆罗门教中有矛盾的教理,也没有更好的信仰可以取代了。

尤其时当战乱的此刻,有一个大地主能保住家产,他确信全由于祖先的庇荫以及平日对神的虔敬信仰。

「桠那,因为我们家历代都信奉这个神,我们才有今天的好日子过,你就是有疑问,也希望你能遵照我们家过去的习惯来信神。抛弃信仰有什么好处呢?你应该想想其他婆罗门种会有什么反应。我是不了解商人和奴隶这两阶级的情形是如何,我只知道我们是道地的婆罗门种。你起码应该顾一点面子。我也不大懂神的事,但守住老规矩总不会错的。你想你今天有这么多奴婢侍候你,不就是老祖宗留下的产业给我们过好日子的吗?你还是快快乐乐地过你的日子吧!」

桠那很了解父亲的心意。

他也很感激父亲在为他的前途着想,在为他祈福。

但是对于现实中地主与佃农间不平等的关系,他仍留有疑惑。

同样是人,为什么会产生这样的不平等。

虽然自家拥有广大的土地,但土地的种植,粮食的生产,都是佃农用下去的劳力。

佃农们终年到头就在工作,不停地工作。

如果是为自己来做这些工作,日子或许好过一点。如果他们能够觉察到生存的目的与喜悦,他们不就会更加重视自己吗?

桠那对于这个问题很感茫然,他见自己高踞佃农之上,过着舒适的生活,有着一种矛盾的情绪。

他放眼观察大自然,只要是有生命的东西,都可以平等地承受神的光芒。

不论是太阳、水、土地等,还有地上长出来的粮食,都不是某个特定的人所能独占的。

但是婆罗门种却自称自己才是神的使者,这非常可笑。

不论信仰的形式如何,这种想法不是很奇怪吗?

他越想让父亲了解自己的想法时,越感觉出问题的矛盾处而无法详作分析。他知道要让父亲了解自己,现在不是时候,自己还没有可资说服的能力。

桠那因此心思一转,露出笑脸说:

「爸爸,您请放心,不管我有什么样的疑问,家里的信仰是代代传下来的,我想其中必定有几分道理。请您们不要再为我去拜神了,我会照您的意思去做的。」父亲闻言,才将瑟瑟不安的心放下来。

后来桠那为了心中的疑窦,曾去拜访大婆罗门和大仙人。

但是其结论都是一样。

因为「传统」一词是婆罗门教唯一可资依据的。

桠那即认定没有人能解答他的疑虑,他就想尝试禅定的方法。

修行场上,到处有人操着肉体的苦行。在树林内,也有许多修行者正藉由冥想来洞察自己。

桠那不知如何正确地冥想,他只有试着去体会其中的奥妙。

他像别人一样地坐在林子中。

由于腰部和足部逐渐麻痹,他尝到了坐禅的痛苦。

后来他向附近的修行者求教,再加上自己细心的思索,终于学会了如何持续较长时间的冥想。

随着冥想的深入,他觉得自己似乎可以看见什么,但再继续冥想时,又什么都看不见了。

\section{桠那成婚}\label{sec4.2}

桠那的父亲虽然愿意相信桠那的承诺,但是发现桠那的行径越来越怪异,算算他已二十岁,认为成家可以把他的心安下来,于是着手计划他的婚事。

有几位很适当的人选,但是桠那都回说自己太年轻而拒绝了父母亲的提议。他知道父母亲的用意何在,有好几次都有着妥协的念头,但又深深感到不安。他索性远离家乡,到邻国的毗舍离城,或朝更远的帕拉那西都城进发。他要去访各处的修行道场。

虽然仍是一无所获,但他在修行场上听到一件传闻。大家传说不久的将来,会有一真正得道的人出现,这个人是具有三十二圆满相的佛陀,是真正的救世主。

桠那听后,就立志要做这个人的弟子,以解脱人生的苦恼。

远行的目的虽然没有达成,返抵家门后,他为了安慰父母,不断努力地工作着。他非常敬爱自己的双亲,同时对奴仆也很和善。

他加入佃农的行列,跟他们一同卖力地工作,因此赢得了佃农们的拥戴,大家都称他「桠那大人」。

做父母的见他如此辛勤工作,反而更加不安了。

他们担心桠那不肯成家,意味着将离他们而去。

桠那的朋友们,大多已经结婚生子,过着安定的生活,他们不明白桠那为何要拒绝成家呢。

「年纪轻」,「还不到时候」等推词,都不能算是理由。那么到底是怎么回事呢?做双亲的禁不住表现出不安的容色。但是这种事又是勉强不来的。

桠那在二十三岁那一年,去探访邻村一个知己裘达尼耶。

「好久不见了,裘达尼耶。……今天我来是想麻烦您父亲为我塑一个美女像,令尊在吗?」

「桠那,美女像虽然好,但是不及一个太太来得实际。怎么说,现在就剩下你这个单身汉了,不要想太多了。你看我已经有两个孩子了,工作也更起劲了,小孩子实在很可爱。……」

「是这样吗?我今天来此的目的,就是想为我心目中的女人塑一个像。」「为什么需要塑个像呢?你真是个怪人。」

「我不过是想塑个像,好让我父母为我物色对象时有个依据。只要他们一发现有这样一个女人时,我一定结婚。」

「哈哈,请稍等一下。」

裘达尼耶进入一个房间。

房间内传来一阵谈笑声,接着一位大约五十多岁,面貌慈蔼的老人笑着走出来。裘达尼耶也笑着向桠那点头示意。老人一见桠那,就笑说:

「久违了,桠那先生,令尊好吗?您的要求可不简单,这个像不容易塑的,你心目中的美人容貌与风度是怎样的?」

老人一面说,一面注视着桠那的脸。

桠那是很认真的,对他而言,这个美女是他苦心思索的结晶。

提到婆罗门迦叶这家人,附近没有人不知道。

裘达尼耶的父亲也非常敬重康毗拉。

康毗拉是个慈善家,经常布施粮食、衣物和药草给孤苦贫病之人。

桠那见老人提起自己的父亲,恭敬地回答说:

「托您的福,家父身体很好。他希望我早日成婚,但一直没有找到合适的人选,所以我想塑一个像,把我理想的伴侣塑出来,给他们一个参考。我理想中的女人,要像天界的仙女一样,容貌美丽,穿着高级丝绸。塑像的费用,全凭您的意思,就此拜托您了。」

一提到仙女,就使人联想到遨游太虚,眉清目秀,气质超凡的那种美女。她们具有尘世间女人所望尘莫及的美。她们赞美佛,并把佛的心巧妙地表现在音乐中,又将天上的喜悦和安祥散播到人间。

康毗拉不知道桠那喜欢哪一类型的女人,然身为艺术家的老裘达尼耶却能洞悉桠那的心愿。

「我知道了,不过要多给我点时间,等我把容貌姿态完成到某种程度时,我一定会跟你连络,尽管放心好了。」

裘达尼耶笑着说:

「桠那,太好了,爸爸既答应了,我就可以安心了。」

「我也放心了,谢谢您,伯父。」

裘达尼耶继续说:

「你难得来一趟,今晚就住在这里,我们好好聊一聊。……」

「谢谢您,我也想留下来,不过实在是有事,不得不去了,下次再找机会好好聊吧!」

裘达尼耶和太太一同送桠那出来。

「桠那,等人像做好后我再跟你连络,路上要小心。「希望你不要跟别人提起我塑人像的事,到时候直接跟我连络好了。再见!」三人挥手道别。

此时太阳仍高高悬在空中。

桠那想花四、五天来做自我反省。



他想学山中的修行者那样地独自修行。

这样想着,他走进森林中。

当他选定好座位,便开始搜集柴薪,准备迎接夜晚的来临。

这个林子很像他过去访过的毗舍离郊外道场,因此不由得兴起一股怀念之情。此时太阳已完全没入西山。

起伏有致的山丘,漆黑一片,树林顶天矗立,就像怪物般慑人心魂。

桠那不断使柴火燃烧着。

他整顿好恰当的坐姿,以便进入冥想。

杂念出人意表地纷纷涌现出来,心湖似起漩涡般,久久不能平静。

原来桠那不断在心中嘟哝着有关塑像的事情。

「即使有了这么一尊美人像,我想也不见得会找到一模一样的美人。再说结婚之后免不了要生儿育女,这样会妨碍我出家。嗯!还是不结婚的好。」

「这样虽然很对不起父母,但是他们只要看到美人像,就会了解我了。」

「裘达尼耶不了解我,所以会觉得我很奇怪。大概是因为没有人会像我一样,为物色新娘还来塑一个理想的人像吧!但是又有什么其他的办法来让父母了解我的想法呢?相信这样一来,他们一定会死心了吧!」

桠那由于一个接一个的杂念涌上心头而停止了冥想,并睁开双眼。
薪火仍然烧着,火焰不断向上冒着。
由于火光反射到脸上,使得桠那看来很像一个怪物。但桠那此时的心已完全平静下来。

星星在夜空中闪烁着,桠那燃起一股热望,想纵身飞向星光深处。
如此凝望星星的时候,他的心深深为大自然的神秘所吸引,种种杂念也倏尔消失得无影无踪了。
然后他将视线移向薪火,此时薪火散出噼啪的声响,这声响似在努力把他唤回现实。

燃烧着的火焰,的确是活生生地存在着,它是生的象征。枯木与干枝叶都不言不语的,但是只要将它们点燃,它们就能把火焰喷向高空,夹着红、黄、蓝等色彩,炫耀着自己的存在。同时,它们还向四面八方放出光亮。

人之所以有体温,也是由于人体内的某些元素在燃烧的结果。

没有体温,人就不能成其为人,只能算是一个没有生命的空壳。

生命,也就是火。

生存的人,由于生命之火而成长。

火是一种现实。火是有生物的明证。

但是火焰本身有生命吗?

隐藏于火焰中的那样神秘的东西,制造出火焰,赋予了生命。那是什么东西呢?桠那凝视火焰,整个人重回现实中。

然而他能在衣食无虞缺乏的富庶的现实中洞察自己吗?

一个人真的能在舒适享乐的环境中认清自己吗?只怕在不知不觉间已越来越接近地狱之门了。

他到处探访明师,但是一无所获。他想,藉由严厉的苦修来祭祀婆罗门神,是否已远离了正道?

那么真正的开悟之道又在那里呢?

无数的念头就这样在桠那的心头消长升沉。

狗的吠啸声,远从树林那头传来。

夜禽高亢的鸣叫声,阵阵传入桠那的耳里。

森林的夜,是如此森萧恐怖,使人打心底里升起莫名的孤独感。

天色在桠那的思绪中渐渐亮起来。

一夜的苦思,是使人极感疲劳的,但奇怪的是,桠那一点也不觉得累。

他此刻的身心,就像东升的旭日和晨间清爽的空气一样。

桠那在森林中度过了三天,虽然在冥想中并没有悟到什么道理,但他仍怀着一颗愉悦的心返回家中。

因为他早己告诉家人、佃农及仆役们他可能的归期,故大家都正以欢欣的心情迎接他的来归。

一个老仆人一看到桠那,就赶忙超前报告说:

「少爷回来了,见到您神采奕奕的样子,真叫人惊喜。您不在的时候,我已把工作都做完了。」

桠那曾嘱咐下面的人在芒果园筑一道围墙,不然常会有人肆无忌惮地潜入园中偷食芒果。

不告而取是很不对的,这种行为不但藐视了神圣的劳力,甚且造下己身的堕落之罪。

为了防范这类犯罪的行为,并减除他人制造罪恶的机会,桠那认为有筑围墙的必要,因而令人克期把围墙筑起来。

几十天之后,裘达尼耶因美人像的事情来拜访桠那。他来告诉桠那人像即将完成的消息。

桠那立刻策马来到裘达尼耶的家。

人像雕得非常漂亮,同时身上披着好看的衣服。

「怎么样?桠那先生,你还满意吧?」老裘达耶尼先生语调中充满了自信。

裘达耶尼在一旁打趣道:

「这真是一个绝代佳人,我看要找这样一个女人,会把你父母给难倒了。」桠那注视着木雕像,很满意地点点头说:

「做的真好,就跟我心目中的女子一模一样,真是谢谢了!」

「这还不算完成了,等脸部和手脚化好妆后,就更好看了。大约还要两三天,等完全做好后,我会叫小儿为你送去。」

木像高约五十公分,脸部的轮廓丰盈,面型修长,一双凤眼,有着挺直的鼻梁,嘴角略带笑意。

长发垂肩。

身上穿着特制的喀西薄绸,如果面部再薄施脂粉的话,真不知有多美。

「和我想象中的女子完全一样,太好了,谢谢您啊!」

裘达尼耶的父亲很欣慰地望着桠那,并高兴地点头示意。

几天后,桠那将人像带到双亲面前。

父母亲先是不了解他此举的用意何在,等听完桠那的解释,不禁大吃一惊。桠那认真地说:

「妈妈,您的媳妇一定要能跟这人像长得一模一样的才好,请您们就在同种姓的家族中为我物色一位吧!」

两位长上听得目瞪口呆。

虽然两位老人家答应了他的要求,但心想天底下怎会有这样的绝代佳人,他们实在不明了自己的孩子在想什么。

木雕像默默注视着两位老人家,皮肤晶莹剔透,嘴角微跷,一副欲语还羞的模样。

桠那见到双亲的疑惑表情,心中有些不忍,两眼低垂着。

亲子三人隔着木像,默默无语。

做母亲的终于打破僵局,对桠那说:

「就依你的意思,我们一定尽力找到像这人像一样温柔美丽的女子。」

桠那想,如果家里有这样一位女子,妈妈也会很高兴的。天底广阔,不怕找不到这样的人,就是因为有,这木像才能雕得出来,如果说世上没有的东西,人们是无法想象得到的。

做父亲的交抱着双臂,认为桠那的母亲在说空话。

他无论如何没有想到桠那打算出家。

桠那的母亲则为了顺儿子的心意,不时把木像拿出来展示给在家里出入的婆罗门修行僧,希望他们代为寻找这样一个女子,同时许下了重金酬谢的诺言。

全家人都为着寻找桠那心目中的妻子而忙碌着。

一日,一位名\xpinyin*{竦}陀利的修行僧来看那的母亲。

因为他知道某处有—个和木像所雕完全一样的女子。

他愿意带桠那的母亲去看。

桠那的母亲欢喜雀跃。

桠那的父母因而和竦陀利启程前往邻国的拿烂腊城。

城内有一位大商贾名麻夏的女儿巴漏\xpinyin*{喇}就是他们此行探访的对象。

两人在麻夏家看到她。

诚如竦陀利所说,不论是容貌、姿态,都像是木雕像骤然间有了生命一样。

「麻夏先生,我是邻国的康毗拉,此番前来,是想娶您家的小姐为媳妇。这真是不好意思,因为您家小姐正是我儿心目中女子,他声言非这样的女子是终身不娶的了。我连日来四处打听,如今终于见到您家的小姐,希望您能成全我们父子的心愿,我改日必带小儿前来拜望您。」

康毗拉客气地恳求着。

「您是迦叶家的康毗拉先生吗?从那么远的地方赶来寒舍,真是失迎。我很早就听说您是慈悲的大善人,对令公子的事我也略知一二。我对这件事绝无异议,只是还得征求小女的意见,一时也无法答覆您。希望您能给我一点时间。」

麻夏夫妇对于康毗拉的造访,认为是天赐良缘,感到满心的喜悦,心想无论如何得好好劝导女儿,好结成这门姻缘。

麻夏夫妇对女儿提到桠那的事,并将康毗拉夫妇随身携来的木雕像拿给她看。巴漏喇看到木雕像,不禁大吃一惊。她以为她的替身站在面前。

世上真有如此不可思议的事情。自己不但没有见过叫桠那的人,也不曾跟他说过话。再说木雕像是雕刻师所制,也不是出自他本人之手。

麻夏夫妇殷切地对女儿说:

「康毗拉夫妇的名声你是知道的,他的儿子希望娶你这样的人为妻。真奇怪,这个人像竟会这么像你,这是求之不得的事,如果你答应这件婚事,他们马上就会带桠那来我们这里、你不用马上做答覆,考虑考虑也好。」

巴漏喇见双亲对这件事如此热衷,并声称是一件求之不得的事,也有些心动。她曾有过终生不嫁的念头,甚至想到情况许可,她要出家以济渡可怜的众生,她为什么会有这种意图呢?这事发生在两年前她与父亲到\xpinyin*{憍桑毗}城去旅行。在憍桑毗城,有个痳疯病人聚集的地方。

父亲不经意地带着巴漏喇从该处经过。

巴漏喇看到痳疯病患的可怜相,心头震惊不已。

病患们在深谷内有限的平地上,盖着简陋的小屋以遮蔽雨露,自己耕作或藉善心人的布施来渡日。

那些或缺鼻子,或手指溃烂,或头发脱落得男女不分的人们,对明天丝毫不存希望,只是垂头丧气地苟延残喘。

她因为站在半山腰往下看,所以并没有看清楚,是后来从父亲那儿知道下面是痳疯病人的聚集地。巴漏喇至此了解到这世上原来竟还有活地狱的存在。她整颗心都为之冻结起来了。

同时在旅途中,她还看到一些因天灾人祸而失去父母的穷孩子们,那些孩子状至可怜地沿门乞讨。她因此更下定了决心要帮助这些可怜的人们。

未出外旅行前,巴漏喇尽情享受着家庭的温暖,只纯真地企盼将来有个美满的归宿,组织一个幸福的家庭。

但自此以后,她梦醒了,理想也随之改变。

她不再认为人生只有快乐,而开始对人生产生许多疑问。

从父母亲那儿得知桠那来提亲的事,她本想一口回绝,但是见到那尊与自己一模一样的雕像后,好奇地想见见桠那。

麻夏夫妇见巴漏喇也颇有意,故而放下了瑟瑟不安的心。他们立即将这消息告知康毗拉夫妇。

康毗拉夫妇高兴万分。

两人一回到家,就将这好消息告诉桠那。

「桠那,你理想中的女子已经找到了,你最好亲自去看看,真是又漂亮,又温柔。」

桠那闻言,不禁感到黯然。

一件看似不可能的事,竟在短时间内实现了。

世界上真有这样完美的女子吗?

他心想:

「我并不是真有什么理想中的女子,造这个人像,只是想让你们死心,我总不能太让您们伤心。要找像这尊雕像的女人是很不容易的啊。这究竟是怎么一回事呢?」

事已至此,桠那无法再逃避了。总之,父母亲是不了解他的心意的。

他依父母的意思去见巴漏喇。

他以木雕像做为订婚的信物,正式送给巴漏喇。

就巴漏喇而言,能嫁一个一心一意只认自己是理想人物的丈夫,也很教她安慰的。

于是桠那与巴漏喇在不可思议的缘份牵引下结为夫妻。

此时桠那二十三岁,巴漏喇十六岁。

\section{精神夫妻}\label{sec4.3}

桠那与巴漏喇在亲长、族人、朋友以及无数佃农、仆役的祝福声中,举行了盛大的结婚仪式。

酒宴持续了好几天,人人欢欣鼓舞地庆贺着。二人开始踏上人生的新旅程。

结婚当夜,桠那将自己的计划告诉巴漏喇。他的神圣计划是出家,而出家的最终目的是领悟人生至理,然后拯救众生。他表明今天之所以走结婚的路,完全是为了使父母安心。只要父母能了解他的心意,即使是翌晨,他也会毅然决然去出家的。起初桠那实在不知如何开口说出自己的想法。

对女人来说,结婚生子是她们的梦,甜美的家庭更是她们生命的寄托处。

在新婚之夜就提到出家的事,或许会被对方视为一种男性自尊自大的表现,但桠那实在是不得已。

既然自己早有出家的打算,则这次的婚姻不但伤害了双亲,连带伤害到巴漏喇。举行仪式是很不应该的,这件事应在自己的家庭中设法解决,不应该牵连到巴漏喇。

结婚、出家是两件背道而驰的事,如今既结了婚,又想到出家,不是显现出自己的矛盾吗?

他诚诚恳恳地详细道出自己的心意,希望巴漏喇能谅解自己。

当他说话的当儿,一直不敢正视巴漏喇。

他微低着头,努力将自己的想法解说详尽。讲完最后一句时,他吐着气战战兢兢地瞥巴漏喇一眼。他看到巴漏喇含着泪水的双眼。

他以为巴漏喇为他的想法感到悲伤。

但是巴漏喇静静地握着桠那的双手,平静地说:

「桠那,你说得很有道理,事实上我也早有出家的念头。当初我和父亲出外旅行时,看到了许多不幸的人,才知道这人生残酷的一面。从那时起,我就一直希望自己能有机会帮助他们这些受苦受难的人。你来提亲时,我确实非常矛盾,害怕自己因有了家累而无法全心全意照顾那些人。但是从另一个角度来看,我是个女人,对婚姻也有过美好的憧憬,因此暗想机缘成熟后再出家也不迟。……尤其看到你们送来的木雕像,我深深觉得你必是一个可信赖的人,将来如果我想出家,相信你也会谅解的。真想不到你也有出家的念头,我这才敢把我的想法也告诉你,你该不会觉得我野心很大吧!」

桠那听完上面一番话,欢愉地注视着巴漏喇,他深深感谢上苍赐予他如此完美且志同道合的女子做伴侣。

一对以精神的爱来维系的夫妇在这一夜诞生了。

他们不行一般的夫妻之道,但感情却融洽无间。

人人都在期盼他二人的爱情结晶,桠那的母亲尤其心焦。

这样过了四年。

做父母的对巴漏喇没有生育一事很感失望,同时开始疑惑,为什么他们会没有孩子。

父母亲终日向婆罗门神祝祷,祈祷麟儿早日降临。

日子在不知不觉间逝去。康毗拉含着未能抱到孙子的遗憾而骤然去世。

桠那的母亲伤心欲绝。全家全族的人都笼罩在悲哀中。

眼看着迦叶家族将要在桠那这一代结束了。

母亲偶尔会委婉地向巴漏喇探听详情。

巴漏喇则每在辞穷之际就转移话题,以免触及母亲的伤心处。无论如何,她与桠那之间的秘密是不能让母亲知道的。

她只好继续这样瞒下去。

巴漏喇觉得进退两难,好几次都想说出事实的真相。

迦叶家族由于失去了康毗拉而开始走向衰势。更由于这家族一直没有新生的婴儿来延续,桠那深知他母亲悲痛的情怀。他真不知该如何是好。一向勤劳振奋的桠那,此时开始怠慢下来。

他耽于沉思的日子越来越多了。

出家修道的意念也越来越强。

他终于决定将计划中的第一阶段付诸实行。

他和妻子商量过之后,开始将一大块耕地一一划分清楚,以便日后分赠给佃农们耕作。

这整个计划在母亲健在时,当然是要保守紧密的。

在取得妻子的谅解后,桠那于某日召集了所有的仆役和佃农,告诉他们划分土地的计划。

「我今天找大家来,目的在表达我对你们的谢意,你们为迦叶家族所做的种种贡献,我非常感念。希望你们能继续地帮助我,就像家父在世时一样。

各位佃户,我现在决定将土地明白地划分一下,各位对于分配以外的田地可以不负任何责任,只要你能在自己的区域内勤奋耕作,你必有很大的收获。

至于芒果园,也将有人平均分担其中的工作,希望大家认清各自的职责。

负责将农产品运送到城里的人,也请经常和商人取得连系,重新订定一项更有效率的运销计划。

我希望大家能在自己的工作岗位上负起责任,同心协力地把工作做好。」事情宣布过后,土地很快地规划好,桠那并从旁指导他们如何独立生产。佃农们的工作效率因而提高,收获也比往日丰盛。

此时桠那三十三岁。



神圣的时刻不期来临。

受人爱戴的母亲在平静中谢世。

一直没有盼到孙子,她非常遗憾。

桠那对母亲的死,悲恸不已。

母亲在世时,他曾想过坦白地把自己的情形告知母亲,以求母亲的谅解,但是他深恐母亲无法接受这事实而更增忧愁与悲伤。

葬礼结束后,桠那庆幸自己没有透露事实的真相,他认为还是不透露的好。巴漏喇一直都是持着诚挚的态度来孝敬老人家,并时时设法安慰她。

她凡事百依百顺,如影随形地终日陪侍在老人家身侧。

母子二人均为巴漏喇勤恳的态度所感动。老人家临终时对巴漏喇说:

「虽然我很遗憾,没有看到自己的孙子,但是我对你无微不至的照顾非常感谢,我是这么任性,许多地方多亏你了,桠那也多赖你关照了。……」

老人家停止呼吸时那一刹那,巴漏喇忍不住放声大哭。

她为自己没有尽到媳妇的本份,没能带给婆母真正的快乐而深感歉疚。

心性纯良的桠那也不忍再见这一场面,而奔进房内号啕恸哭。

办完母丧的第四十九天,桠那和巴浦喇双双剃掉头发,穿上僧衣,准备出家。佃农们被召集在庭院里,看到二人的装扮,不禁骚动起来。

桠那和巴漏喇并肩立于大众面前。

桠那一面以手示意大家安静,一面用缓和的语调说:

「各位佃户请安静。……我们夫妻自落发起,算是出家人了。记得六年前,我为你们划分好了田界,大家也都已负起了耕种的责任,逐渐知道如何自立。今后希望你们继续珍视自己分得的土地,勤奋地耕作。我将迦叶家族的财产平均分给各位,你们各人工作范围内的土地已经算你们所有的了。从今以后。我夫妇二人不再过问你们的工作与收入,我们要以修行渡此余生。请大家自己保重了。……」佃农间再度响起嘈嚷之声,接着发出一阵阵的喊叫声。

「桠那先生……」

有的人甚至痛哭失声。庭院内顿时一片混乱。

一位资深的老佃农,一面擦着眼泪,一面恳求桠那:

「大爷,请不要离弃我们,我们绝不会给您增添麻烦。您要知道,这是我们大家的意思呀!……」

「我了解你们的心情。但是六年前你们已经知道怎样独立耕作了,我就是不在你们身边,大家只要彼此合作,还是可以和往常一样把事情处理好的,说不定会比以前更好。希望大家不要再难过了,打起精神来,好好地护守自己的土地。

优陀,你一直都跟随着我,应该了解我的想法。请你就做大家的参谋。各位如果有不懂的地方,可以和优陀商量。希望大家好好地干。」

优陀听完这一番话,更是涕泗纵横,泣不成声,他断断续续地说:

「大爷…就让我跟您一起去吧!我可以好好侍候您们,什么苦我都愿意吃,求求您……。」

「优陀,你还不明白吗?你留在这里可以做大家的参谋,这是你的职责,除你之外,你还希望谁来担当这重任?

各位,请多多体谅我的心情。」

桠那很吃力地说到这里,他再也说不出话来。

身旁的巴漏喇,也正在和仆役们殷殷话别。

桠那见此情景,悲伤莫名,只楞楞地站着。

如此几番拖延,两人终于离开大众,走出家门。

渐行渐远,桠那顿时觉得以前的种种就像是一个遥远的记忆。

他细细思索起来,迦叶家之所以能世代繁衍,主要是建立在佃农和仆役们的牺牲上。

耕地因佃农的辛勤耕作而显其价值,那么将他们血汗的结晶还给他们又有什么好奇怪的呢?

他们迦叶家的祖先包括他父亲在内,在离开这片土地时,有谁带去了寸土片瓦的?

这些上千的佃农们,不应做更多的牺牲。

由于土地的归属,将唤起他们对工作的热忱,以及对未来的希望。

桠那自忖自己所做的决定是正确的,他相信历代的祖宗们对他这样的处置当不致不满。

母亲离世只有四十九天,她对我这种处置,或许会有不满吧!

但是我们早晚要离开这人世。

穿着僧衣的桠那,像放下重担似的,心情为之一爽。离故乡愈远,他的步伐越轻松。

为什么桠那要等母丧四十九天后,才采取这种措施呢?

自古就有人传说,人在咽下最后一口气的四十九天内,灵魂不会离开居所。当时印度也流传这种说法。

这是婆罗门教传出来的说法。

对于四十九天的说法,今昔略有差异。

大致来说,死者的灵魂无论对现世间有多大的执着,一过了二十一天就必须回到意识界。

然后在另二十八天内,依各人灵魂的状态,再分别定居于所谓的天国或地狱。这段期间,如果现世间的亲属做出使死者灵魂不得安宁的事情,则灵魂虽一度离开阳世,也会再受到阳世中他所执着的事引导,而成为「自(地)缚灵」,对阳世的人造成种种影响。

因此,遇有丧事时,一切与死者有关的各种处置,都是过了四十九天以后再进行。

譬如说,死者才断气不久,家中的亲属间就为了争夺财产而闹出纷争,则死者的灵魂会因舍不得自己的财产而拚命设法阻拦。这往往就是许多意外事件和灾难发生的根源。

又有些灵魂,即使过了四十九天,也仍会在阳世间变成「自缚灵」,但这种情形非常少。

近来由于物欲横流,很多人不相信意识界的存在,因此除了自杀或因痛苦而死的人之外,自然又产生许多对阳世执着的自缚灵,经常地骚扰人类。

所以说,这人世间可说是越来越狂乱了。

桠那今后将居无定所,他的家就是到处可提供他居处的大自然。

大自然没有界限,不会使遨翔于其间的物类发生任何争执。像这样和平安详的所在就是他要住的居处与庭院。

不受任何人的束缚,既没有可被夺取的东西,也没有想要舍弃的东西。

身心日趋轻松自在,无形中了解到过去令自己苦恼的物质等,实际上是空虚不实的。

就是本身的肉体也不能为自己所有,早晚得留在世界,回归大自然。

领悟了这层道理之后,人就不会再为外在的,不能永久存在的东西所迷惑,因而能从欲望的痛苦中超拔出来。

桠那反覆在心中思量着,他确信自己的行为是正确的。

另一方面,一直和桠那在一起的巴漏喇,此时已二十六岁,朝气蓬勃的。

她觉得和桠那结婚是一件真正幸福的事,同时她由衷地敬佩桠那。

直至此刻,她始终守护着、照顾着桠那。她感觉桠那是她心灵的伴侣,是个得人信赖,具有高度智慧和勇气的大丈夫。

十年来,两人虽共处于一个屋顶下,却从未有过肌肤之亲,完全过着兄妹般的生活。

每天过着超越情欲的生活,她回视自己,想自己总不致是精神或肉体上有什么缺陷吧!

\section{夫妇一道出家}\label{sec4.4}

巴漏喇的家世不如桠那的显赫,不过她的父亲倒有好几个太太。

巴漏喇的母亲是麻夏的第一夫人。其他有第二、第三、第四夫人负责侍候麻夏。此外还有其他的夫人。

一个有生活能力的男子,在当时的社会是可以拥有许多妻子的。甚至于妻妾越多,越显示该男子的能力。

在这种情况下,女人虽然只能算是一个附属品,什么事都不能做主,但为了获取安定的生活,或满足某种虚荣心,多数女人也都甘于接受这个事实,尤其当她身为某一豪族的妾时,还会有着无上光荣的心情。

一夫多妻的制度下,隐伏着男尊女卑的观念,只把女人当成男人的工具。因为女人普遍对男人强壮的臂力存着憧憬,认为力量就是一切,力量代表了公理。这种弱肉强食的动物界生存原则也一样存在于人类的社会中。

力量代表了公理,这种观念直到今天仍滞留在人们的心底,并显现于各方面。不论是政治、经济、教育,就是国与国间的外交,也显露着弱国无外交的趋势。虽然人们高喊伸张正义,但最后王牌仍决定在一个「力」字上。

虽说扩充军备无补于政况,但是任何以美、苏为核心的小国就是付出莫大的牺牲,也要倾力于这方面。再说经济活动也是以竞争为原则的。教育设施上更是以才能教育,天才教育等的培育科技人才的教育为主干。

从臂力到头脑……,古代和现代在力量的利用上,只不过是形式的更替罢了。

女人对男人的世界有着无限的向往,她们愿意时刻追随在左右。

虽然今天一夫一妻的制度保障了女人的权益,但是在彼此的心理上只要稍存上述那样的念头,男女注定就无法平等了。

想要摒除这种男尊女卑的观念,唯有让这观念在人的六根(译注:即指「眼、耳、鼻、舌、身、意」六者。)上消失。但一般来说,首先应更正力量即公理的错误观念。

从「力量即公理」所引发出来的烦恼,虽根植于对权力、地位、名誉等向往的自我保存意识,但事实上,己身的孤独感以及与人相处时的敌对感、竞争心更强化了这种烦恼。

烦恼带给人空虚、无助的感觉,争先恐后争取到的名望与地位,徒然加重了肩头的负担,最后使心神超越于纷扰中。

心灵若想获得安逸,唯有袪除自己与他人之间的藩篱,创造出一个互助互爱的和谐社会,这样的社会,正是天堂的写照。

此外,女性若想提升自己的性灵,必须能扫除自心对肉体的爱恋。究其原因,

女性之所以将男性视为可依赖的对象,即在于女性对本身肉体有所偏爱所致。

女子在养育子女时,常将子女视为自己的化身,尽心养育他们,无非是一种对己身偏爱的延伸罢了。

男女的数量,无论在现实界或意识界,都是一比一相等的,绝没有男少女多的现象。

如今现实界中的男女比例之所以稍有不平卫的现象,是由于战争等因素造成。但大致说来,男女出生时本是平等的。一夫一妻的搭配原是上天的旨意。

也有人认为要想使男女具有平等的地位,必须使两者在经济上形成平等,其实这只是今日社会上的经济结构所造成的观念。这观念基本上就不合理,如此说来持家和养育孩子的责任又该由谁来负担呢?

对人而言情感教育是不可或缺的,而家庭在情感教育方面担任着重要的角色。

亲子关系是一种前缘注定的事,这种缘份又须透过家庭而被肯定。

如果每一个新生儿都由公家特设的机构来养育,自然就没有家庭的存在,但如此一来,新生儿在教条式的教育下人格都将趋于僵硬,思想易走极端,最后是替社会制造混乱。

女子的职责就在如何正确地养育自己的下一代,这是和男子的职责不一样的。也就是说由于女子生养儿女,教育儿女,人类才能绵延不绝地保持命脉。而男子,则对现实社会具有特殊的义务和责任。

弥勒因为主张对未来理想社会的创建而被尊为未来佛。此一思想即根源于他确信女人具有创造理想社会的能力。

在巴漏喇的家庭中,她的父亲即由许多妻妾围绕着侍奉的。

这虽然是当时的社会所视之当然的一个现象,但由于男女的结合原应以一对一的形式产生,故而一旦不合乎这一自然定律,女性间就产生了丑陋无比的纷争。因而巴漏喇的母亲与父亲之间总存着一股不妥协的暗流,随时有破堤暴发的可能。

巴漏喇自幼处于这股暗流中,故在身为女性的觉悟上,她深深感到女子的尊严受到伤害。

不知不觉间,她对女人充满了怜悯的情怀。

母亲平时的话题,总离不开对父亲的不满与愤怨。

父亲的不忠的爱情,也残酷地伤害了母亲。

巴漏喇对自己的母亲也爱莫能助,她只能与母亲站在同一立场上默默忍受着现实的残酷,除凡事顺从母亲外,她无能为力。

虽然她在物质生活上不虞匮乏,但这种家庭背景是造成她日后潜心出家的原因之一。

她很早就想到,即使日后自己嫁为人妇,有了归宿,谁又能担保自己的丈夫不会将小夫人一一带进家门?因此女人的幸福说来是很虚幻的,随着时间的迁移,自己难保不和母亲一样,为了丈夫的移情别恋而苦恼不安。

为了避免此一苦恼,唯有一辈子守独身,专心地贡献力量于需要帮助的人们。

巴漏喇由于有这样的独身思想,加之与丈夫又没有实质的夫妻关系,故她对自己的丈夫丝毫没有独占欲,也不致产生忌妒心。

桠那经常出外旅行,她虽然会关怀桠那的安危,但不曾因桠那的外出而感到寂寞或不安。

桠那将世代累积的财产平分给佃农,巴漏喇看到佃农们喜悦的神倩,心头也感到十分的快慰。

由于少女时代对环境的体悟,加之与桠那经历了一段清净的婚姻生活之后,她深深理解到痛苦是由于执着的行为与思想所引起的。

但是巴漏喇仍有许多困惑于心的问题,譬如她很重视家庭的问题,以及人类社会延续的问题。换言之,她不了解爱的真谛何在。

不制造执着的因子固然非常重要,但人们是不是只要避开它就真的不再有执着吗?

那么结婚的意义何在?夫妇又是什么?

男女两性的机能,在现实界是不可或缺的配合,此配合更适用于万事万物,它们藉着天地间一切阴阳机能的调和而使现象界的万般现象回转不息。

婚姻即是阴阳调合的具体形式。

男女各有其职务与特性,两性结合则促使二者在精神上和肉体上的成长。如果男女不相结合,各自活动,就不会有所谓的人类社会存在。

佛国土(理想世界)的存在,首先必存在于个人的心中,然后将之实现于现象界。现象界的存在,始于男女两性的结合。

因此婚姻为上苍所应允,没有婚姻,理想的世界不可能在现象界中出现。

圣经上亚当和夏娃,就是人类社会存在的象征,人类社会的谐调根源于两性生活的谐调。

以男女的结合为核心,藉由孩子,使家庭成为宇宙中的一员而开始运转,朝着使社会调和的方向而努力。

没有家庭,没有夫妻合理关系的社会,人们不可能在其中安定地生活着。

因此人类藉着婚姻与家庭,连绵不绝地生存于此世间,同时更努力地朝更高层次的调和发展,这是人类的使命。

人类背负着肉体生活着,必然会对事物起执着,但是如果了解到一切外在现象后面的空相,就能领悟为物缠缚是多们愚昧。

唯有脱离执着的痛苦,身心及周遭的调和之门才得以敞开。

通常「爱」都是由男女两性发展而来。

我们所以会对异性向往,是因为对方拥有自己所没有或自己感觉不够的东西。当然二者间若无互相心怡的本质,也不会互相倾慕的,因之也不会成为人生旅途上互相扶持的伴侣了。

爱,正是从这种互相扶持,互相弥补,使对方复苏的关系中所产生。它同时可以发展为对邻人、社会国家以及全体人类的爱。

不管社会如何发展、变化,人类如何增加,男女两性的基础绝不会崩溃的。

因此爱是由男女两性的结合为起点,再加以发展、扩张的,如此说来,爱就是促进现实世界和谐安宁所不可少的神光了。

「巴漏喇与桠那并肩走着,同时回顾这六年来的婚姻生活,她突然在精神上、肉体上感到一片空茫。

她自己也不明白究竟是什么使她感到空虚,出家的意念本来自她对环境的觉悟。那个称做丈夫的人就在眼前,但她未曾真正了解丈夫对自己的意义何在,不禁有种缺憾的感觉。

因为迦叶家的主人出家了,一些与桠那、巴漏喇熟识的商贾们,沿途闻讯赶来相送,眼中含着依依不舍的泪水。

二人来到村庄的南北界限处。

彼此无言相对,还是桠那先开口:

「巴漏喇,两个人一起出家修行,恐怕会有妨碍。但是我又不放心你一个人旅行,你一定要好好注意身体,希望你日后成为一个优秀的女修行。我也要单枪匹马去寻访一位修行人。我要往南方去,你就走东北方吧,此路通往帕拉那西,过了河就可到毗舍离,那儿有许多女修行。」

「桠那,也请你多保重,你开悟后,不要忘记来看我,来引导我。我知道离别是很使人寂寞的,但我会尽力去克服,并且努力地修行。你要到哪一个修行场呢?」

「我想到摩竭陀国的拉迦库利哈去看看,也想去探访一下伽耶达那的修行场。」

二人紧紧地握着手,同时依依不舍地互望着,除了互道珍重外,眼神中更透着几许约定再见的祈盼。

巴漏喇的眼中闪着异样的光芒,但她立刻转身直往东北之道前进。

桠那在后目送。巴漏喇也不时回身,并挥动小手,然后渐行渐远。

不久,巴漏喇消失在起伏平缓的森林中。

就在她消失的那一刹那间,桠那控制不住地失声痛哭起来。

六年来,桠那在精神上可说比任何一个做丈夫的都要怜恤妻子,他有股按捺不住的冲动,想要把巴漏喇追回来。

桠那在心中喊道:

「巴漏喇,你要原谅我,希望你能看透这个道理,这也是一种因缘哪,请多多保重自己!」

他目不转睛地望着吞噬了巴漏喇的黑林子。

当他的心境略趋平静时,这才掉过头朝南方行去。

头两天的野地露宿,比他意想中的要痛苦多了。这和从前外出野宿时的滋味完全不同,因而激起了他强烈的思乡之情,更加深了对巴漏喇的思念。

他本是一心一意要出家,一旦如愿以偿,又有一种被抛置于茫茫大海的感觉,目前虽然海阔天空,有着未曾享过的自由,但他一时之间不知如何适应这自由的情况。

他从未觉察过出家的生活是如此残酷的一项挑战。

第五天以后,桠那的情绪才稍稍平复。

由于一股强烈的求道的心,他很快地和大自然融合在一起。

此时,佛陀正在竹林精舍东北角的拿兰陀林中禅定。

林木茂密,层层遮起阳光,是一处理想的修行所。

佛陀心想,那个修行者应该要到了。

佛陀禅定时,就像树根包着大地一般,端严稳定。

浑圆宽广的后光,重重包围着佛陀,周遭闪耀着金黄色的光芒。

一个修行者悄悄来到佛陀面前,突地匍伏于地面,双掌合举于头顶。

男子一面膜拜,一面急切地说:

「伟大的悟道者,大慈大悲的救世主,请收在下为弟子。……」

汗水、尘土加上泪水,桠那一脸肮脏,慌乱地发着抖。

佛陀睁开眼,安祥地看着他。

「修行人哪,请抬起头来。」

「啊!佛陀,真高兴能拜见您。我是拿兰陀东北方梯尔大城是婆罗门,名叫丕葩利•桠那。佛陀,请您引导我。」

「桠那,我是释迦族的悉达多修行者,我知道你会前来求道,特在这里等你。」

佛陀的话,字字如千金之鼎,撞击着桠那的心。

虽然桠那落发不过才几天,但他出家的意念却由来已久,如今和佛陀相会于此,算是达成一椿宿愿,因此佛陀不管说什么,他立刻能心领神会。

他不断颤抖着,心中充满着无上的喜悦。

桠那从随身的袋中取出一件崭新的僧衣,以供奉佛陀。

「佛陀,请您收下这件僧衣,这代表我对您的一番敬意。」

桠那刚才看到禅定中的佛陀,周身为金光所围绕,即使一袭粗布僧衣,也显得那样庄严美丽。

佛陀身后泛光的情形,正好与途中修行僧们传说的情形一样,那些修行僧说这是真正悟道后的表征。因此桠那能肯定眼前这人就是人人尊称的佛陀。

他感觉佛陀在外表的光就已具备有三十二种庄严相,自己的心思也早被佛陀所看穿。

「丕葩利•桠那,你的妻子到巴达利•盖马去求道了吗?」

「是的,佛陀,正是那个方向!」

「你自律甚严,一心求道,甚至把财产、领地等都从执着中抛掉,心中更因遇见我而充满了喜悦。千万要注意,今后可不要忘了当初那种求道的心志,就像疤结好而忘了创痛一般。你谦虚求道的心,必能使你成为一个伟大的修行者。」

「啊!佛陀!佛陀!谢谢!谢谢!」

他由于感动之情的奔流而放声痛哭起来。

随着激动的感情,他领会到自己和佛陀在前世就有过师生关系。

想要进入佛陀的僧团,至少要在山中修行一个礼拜。但丕葩利•桠涉的情形特殊。

桠那很快就达到阿罗汉的境地,他能以古语叙述当时的佛法。

佛陀也操着令人怀念的古语,他由衷地为桠那转生今世而高兴,他也为两人能相会于今世而流泪。

桠那和佛陀偕伴修行七天后,佛陀将身上的僧衣脱给他,并对他说:

「桠那,我把过去所穿的僧衣送给你,这是我悟道时所穿,希望你也能早日达到和宇宙合一的境界。」

桠那伏地接过僧衣,发誓要以佛陀所达的境界为目标,努力地修成正果。

回到竹林精舍后,佛陀立刻将桠那介绍给大众。

「诸位修行人,我现在为你们介绍一位修行人,他名叫丕葩利•桠那;现在已经领悟了佛法,达到阿罗汉的境地,能洞悉过去和现在。他早晚要修成正果来向更多的人传布佛法。希望你们也能好好地实践八正道,将心量拓展开来。」

佛陀的话,一字一句地传入弟子心中。

当初舍利弗和目犍连被佛陀重视时,曾引起弟子们的嫉妒和不满,但佛陀的预言,经过时日的变迁,一一被证实了,因此现在他们都能以亲切的态度来接纳桠那。

桠那成为教团的一份子之后,各方面的表现都很卓越,弟子们对佛陀伟大的预言,更增长了敬畏之念。

桠那皈依佛陀时,是当佛陀四十岁,亦即悟道后的第四年。

桠那与教团中的僧众相处很融洽,甚至成为众人的指导者。

他的名字已改为摩诃迦叶。

与桠那分别之后的巴漏喇,由巴达利盖马进入憍萨罗国,在该国境内的一处道场修行。这道场聚集了许多婆罗门的女修行者。到这里修行的女人,必定要是婆罗门出身,且志节坚定的。在日常生活中,要持守戒律,操演仪式,并祭祀婆罗门神。在这里,婆罗门种特有的优越意识是很浓厚的,巴漏喇为种姓不平等的现象而深深苦恼着。

皈依佛陀的人越来越多了。

皈依的人既然增加,就应好好组织起来,然而一有组织,许多事就不免流于形式了,因此佛陀布道的大目标——使众生脱离生老病死的苦恼,以建立完美的自我——也就有了形式化之虞。

佛陀因此深感戒律的必要性,为了有效管理僧团,戒律一项是不可或缺的。尤其在摩竭陀国境内,有各式各样的宗教团体存在着,各教团间为了伸张自己势力,常不惜使用各种手段来打击其他的教团,甚至派人潜入某教团,以制造纠纷。

当时各教团间的信仰,多注重于藉重神力的他力信仰。

神人分割,人们采取偶像崇拜的信仰形式,因此信仰与争执成为该宗教的表里。

佛陀所传扬的教义,则在于以大自然为仿效的对象,使人与宇宙合而为一。在生活上,则表现出和太阳一样的宽宏而慈悲的心怀。

但佛陀的教团由于人数日众,组织扩大,渐渐有了形式化的危机。

有的人一踏出教团,立刻被周遭他力信仰的气氛所感染而有了歪曲的见解,佛陀有鉴于此,更觉得戒律有设立的必要。

要想进入佛陀的教团,首先必须先皈依三宝。有些在其他地方加入教团的人,往往有不皈依三宝的现象,这种人很容易就会走错方向。因此佛陀认为应以竹林精舍为中心,首先决定一位已受有具足戒的修行僧来做指导者,如果这位修行僧出缺时,则应有一位代理人来负责指派各地区的负责人。因此有人要入教团时,就得先经过既定的程序,而入团时的年龄一律当在二十岁以上。

\chapter{第五章\ 巴巴里十七个弟子的皈依}\label{ch5}

\section{师徒分离}\label{sec5.1}

帕拉那西郊外的某婆罗城内,大家均在谈论佛陀的事情。

因为耶萨的父母亲一个在家的男女弟子——正在宣扬佛陀的正法。

他二人在婆罗门教徒中是有名望的人,同时二人是当地的大富豪,他们对修行者,皆能不分彼此地慷慨布施。他们同时还对中途落脚的修行者们传布佛陀的教义。

大婆罗门巴巴里从他们那儿听到佛陀的事。其内容非常精辟,尤其谈到如何脱离生老病死的痛苦时,可看出演说者真不愧是佛门的弟子。

尤其令巴巴里感动的是,他二人的独身子耶萨早已皈依佛陀门下,并达于阿罗汉的境地,同时在问众生传法。为此缘故,巴巴里非常渴望能拜见佛陀。

巴巴里虽然有这一个心愿,但以自己达于一百二十的高龄,若想走到摩竭陀国,恐怕需要一年的时间。

他正在犹豫不决时,眼见城里即将举行祭祀大典。

一年一度的祭典,是为颂赞婆罗门神而举行,是人们彼此祝福并欢欣鼓舞的日子。

巴巴里是祭典的主办人,他在顺利地主持完祭祀后,回到他所在的道场。

正有待喘息之余,一位衣衫褴褛的男子来拜访他。该男子的身体有些浮肿,头发布满了灰尘,脸上更是污垢四处,就像是一个乞丐。

但他声称自己是婆罗门的修行僧。

巴巴里请他坐下,并以稀饭给他果腹。

他以炯炯发光的眼睛盯着巴巴里说:

「祭典是如此盛大,人们奉献的财宝一定相当多吧!来,分一点财宝给我,这是我此来的目的。」

这番话与他的身份极不相称,因此巴巴里愣住了,一时不知如何回答。

吸了一口气之后,巴巴里放稳声调说:

「祭典上的赠品早已全部分给大家了,这里什么也没有,很抱歉不能再分给你任何东西。」

「呵,那么多的供品已经没有了,不用说你怀里也没有了,那没有关系,我将在七日之内施一种咒语,让你的头裂成七份,教你凄惨而死,你就把这当作神的惩罚吧!」

修行者说着,就面向巴巴里,把手放在他额上以遮住阳光,口中并念念有辞,然后就转身离开了。

巴巴里从未经历过这种事,心里非常恐慌。

不论是谁,被施过咒语后,心情总会抑郁不安的。

巴巴里很苦恼,连饭都吃不下。

为了稳定情绪,他到树林中去修禅定,但是徒劳无功。

加之主办祭祀,使他备感疲劳,他不知不觉间睡着了。

第二天一早,他被耳边的声音惊醒。

「巴巴里,你苦恼什么?难道你还没有除掉执着心吗?什么是可怕的?你惊讶什么?那个婆罗门无赖想要你的财产,难道你对头陀七分这件事一点都不明白吗?

巴巴里问那个声音求援道:

「婆罗门神呀,请指示我一条了解之道吧!」

「你还在拘泥吗?死有那么可怕吗?」

「弟子们也跟我一样,对于神的处罚感到非常苦恼,请指示我一条悟道的路子。」

巴巴里抬起头诚敬地聆听那声音。

「憍萨罗国的释迦牟尼•悉达多太子已出家修得正果,目前正在摩谒陀国境内说法,他己了悟因果之法,正在普渡众生。」

天空中一回响着这样的声音。

巴巴里又是大吃一惊。

因为这一说法与帕拉那西那两位佛陀的在家弟子所说的不谋而合。

巴巴里立刻将大弟子们召集而来。

弟子们不知发生了什么事,都面现惊讶之色。

照理说祭祀过后是最闲暇的时候,为何师父要如此急急召唤呢?

「希望你们好好听我说,我虽然出自婆罗门家系,一直学的是婆罗门的圣典,同时依传说的方法修行到现在,但是不明白的事情仍然很多,我自知距离悟道还相当遥远。昨天那个贪心的婆罗门僧问我要供品,我没有办法给他,他就诅咒我在七日之内遭神的处罚。我苦思不得其解,今晨听到神的指示,知道有一位佛陀正在说法。……」

由于巴巴里的语气十分诚恳,弟子们深信不疑地谛听着。

「对于婆罗门的信仰习惯以及祭祀方式,相信大家也都早有疑问,对这些疑问,我也无法回答,为什么呢?因为婆罗门教已完全成为一种理论,而佛陀则有看穿一切的法力,因为他把握了我们真正的『心』。如果他是一位真正的佛陀,他必能看穿大家的心事,希望你们能问佛陀学习我所不能教给你们的那个『心』的道理,将来好成为真正的修行者,达到大澈大悟的境界。」

巴巴里边说,边扫视弟子们的脸,一面物色到佛陀门下修行的人选。

「频迦,希望你做我的代理人,率领我所挑选的弟子到摩谒陀国的拉迦库利哈去。」

频迦的体形魁梧,力气充盛,有足够的能力带领弟子长途跋涉。

弟子们都很好奇将有哪些人被挑选出来,十分注意巴巴里说的话。

「在法之前,没有男女之别,一切都是平等的存在。」巴巴里边说,边将视线投向他那年轻的姪女埋托勒呀身上。

「埋托勒呀,你从小就很热心地修习婆罗门的教义,是个很优秀的女修行,同时你也认真地教导了婆罗门的小孩子们学习严格的祭事。你保守独身,一心过着苦修和冥想的生活。摩谒陀国虽然很远,你是不是想去呢?」

「我想去,请让我去吧!」

她毫不迟疑地回答道。

巴巴里微笑道:

「年轻人就应该有吃苦的精神。」

巴巴里接着又点名道:

「沙罗兰陀、波萨罗、陀提耶,你们也一起去好吗?」

三个女子很高与地答应了。

之后巴巴里又环顾大众。

三位女子都比埋托勒呀大三、四岁,平时也都精勤修行,同时四人的交情非常好。接着蒙迦罗闍被指名。

他对婆罗门的圣典很有研究,因此巴巴里认为在探讨婆罗门的真正价值上,让他去接近佛陀,是非常有意义的安排。

蒙迦罗闍高兴,但犹疑地说:

「谢谢老师的好意,虽然老师称这个人为佛陀,但我不觉得他会比老师您,或摩诃•伊西(大仙人)更伟大。」

巴巴里回答道:

「那只是你的想法罢了。在没有经历实际状况之前,只凭想象来下断语,正是你一向具有的缺点,我们必须从各类经验中获取事实的真相。」

巴巴里接着又一一点出姓名。

「阿羁陀、蔑陀古、多达加、赫摩迦,你们都还年轻,正是需要多方磨练的时候。迦陀康尼,你也一起去,优陀耶、迦波,你们也跟着去,要知道好好修行。一路上不但有险峻的山谷,也有毒蛇猛兽,希望你们要时时彼此照应。此外还可能遇上山贼,不过山贼多半不会对身无分文的修行人下手,女修行们最好做男子的装扮,男修行们自然要尽力保护她们。

碰到佛陀时,要小心观察他的言行,他若是真正的佛陀,当会看穿你们的心事,譬如说我是谁,你们这一大群人为何会去见他,我有什么烦恼,甚至我的头颅要分割的事,相信他都会明了的。

婆多罗吠陀、彭那迦、优婆悉跋、难陀,你们也一起去。被点到名字的人,都到前面来。」

十七人立刻齐集在巴巴里的面前。

巴巴里环视跟前的十七个弟子,心中升起无限的感慨。

如果自己还年轻,必然会带领他们去拜见佛陀,现在这把年纪是心有余而力不足了。这十七个优秀的弟子们即将离开自己,要到一个完全陌生的环境去开创学问的天地,他的心头掩不住有着难舍与寂寞的感觉。就像与子女们分离一般,巴巴里抑制着痛苦的心情。

「现在在我面前的弟子们,都是很懂自律的人,能行严厉的修行生活,且经常做冥想的功夫,在转生的过程中能成为出色的婆罗门修行者。我平时所传授的,只能算是一种婆罗门的知识,但是佛陀却能教导你们如何锻练自己的心性,来真正行在正道上。当你们遇到真正的佛陀时,一定要求他收留你们。频迦,你要经常与我联络。」

巴巴里说着,声音不由得颤抖起来,眼眶充满着泪水。

埋托勒呀等三位女婆罗门见状,也抽抽搐搐地流下泪来。

他们知道与师父这一别不知何日再相聚,以师父这样的高龄,他们有着诀别的感觉。

但是为了使弟子们真正完成自我而去皈依佛陀,巴巴里的这番慈悲之意,也深深打动弟子们的心。

「既经决定,你们就在明晨动身吧!在第一声鸡鸣时分就到道场来集合,然后再一起出发。」

当时的旅行,泰半要靠人的一双脚,如果是短短的一两天路程还好,但此行是到完全陌生的另一国度,一路上不知有多少险阻的山川,同时炙热的太阳使行程更加艰巨。

因此在出远门的同时,就具备了生离死别的意义了。

治安良好的城镇或村庄,自然不成问题,一旦踏出乡里,就等于进入老虎、土狼、毒蛇等猛兽的国度,同时还要应付山贼的出没。

旅行是一件与死亡毗邻的险事,因此十七人在准备上路前,心情一直是很慌乱的。

第二天第一声鸡啼声响时,大家都已聚集在道场上。

巴巴里身着鹿皮僧衣,头发梳理整齐,正在场上等候,大家纷纷趋前请安。

巴巴里合掌为每一个祝福。

师徒间默默无言,只任由一股暖流在彼此的心底交流着。

一行十七人离开道场后,越过几座小丘后,一直向北进发。

巴巴里则站在高处目送他们,一直将他们送出视线之外。

十七人怀着瑟瑟的心情,历经了千辛万苦,终于进入摩竭陀国境。

他们眼中看到的,耳中听到的,都是那么新奇,不由兴起驻足观赏的欲望。

但是他们随及想到去见佛陀才是此行最重要的目的,他们不能把时光无谓浪费掉。

频迦向过路的修行僧们打听佛陀的住处。

佛陀此刻正在北门外的灵鹫山向大众说法。埋托勒呀和其他三位女修行谈论有关拜见佛陀时可能的景像。

「佛陀究竟是怎样的人呢?」

「听说是迦毗罗卫国的王子,可能具有刹帝利(贵族武将)的威武之气。」

「他不是婆罗门种,也能成为佛陀吗?」

与佛陀见面的时刻越近,他们心中的疑虑也越多。

他们一向就认为一个人要想成为大智大慧的救世主,本身一定要是婆罗门出身,同时经典上也清清楚楚记载着这样的事实。

佛陀正背对着一个大岩石,在向许多修行僧们说法,他的声音有如宏钟,响彻了整个山谷,永远回荡在人们心中。

\section{初见佛陀}\label{sec5.2}

「诸位修行者,光靠严厉的苦行,是没有办法平熄烦恼之火的,因为你们的心已被五官所束缚。固然你对那些苦行有着超人的耐力,但是心中免不了产生『我』的念头,这念头又产生一连串自我中心的思想。

要想免除迷惑,唯有正确地想,正确地做,否则你的心永远无法平静下来。如何评断自己的心行是正确的呢!那是有准则可资依据的。

盲目的修行,会增强扰人的自我意识。

在舒适的环境中,任由自己欲望兹长,是无法使烦恼消除的,因为一个欲望衍生另一个欲望,欲望无止尽地纠结在你生活中,你就像双脚陷入泥沼般,越想拚命挣扎,就越陷得深。你们唯有从欲望中走出来,过一种合乎中道的生活,才是一种解脱之道。……」

频迦一行人并排坐在听众的后面,聆听佛陀的法音。

「埋托勒呀,你看佛陀被金黄色的光圈围绕着,是多庄严美丽啊!」

阿羁陀对坐在身旁全神贯注的埋托勒呀说道。

埋托勒呀也正在惊异于佛陀显现的样子,她奇怪佛陀的脸为何会有重叠的影像,身后又为何有眩目的光芒。

佛陀已知道他们来了,不时将视线扫向他们。

一般人见他们穿着整齐的僧衣,知道是婆罗门种出身,但却不知他们来自何方,所为何来。

唯有佛陀早已洞悉。

约一个半小时之后,佛陀说法完毕,频迦等一行走到佛陀面前,向他请安。「你们真是不远千里而来!」

佛陀微笑着,语调亲切和蔼,与刚才说法时的铿锵有力大异其趣。

十七个人很用心地观察佛陀的一举一动。

蒙迦罗闍对佛陀刚才说的法虽很注意聆听,但他有着很强烈的婆罗门教意识,故佛陀首将视线转向他说道:

「当我们眼睛盯着一桌好菜时,是不知菜的味道好坏的,虽然思索着这些菜的做法,但肚子仍空空的,未曾享用。这不正是所谓的用眼睛来享受好菜吗?

婆罗门僧向来用智识来领悟道理,却不知道人心的尊严与可贵。任何智识应该是藉行为的实践来领悟的。这点你懂吗?」

蒙迦罗闍对佛陀突如其来的训语大吃一惊,立刻领悟到自己的愚昧,因而双手合掌,以头触地,失声痛哭起来。

身后的阿羁陀则在心中念着:

「我的师父正为头颅分割的事烦恼着,伟大的圣仙呀,请为我师父解答其中的奥秘吧!」

「修行者们,你们想知道我是不是真正的悟道者,同时进一步想,学习实践我的法,做我的弟子。一般人都容易把自己学得的东西看成绝对的,但是要知道,那只能表示是你自己能了解的东西罢了,如果因此而轻蔑他人的说法,便使自己的心胸变得更狭窄了。」

佛陀一面注视阿羁陀,一面继续说:

「你们的老师巴巴里,精通婆罗门三圣典,又能确切地修行,可说是位非常杰出的大婆罗门僧。今天他派你们到我这里,来真正学习心性的实践之道,以完成个人的自我。巴巴里是个有勇气的指导者,他因年纪关系无法亲自前来,我也觉得非常遗憾。」

大家听佛陀点出他们的心事,都非常感动。阿羁陀更是声泪俱下。

埋托勒呀坐在最后面,为佛陀的一言一语所震动,一时抑制不住心头的感激之情,泪水沿着面颊不断流下来。

她激动地喊道:

「佛陀,伟大的救世主,请收我为弟子,我要皈依佛陀。」

这是她生平第一遭的经验,感激之情使她全身颤抖,内心如潮水般澎湃着。佛陀伟大的光,打开了她的心扉。

阿羁陀为自已先前的幼稚想法感到羞惭,他走到佛陀面前说:「请你原谅我!」

说着就伏在佛陀的脚边礼拜着,接着说:

「我们的师父被人诅咒了,他的头颅将被分割为七,请救救我们的师父吧!」

佛陀注视着阿羁陀的脸,回答道:

「人的无知,可由这件事看出来。肉体是无常的,既可说是自己的,也可说不是自己的。当初肉体得自双亲,等时日一到又将归还大自然。一个人本应当透过己身的五官使自己的人生经验更丰富,使心境更宽广,但是许多人却迷失在五官的欲望中,产生许多苦恼。

任何人都没有办法永久保有肉体,既然如此,我们对这个肉体又有什么好执着的呢?你们的师父巴巴里从婆罗门教中知道头颅分割的事,却不了解如果自己行得正,那咒语最后终会回到施咒人的身上。唯有在阴沉污秽的地方才容易长出蛆来,太阳灿烂照耀着的地方,才会飞舞着美丽的蝴蝶与蜜蜂。你们何曾看过蛆生长在明亮干燥的地方?

由于自己的心地与行为有了偏差,才产生受罚的念头,心行若能守住中道,过着充满光明的生活,就不会受他人诅咒的坏影响了。

无知就是一种缺乏智慧的现象。

一个人只凭知识是无法使心神安泰的,如果将已获得的知识配合上身心的实践与体验,才能将知识转化为更深一层的智慧。

那个咒语不会在巴巴里身上应验的。

如果是符合自己的现况,尽管对方施予什么样的咒语,都不会产生效验的。」

佛陀说到这里,指着频迦说:

「你去向你师父传达我的这一番言语,好吗?」

这十七个人,立刻褊袒右肩,伏在佛陀的脚边。

佛陀用慈爱的音调安慰他们道:

「希望巴巴里能安逸地生活着,并祝他安享高寿。……」

佛陀注视着他们,接着说:

「婆罗门的修行者,我将解答你们心中的疑问,不论什么问题都可以提出来,不要拘束,心情放轻松,一个个问。」

蒙迦罗闍首先发出了疑问。

「我曾在心中两度向您发问了,但都没有得到任何回答,请您现在答覆我好吗?」

「你总会说话吧!不能要我一个人一一看穿你们的心事来回答你们的问题。你不以为利用语言来诉说心中的事,是谋求思想沟通的最佳途径吗?如果言语是多余的,那么整个人类社会将有许多不便,事务也将无法顺利地进行,是不是这样呢?」

蒙迦罗闍红着脸说:

「真对不起您,我错了。」他低下头,清清楚楚地说出自己的疑问:

「关于现象界和意识界,以及梵天和印度诸神的世界,请您描述给我听,此外,为了不被神所迷惑,人该如何做?」

「蒙迦罗闇,你是否将肉体视为一切?」

「不,不是那样的。」

「那么肉体以外,还有什么呢?」

「你身为婆罗门种,出身圣职之家,经常要祭祀诸神吧!但太阳光只为你们带来温暖吗?一切保存肉体的环境都为你才有的吗?不是这样的吧!首陀罗(奴隶)、吠舍(工商业者)、刹帝利(武士)不也都平等地受到太阳的恩泽吗?

大自然对人类并无所求,它公平地施予万物生存之道。这就是慈悲的表现。在大自然慈悲的胸怀下,人才能保有肉体生存下去,因此大自然的恩惠就是一种慈悲的凝结。

蒙迦罗闇啊!人之所以具有如大自然般施恩的根本意识,就在于有空的世界,即意识界,也就是心的世界。

这和透过你身体,使你身体行动的意识作用是一样的,当那个我们肉眼看不到的意识界和我们的意识产生关联后,才有我们的存在。

支配肉体的那个意识,似空实真,它就是不生不灭的人类真实的本身。

蒙迦罗闍,你睡着时,耳孔虽然开着,但能听到他人说话吗?脑中记得那些话吗?没有吧!没有了支配身体的意识作用,身体上的感官就无法发生作用了。

肉体的支配者,可说是真正的自己。梵天、诸神都存在于意识界,那个肉眼不见但却实际存在的世界。

肉体在自然界中调适地存在着。一个人死亡时,就是与自己肉体告别之时,因此肉体是无常的。

要消除对死亡的恐惧,就必须消除对生与死所存在的差别观念。

执着于肉体上的官能,就会产生对死亡的恐惧,又如果以为『生』只存在于这一世,自然就对『死』感到迷惑了。

我们应当舍弃这外在的假我。因为假我是很自私的,常产生不顾他人利害死活的冷漠。

我们如果能超越生和死,我们就看不到死神。

梵天界是光明的世界,那儿充满了无法言喻的欢愉与和平。意识界的光明与否,实与一个人心胸的宽狭成正比。将来自己要归宿的世界,实在是自己造就出来的。

所以,认为五官无法接触的事物就是不存在,这是很愚昧的想法。」

被誉为婆罗门理论家的蒙迦罗闍,在佛陀面前噤若寒蝉。

埋托勒呀道出了她久藏心底的问题:

「佛陀,怎样的人才能真正做到不被动摇,不被污染且能超越一切的欲望呢?」

佛陀慈祥地注视她说:

「埋托勒呀,许多人变成欲望的俘虏,抱怨自己欲望不能满足,于是心中充满了愤怨,忌妒等,如此心志动摇,随俗浮沉,变成一个极度苦恼的人。

只有看破欲望为害的道理,恪守中道,过着少欲知足的生活,即使对情欲,也时时存着纯洁的念头,经常规律地思维,心中充满不偏不倚的思念,如此才能冀望心灵平静,安如磐石,如此才不会轻易为外界的纷扰所动摇。经常把慈爱施予他人,过着和善的生活,心灵才不致被世俗所污染。又因为藉着对正法不断的实践与体验,而能增长自己的智慧,达到超越一切的境界。真正称得上是人的人,就是指对这种道理能领悟的人。」

「啊!我明白了,谢谢您!」

她从而了解到婆罗门教只重祭祀仪式而没有确切实践法规的信仰形式是错误的。她深深感到佛陀的法才是人人应走的正道。

于是她下足决心要继续修习佛陀的正法。

接着她又问:

「佛陀!刹帝利和婆罗门都曾为诸神做过许多供奉,但这供奉的意义何在?尤其婆罗门在祭祀时奉献了许多供品给神,请为我解答这疑问。」

澎那迦也有同样的疑问,他说:

「我也有同样的疑惑,诸神真的喜欢这些供奉吗?是听取供奉人的心愿呢?还是给予供奉人心愿呢?」

佛陀回答二人道:

「许多人都为了祈求自身的幸福而来供奉诸神,有的人希望解除眼前的不幸,有的人则希望保有现前的幸福。人各有其愿望。但他们不了解苦恼完全是自己的心和行为所制造出来的。因此,我们要知道排除苦恼的原因是很重要的,其根本的苦因不去除,人就会为苦恼一直纠缠着,那样就是供奉再多的供品也是没有用的。

真正的供奉不只是把物品献给神,而在于人们实践正确的道。不论你如何赞美神,你在供奉时只想着追求欲望的满足,那是毫无用处的。也就是说,你不认清真正苦因之所在,无论多么热心地祭神,也没有办法了脱生死。

在我们这个世界,没有人能独自生存的,必须藉着与他人的相互关系而生活。我们如果了解这一点,就能抛弃自私自利的念头,而去追求和谐的,与人互助合作的生存之道。人际关系在人的社会中是相当重要的课题。

能供奉活着的可怜人,才称得上是真正的供奉。

人与人互相帮助,互相谐调的努力和行为,就是一种对大自然心存感谢的具体表现。与人和谐相处,就是报答大自然的一种大供奉,这种供奉才是神所称许的。」

埋托勒呀听后频频点头。

澎迦那接着问道:

「许多人都热衷于供奉,但只靠供奉又无法超越生死,那么在诸神和人的世界中,哪一种人才能超越生死呢?」

「一个了解人生,对人生任何现象都不动摇信念,已达涅槃境地的人,他已没有了虚饰之心,也没有烦恼,凡事都能安心知足,这样的人就能了脱生死。」

「我明白了。」

澎迦那回答道。由于佛陀的话非常浅显易懂,使听讲的人一下就明白他话中的意思,但是若想真正由心灵深处领悟生死之道,那得经过一段长时间的修行才能达到的。

接着蔑陀古问道:

「我们的苦恼究竟的因何而来的呢?」

佛陀回答道:

「这是一个很好的问题,我就我所领悟的道理来说给你们听。今天在座的各位修行人,都自己体验过人生的苦恼,人在一生下来就被种种苦恼所缠绕着。产生苦恼最基本的原因,即在于人不明白人生的目的与使命,因而在假我中制造许多无知的思想、行为,过着盲目的生活。

如果我们在日常生活时时以「正法」做为思想和行为的指针,就能脱离假我的控制,避免假我产生的苦恼,也就不会盲目地过着充满欲念的生活。

抛弃苦、乐的两种极端思绪,经常行止观的功夫,好好端正自己的心行,实践正道,就能超越世上的一切苦恼,也就能从生、老、病、死中求得解脱。

一切的苦恼均源于偏颇的思维与行为,了解这一点是相当重要的。」

蔑陀古由于佛陀的这一番解说,了解到正确的心行中才有解脱之道。

\section{泉涌般的智慧}\label{sec5.3}

所谓正确的心,即显现在宇宙间的不偏不倚的现象中。若要将中道之法应用于生活中,唯有经常反躬自省,不断观察自己既往的言行与思想是否合于正道,一旦发现有偏差时,就应立刻自劝自心,以期再次行动时能避免重蹈覆辙。

中道之法就是远离欲念而与大自然合而为一的一种做法。

佛陀所悟得的道理,即在于与宇宙合而为一的中道的思想与行为,就是解脱生死之道。

藉着这一觉悟,就能理解过去无法觉察到的苦恼的根源,因而及时从苦恼中得到解脱。

因此佛陀的弟子或信奉的人,就是那些能实践佛陀中道思想的人。

佛陀继续对蔑陀古说:

「不论东西南北任何地方,都不能让各种现象与执着动摇你的心,占据你的心。一个立志求道的修行人,如果能集中思维,不懈怠地做反省与冥想的功夫,早晚必能离开苦恼与悲哀,了脱生死。蔑陀古,所谓八正道,就是解脱一切苦恼的捷径。」

蔑陀古感激地说:

「我诚心领受您所教诲的法。佛陀您就是解脱了一切苦恼的伟大圣者,相信实践佛法的人,必能从苦恼中解脱出来。我也要皈依佛陀,请佛陀引导我。」

他由于无法控制心中的激动,一时不知如何再讲下去。

在附近的修行者们,也不知什么时候又都聚集过来聆听佛陀说法,谁也没有离开的意思。

太阳渐渐西斜,红霞满照天空。当太阳西沉后,顿时有寒气袭来。于是修行僧们在座中生起柴火。

围着熊熊的火焰,佛陀继续说:

「宇宙的大法,称为真理;领悟实践此真理的人即是圣者。圣者无欲,充满慈悲,对于众生,他施予无限的爱与慈悲。

圣者没有欲念,对爱欲与生存更没有执着之心,尤其对人生的各种障碍都能忍辱行之,不用暴力,也没有疑惑,最后终能达到不生不死的彼岸。」

「谢谢您!佛陀!」

蔑陀古不知该说什么,只能掩面哭泣。

喜悦的泪,尽情地沿着面颊流下来,清洗着他的心灵。

坐在蔑陀古身旁的优婆悉跋说:

「佛陀,请指引我,我虽然相信您所说的,但我以为超越生死不是一件简单易行的事,人究竟当凭借什么才能超脱人生航路上苦与乐的激流呢?」

佛陀看着优婆悉跋回答道:

「抛弃五官所引致的烦恼,不为一切所束缚,以正法做为心行的指南,如此必能超越苦乐的激流。斩断五官产生的爱欲,远离一切杂念,努力过清净的生活。」

「以佛法为依凭,抛弃一切欲念。是不是解脱苦恼后的人们,就能永远安享清净快乐的境界?」

「不错。但要知道,虽然是依正法生活,如果只将正法视为一种知识来了解而没有实际去奉行,则即使曾处于高远的境界,那境界也是无法持久的。

不经实践的法,容易蒙上不洁的尘埃。

不经实践的法,就像山间的滚石,随时会坠落深崖。

因此不要怠忽职志,要经常督促自己努力向道。」

优婆悉跋由此了解正法严峻的一面。

他领悟到所谓的皈依法,即在于以法为根据,严厉地督促自己精进向道,并随时将正法应用于日常生活中。

「我想再请教您,那些依法精进的人,他们的心是否能停留在和谐的境界,而且能自然达到解脱的境界?又解脱后的人就不会转生了吗?」

「优婆悉跋,肉体存在于这大自然中。

此肉体由幼年而少年,再成为青年、壮年、老年,最后又回归大自然中。

但是支配肉体的意识,将回到意识界,不久又由于缘生的关系而转生到现象界中。

悟了道的人能自由地往来意识界和现象界,因而也就从转生轮回中解脱了出来。

优婆悉跋,我们有时在旅途中。或骑马,或骑象,或坐渡船,虽然乘坐的交通工具不同,但乘坐的这个我没有变,最后终能达到目的地。

肉体就是人生航路中的交通工具。

如果驾驭这交通工具的心灵,能不断地体验正法,就能与自然相谐调而达于真正安适的境地。

导致生、老、病、死的迷惑和痛苦的真正原因,就在于人们误把肉体视为真正的自己,而不了解除肉体外内在还有一个不生不灭的自己。」

接着,多达迦问道:

「自己由于听取佛陀的教诲,再藉着实践,是不是也能达到最高的境地?」佛陀微笑道:

「多达迦,你要热心地学习我说的法。

你应当正确地评断事理,正确地思维事理,正确地表达正理,正确地工作,正确地行谋生之道,正确在向正道精进,正确地把念头置于正道上,正确地反省和冥想。你当惕励自己以此八条正道来实践自我。

当你经过实践而达于开悟的境界时,就会明白宇宙万物就是自己本身。

也就是说,宇宙就在自己的心中,也就能了解到肉体不过是宇宙的缩影。

因此也就不会再以渺小的五官官能来判断事理,来制造烦恼。

体验到至高的境界时,就自然而然地会觉察到我们现有的肉体是多么污秽,比泥沼的污浊尤有甚之。

当我们的意识扩展到如宇宙那般宽广时,便能领略到一切物质的欲望是多么愚昧。

物质的欲望永无止境,然而物质本身却是有限的存在,但没有人认真区别过这一点。

对肉体执着的人,便也执着于物。他们没有认清肉体是不能永久存在的。

当我们看穿了无常的道理后,便能了解被欲望所摆布的人,是多么的可怜。

你们当好好学习法,实践法。」

佛陀停顿下来,注视围绕在身边的人,然后指着赫摩迦说:

「这位修行者,你有什么疑问吗?请提出来没有关系。」

赫摩迦冷不防被指点到,稍稍迟疑了一会儿,然后谦虚地说出自己感想。

「佛陀,您的教诲我一定铭记在心。过去我所学的,只是一些祭祀方面的事宜,同时依前辈的看法,能这样修行就已经很够了。至于这样做,将来会有怎样的成就,经典上的记载也都很空洞而不实际。

不但如此,就是某地的大仙人是如何修行,其修行情况如何,也都是大家猜疑议论的话题。

我对这种现象一直很疑惑。

佛陀,您的道是经过亲自体验而悟得的,我们如果照着修行,应当不会产生疑问。

所以我确信你所说的道,并且愿意努力去实践,以达到解脱的境地。

实在很感谢您对我们的教诲。」

「赫摩迦,如果我们的心不为现象界的形形色色所动摇,时时除去心上的贪念与欲望,就能达到悟的境地。

在混乱的人世间能除掉种种执着的人,就能领悟这个道理,他们的心也就能恒常地得到安逸。

好好地以正道去做心灵的粮食吧。」

佛陀说完后,又鼓励女婆罗门陀提耶发问。

陀提耶也许是由于长途旅行,而日久曝晒于太阳下的缘故,已看不出是个女子。陀提耶用温婉柔美的声调问道:

「佛陀!如果一个人的心中已经没有一点爱欲与贪念,并时常以正道为行为的依据,成为一个毫无烦恼的人,那么他这种解脱可说是怎样的情形?」

佛陀回答道:

「他已得到了至上的解脱。」

「得到这种解脱,心灵不再受欲望所控制,是不是智慧之门就会开启呢?

「这种人不会再起任何欲念。也可以说,无论是谁,都具有佛性,即使智慧之门还没有开启,但他本身已具备了极髙的智慧。只要我们的心不再为世俗之物所迷惑,这本来具有的智慧就会发挥出来。」

佛陀慈蔼地解答道。

坐在陀提耶身旁的波萨罗接着道:

「佛陀!没有物欲,不为一切世俗现象所摆布的人,他的智慧是如何?」

「悟了道的人了解一切知识活动的层次,同时也知道一切的苦乐都关系到心的受束缚与否。

这样的人,他的智慧如泉水的涌现,永无止境。

波萨罗,你当好好修行,以求达到那种境界。」

「是的,谢谢您!」

佛陀在回答各人的发问时,态度是那么认真而诚恳,同时随着每个人的心性来做适当的解答。

十七个人在接近佛陀之后,渐渐了解到一些从前所没有经历过的道理。他们欣喜若狂地彼此祝贺,他们更感谢师父为他们做这样的安排。

如果失掉这个接近佛陀的机会,他们将无法体验到这种闻道后的喜悦。

他们是一群修习过婆罗门教义的修行者,故而佛陀的一言一语都能很真切地进入他们心中。

佛陀的言语就像一道光,清澈了他们的心。同时每一个字,又好像化成了无数的字句,在他们心中散开。

佛陀的每一句话,都让他们愈嚼愈有味。

现在众人已暂时没有疑问。

寂静之流回旋在他们之间。

当此之时,婆多罗吠陀,泪眼婆娑地说道:

「佛陀啊!您能抛弃尊贵的王室地位,断绝一切爱欲,而达致智慧的化身。您将法灯赐给我们这些处在迷惘与黑暗中的人,我由衷地感谢您。

现在,我们希望能皈依您,做教团中的一份子,接受您的教导。」

说完,他便合掌问佛陀顶礼膜拜。

佛陀看着他,慈祥地说:

「婆多罗吠陀,人心因五官的烦恼、迷惑而变化无穷,一时产生的欲念即使能满足,另一个新的欲念又会接踵而至,使你永远达不到安逸的境地。

内心失去准绳的人,很容易为假我产生的心魔所控制,因而过着苦恼的人生。你们要好好地去领悟正法,去实践正法,不要在世间的万般现象中产生执着。为了保护自己的心,不使它被一切欲望所破坏,万不要吝惜你的努力与实践正道上。

实践正道是需要勇气的,而将平日点点滴滴的努力累积起来将更为重要。凭着内在丰富的智慧,除去执着的根源,是达到安逸之境的捷径。

「谢谢!佛陀,谢谢!」

婆多罗吠陀一叠声地称谢着,身体不住颤抖着。

坐在众人后面而几乎不见身影的难陀,此时举手发问道:

「佛陀!世上有所多被尊为圣人或自称圣人的人,是指他们的智识卓越呢?还是指他们因苦修而留下卓越的成绩呢?」

「难陀!真正的圣人,是指能脱离一切执着而已经没有了苦恼与欲望的人,同时还能替身心患病的人点燃慈悲的法灯,让他们获得真正的平静。

一个人因研究学问而向众人宣说自己获得的见解,如果他执着于假我,做出虚假的行为,那么他只是一个具有知识的形体罢了,他没有真正的智慧,他只会在人间播下纷争的种子。

知识是靠观察与听闻而来,大体上说是由外而来。

因此知识的范围似广实窄。

但是智慧,则涌自我们的心中,同时是无穷无尽的。

智慧能应付瞬息变化的种种情事,但是却不为变化所动摇,甚而能超越一切事象之外。

智慧产生自不间断的正确行为。智慧能给予人们慈悲,如处在黑暗中的人点燃法灯。

所谓圣人,应当是一完成自我后继之为他人点燃法灯的人。」

「哦!我明白了,谢谢。另外有一个问题想请教。像我们这种出身婆罗门家庭而以祭事为生活中心的人,长久以来就被灌输了一个观念,就是我们只要能完成宗教工作,就能达到清浄安乐的境地。

有些前辈们更教导找扪透过严厉的苦行来达成目的。甚至还有人说只要凭着学问就可达到清浄的境地,请问这些说法能使人们由生死中获得解脱吗?」

「难陀,没有人能以那样单纯的方式达到悟境的。无论你用怎样珍贵的宝物或怎样丰盛的食品来供奉神,都无法打开一条光明之道的。

为什么呢?其实我们对神还有什么祈求呢?神已赐给我们一切的东西,譬如大地、水、空气、阳光、热等,几乎我们所赖以生存的种种元素,神都毫不吝惜地赠给我们了。如果谁觉得没有领受到这些,那只表示他自已没有创造出能领受它们的生活与环境罢了。

不靠头脑去理解道理,不靠身体去实践道理,人们终究无法培育出清浄的本心的。

要知道单靠外来的力量是没有办法解脱烦恼与执着的。

已出家修行的人们,虽不能说都是受生死束缚的苦恼人,但只要自己感受到这方面的苦恼,就应当惕励自己抛弃一切的错误观念,以及错误的行为等,舍弃欲念,在心中不受污染的情形下,自然就能超越人生旅途中可能遭致的严酷波折。

又如果能不为传统、习惯及教条所摆弄,经常生起疑问和探讨的心,运用内在智慧及勇气,这样的人最后终能领悟无边的真理。」

「佛陀,我明白了。藉着疑问和探讨,迟早可领悟到真理,我今后必定会朝这方面去努力。」

他们的问题和一般平民不一样,大多能触及问题的核心。婆罗门经典中本来也是阐述神义的,同时他们也都对经典有过相当的研究,但究竟只限于在知的一面的了解,目前他们的课题是,如何使经典中的神义复活起来,在他们的身上发扬光大。因此佛陀的话,在他们每个人的心上产生了无数的回响。每一宗教都提到「罚」,表明了神的愤怒。

人们因害怕触犯了神明,故而不敢对教义起疑问或追究探讨,只一味在畏神、敬神和祭祀神。

在婆罗门教漫长的宗教历史中,关于神罚的传说层出不穷,以致产生今天婆罗门教冥顽不化的固定形式。

婆罗门教原以梵天的神理为依据,但由于传教者的拘泥形式,使整个教义如尘封般教人无法明了。僧侣们唯恐有冒犯神的举措而一味热衷于祭事,渐渐地,婆罗门教变成一个只重祭祀的宗教。

沙罗阑陀一直对这现象抱持疑问。因此她问佛陀道:

「婆罗门教的教师们一再地告诫我们,如果怠忽祭事,一定会遭神的惩罚,但是神真的会惩罚这种事吗?请您指点我。」

「沙罗阑陀,你的双亲健在吗?」

佛陀温和地注视着阑陀。

「是的,我父母都健在。」

「你有兄弟吗?」

「我有弟弟和姐姐。」

「你想想看,你的父母曾不顾你的死活而陷你于不幸吗?」

「当然没有,他们一直很关心我。」

「沙罗阑陀,神就像人的父母,岂会盲目而严厉地把惩罚加在自己子女的身上呢?

惩罚是当自己的心和行为有错误时,由自己本身所制造出来的。神不会随便惩罚人。」

沙罗阑陀经佛陀这一点醒,立刻领悟过来。

神会发怒或会施罚,是宗教指导者为了便利传教而造出来的一种说词。利用神来威吓人类。有时候是传教者为了确立自己在信徒们心中的地位而强调神罚的说法。

迦陀康尼也接着问:

「佛陀,我想知道,您为什么要否定爱欲呢?」

「因为爱欲能使人的心化为一团火焰而将原有的佛性毁掉。爱欲之火一旦燃烧起来,人们就会失去自制之力,为了贪一时的快乐而走上错误的道路。再说爱欲会产生独占欲,使人成为欲望的俘虏。

在见到某一异性时,虽然没有实际接触,但心中已为之产生欲念,这也等于与对方通奸了一般。

爱欲会在心中投下阴影。

心中只要有阴影,必然会遮去原来的灵光,苦恼即由此产生。

因此,抑制爱欲并远离它,便可得到安逸与宁静。

但是,也不能因此就在心中造成束缚。

因为束缚会在心上留下一个结,这个结早晚会成为具体的行为展现出来。

所以为了去除烦恼的根,必需有勇气和决心。」

「佛陀,在家的修行者,又当如何处理爱欲呢?」

这个问题引起了众人的注意。

自离开巴巴里以来,他一路上与同伴讨论这方面的问题,但结论不外是「去请教佛陀」。

他此刻满脸通红,等待佛陀的回答。

「在家人的夫妻关系非常重要。只要行事得宜,很少有因爱欲而产生迷惑的。但是如果夫妻关系牵涉到第三者时,其中一人就会为嫉妒和憎恨所煎熬着。如此嫉妒与憎恨将腐蚀两者的关系而使当事人感受无尽的苦恼。

在一夫一妻的家庭中,知足而和谐的爱的生活是被允许的。也就是说,心中常存理智的人,丝毫不用担忧这个问题。

另一方面,一个本来立志独身修行的人,却变成了爱欲的俘虏,无疑地会被苦恼所缠,最后终将远离悟境。

一个曾有过爱欲的人,如能时常反省自心,领悟正道,不再犯同样的错误,日久也能使自己的心灵达于安泰、和平的境地。」

「佛陀,我想我明白了。

爱欲是藉由五官而产生出来的,更是一切烦恼的根本,真谢谢您!」

迦陀康尼逐渐恢复平静,自觉心中所隐藏的对异性的憧憬,也逐渐在缩小,而至消失。

接着脸孔长长的迦波询问佛陀说:

「佛陀,我们应当如何济助那些在恶劣环境下苦恼的人,或在生死间徬徨的人?」

「迦波!在苦恼的泥沼中喘息的人,是为什么而苦恼呢?唯有努力除去苦恼的根源,依据正法来去除心中的执着,抛掉一切的束缚。

在修正了身心和行为的错误后,人生才会充满光明,整个人才能脱离魔鬼的控制而获得真正的自由。」

「佛陀,我知道了。我应该向正道迈进,先完成自我,再去为煎熬痛苦的人解决问题,在他们心上点燃法灯。谢谢您!」

最后是身体健壮,领导众人前来的频迦发问。

「我已年迈,双眼模糊,又患着重听,即使有心藉着以往的修行功夫,也无法使心灵获得安逸了。我不想在未成道之前就离开人世,请您指引我一条路。」

「迷惑于五官的人,首先应知道心的性质,要时时督责自己不沉溺于感情中,不被爱欲所玩弄。

不应该为着自己得到的一点知识就骄矜自满,应经常藉着正确的道理来调和心性。在思想和行为上都离不开正道,并过着清浄的生活,日久必能培养出更丰满的心。

如果你已为自己的衰老而苦恼,就更应该勤奋地去实践正道,设法抛弃一切欲望,守住一颗清浄的心。如此日积月累,必有领悟人生真谛的一天。

你日后要回巴巴里身边,向他报告在此的经歴,也算是向老师报恩。巴巴里正在等你回去呢,你要好好修行,以便回去看望他老人家。」

「佛陀,真谢谢您教给我们实践的道理。从前我总想好好研究婆罗门教的经典,结果发现里面尽是一些理论而没有实践的方法。现在我已清清楚楚地明了到心的伟大,回到故乡后,我一定要把这一点向老师报告。我不但自己去实践您的教诲,也要将佛法传给其他的婆罗门知道,为他们点燃法灯。」

佛陀点头称赞道:

「你能这样想,非常好。」然后佛陀再环视围绕在身边的人,又说:

「今天的问题就暂时问到这里,请各位好好体会每一个问题及其答案,切实将法做为心行的依据,如此必能走上解脱之道,达到安逸的彼岸。希望各位不要懈怠,要努力去做修行的功夫。」

佛陀和十七位婆罗门的谈话,就此结束。

他们对人生有着各式各样的看法,同时有着无法解决的难题。在遇见佛陀前,他们不知如何面对那些问题,也不知如何使自己的思想往前迈进。

遇到佛陀后,他们了解到心的伟大。

他们虽是巴巴里在弟子中挑选出来的精英人选,但平时在巴巴里的教诲下,也只是将经典上的东西囫囵吞枣似地接受下来,每遇难题,就相互争论,但也争论不出个所以然来,最后还是不求甚解,不了了之。

关于知识的吸收,他们一向很自负,皆属博闻强记者,从未输给其他的修行者。但是现在遇到佛陀,才知道自己一向所知道的是多么空洞而无实物,尤其当他们的智慧渐从心底涌现时,更是惊讶不置。

太阳已完全西沉,灵鹫山与天际的分界限已越来越模糊,他们的周围,弥漫着昏暗的凄清。

为佛陀和大众带来温暖的薪火,正猛烈地燃烧着。

每个人的脸庞,都辉映着薪火的反光。

佛陀考虑到四个女子的处境。

因为当此时,佛陀的僧团尚没有女修行入团的例子,出家人全限于男子。但这四个人自过去世就和佛陀有着深厚的缘份,她们此刻虽然身为女性,万没有不能修行的道理,只要志节坚定,是不会输给男子的。

在佛陀之前,人人平等,自然包括性别的平等在内。人类社会由于男女的正确生活方式而得以成立。一个教团如果不允许这种性别差异的存在,就是不自然。

固然说佛陀的僧团皆由出家僧所组成,出家意味着修行传道,但如果佛法只限于使出家人受益,则法不论历经多少时日,都将无法存在于人类社会中。

法不是少数人的专利,是全人类应当共享的。

那是上天的慈悲赐予。

因此佛法的宣扬,要能及于社会大众,才能显其价值,使人类社会达于真正理想的境界。

佛陀的僧团即是使法及于全人类的先锋组织,世俗的弟子们,亦能因他们而将法的种子散播开来。

由此意义观之,若将男女的能力加以区别,就是一种不自然的作为,凡志操坚定的女子,仍可成为传道的中坚份子。

于是佛陀允许这四个女子加入僧团,但指定她们与男修行们分开修行,为避免他们因接触频仍而产生不清浄的念头。同时她们负责在佛陀说法前去召集大众。

除频迦外,十六位修行者很快地理解了佛陀之法,而一一地达到阿罗汉的境地。

频迦则在经过几天的休息后,回到巴巴里身边。

一见到巴巴里,频迦就迫不及待地告诉说道:

「老师,释迦牟尼是真正大澈大悟的救世主,他拥有清浄而充满智慧的僧团,他所说的法皆属于他个人的亲身体验。

他领悟到少欲知足而没有执着的生存之理,他能立刻视破对方的真伪。

他是如此清浄圣洁,同时态度一点也不骄矜自满,他始终是那么温和亲切,对我们的问话都不厌其详地加以解答。

他能自由自任地透视过去、现在和未来的一切,他的心中已没有的阴影,他能遨游于宽广的实在界中,更由于他的智慧力,他已超脱了生死的轮回,断掉了世俗一切的烦恼。心境是如此平和,不会轻易为身边发生的一切而动摇。他说出来的话,一宇一句都是真理。他真是一位伟大的佛陀啊!

佛陀说话完全凭借他自己内心涌现出来的智慧,他不需要引经据典。

我总算找到了一条真实的道路。

我现在就像是久旱逢甘霖般,心中的干涸已被佛陀的法所浸润。

佛陀的周身时时被一层层美丽的、淡黄色的金光所包围着。

佛陀是个凝聚了所有的智慧、宁静、安泰和慈爱的人。

佛陀除去了我心中的阴霾,他的法,使我的心中充满了光明,根绝了我心中的恶念,消灭了一切的灾危。

他使我感受到从所未有的安泰。

我由衷感激老师您挑选我前去接近佛陀的一番恩泽。」

频迦一口气把话说到这里。巴巴里则边听边点着头,心中涌起无限的感慨。他想,自己如果不是年龄这么大而不艮于行,也就有机会亲自见到这样一位伟大的觉者了。

但他继之一想,自己的志向也能由于弟子们的皈依正法而得到报偿。

「频迦,你为何离开佛陀而回来了呢?我已不久于人世,没有什么东西好传授给你了。

我虽然先前说过你要回来报告经过情形,但你如果继续留在佛陀身边修道,不是更好吗?」

「老师,您这就错了;佛陀的法已在我心中生根了,它正奇妙地发挥作用哩,它已使我从一切执着中超脱出来。

我能以法做为根据,用心眼来礼拜佛陀,因此不论我身在何处,我都能与佛陀心心相印。

我的年纪也不小了,无法再回到佛陀身边,从今以后我会依佛法继续精进。」巴巴里闻言,非常高兴,当下弯身趋前,握住频迦的双手说:

「点燃心中的法灯,脱离一切苦恼以达彼岸,真是太好了。我也要紧记佛陀的教诲,努力精进,以到达涅槃的境地。」

说着,说着,巴巴里禁不住流下眼泪。

「老师,真正的佛陀,他的心就是上天的心,他就是慈悲的化身,使法原原本本地显现出来,毫无粗俗的态度。

佛陀领悟到一切现象的因果关系,因此能明快地为我们解答一切疑难。

老师啊!我真感谢您推荐我去皈依伟大的佛陀。」





频迦说完,也在巴巴里膝下痛哭起来。

巴巴里任由频迦哭着,也不劝止。他的眼光落在频迦的背上。

巴巴里的面貌,显得如此细微而高雅。

频迦的身裁魁梧,仪表也相当堂皇,因此二人在一起,简直分不清谁是老师,谁是弟子。

巴巴里常年拥有众多的弟子,他的知识丰富,是其他婆罗门难望其项背的。他拥有极慈悲的心怀,时时关怀弟子们的起居与健康情形。

如果不是他坚决主张让弟子们去皈依佛陀,弟子们是不会轻易离开他的。

与其说他们有着师徒的关系,倒不如说他们像亲子般彼此紧密地连系着。任何细微小事,弟子们都不忘请示老师。

因此不但是重回故里的频迦,就是那十六个已皈依佛陀的弟子也日夜悬念着家乡的巴巴里,他们因而更努力地修行。

巴巴里由频迦这儿认识了法,以法做为心中的支柱,纠正了往日不自觉犯下的错误。

不久,他也将思想提升到更高的层次。

巴巴里,后来被称为伟大的婆罗门之师——阿闳如来。

回归故里的频迦,也因不断反省改正了自己的行为,最后也达到阿罗汉的境地。其后,他开始在婆罗门种间传扬佛法。

帕拉那西的修行者们因频迦的说法而将法做为精神粮食,在精神上皈依了佛陀的僧团。

他经常与耶萨保持连系,并在喀西国为众生说法,再度将佛法传扬开来。

\chapter{第六章\ 教团的产生}\label{ch6}

\section{祗园精舍的捐默}\label{sec6.1}

在憍萨罗国都城舍卫城的大富豪中,有一位被称为「给孤独长者」的人。他名叫须达多,颇受憍萨罗王的信赖,同时也受到众人的景仰。

须达多为一些因战争而失去父母的孤儿们建了一所孤儿院,同时经常行着类似的慈善活动。

某日,他远至摩竭陀国拜访他的义兄迦兰陀。

当他走进迦兰陀的宅院,看见许多仆役与佃农正在庭院中忙碌地工作着,好像在迎接什么人。

他想可能是迎接频婆娑罗王吧。

以往每当他一进门,就立刻有人迎上前来,今天却没有一个人看到他。

他望着来往忙碌的人。突然有人拍了一下他的肩膀,他掉转头,原来是迦兰陀本人。迦兰陀笑着说:

「呀,兄弟老远从憍萨罗国来,失迎,失迎,真是抱歉,你一定累了吧,赶快到里间来歇息。」

须达多轻轻点头道:

「老兄,院里这么忙碌是为了什么事?是不是频婆娑罗王要下塌此地?看他们都在堆柴火,这时来打搅。	」

「 哪里的话,你来得正好,不是大王要来,我们正在迎接佛陀。我要布施餐食给一千七百位的僧众,今天就是布施的日子,所以正忙着。」

须达多露出一脸的惊讶。

「你是说真正的阿波搂门依迭•秀巴腊吗?」

「正是,真正的阿婆罗珂啻•修婆罗,就是指在婆罗门圣典中提到的那位能洞视过去、现在和未来的悟道者----佛陀。

「我也常听国内的一些婆罗门修行人说伟大的秀巴腊出现了,想不到真有这回事。这机缘真正太难得了。」

须达多一直就希望能见到佛陀一类的圣者,如今听到义兄肯定的言谈,心中有如波涛般汹涌着。

须达多被请至宅邸大厅。

二人面对面坐下,喝茶、歇息。

「老兄,佛陀教导众生哪些方面的事?」

「他直接指出人生苦恼的原因何在,同时教我们如何袪除这苦因。他指出八条正道,要众生去实践,只要一心一意向正道走去,终有解脱苦恼的一天。八条正道能通达到人的本性中,依这本性活下去,日久就能体会到佛心,于是就能从痛苦与悲哀中真正超脱出来。人的苦恼皆起于假我的欲望。我坚信佛教所教的是真正的道,我想我如今之所以能过如此安适的生活,都因为我在日常生活中已活用了佛陀所教的法理。」

「不错,至善无伪的心确实深藏于我们的内心深处,那个心是很重要呢!

「是啊。如果说只有婆罗门种姓才是神的使者,这真是大错特错。过去我只对婆罗门行布施,但是佛陀教导我们,说人是平等而没有差别的。我认为这才是真正的道理。

婆罗门有名的修行者优楼频罗迦叶圣仙也都舍弃了自己的信仰而皈依了佛陀。他真是个了不起的人,他的弟子们也都一齐皈依了佛陀,真不可思议。」

「哦?优楼频罗迦叶舍弃了自己的道场去皈依佛陀吗?	既然是这样,这个





佛陀一定是真的了。他是哪一国人?」

「是迦毗罗卫的悉达多太子,是个刹帝利出身。听说悟道前吃了不少苦头。」须达多自始至终都全神贯注于义兄所说有关佛陀的一切。

他不断思忖着,虽然他很早就听说婆罗门中会出现秀巴膜,自己也一向对婆罗门抱着这样的期望。但他眼见许多顽固不化的婆罗门,只知沉浸于自己优越的地位中行礼如仪,这样的组织中可能出现真正的觉者吗?尤其他们在修行时,都是藉助他力,只图君临于其他种姓之上,这都不是神所要揭示的教义,即使婆罗门的教义也深具意义,但婆罗门僧的行为跟教义也不一致,他们给人的感觉只像是没有灵魂的木偶罢了。

真正的教义往往并不显现于宗教固定的形式中,很可能会在一个门外汉上显现。

这个佛陀必定是真实不虚的了。如此一想,须达多兴起了一见佛陀的渴念。「老兄,佛陀在哪里说法?我也非常想见他。」

迦兰陀从须达多的眼神中已看出他的心事。

「我在竹园内捐建了一座竹林精舍给佛陀和他弟子们。频婆娑罗王跟我都认为佛陀需要一处设备完善的布道场。为了让更多的众生听法,你也布施一座精舍如何?」

「我正有此意。明天一早我想亲自到竹林精舍去见佛陀。老兄,我此趟能见到真正的佛陀,真比挖到宝物还要高兴。佛陀真是这世上无上的至宝啊。我真要好好谢谢你。」

须达多只是从迦兰陀这儿听到有关佛陀的事,就已经兴奋得不能自已,脑中浮现出即将布施的精舍的形象及地理位置。

在迦兰陀宅中住了一宿,翌晨第一声鸡啼时,须达多即从床上跃起,整完衣装即向竹林精舍出发。满布雾水的野草地很快就沾湿了须达多的双脚。四周弥漫着浓雾。

一踏进竹林,须达多犹如置身幻境般,不辨东西,一刹那间,孤独与落寞袭击全身,他每天在众人的簇拥下过着优越的生活,只有在单独决定某事或入晚上床睡觉的一瞬间,才会感到孤独。

在一天之中,会产生孤独感的时刻是相当短暂的,在他的周围,不分昼夜,总是围绕着许多人。因此,像此刻,独自置身一片茫茫的为浓雾笼罩的阴湿林中,是他从未有过的经验。孤独的感觉一阵一阵地袭来。

由于夜雾的浸沾,竹林的叶片都沉甸甸地垂着。

他不自禁地想到,人究竟本是孤独的呢?还是群居的呢?

一个真实的人,究竟应过那一种生活呢?

就在他的右前方,突然出现一个朦朦胧胧的身影。原来是个修行人正迎面向他走来。

是梦?是真?他竟然看到这位修行人的头部为光圈所环绕。

一个念头闪过脑际:这个人难道就是佛陀吗?

他开始心神不宁起来。

待走到修行人面前,他不自觉地喊道:

「佛陀!佛陀!秀巴腊!秀巴腊!」

佛陀回答道:

「喔!你这么早就来访了吗?刚才在林中散步时,就感觉这里有人。你是憍萨罗国的给孤独长者吗?」

须达多很奇怪佛陀知道他的名字,于是回答道:

「是的,我本名叫须达多,大家称我给孤独。迦兰陀是我的义兄。今天很荣幸在这里遇见您。」

他伏在地上,于顶上合掌,以示敬意。

佛陀和霭地看着态度真挚的须达多,说:

「快起来吧,到精舍来。能对说法的人行布施,功德是不可限量的。在心中燃有法灯的人,就能转生于极乐的境界,你哥哥是个了不起的施主。」

「佛陀,我也想在憍萨罗国捐建精舍,希望你也能到舍卫城来开导众生。请佛陀接受我的布施,佛陀……」

「须达多,能为说法的人提供场所,这功德是非常大的,这功德将使你的家人也能获受法益,你的子孙将得以繁衍,天堂之门也将为你而开。」

须达多将佛陀的话铭记在心。当他和佛陀商量好建筑精舍的程序后,回到迦兰陀的宅第。

「老兄,佛陀真如你所说能一眼看穿一切。关于买卖的事,我们改天再谈了,我要赶回去找一处可以建精舍的地方。我这就告辞了,真要谢谢你了。」

「你要捐建精舍,真是太好了,但也用不着这么急嘛,在我这里歇息几天再上路吧!你不是很累吗?」

「老兄,真是奇怪,我现在一点也不累,大概是受了佛陀的保佑吧!你既然这么说,那我明天一早再出发好了。」

「那太好了,今晚可要好好地休息了。」

二人又谈了一会儿有关佛陀的事,然后各自就寝。

第二天一早,须达多就上路回国了。

一抵家门,就开始四出寻找捐建精舍的用地。

他看了许多地方,最后发现只有祇陀亲王所拥有的园林最合适。

于是须达多不假思索地就去见亲王,希望能说服亲王出让部份的园林。

当他见到亲王,免不了将自己的一番心意呈明,同时描绘佛陀传教的事迹,以及说法救渡众生的重要性等。

祇陀亲王由于太喜爱这座园林,而无法感受到须达多热忱捐献的心清。最后他甚至不耐烦地吓阻须达多道:

「须达多,你实在非常唠叨。为了一个叫释迦牟尼的人,就要我出让我心爱的园林。我可是一个有钱有势的人,你要认清楚。如果说这个释迦牟尼要献一块土地给我,那还说得过去,现在竟要我来出让土地给他,你不是在说梦话吗?你如果真有诚心,你就去拿财宝来铺满我的园子,等铺满了再说。」

亲王只不过是想用这种手法来打消须达多的念头。

不料须达多不以为意地回答道:

「亲王,我明天就用财宝来铺这个园子,请您多多包涵了。」

亲王十分不解,对须达多说:

「你真是一个多事的人,你就试试看吧!如果你做不到,我就让你的脑袋和身子分家,听到没有?须达多?」

须达多回到家里,立刻将家中的金银财宝搬出来,然后用车子运到园林,开始将财宝铺在林地上。

看到这一壮观的景象,亲王也不知如何是好。

在他的眼前,各式各样的金银财宝闪着亮光。

他对须达多的富有早有所闻,只是未曾亲见他财富的众多,同时他对须达多那种一掷千金的慷慨作风也很感赞佩。须达多此刻撒金子就像撒水一样,态度冷静而平和。

须达多能如此轻视财富,足见是个不同凡响的人。

祇陀亲王终于忍不住说话了:

「须达多,我现在了解你的心意了,同时非常佩服你为义舍财的慷慨行为,就将那些财宝用来建造精舍吧!这座园林任你使用吧!」

「从一切执着中脱离时,就充满光明--」佛陀所教诲的道理,须达多在这里得到印证了。

用地既经决定,接下来就是精舍的建筑问题。

他找来了从前建筑竹林精舍的木匠,并在本地雇用了一些出色的建屋人才。工程的负责人,经佛陀指派了目犍连前来。

因为所建造的是佛陀将驻足说法的场所,所以建筑材料供应自许多地方,许多人也都很高兴地参与了精舍的各类建造工作。

精舍就在众人的协力下完成,是如此的美仑美奂。

通常像这类浩大的建筑工程,在建造过程中难免会有人因不慎而遭意外的伤亡,但是在建造这座精舍时,就没有这类事情发生。连当初漠不关心的祇陀亲王也捐出大门及园林中所有的树木,就是波斯匿王,也成为精舍建造时的支援者。

憍萨罗国王-波斯匿王—对佛事的赞助很快就传遍全国上下。

净饭王很快地也风闻了此事。

这消息来自阿私陀仙人的外甥卡迦那。

净饭王因着爱子的种种传闻,又升起无限的亲子私情。

他立刻派出使者,传达他的思念之意。

但是佛陀的回答仍和初出家时一样,非常坚决,他说:

「相聚的时机尚未成熟……」

佛陀一点也没有伤害父王的意思,只因目前他实在分身乏术,要为广大的皈依众说法,他知道等时机一到,他总会回迦毗罗卫城一行。

佛陀再次向来使说明自己此刻无法随行的苦衷。

净饭王得知这样的答案后,心中久久不能释怀。因为在他心中,他仍念念不忘佛陀那做为太子的身份,他很执着地认为佛陀是他的儿子,同时更想着再度生活在一起的想望。

\section{雨季来临}\label{sec6.2}

印度是个突出印度洋上的倒三角形半岛,因为是半岛,常被误认为是个小国家。但是她的面积约有四百四十二万平方公里,人口约达六亿,由此可见印度并非一个单纯的半岛国。她的北边是具有「世界屋脊」之称的喜马拉雅、喀拉昆仑大山脉,东边是阿拉干大密林地带,西边绵亘着不毛的兴都库什山脉,成为自然的大屏障。

依地形与地质来看,印度大致可区分为南部的德干高原,中部的印度斯坦高原,以及北部的喜马拉雅山区三部份。这些地域、气候风土互异,生活方式及语言系统也迥然相异。最古的居民是生活在锡兰和中印度密林的未开化民族,接着南部有很多的Dravider系人来居住。到纪元前二千年左右,阿利安系出入于西北部。同时西藏、缅甸系的黄色人种也逐渐移来,故而印度的人种也是相当复杂的。

在语言上,印度语共有八百多种之多,今日公认的有七十种,其中主要的语言又被浓缩为十五种,包括英语在内。

语言大致分为欧洲语(阿利安语)、Dravida语Munda诸语、西藏语、以及缅甸语等。其中常用的语言是阿利安诸语和Dravida语(以锡兰和德干高原为中心)。

语言不统一,容易造成意志的分歧,更是发生争执的主因,因此印度在成为英国殖民地以前,经常有战乱,纷争无已时。

印度的历史始于阿利安人移居印度河三角洲,约纪元前二千年至一千五百年。阿利安人最先是从埃及、希腊、伊朗等地迁移而来,在西巴基斯坦北部的印度河三角洲,就靠农耕定居下来,如此再渐渐移至印度的北部和西部。

阿利安人因种种自然现象而认识了神,不久就创出婆罗门教,并以梨俱吠陀这一基本经典在印度河及恒河流域教化众生。

但是日久以后,婆罗门僧拘泥于经典中的教义而自视非凡,以其权势将社会化分为四个种姓,形成严格的阶级制度。

就过往的人类史来看,当伟大的圣者出现后,人们就有机会处在较统一而和谐的社会中,然而一旦其灵光升天后,地上立刻又呈现一片混乱。而人类的行为就离正道越来越远。所以许多人观此景象,总不禁要哀叹一声,怀疑人类为何还不被上主所拯救。

这不是上主所能拯救的,这一切现象的成因皆来自人们自己带来的「业」。「业」是人类对物起执着并造作出行为的一种恶性循环现象。如果个人的「业」,在转生过程中不立刻加以修正,就无法脱离造成的各种轮回,扩及整个人类社会,若多数人皆不知修正自己的「业」,社会的混乱现象就不会消失。

「业」在遇到机缘成熟时,才会让人觉察它的存在,因此许多人在不自觉的状况下造业,也在不自觉的状况下收受苦果。在佛经上,因而将人世化分为所谓的「正法时代」与「末法时代」。

正法时代是圣人出世,教化众生,使众生了解「业」的真相,使人们不断修正自己业缘的一个渐入条理的时代。反之,末法时代的人,则多半忽视「业」的存在,并从而不断造作新业,将整个社会搅得昏天黑地。所以说,在末法时代生活的众生,要想接近正道,也是很不容易的。

说起来,人类生存的目的,即在于不断在转生过程中修正自己以往盲目造下的「业」,以期脱出生死的轮回,使自己的灵魂真正得到解脱,否则就只有再度盲目地陷入社会众人盲目造下的「业」中,在末法时代过着暗无天日的生活。

末法时代之所以混乱,另一原因即在人口的激增。今日较百年前增加了近四倍的人口,今日人口的增加率,更是千年前人口增加率所无法企及者。由于地球上的人数日众,面临到粮食及居地的分配的问题。因为粮食的增加永远赶不上人数增加的速度,而地球上的可居地也只有一定的限量,故而末法时代的众生不时要起争执,国与国之间的政治关系益形复杂,都跟人口的激增有直接的关系。

此外,就灵魂存在的观点来看,灵魂存在现象界中的时日越久,越能获取生存的经验以达和谐的境地,反之,就易受现象界中种种的物象愚弄而过着是非颠倒的生活。

这是一个很浅显的道理,反观我们本身,也可以看到这个道理。当我们在幼年期与少年期时,心地纯洁,对社会众人之事非常懵懂,整日关切的只是围绕在自己身边的事,非常的自我中心,但是年事渐长,进入青年、壮年而至老年,就能把眼光从自己身上移向他人,以致其他的社会,了解到众人与自己的密切关系,也才能把握住人生的面貌,知道如何与人相处,如何过较合理较和谐的生活。

当然就人的个体而言,一个人在生长的过程中,是属理性的,还是感情的,还是跟他个人在转生过程中的修行程度有关的。换句话说,一个人的处世经验深浅与否,还是看他对己身「业」的修正有否努力而定的。

同理,藉由人类社会的经济问题和社会问题来看,一个社会的是否安定调和,实跟它的组成份子的灵魂有关。

人口的激增,意味着人类社会目前正充斥着许多处世经验很浅的灵魂,亦即自我中心的灵魂数极众,故而社会易趋于混乱,也是可以理解的。

正法与末法,在人类历史的演进过程中,有如钟摆的摆动,是一种循环的现象。

人口的增加将至顶点,那是可以预料得到的后果。有许多灵魂由于处世经验愈深而能修成道果,转生菩萨界,这情况也是可以想见的。所以在佛经或其他圣典中,预期到理想国土(佛国土、净土)的应运而生,绝不是一种空想,也可以说,圣典已描绘出人类未来世界的蓝图,让人类照着一步步实现罢了。

菩萨界的众生,即是有着菩萨心肠的人,在人类社会各阶层中试图引导人们走向调和之境。

佛陀出生于二千五百余年前,而耶稣传道,亦在二千年前,至今他们的正法仍被人类转相传述着。他们的目标也无非是在领导人们到达理想的国土。今世的我们担负了此一职务,未尝不可说就是未来菩萨界的一份子。

有的人由于正法与末法的反覆出现,就视人类的存在为一持续的现象(不再有所变化),这种看法是相当没有远见的。人来自自然,终将回归自然,这现象就是再幼稚的灵魂也能理解的。但是人类,由于套上了肉体这一外壳,心灵就完全被蒙蔽了,因而不明是非,更由于累世积聚的肉体经验而更加的黑白颠倒。所以一个人如果能在肉体之下了解到心灵的伟大,将心灵与肉体的关系了解透彻,就能领受到身心和谐的美妙,灵魂便能达到物我两忘的绝佳境界。

言归正传,印度属于热带性气候,雨季在六月至十月。十一月至五月为干季,而干季又分为冷凉和暑热两期,尤以三月到五月之间最为酷热。

雨季的开始,依地区而异,降雨量也依地形和风向而有多寡,譬如以喜马拉雅山系为中心的尼泊尔及印度北部降雨量就很多。在雨季期较长的地方,降雨量可达二千公厘以上。

佛陀所在的北印度降雨量相当多,到了雨季,布道活动就无法进行。

乌云开始聚集于灵鹫山头,当乌云一遮住阳光,地上立刻灰蒙蒙一片,周围的树木随风沙沙作响。

佛陀背依大石,正在禅定。思念着远方的舍卫城,而舍卫城离他的家乡迦毗罗卫国很近。

目犍连所督导的精舍建筑工程,此刻正在进行中。

佛陀不断揣想以后布道的情形。

由于特殊的因由,波斯匿王(憍萨罗国王)及祇陀亲王也都提供了许多建材,使得精舍的建筑工程进行得相当顺利。

精舍的东侧,拟建造一个东门,这是祇陀亲王捐建的。这个精舍距迦毗罗卫国很近,故佛陀时时在悬念中,希望精舍能顺利如期完成。

佛陀轻合双眼,意识明了于舍卫城的一切。此时舍利弗走近佛陀身边,将厚僧衣搭在佛陀的肩上,并轻声说道:

「佛陀,快下雨了。我想今天还是下山回竹林精舍去吧……」

佛陀睁开眼睛,凝视着前方说:

「外出游化的弟子们也都回到精舍了吧!我们在天黑前回去。」

说完,就从位上站起来。

周围已暗下来。天空布满乌云,眼看就要下雨了。

「等雨季过后,我们要往憍萨罗国去一趟,这是一段长途旅行,舍利弗,你就安排一下同行的人。」

佛陀边走,边笑着回头对舍利弗说。舍利弗回答道:

「好的,我回去跟憍陈如商量一下,等安排好再向您报告。」

随行的憍陈如听舍利弗如此说,就回过头,与舍利弗交换了会意的眼色。

三人在下山的途中遇到了大雨。

大颗大颗的雨,猛烈地敲击在干涸的红土上。大地立刻为一层砂烟所包围,由地面扬起的砂烟,毫不留情地袭人的眼睛、嘴巴和鼻子。如果脸部不用头巾裹住,就有不适的感觉。加之地上泥泞滑溜,更使人举步艰难。在雨季期间,像这样的倾盆大雨会不断地下着,山路往往也被雨水打成瀑布一般,道路立刻变成河川,平原地带更为洪水所泛滥,人与动物皆有被冲走的可能。平时空旷平坦的草原,此时满布浊流,故不谙地形的人,是不宜于雨季中出门的。

雨点就这样无情地打在三人的头和肩上。

竹林精舍遥遥在望,三人的心情未受到大雨的影响。

「耶萨每隔三个月会来和我碰一次面,你们可曾听说他由帕拉那西回来了吗?

「佛陀,耶萨是该回来了。耶萨的弟子们已跟我连络过。耶萨大概会在今明两天内到达。」

憍陈如恭谨地回答着。

武士出身的憍陈如眼光锐利,体格魁梧,气宇不凡,已完全拥有修行者的安泰,连说话的语气都有了大大的改变。

人常由心的改变而有所改变,心是诸相的根源。

这就是憍陈如跟随佛陀修行了十二年所得的成果吧!

佛陀的另一弟子比佛陀早一步回到竹林精舍,并向留守在精舍中的弟子报告了佛陀归来的消息。

于是精舍中的弟子们都齐集在大门口迎接佛陀。

穿着黄色僧衣的弟子们排列整齐地跪在大门边。

佛陀遥望到此一感人的情景,快步通过竹林来到众人面前。

「大家辛苦了,请快到里面去,不要淋雨了。来吧,大家快起来,到屋子里来。」

佛陀即领着大众走进精舍。

阿舍婆誓立刻取水来替佛陀洗濯沾满泥尘的双脚,并在前方为佛陀的入座开路。

弟子们立刻一个接一个地列队坐在佛陀面前。

此时雨势更猛了。屋顶哗啦啦地发出倾盆之声。在精舍未建造以前,佛陀和弟子们只能在空间有限的岩洞内避雨。由于空间有限,故而许多弟子都散居于其他岩洞中。如此在雨季中,许多弟子们便没有机会听佛陀说法了。

现在有了精舍,弟子们就能风雨无阻地听受佛法,获受无上的法益。

大自然有时之于人,的确是严厉无比,但是人类如果能巧妙地与大自然相处的话,不但能完完全全地感受到大自然的慈悲,同时也能由此了解自然的真谛。在无法遮蔽风雨的环境下是很难与大自然和谐相处的,然而人类究竟异于禽兽,他与生俱来有一种创造的本能,因此能在险恶的环境中发挥这一创造力,从而改善了自己衣食住行方面的条件,得与大自然和谐相处。

「在游化途中有否碰到生病或被毒蛇咬伤一类的事?」

佛陀一面环视大众,一面慈祥地发出询问。

千百个修行僧经此一问,互使会心的眼色,接着望向脸色红润的佛陀,兴味盘然地争相报告个人游化的经历。所幸人人都平安无恙。

「大家经过长途的跋涉,一定很累了吧!有的话一时也说不完,今天还是让各位先好好休息。

调和健全的身心,是法的根本原则。希望各位能在这段雨季期间好好将自己游化时所得的体验做为心灵的粮食,努力塑造自己,使自己的心量更宽广,心性更慈悲。

明天以后,再为各位解答一切疑问,今天就到此结束,大家各自休息去吧!」大厅内由于灯烛的照射,明亮无比。

佛陀脸上的慈光与灯光相辉映,周身依然为层层柔和的金光包围着。

佛陀此时光灿无比,修行僧们见此光景,都不自觉地双手合掌,面现虔诚。

\section{精舍内的法会}\label{sec6.3}

整个精舍笼罩着佛陀的光明,因而精舍上下充满一片蓬勃的朝气。

向佛陀合掌致敬的修行僧们,精神肃穆,大厅内寂静无声。

「舍利弗,憍陈如,丕葩利•桠那你们集合各组的负责人,一同商量今后的行事计划。其他的人可以解散了。」

佛陀说完,起身进入里间。

舍利弗等依言将各组负责人召集前来,其余的人就各自回房了。

在竹林精舍中,不论是饮食起居或洒扫洗濯等工作,都由修行僧们自己担当。并没有刻意使新加入者担任某一特定的工作。因此即使是老前辈,只要有空,也有担任清扫和烹饪等工作的机会。

这是其他教团所没有的现象。佛陀特重个人的自由,他不主张用特定的规章来束缚人,他认为如果是那样,就等于把人心也束缚了,如此就没有机会了解到佛心。佛陀因自身的体验而领悟到这个道理,他很怕他的教理会流于形式,故特重自我尊重的做法。

同时若培养修行僧们的自律行为,就能在无形中激发各人潜在的创造力,从而产生众生平等的观念。

雨势越来越猛了。倾盆而注的雨,将精舍团团围住,视野为浓浓的水雾所遮断。道路成为激流的河道,宛如狂舞着的生物。在此壮观抑或恐怖的自然景况下,所有的人、动物、植物只有寂静地置身在时间的潮流中,任由大自然喧腾着。

佛陀回到房内,一面倾听激烈的雨声,一面惦记着尚未回到精舍的弟子们。他静静地躺着。

翌晨,佛陀醒来,不为雨声所动地进入禅定三昧。

在三昧境界中,佛陀感到无比的安适,他将打在屋顶上的雨声视为天女有节奏的琴韵,在其中享受着乐声的乐趣。

佛陀倾耳谛听,这样悦耳的和声,是他曾经在禅室中体验过的。

用肉眼来观看,像这样激烈的雨,无异于狂呼怒号的野兽,使人心悸,但是用心灵的耳朵来倾听时,它们的旋律就像来自天庭的仙乐一般。

在漫长的干旱季节里,草木枯萎,地上的生物相继衰亡,然而激烈的洪流,又会把动植物冲走。一滴水的跃动与否,操纵了地上物的生杀大权。因此人类当悉心探讨宇宙间的奥秘。

佛陀倾听着仙乐,不觉绽开和悦的笑容。

有人敲门。

「请进来。」

佛陀依然闭着眼睛。

阿舍婆誓推开门进来。

「噢!阿舍婆誓,昨晚睡得好吗?	」

「好,佛陀。也许是在看不见星星的屋顶下睡觉吧,我睡得非常熟。这跟从前的武士生活大大不同,我现在每天都在平和的心情下渡过,心情很愉快。佛陀,这要谢谢您!」

十二年前的阿舍婆誓是守卫迦毗罗卫国的武士,体力充沛,武术高明。

他负责监视入城的商人和杂役等,并防备从他国潜入的间谍和地下工作人员。因此他必须时时提高警觉,终日将神经抽得紧紧的。

但是今天,他的生活中已经没有敌人,外表也不再去刻意地修饰,就像一个赤裸裸的新生儿,一切都焕然一新了。

「阿舍婆誓,端正的思想与行为,会改变人的价值观,使人培育出更丰盛的心。因为没有了苦恼,心神始终安泰,光明照耀着前途,修行的喜悦,一旦融入生活中,生活便一天天地充实起来。

法存在于实践之中,因之法灯在心中点燃后,我们就能处在不生不灭的境界中。

心中没有法灯的人,都由于有着错误的思想与行为,终生过着苦恼的生活。我们的教团有一目的,即在于救助那些对世事有着错误观念的人,将心灵的粮食赐予大众。教团中的僧众,必须像太阳一样,将光明不分彼此地赐给大众。

要时时点燃心中的法灯,不要陷于骄矜自满之中。」

阿舍婆誓跪在佛陀跟前,频频点头说:

「佛陀,您就像太阳一样无分厚薄地用法灯点燃了失道者的心,而我却只沉浸在自己的法喜之中,在这段雨季期间,我要努力精进,将心量扩展开来。」

正当他咀嚼着佛陀的言语时,他想起一件事。

「啊,佛陀,弟子们已在会场上集合好了,恭候佛陀前去说法。」

「耶萨还没有到吗?」

佛陀说着站起身来,往会场走去。

穿廊衔接会场与佛陀的房间。由于吹进来的雨丝弄湿了走廊,唯有佛陀走过的地方是干的。

阿舍婆誓小心翼翼地跟在佛陀后面。

在会场上静候着的修行僧们,见佛陀出来,立刻以最尊敬的礼仪来迎接他。佛陀一面注视大众,一面以心眼审视他们心灵的调和状况。

然后佛陀微点着头坐下来。

「修行僧们,请抬起头来!」

在会场最前排的有丕葩利桠那、优楼频罗迦叶兄弟、提舍波、舍利弗、憍陈如、阿舍婆誓、多达加及跋提。

由于他们的心灵都很柔和,头上均闪着光亮,同时周身为光芒所围绕。

坐在后方的修行僧们,头部周围也淡淡地泛着光芒,整个会场为一片庄严的气氛所包围着。

佛陀身后的光格外明亮,和昨日一样,修行僧们为这耀眼的光合掌致敬。

「依据正法来游化并过着正常生活的修行僧们,当可真正体会到生命的意义。希望汝等奋发精进于正道。

但是你们中间还有人无法舍弃旧日的修行方式,只图以头脑来理解正法,而忘记用心去实践,这样子是不会得到真正领悟的,以致心中仍充满着许多苦恼。

我所说的法,就是为人之道。如果能时时刻刻身体力行之,日久必能豁然开朗,了除心中的晦暗。因为佛的慈悲能产生光明,这光明使你的心灵世界得到调适而让你领受到真正的舒泰。

在你们的自我中,存在着善我和假我。一切的苦恼皆来自假我。

有时候看似假我的种种需求已能获得满足,但事实上,假我是执着的大包袱,是苦恼的因子。

不袪除苦恼的根源,就无法从人生的苦恼中解脱出来。

因此,当伪我中的自我保存欲望能善加控制时,你就能自然而然地感觉到善我中佛性的存在。

修行僧们,看看那些经历风吹雨打的草木,瞧瞧那笔直延伸的竹子。它们忍耐自然的无情,并在其中美丽地成长,那是因为它们并不违逆自然。同时,也因为它们的根厚实地扎在地底,它们才得以耐住一切横逆。

不但如此,它们彼此谦让,并互相扶持,如此延续了它们自己的生命。汝等要仔细体味这个事实。

你们若能领悟正法,就不会让五官的烦恼来摆布,也就能正确地了解一切事象。这样在心中,就不会再将苦恼的毒素漫延开来。

不论是看到或听到任何悖理的事象,只要能依正法来思考,则此心必不会随意被动摇,也不会产生任何悖理的行为来。

要了解,不退让的求道心,是来自对正法的实践。」佛陀的这一番话,在有体验的修行僧心中扎下厚实的根。

佛陀继续说道:

「因此我们要知道,生活的规律不在五官中显现,而是在乎我们的心。

心是一切的根本,若与一脱离正道的心为伍生活着,势必就要与痛苦缠结在一起,就如同身上背负着一个永无法卸下的沉重包袱。

那样的人生,有如身负重物上坡一般。

相反的,如果以中道之心,过着以善我为中心的生活,则心中的安宁与喜悦就会如影随形般使自己受用不尽,甚至能与光合为一体,生活中充满了光明的远景。

当你们不意受到他人的凌辱、诋毁、或怨恨时,如果因此而产生怨恼与不安,成为对方言行的奴隶,你们也终将无法从烦恼的情结中解脱出来。

宽容即是慈悲的表现,是引导我们从愤怨中走出来的佛光,你们一定要试图了解一切事象产生的原因,不要中了愤怨的毒。

愤恨不能以愤恨来回报,因为如此一来,心中的愤恨永远难消。

此刻外面正在下着倾盆大雨,如果这房子的屋顶盖得不坚实,就会有漏雨的现象,这房子也就失去了它应有的功能。

同理,如果我们不虔诚修行,心中就易产生贪念,使我们堕落。

有智慧的人,绝不让烦恼的火焰燃烧自己,他的生活中,必有正法做支柱。

你们当藉着勇气、毅力和智慧,控制心中的假我,以正确的善我之心来生活,这样总有抵达光明悟境的一天。

一个人在知足的时候,就能获得意想不到的财宝。

这财宝就是你们在转生过程中所应体验到的人生智慧,那是安泰的泉源。这一智慧之门一旦开放,一切的迷惘与烦恼都将抛入九霄云外。

人在此时,也将能领略到那不生不灭的生之原理,了解到此刻映入眼帘的一切有形之物,终将归于灭绝。

厌倦了被欲望玩弄而一心想走正道的人们,不久就能开启智慧的财宝之门,到达觉悟的境地。

制弓的人必同时配制笔直的箭,弓和箭的巧妙运用得以发挥较大的威力。你们的生活也应如此,要先端正心性,生活才得以确立。

但是人的心经常会被欲望占据着,因此也就很难达于调和。欲望又使人成为愤恨与贪婪的奴隶,是人心安适的大敌。

只会堆砌美丽的辞藻而不身体力行的人,就像没有香味的花,蜜蜂和蝴蝶是不会来眷顾的。花儿香气四溢,生气蓬勃,才能吸引蜜蜂与蝴蝶,从而共生共长。实践正法可以使人重获新生,周围的一切也将明朗开阔。

你们应当藉着光明圆满的心灵,来点燃其他众生心中的法灯。

莲花、水仙和百合等,能放出沁人的芬芳,然而正法所散发出来的香味更无与伦比,希望大家都能使自己成为散发芬芳的人。

在座的人,有没有在游化时尝过失眠的滋味?无法成眠的那一夜,似乎特别长。

同样的,对于尚未渡过生死大海以达安宁快乐的彼岸的人而言,日子就有如崇山峻岭般难于攀登爬越。不了解正法的人,盲目地走在漫长的人生旅途上,常觉苦闷与迷惘。

修行时最好和品学比自己优秀或相当的人为伍,因为与劣友相处,容易堕落而不自知。当知道劣友比猛兽更可怕。

森林里的野兽或许会毁灭你们的肉体,但却无法毁坏你们的心,但是坏朋友却能毒害你们的心灵,因此你们当密切注意身边是否有带坏你们的人,交朋友时千万要谨慎。」

许多修行僧听佛陀说法时,都不住地点着头。

佛陀见他们都能领会自己的意思,也很满意地微笑着。

外面的雨,仍在不停地下着。雨势或强或弱,每当佛陀停下来时,似乎雨势就如千军万马般奔腾起来。

会场后面有着不安的骚动。

原来顶着风雨前来的耶萨进入会场,佛陀业已瞥见。

「喔!是耶萨回来了!」

憨直的耶萨,擦了擦头脸,立刻去到佛陀面前深深顶礼,并说:「佛陀,您这一向好吗?我与弟兄们都平安无事,请您安心。见佛陀圣体安康,不胜欣慰。」

耶萨掩不住自己的喜悦之清,不知是雨水,抑是泪水,正在他脸颊上交错纵流着。

「平安回来就好了!赶快去换衣服,你们一路辛苦了!」

接着佛陀对和耶萨同行的弟子们也说了些慰藉的话。

「听说巴巴里先生的弟子们也皈依佛陀了,巴巴里先生一定很高兴,真感谢您,佛陀!」

耶萨说完,再次深深的行礼。

在场听法的弟子们也都由衷地欢迎风雨中归来的耶萨一行人。

耶萨在此之前,在喀西国传扬佛法,他以帕拉那西为中心,向许多在家人或婆罗门修行者宣说佛陀的教义。

一行人在阿舍婆誓的房内更衣,然后以洁净清爽的模样再度回到会场。

佛陀继续在说法。

「无论你拥有多少财富,或拥有多么贤惠的妻子,多么可爱的子女,若任凭欲望之流冲激自身,使自己身陷不知足的苦恼中,这苦恼将永无止境。

肉体看来是自己的,但并不是自己的,随着时间的流逝,终究要回归大地,更何况是身外的子女、妻子与财产呢?

这些身外之物,不过是我们在转生的过程中,由因缘的和合而产生的。每个人的一生中,都有着各自的使命与目的,关于这一点,希望大家能重新加以体认。

那个永远属于自己的,就是生命,也就是心灵,除此而外,没有东西是真正属于自己。

虽然说自己有个心爱的孩子,但是这个孩子也会在成长的过程中与父母的意念背道而驰,因为孩子也有孩子自己的个性与灵魂。

双亲为了养育子女,就像太阳之于大地,是不求报偿地付出爱心的,而为人子女者,也会回报双亲的这种慈爱之念。

在亲子的和谐关系中,存在着人的道,这是人类活下去应尽的义务。

怀着感谢的心,将感谢表现在报恩的行动上。孝养父母是为人子者应有的行为,倘若无法做到这一点,则人类的体系即将瓦解。

既然自己是社会的一份子,就应贡献一己之力使社会更趋和谐,如此由于正法被大众所实践,理想的佛国土就有被创建的一天。

社会并非一个能满足一己欲望的地方,万物藉由互助而取得谐调,进而能过和平与安逸的生活。

有智慧的人能深刻了解到知识的无涯而日益谦逊,只有愚昧的人才会利用肤浅的知识来夸耀自己。

在婆罗门的修行僧众中,有许多人精通吠陀和奥义书,知识可说相当丰富,但是并不能将之付诸实践,故始终无法跨出知识的栏栅。

真正的智慧是来自心灵深处,我们称之为「般若波罗密多」,这种智慧绝不是一般世俗的小聪小慧所可比拟,这种智慧是从对法的实践而后获得宁谧、感恩与调适的心性中涌现出来的。

你们不能迷恋知识,就是把知识的领域扩大了,也不能保证你们的心一定会变得丰盈圆满,相反的,会有可能增加迷惑、不安与混淆。知识只能说是经由智慧而产生,但并非智慧本身。」

佛陀充满了恢宏的气魄,说出来的道理,很能触及问题的核心。会场一片肃静,只有佛光在周遭流动着。

「智慧的涌现在于对法的实践,除非真正除去心中不调和的阴霾,否则无明永远陷自己于苦境。

想除去心中的阴晦,必须时时以法为尺度来审顾自己的思想与行为,随时修正自己的错误,这就是反省之道,同时经常行在正道上,不再在心中制造阴晦。

行走在正道上,就可以控制烦恼的,虚假的自我。

以在你们心中本来就有的,不对自己说谎的善我的心去修行,才可称得上是真正的修行者。

此心不断奔驰外物的话,就会使自己落入游荡不安的境遇中。在那里,只有千仞之谷,只有无尽的烦恼。

恶业虽然不会立刻出现恶报,但就像陷于灰堆中的火种一样,不知何时就会因一阵风的吹袭而延烧开来,甚至烧毁整座山林。

愚昧的人经常为各位的贪念所困,执着于物质、财宝、情愁等而折磨着自己。修行人是不会看重名利的,他们并能不断地实行利他的行为,舍弃自我保存的欲望,常住在安泰的生活中。

饮用法水的人,由于心灵经过洗涤,不会被外物所束缚,能够安适地将视线推至极远的地方。

我们看灵鹫山附近的岩石地带,任凭风怎样地吹,岩石一点也不动摇,这就是安住于自然之中,不为物所动的道理。

身为一个修行者,也要像岩石一样,心不为毁诱、赞赏所动,倘使心灵稍有动摇现象,必会发生异常的作用。

对于任何言行,均要仔细地加以观察、分析,能据实地说,正确地做。进而在生活中努力精进于正道,不断地祷念,不怠忽反省,使心灵保持圆满丰盈,以享受禅定之乐。

舍弃沉溺于享乐的行为,遇到任何使自己痛苦的人事,都要能先究明前因后果,消弭引发痛苦的根由,在心中继续维护法灯的明亮。

可是深明此理,真正到达彼岸的人却很少,许多人仍然执着、傍徨于这个无常的世界。

你们必须在转生的过程中习得伟大的智慧,要懂得知足和与世无争的处世艺术,怀抱如天空般清澄宽广的胸襟,只要有一颗知足的心,就有从生死轮回中解脱的希望。

修行者应该了解「安泰」一词,那是胜过百万卷书本的生存之理。真正的解脱之道,不在于读多少书,说多少话,而在于有否慈悲的心肠。

还有一点应该铭记在心的,就是努力克服心中虚假的自我,那样将胜过在战场上击溃百万的大敌。就是再大的堤防,也有被侵蚀而致崩溃的时候,我们的心应当更加小心防护才是。心的苦恼,唯有借助心理治疗,才有得到解脱的可能。

有智慧的人都知道应尽速设法从这一世的业火中逃脱,以脱离充满愤怨、苦恼、诽谤、贪婪等的漩涡。我们的心一旦被业火燃及,就注定要在火中忍受煎熬之苦。

善心就是你们的主人,也就是永不会灭失的自己。为了不失去那个真正的自己,一定要远离业火。

一个人只有在真正爱自己的时候,才能得到安宁与舒泰,同时心中产生的喜悦也格外能感染他人。爱自己的人也一定知道如何爱他人。

如何才会爱自己?首先要修养自身,以正法为依据,确立起自我。并且要确立起那个虽被业火侵袭但也能将火媳灭的自己。

外出游化,散播正法的种子,若竟疏忽了对心灵的开拓,那就好比在贫脊的土地上播种般,终究不会有多大的收获。

智慧、毅力和勇气。

是确立自我,使人从迷惑的深渊中到达彼岸的凭借。

不能依靠人。

任何事都不要归因于人。

你有了善恶的报应,这都是因为自己的心和行造成的,千万不可归咎于他人。你们修行的目的,是改善自己,不是战胜他人。」

佛陀说到这里,暂时停顿下来,眼光缓缓地环视着四周。

佛性的种子已撒在修行僧们的心田。有些人已了然于心,有些人则仍在思索话中的意义。

场内一片寂静,大家都期待佛陀继续说下去。当佛陀说法时,他们一点也没有听到屋外的雨声,但当佛陀歇息时,落在竹丛间的噼啪雨声阵阵传过来。

屋外是狂风夹着暴雨,屋内则寂静无声。

这是一个强烈的对比,外面有如地狱,里面就像梵天佳境。

竹林精舍虽在风雨交加中,但会场上的每一个心灵都如湖水般平静无波。修行僧中渐有人领悟到此心即法的道理。

不一会儿,佛陀谈到日后游化的事情。

「据目犍连的报告,舍卫城的精舍将在雨季结束后落成。

所以我计划在雨季过后,我们就往舍卫城去游化。

希望上了年纪的优楼频罗迦叶兄弟和几十位弟子,能继续留守在竹林精舍内。」

「是的,我想我们是不能长途跋渉了,我们会依您的指示,继续在精舍内修禅定,磨炼身心。」

年长的优楼频罗迦叶代表其他两个兄弟及弟子们,走到佛陀面前,向佛陀顶礼道:

「我们会好好留守的。」

迦叶三兄弟都恨不能自己是年轻人,如此就能和大家同行了。

「跟你们一起留守的人,就由你们弟兄三人自行决定吧!」

「好的,我们会在新进的同道中去决定。」

迦叶兄弟本是伽耶山的大仙人,是拜火教的教主,但是现在全都抛弃了旧日的信仰,归投到佛陀门下,心中充满了宁谧与喜悦。你一旦知道他们曾是拜火教的教主,也会不由得怀疑这件事的真实性。

「除了迦叶兄弟所决定的人留守此处外,其他的人就一起到憍萨罗国去一趟吧!

在这两季中,希望各位能尽力端正己心,并锻炼出能耐长途跋渉的体格。

桠那、舍利弗、阿舍婆誓、跋提、耶萨、婆罗多尼耶、憍陈如,你们在散会后到我房间来。」

佛陀把他们召集起来,为的是进一步商讨游化时的工作分配问题。

在分别决定好游化时的负责人后,佛陀又对他们这些指导者详细指示修行的方法。

许多弟子们都在定中预见了尚未到过的舍卫城的精舍。

住何国家均允许修行者自由出入境。

因此他们只待雨季一过,就随时向憍萨罗国进发。

他们不时地洽商计划各类事宜。

滴滴答答的日子,在不知不觉间过去了。

从舍卫城传来了消息。

佛陀正在房内坐禅,当他从定中醒来,舍利弗带着使者进来。

「佛陀,我名叫优陀耶。是须达多长者派来。舍卫城的精舍已经落成,正在等待佛陀的光临。同时波斯匿王也正在热切地盼望佛陀前去,请佛陀指示一个明确的日期。」

「优陀耶,你一路辛苦了。我们大约会在两个月后出发,因为我还要一面游化,一面说法,所以等决定了确切的日程后,会再跟你们联络的。」

「是,佛陀。我先解释一下精舍的结构好吗?」

说完,优陀耶就在佛陀面前展开一方绘着精舍图样的绢布。

「东边那道很出色的大门是捐献土地的祇陀亲王设计捐建的。大家都说此门很适于用来迎接佛陀。

在四边各有修行者的宿舍。佛陀的房间在东南边。

中央是说法的殿坛,东边有仓库。」

佛陀边听边依言看着设计图,然后点头称道:

「嗯!构想相当别致出色!」

「佛陀请为这精舍取一个名字吧!这是须达多长者要我代为请求的。」

佛陀沉思了一会儿,说:

「这精舍有劳祇陀亲王和须达多的布施,我就将他取名为祇树给孤独园吧,简称祇园精舍好了。」

「啊,真是太好了,就这么决定吧!」

身边的舍利弗也很开心地微笑着,他正在反覆想着这个新精舍的名字。

「优陀耶,你一路上一定很辛苦,今天就住在此地好好休息吧,这里的生活方式虽与你的不同,但你尽管安心地休息。」

「谢谢您,佛陀。舍妹就住竹林精舍附近,我今天可以住在她那里。」

「须达多大概没有对你讲清楚,你不用客气,尽管住下来,明天再到你妹妹那儿也是一样的。」

「家主人对我说过,既到了佛陀这儿,就应该将佛陀之法做为自己心灵的粮食。」

「你既知道了这个道理,你想要拒绝,是觉得这里的粗茶淡饭不合你的胃口吗?……」

佛陀一面看着拘谨的优陀耶,一面慈祥地笑着。

「没有这回事,能够来到这里,真是我的荣幸。」

优陀耶原先把佛陀想成一个非常严厉的人。如今才发现佛陀是如此慈悲、亲切而和霭。但他一时间也不知如何是好。

「舍利弗,找个人好好招呼优陀耶,给他安排了宿舍。」

「好的,佛陀。」

舍利弗领着优陀耶走出佛陀的房间。

不久,就有人为优陀耶送来蔬菜稀饭,优陀耶连称好吃,一下子吃下好几碗。舍利弗又带他到精舍内,并详细地为他解说佛陀的教义。

优陀耶全神贯注听讲着,不知夜已深沉。

接下来的两天,优陀耶都非常仔细地听佛陀说法,于是他在精舍停留了三天。在此停留的三天期间,优陀耶也好好地反省了三天,至此才知道从前是如此的懵懵懂懂。

他由衷地向佛陀道谢,然后才离开竹林精舍。

\section{舍卫城之旅}\label{sec6.4}

到舍卫城的旅行表终于订出来了,以舍利弗和丕葩利桠那等三十六人为领队的各小组也都编列出来了。

在旅途中,他们大多露宿于树林或森林中,并没有想到投宿旅店或民家。

一般团体在编组或订定行程时,多半会出现意见冲突的现象,但是在他们的绵密计划中,各组间丝毫没有冲突发生。

佛陀的僧团在决定游化行程时,丝毫未考虑到住宿的问题,这正是他们能安享至乐的原因。

一路上的餐食也以乞食为主,不需要带什么东西。沿途均不乏民家的布施,他们吃到各种口味不同的食物,在营养方面也没有偏颇或失调的情形。在当时的印度社会,僧侣的地位非常特殊,常被视为是较接近天国的一群人,因而僧侣也多半有着类似的优越感。

所以在这样一种社会背景中,婆罗门家系中如有担任圣职的,均会受到人们的尊崇。虽然宗教林立,但由于因袭传统,婆罗门始终有着压倒性的势力。

婆罗门家庭中的成员,从小就接受圣职教育,被称为沙弥。至十二、三岁以后,开始跟随长老们学习吠陀或圣典,体验僧团生活。

当修行生活结束后,就开始营家庭生活,结婚生子,直到四十多岁,才再进入山林做进一步的修行。

在此时期,有人专心于瑜珈的肉体修行,有人则祭祀火神阿克尼,因各人的志趣而有不同的修行方式。

当大祭典来临时,他们就由山林出来,司掌传统的祭事,也有人再度回到家庭中。

在这一段修行期,有许多人是夫妇一起修行的。

夫妇生活又如何呢?原则上在此时期,夫妇俩各自从事修行,断绝了一切的交往。

在这个阶段的修行,是禁止至各地游化或乞食的,只能在某一教团中磨炼自己,食物则从家中供应。此时期的修行,有许多的限制。

最后一个阶段的修行者,称沙罗门,年龄已在七十岁以上,可以参加正式的游化。

从乞食到处理身边一切的杂事,虽不完全靠自己,但实际是在过着与社会完全隔离的生活。佛陀在世时的印度,也有例外的情形,有修行的长老,或主办祭事,或出入宫庭,成为圣典的指导者。

由于长老的教导,受教者的知识得以更加丰富,进而了解到各种人生经验。民家对于来到檐下的沙罗门,没有不热心布施的,对修行者施食被视为对神的间接供奉而有大功德的。此类布施被肯定能得到永恒的安乐。

婆罗门教义由西方传来,由来已久,婆罗门僧修行的最终目的,在于得到观自在的能力。

也就是说,要具有伟大的超灵能力,以便引导众生,就是他们修行的目的。当时的社会,并不像今日这般文明与进步,同时还必须在残酷的自然环境中求生存,因此如何透过自然与人的谐调关系来解脱生存的痛苦,也就是当时的人修行的目的吧!

人们完全生活在神的救助下,故而时时刻刻不忘向神祈祷,想藉助神力以克服艰难的生存环境。

但是如前所述,光靠知识的领悟是不可能达到觉悟之境的。

因此有人盲目地信奉了各种神明,徒然地奔驰于对灵力的追求。

佛陀不指导人追求灵力的方法。

唯有此心在大彻大悟后,才能获得真正的宁谧,欲达此目的,必须经过漫长的修行过程,而修行之道是无穷尽的。因此佛陀昭示出八条正道,使人长住于安定中而免于崩溃的危险。

肉体虽然只是灵魂的舟车,我们也不宜疏忽它,否则人类的生活亦将无以为继。

所谓中道,即在于心灵与肉体达于谐调,因此心灵是不能与生活脱节的。人类为了克服严酷的自然环境,因而设法以种种方式来改善自然环境,于是人类努力于运用自己的智慧。

但是当时人们都认为自然环境是为神明所主宰,若想超越,不离开肉体的束缚是办不到的,因此也就想到开发灵力的路子。

由此一来,人们不但疏忽了修行之道,甚至于漠视了现实的生活,只一心一意追求超灵的能力。

所以到处可以看到被魔控制的废人。

另一方面,优越高傲的婆罗门,又只知竞遂于知识的多寡,失去了人们的信赖。也可以说,这是阶级意识与权势腐蚀了婆罗门僧的心志所导致的一个现象。偏重于追求知识或超灵能力的修行方式,都不是正确的。

中道就是对两者都不偏执的一种中正之道。

中道表现于对现实的努力,以及使心灵趋向丰盈的实践上。

阶级制度深深箝制着当时的印度社会。婆罗门僧盘腿高踞于首陀罗(奴隶)与吠舍(商贾,老百姓)之上,不发一言地就将印度人民控制于股掌之间。

佛陀的教义即针对此一弊病而引申出来,他严厉地批判了当时的阶级制度。

一个奴隶的生活往往不如一个畜牲,家畜尚且居住在廏舍内。

而一个奴隶则被迫过着野宿的生活,他必须从事耕织或做些为人轻贱的杂役才能维持生活。

奴隶的生活不但一贫如洗,甚至经常受到监视,毫无自由行动的权利。

当时的印度,各城邦间连年争战。有许多战败的贵族之士或武士,就会逃至山林中,一夜之间变成修行者或山贼。战败的命运是很惨的,若不设法逃脱就有遭受灭族厄运的可能。

在佛陀的僧团中,出家人一律过着乞食和游化的生活。整个僧团的成员,平均年龄都很轻,二、三十岁的人占大部份,年长的人则屈指可数。

至于婆罗门的乞食沙罗门如前述都是年届六十以上的高龄长者,人们对于沙罗门的乞食和游化均会毫不吝惜地予以供养,只要沙罗门一站在屋檐下,商家或农家会立刻辨认出而殷勤地搬出食物来。

佛陀的僧团也行乞食游化的生活方式,但是成员皆属年轻人,依年龄看,他们只相当于婆罗门第二阶段的修行僧而已。就是佛陀本身,让人看起来,也是如此年轻。

因此佛陀的僧团一旦到达婆罗门很多的地方,经常会乞讨不到东西,虽然有时也会得到些许布施,但是不像婆罗门的沙罗门那样受欢迎。

布施的观念来自人们感激现实的生活,并祈求未来生活的安定。安定的生活被认为是受赐于神明,以及对神的祭祀。因而他们认为对年长的婆罗门行布施,即是间接对神行供奉,将有意想不到的功德。

由于这一社会背景,佛陀的年轻僧众们在乞食时,经常遭到拒绝。可见修行之道也并不一定平坦易行的。

等到佛陀的感召力一天大似一天,各地人们对佛陀的僧团有了普遍的认识与接纳,情况才有所改善,但是北方偏远的农村小镇,对他们仍是冷漠相向的。因此僧众们大多是以山林中的水果充饥。

在离开精舍的第五十一天,佛陀与弟子陆续进入憍萨罗国境。三十六组僧众先后抵达,他们中有人在乞食游化时遭到困难,这是可想而知的。

佛陀和舍利弗一组人朝拿兰陀的方向继续前进。同行者还有巴巴里的女弟子埋托勒呀,波萨罗及陀提耶。

佛陀一行人计划在拿兰陀举行一场街头布法。

埋托勒呀、波萨罗及陀提耶等被指派担任敲钟和召集听众的职务。

拿兰陀也是舍利弗和目犍连的出生地,所以舍利弗和弟子们能藉助自己的家长和亲友来传播佛陀说法的消息。

佛陀在绿意盎然的拿兰陀森林内休息,弟子们也在他身边坐下来。

佛陀对坐在一旁的舍利弗说:

「明天傍晚我要在镇上的广场上说法,所以明天一大早就要开始乞食。说法前,埋托勒呀和波萨罗会敲钟通知镇上的人。」

第二天,佛陀先众人而起身,静静地离开林子,往村庄走去。

天色微暗,星星闪烁于天际。

田园的一端,已袅袅升起白色的炊烟。

田园的景致异常平静而祥和。

当佛陀从一家农舍前经过时,一个妇女立刻捧着装满稀饭的土锅从里间出来。

「修行的师父,请接受我的供养。」

她手捧土锅,跑步过来。

佛陀闻声,立刻停下来。

「真是谢谢,您真好!」佛陀说着,就将左手上的钵递过去。

妇人将钵接过来,就谨慎地将刚煮好的稀饭倒满一钵。

稀饭冒着热气,香味直逼而来,勾起人的食欲。

佛陀默默地注视着身边发生的一切。

此刻钵中盛着的是粗糙的蔬菜稀饭,以往的悉达多太子是绝没有机会吃到这么粗劣的食物的。

但是一般农民就是以此类粗食渡日的,有时甚至只吃些粟稗或野菜。能吃到稀饭,已属相当不错了。

境遇较差的农人,难得吃到米。

佛陀自出家修行以来,一直就甘于粗茶淡饭。

眼前受供的虽是粗糙的稀饭,他满心欢喜地接受下来。

妇人将装满稀饭的钵捧至佛陀面前。她曲膝仰望着佛陀。从这一角度看佛陀,妇人始发现佛陀是个年轻僧侣。

她先前误以为佛陀是年老的沙罗门。

在昏暗的天色中,佛陀如附着于大地般稳重而庄严,使妇人觉得他是一位年老的婆罗门僧。

她在讶异中继续凝视着佛陀,突然发现佛陀周围有光闪现。

一刹那间,她不由自主地拜倒在佛陀的跟前,嘴中不住喃喃道:

「秀巴腊,秀巴腊……	」

同时全身颤抖着。

「女施主,谢谢你的好意,你热心地布施,必会得到恒常的幸福。好好侍奉丈夫,养育孩子,享受你幸福的人生吧!」

「谢谢您,秀巴腊,我现在有如卸下了包袱一样地感到轻松,实在是感激您!」

妇人说完,就目送佛陀离去,直到佛陀消失在她视线之外。

佛陀又顺道到舍利弗的俗家,当他重回林中,天色已大亮。

佛陀与舍利弗分食了刚刚乞得的稀饭,并详细告诉他一切经过情形。

接着,又教导他有关慈悲的可贵道理。

「舍利弗在婆罗门的习惯中,布施食物是非常好的传统习惯。布施的人中,有的很真诚,有的很勉强。

由衷的布施,不但可以与他人的心相契合,同时这美德能照亮周遭的人。如果在布施时,只计较布施的多寡与功德的大小,那就成为一种欲望与执着。不乐于布施的人更因而在心中产生不平之气。

你们这些修行中的人格外要注意,除了真实的东西之外,是找不到光明的。」「谢谢您的教诲,我必努力精进于真实的佛法,传给弟子们,并济渡众生。」舍利弗一面喝着佛陀给他的稀饭,一面想着布施的个中道理。

山林在晨曦的沐浴下,益显朝气。

阳光穿过树丛,照耀着大地。

小鸟啼唱,野花怒放。

舍利弗吃完稀饭,对佛陀说:

「佛陀,为了今晚的说法,我想先到镇上巡视一番,我先告退了。」

舍利弗留下几个年轻的僧众陪侍佛陀,自己独自离去。

佛陀微笑着目送舍利弗。

在拿兰陀有舍利弗旧日的老师删闍耶的道场。

舍利弗曾在那里修行求知,道场中经常有师生论战的情形,有许多人是强词夺理。

舍利弗当初很不满这里的情形,因而和目犍连双双离开,另谋修行之途。佛陀料定舍利弗会与旧日的师友见面的,他同时料定,只要他们提出挑战,舍利弗绝不会输给他们。原因是舍利弗已不再为类似无意义的论战所迷惑了。

佛陀轻合双眼,开始反省自己的思想与行为。

他想审视一下自己有没有傲慢的心思,同时不论在说法或开导弟子时,有没有说错或做错什么,又,在生活中有没有偷懒的行为,凡此种种,他审慎地检视着心中的尘埃与污垢。

然后他发现自己对舍卫城之行并没有任何希冀,心中有的只是安祥与平和。佛陀进入禅定三昧。

留在林中的修行僧也远远地陪侍着佛陀,并各自进入冥想,以此得到心灵的静定。

\section{街头布道}\label{sec6.5}

村落里很热闹。

佛陀的弟子们四处敲打大鼓或鸣钟。

路上的行人及婆罗门的僧侣们也被钟声或鼓声所吸引,停下来观看。

佛陀的弟子们向大众说明,今天晚上,伟大的佛陀要在这儿说法。机会难得,希望大家不要错过。

到了预定的时间。各种阶层,各式各样的人们都聚集到城里的广场上。

在佛陀开始说法之前,拿兰陀出身的舍利弗首先站起来解说佛陀的教义是什么,又陈述了自己皈依佛陀的经过。

因为城里的人们都已闻知佛陀是迦毗罗卫人,是一位秀巴腊,四处向众生说道。因此,究竟佛陀悉达多是个怎么样的人物,说些什么话呢,连婆罗门的僧侣们也露出好奇的神情。

在听众里,也有人在摩拳擦掌地等待着,因为这儿是婆罗门的地盘,如果佛陀说出不顺耳的话,就要给佛陀难堪。

在舍利弗的主持下,佛陀开始进行说法。

「诸位众生!

请看看旅人。旅人走在先人所筑的路上,前往目的地去。假使没有先人所留下来的道路,山野均为草木遮掩,要走无路之路到目的地去,那是非常艰苦的。

但是,因为先人的智慧,在黑暗中点燃炬火,修筑桥梁,开辟道路,使后世的人们旅行时,非常便利。

人生航路的旅行,也是和这个一样。先人的教诲使我们能避免这些苦难。心地坦诚就能度过更丰饶而光明的人生。

虽然如此,人类一生就像在黑暗中旅行,必须在烦忧中痛苦挣扎的情景,非常多。

为什么不单纯地接受先人留下的光呢?

为什么不想度过安泰的一生呢?

因为,人们的心常被烦恼控制,把心卖给了欲望。

而且,先人的教诲没有能正确无误地留传后世。后世的人们把教诲歪曲,所以人们越发失去光明,迷失了方向。

知与意本来是因欲望而产生作用。若是失去本心,知与意就会任怠而行。

以动物来说,猿猴也是有知的动物。它们由微小的知发出作用,因此,要捕捉牠们,就利用猿猴的小知就行了。

猴子窥见圆罐,看到里面有水果,就想伸手拿出来,于是将手伸进罐口,然后抓起水果。

然而,罐口狭窄,手本来是勉强才得以伸入的,现在猴子想把水果拿出来,握着水果的手就卡在窄小的罐口,拔不出来。

如果猴子放开水果,手就能够拿出来,但是,它舍不得放弃水果。

终于,猴子被欲望所征服,而被人们捕捉。人世的烦恼和这种情形相同,如果脱离愁望的束缚,就能避免身之破灭。

其实,水果到处都有,遍生山野。但是,因为小知的作用,心被欲望所摆弄,所以遭至那样的结果。

有智慧的人,领悟了被物所缚的愚昧。所以聆听自己本心善我的声音,正确地生活于世间。

众生啊!

肉体的五官所触及的一切外物,都是无常的。你们应当依照善我的本心去生活,努力地服务大众!

当人们遵行善我的本心行动时,人们就全部变成佛子。充满调和、平等的社会,就会自然建立。

大自然本来是无差别的,平等的给予我们生活的条件。

所有的不平等都是因为人们的心受欲望玩弄,而建造了各式各样的围墙或阶级啊。……」		

佛陀说法的时候,常使用「方便」之语(佛家语)。因为当时的人们,识字的人很少,文盲很多。所以佛陀引用许多浅近的例子,或人们所熟习的周遭事物为比喻,阐明神理。这样的话,任何人都能很容易的了解。

四周围观的听众们,都被佛陀赤诚的言语吸引住了,没有一个人想起别的事而中途离去。

佛陀的弟子们也散在听众之中,因为他们一向把佛陀的言语当作喂养心灵的粮食,因此,一动也不励的仔细倾听。

佛陀说法结束之后,开始接受人们的询问。

这时,有一位婆罗门修行者,用亢奋的语调问道:

「悉达多,你的弟子们都称你为秀巴腊,尊你为佛陀。非常尊敬你。但是,这种称呼使人疑惑,因为无论到那一个国家去宣道,就自称是真正的秀巴腊,或是说我就是真正的佛陀,像这样的人太多了。

我是婆罗门的沙罗门。我认为不是侍奉神的圣职者---武士阶级出身的悉达多自称是秀巴腊和佛陀,那是冒渎了婆罗门的神。

希望悉达多,你自己说明一下,你是怎么样的人?」

婆罗门的修行者,严肃地瞪着佛陀说着。

佛陀露出微笑,婉转地说:

「婆罗门的沙罗门啊!太阳只是你的吗?我认为太阳也是刹帝利、吠舍、首陀罗的,这样错了吗?」

「……	」

「太阳之大是因地上各个阶级而不同吗?」

「太阳只有一个,你不说我也知道。」

「但是,你想要说的就是只有婆罗门才有太阳。其他的人都是没有太阳的人。大自然对待万物一切都是平等的。人不是因为种姓、出身而成为圣者的吧?

简单的说,圣者不是由于驾行正道,解救众生脱出烦恼而成功的吗?

沙罗门啊!把他人的话语生吞活剥的话,就会产生错误的看法及想法。首先,应该要正确地用自己的眼睛加以证实,仔细的思索。如果不这样做,就会因别人所讲的话而产生嫉妒之心,把自己掷于黑暗的洞穴里面!

请看那棵树上生长的芒果,那个芒果的滋味如何你知道吗?」

修行者把眼光转移到佛陀所指的芒果树,无法回答,而默然不语。

「那个芒果的味道,你现在知道吗?」

佛陀继续追问修行者。

「嗯!嗯……若是以外表的色泽来判断,要是不再等几天,味道恐怕还不好。……你到底想说些什么?」

「如果摘下来吃,就很清楚是什么味道。」

「那是当然,摘来吃谁都晓得是什么滋味了!」

「沙罗门啊!我想说的是去实践我所说的法,那么就会知道法的味道啊!听到他人的传言或道理,自己先去做做看,那样的话,就像尝过芒果,是甜的?还是酸的?你完全明白。要先去实践才会知道那是错的,这是对的。

把平日贮存的知识与学问,灵活的运用于日常生活之中,活用体验之后就变成智慧,这不是成为秀巴腊之道吗?」

婆罗门的修行者垂首无语。

他了悟自己的无知而感服于佛陀的言语。

他因为这次的质问而皈依了佛陀的僧团。

这一次的街头讲道,使得许多婆罗门修行者皈依佛陀。

而且各阶层的民众也因为佛陀的说法而幡然觉醒,决心从此实践于生活中。佛陀等一行,这样随处向众生说法,经过巴达利盖马,进入毗舍离城。

这时离开竹林精舍已经有半个月,一直居无所的佛陀,早已了悟随时随地,都是修行之所。因此,从不以为这是在旅行。但是,弟子们之中有人认为竹林精舍才是自己的修行所。所以向同伴说:

「下次回到竹林精舍时,就要彻底修正自己过去所犯的无数在思念上和行为上的错误!」

弟子中对竹林精舍怀有乡愁的人很多。

佛陀说:

「修行僧啊!你们的修行所不是只在竹林精舍。如果是这样,你们的心就已经被对竹林精舍的想念所束缚了!

你们必须要明了你们所至之地,都是你们修行的场所。

现在,我们所在的毗舍离都城,也是你们此时此地,利用宝贵的时间应该修行的地方。

此时此刻就要端正自己。若是现在还不端正自己,那么,越往前走,负担越大,你就会经常与苦恼同居。如果你们的心因为不同的场所而不安定,那不久太阳就要下山,很快的有黑暗的人生在等待着,你们要了解这一点啊!

场所、时间都是无关紧要的,活在现在才是最重要的。

要把依赖明天的心抛弃,想着还有明天的话,就会产生松弛的心,而以放逸的动心度过今日,这样的人就会过着愚蠢的人生。

教你们这些修行僧不应该拘限于这样小的心,整个大自然都是你们的住家,要住在这样的佛陀心内。」

「佛陀,谢谢!我们的思想错了。

应该以此时此地当作自己的修行所,以法为凭依,砥砺自己。」

目犍连的弟子们异口同声地回答佛陀。佛陀弟子以外,一旁的裸体修行者也听到了佛陀的话,其中有一个人问佛陀说:

「悉达多--。		

我曾经由别人那里听说过你的名字。但亲耳听你说法,今天还是第一次。

我们的老师禁止一切杀生。关于杀生这个问题,悉达多你的看法如何?请指教!」

舍利弗的弟子对这个裸体修行者骄傲的态度感到不满,就说:

「梨迦布族的修行者们!

你们对你们的老师是用这种态度求教吗?

佛陀是我们的老师,要先端正言语及态度然后询问,这才合乎修行者之道!」佛陀的弟子们提醒那十几位举止粗鲁的僧侣们。

可是梨迦布族的修行者对舍利弗等人的忠告置之不理,以充满敌意的锐利眼光等待佛陀的回答。

佛陀笑着说:

「舍利弗,

修行者们,见定真实,无差别地抱持平等之心的修行者是谦虚的、没有愤怒的、不自傲的,以法为心的粮食而成长着。

无法见定真实的修行者,认为自己过去所信仰、所修行的道才是绝对的道。而过着盲信、狂信、迷信的生活,并且无法解答心中产生的疑问。这样的话,是无法脱出知识的框子的。你们也曾体验过那种愚昧的修行。

吠罗的弟子们尊敬吠罗为师,这点和你们敬我为师,并无不同。

把正确的法作为知识解悟,获得这个知识生活的时候,真实的智慧就会由内在的心显现出来。然后就能领悟更真实的法之价值。

芒果是什么滋味也要吃了以后才明白。如果吃了未成熟的芒果,嘴里会有涩味,就无法尝知芒果真正的味道。

等待一段时间,吃了成熟的芒果,才会知道芒果的真味。只有到那个时候,你才会说芒果很好吃。

等待他人理解,也是一种慈悲。而不是胡乱地感情用事,说芒果好吃而硬推给人。」

「佛陀!我们懂了!我们没有慈悲之心。」

舍利弗细细咀嚼佛陀的话语,铭记在心,恐怕今后其他弟子也会犯同样的错误!

佛陀的眼光转向摩诃•吠罗的弟子们。

「摩诃•吠罗的弟子们啊!无义的杀生不是正法。作为生命而存在的东西的生存本身就是他们被赋予的权利。

不过,裸行的修行者们!你们为了维持生命也必须喂养肉体吧?我想请教你们是如何喂养肉体呢?」

佛陀含蓄地向裸行者的长老巴巴利陀请教,这是佛陀善用的询问。

巴巴利陀说:

「我们以在途中所得到的芋粥或山野所生的水果为食物。」

「那样子吗?那么!从来不曾使水果腐烂,或是因为过多的芋粥腐败不把它丢弃吗?」

「那是有的,食物太多的时候,因为吃得太多会撑坏肚子,所以曾经丢掉过芋粥。有时也丢弃一些腐烂的水果。

但是,悉达多为什么问这种芝麻小事,我们实在不明白。」

巴巴利陀脸涨得通红,最后的话语因为愤怒而有些颤抖。

「巴巴利陀啊!不要生气,心中有歪曲的念头,是无法领悟我说的话的真义。请你把心平静下来。

因为你们问我有关杀生的事,所以我才问你这些事。

芋具有芋的生命,粥虽为粥仍然具有米的生命。

芋与米都是以大自然的土壤为母亲,吸取营养。由于太阳的恩泽,不久便发芽,吸收空气生长,以绿色装饰自然,同时米与芋时常反覆循环生死,供养我们的生命。

不要忘记,变成你们的血肉的芋和粥,是为了你们而舍弃了生命。水果也是同样地为了成为你们的血、肉、骨而牺牲了生命。

你们糟塌了这些生命,这些生命就等于是死无代价。所以存着感恩的心,以及不糟塌食物,这样的生活是必要的。

因此一粒米也不可以浪费,不要忘记那是农夫血汗的结晶啊!

你们裸行僧禁止杀害蚊、苍蝇、蚋等生物,该不会忘记米、芋也是有生命的东西吧?

而且,你们裸行僧,

一向是以正确的听,正确的看,正确的说作为修行者当行之事。但是,你们心中的思想是自由的吗?如果是自由的,那么,如何去除心中所产生的苦恼呢?

不管五官眼、耳、口、鼻如何调配,但最重要的关键还是在心,心中若是想着各种各样的坏事,这种思想念头就会变成苦恼。

心中苦恼,证明了心中所想的会造成和所行完全相同的结果。

虽然端正了外在,而遗忘了内在。这样去追求真实之道就很困难了!」

佛陀的弟子们不用说,裸行僧们也了悟了心确实是生命最重要的本质。巴巴利陀说:

「秀巴腊(悟道者),谢谢!

秀巴腊的教诲是真实的!

我过去所以成为裸行修行者,就在企图抛开一切的执着,但是那只限于外表,心中依然起伏不定,无法得到安宁。

因为心是一个看不见的世界,以致经常疏忽它,也就永远难以解脱苦恼。脱离执着的只是裸体,而您才是真实的秀巴腊,真谢谢您。请您收我们为弟子吧,即使是当仆人,做粗活也没有关系。」

巴巴利陀是为求道而来的,当然很快就理解了佛陀的教义。经他这么一说,同来的裸体僧也都匐伏在地,由衷地期望能皈依佛陀。

「巴巴利陀,只要是了解并实践佛法的人,一概都是同道和朋友。好好去体会,努力去实践,过不了多久,你们心灵的世界自然就会开启了。」

「秀巴腊,真谢谢您!我们会好好去体会,并且实践的,一定遵从您的教诲。」

当时的婆罗门修行僧,都把悟道者称为「秀巴腊」,将之视为具备观自在力神的化身。

佛陀并不想立刻收他们为弟子。

原因是佛陀目前还在往舍卫城的途中,而裸体僧所奉的迦那教教主是梨迦布族王子,也是毗舍离人。虽然信奉的教不同,但是能了解且实践正法才是最可贵的,依然能通向正确之道。

梨迦布族人,大都器量狭小,迦那王子若知道弟子们转信他教,一定会在教团内掀起风涛。佛陀认为这是应该极力避免的一点。

事实上,如果真实而由衷地领悟了佛陀的法,他们自然也会靠自己的力量来修正性格的。

诚如佛陀的预测,几年以后,当他的弟子在这里说法时,巴巴利陀的同道们己成为毗舍离推动佛教的中心人物。

佛陀一行人,继续在憍萨罗国的各村镇间展开说法活动。他对许多听众说:「诸位听道人,

首先要感谢大自然所赐予的恩惠。

天降甘霖洗涤大地,而大地又给予植物生长的场所,使得大自然是如此美好。在大自然中,一切都平等存在,生物们过着快乐自足的生活。

真实的婆罗门当知道一切平等,并传布普遍的法。在那个法中,没有婆罗门、刹帝利、吠舍、首陀罗的区分。

有的只是一律平等的人们。差别只是依个人的职业、地位而造成的。

佛的教义不由人的知、意而改变,那是普渡全人类的真道。

人不是藉着祭祀而获得拯救的。

要知道,在心中存在着伟大的神佛的慈悲,除此之外,神是不存在的。

诸位听道人,

必须了解你们的心就是永不生灭的自己本身。

那个心就像大自然中的一切,在相互关连的大调和之下,万事万物都是平等存在着的。

有的人过着贫穷的生活,有的人过着富裕的生活。

但是贫穷的人,不管环境多恶劣,他的心未必也贫穷。

千万不能因为自己贫穷而想到盗取他人财物,从欺骗中获取利益。

即使生活贫穷。但是能与人互相帮助,和睦相处,不贪求身外之物,终究还是拥有一颗富有的心。

一个家缠万贯的人,却任凭虚假的自我过着孤独的生活,漠视他人的存在,吝惜自己的财物,总嫌自己财富不足,他的心灵可说是空白一片,益增苦恼了。

诸位听众,

心灵以外的诸般现象都是无常不实的,即使现在拥有许多东西,但毕竟是身外之物。例如金钱、财物等,当你一旦离开人间时,是一分一毫也带不走的。人们往往被这些无常的东西绊住,使自己的心趋于苦恼、狂乱,进而在生活中制造许多纠纷。由于此心盲目而迷惘,加之被情欲所困,人们就不断在苦恼的漩涡中挣扎着。

自己播下的苦恼种子,必由自己来收割。

我们当了解一个人心灵的世界,是由自己主宰的,心是一个伟大的王者,这一认知非常重要。

诸位听众,

在自己的生活中,请即刻停止播散苦恼的种子。

苦恼的种子,计有贪婪、愤怒、毁谤、嫉妒、傲慢、虚荣、不知足、情欲等。诸位听众,

这些种子很快会在心中生根发芽以致茁壮,蒙蔽了心,也遮着了神佛的光明。过去所播下的苦种,成为你现在痛苦的境遇,而且痛苦有增无减。想从其中获得解脱。一是要将苦种连根拔去。如果你只是从表面割除苦茎,恨部仍会萌长新芽,再度结成苦果。

根除苦种,在于对正法的实践,随着实践的过程,光明的种子将不断生根,光明将充满整个心的世界。」

佛陀说的法,摇撼了听众的心灵。患病的人因而觉得病痛痊愈,愚痴的人心中充满了光明,大家都感到一种宁谧的安乐,脸上泛着生气勃勃的光。

听众间,偶而传来因喜悦而发的啜泣声。

佛陀的言语铿锵有力,有着一股不可抗拒的力量。虽然是同一个道理,换一个人说,其情况一定不一样。这是因为佛陀有着坚强的信念和广传的慈爱心所使然。

舍卫城之行,佛陀和弟子们可说是风尘仆仆,未曾歇息过一日。随着旅途的延伸,皈依佛陀的人与日俱增。

但是另一方面,嫉妒佛陀、毁谤佛陀的人,也不在少数,例如婆罗门修行僧。婆罗门教在当时己经有一千八百年的悠久历史,在社会上早已根深蒂固。婆罗门教有许多派别,在祭祀方面也早已流于形式,变成一个僵化顽固的宗教。而信仰者的态度也日渐冷淡,而修行僧们的心多被权力的欲望腐蚀了。在这种信仰下,想找寻真正的道,是非常困难的。

有许多婆罗门僧也想超越这个现象,故而想藉严酷的苦行迅速接近神。

问题是,他们未能真正领略到心的伟大价值,只在不调和的修行生活中,使自己的心灵益趋贫乏。因此他们之中,有许多人分别被名目不一的地狱灵控制着,为人们描述肉眼不及的灵世界,最后使信徒为错误的信仰所遗误。

有许多这样的修行者就自认为自己是伟大的悟道者,或自称自己是佛陀。

特别夸耀传统的婆罗门,一经有人自称佛陀时,都不由分说地提出严厉的抨击。

他们抱着婆罗门圣典,与这班人展开激烈的论战,许多傲慢的婆罗门因此夸耀自己如何如何战胜邪道,又在此情况下,更使许多修行者奔竞于对圣典中理论的吸取,热衷于说理论道,而忘记了对心法的实践。

于是有许多婆罗门攻讦富豪须达多,为他相信刹帝利出身的某国王子的教义,而投下大笔资金在捐建精舍方面。

大婆罗门的司祭们对须达多说:

「须达多,要知道,供奉婆罗门的神才可以得解脱,现在你为了一个来路不明的人,投下大笔的资产,一定会受到头颅七分的惩罚。

你家里的老人家也都信奉婆罗门神,你从小是在这种信仰的环境中长大,你家的财富是婆罗门神所赐的,你改变信仰,对祖先就是一种大不孝的行为,而且你也冒渎了婆罗门,假如你还想保有生命和财产,就不应该迷信外教。

你有好儿女吧!提楼沙和优陀罗的前途你可要好好的考虑到,千万不要相信外道。错误的信仰,也会影响子孙的。」

须达多平静地回答道:

「大仙人,我知道您一向很关心我和我的家人,我真是感激不尽。这次捐建精舍完全是我发自内心乐意做的一件事,这并不表示我不再信奉婆罗门神了。

这精舍只不过用来迎接迦毗罗卫国王子悉达多的,让他有地方住罢了。如果说我是在建造什么祭神的大道场,那工程相当浩大,绝非我能力可及的。

我们是想听悉达多太子讲道,建造的只是一个说法的讲堂。

大仙人,你何不也来听听,祭神的场所不代表一切,重要的是要有说法的地方。」

须达多由于亲自听过佛陀说法,对佛陀解析人生价值的论说极为感动。

他以捐建精舍来表达内心对佛陀的感激,他同时为其他受苦受难的同胞着想,希望他们也能听到佛陀的法音而获受法益。这个愿望即将实现了,所以他天天远望已完工的祇园精舍,脸上不禁露出欣慰的笑容。

在须达多身旁的婆罗门大仙,听了须达多的话,一时不知如何答腔,只随着须达多的目光,眺望远处壮观的东门,心理十分复杂。

须达多则满怀信心地想,这位大仙迟早也会像优楼频罗迦叶兄弟一样,皈依。

须达多从小就对婆罗门教的祭祀抱持怀疑的态度,而且深深感到其中的修行也毫无意义。因为他眼见许多修行者只是过着圣职者的生活,本身并没有真正去做修行工夫,虽然他们的知识都极丰富,但从未照圣典中的教义去实行,离悟道还相当远。

在同一个太阳底下生活,只因为自己出身婆罗门,得以担任圣职,就轻视其他的种姓,陶醉在优越感中,这是很令人反感的。

太阳对一切生物都是平等对待的,从没有任何差别待遇,它赋予大地慈爱的光与热。

须达多认为一个人由于出身而被归纳在某一阶级上,是非常不合情理的。

当此之时,佛陀一行人进入舍卫城。

目犍连和须达多闻风前往迎接。

看见佛陀身上的粗陋僧衣。须达多内心顿时羞愧交集。

须达多所穿的是织锦华服,两下比照,真有天壤之别。

须达多的家仆立刻拿出早经准备好供奉佛陀的名牌丝织僧衣,是须达多的献礼。这时须达多才稍稍宽心,对佛陀说:

「佛陀远道驾临,一路上辛苦了!我们都由衷地期待您的光临,这是弟子的一点心意,请笑纳。」

须达多把僧衣捧至佛陀面前。

「须达多,对说法者行布施,你的子孙也将同受福祉,而你布施的功德也会为后人带来不朽的影响。谢谢你的布施。」

须达多笑道:

「佛陀,请不要客气,假如您肯换上这件僧衣,将是我们最大的光荣。……」

「须达多,多谢你的关怀。我的僧衣虽已脏旧,不过心境确清澄丰盈。人不是靠衣来决定品位高下的,而是靠行为来确立价值的,因此你的好意我已心领。」「佛陀,真对不起,现在容我送您到祇园精舍去安歇。请乘坐轿子。」

「谢谢你周到的安排,我的精神很好,不过有的人可累坏了,轿子就让累的人坐吧!」

佛陀满怀慈悲地说。须达多、目犍连齐声答道:

「是,佛陀,我们会照您的话办。」

他们的眼中闪着泪水。

\section{波斯匿王的皈依}\label{sec6.6}

当佛陀一行人来到东门外,许多民众为了瞻仰佛陀的丰采而在道路两旁供奉了许多鲜花与水果。

佛陀在心中不断为这些民众祈福,他希望民众能因正法而获安宁,能对人性觉醒而过着和平康乐的生活。

祇陀亲王出迎佛陀,并前导佛陀进入精舍。

「佛陀,这位是祇陀亲王。」

须达多为佛陀介绍。

「秀色腊,真诚欢迎您来到敝国。关于您的宏德,我常听须达多提起,今天能亲自面谒,真是太高兴了。」

「亲王,我也由衷地感谢你对祇园精舍所做的贡献,使贵国民众有了听法的地方。」

佛陀再次称谢。

祇陀亲王有着优雅的面貌,但是佛陀却感觉到他心中有着困扰的心事。因为亲王的弟弟波斯匿掌握憍萨罗国的实权。弟弟是正妃的儿子,自己则是第二夫人所出,因而只能做个小城的城主。他对此深感不满。

他目前正在默默忍受着不平的际遇,但他迟早会领悟到这种对权力的执着是愚昧不堪的,使心灵趋于平静。

佛陀确信祇陀亲王总有一天会幡然觉悟的。

祇园精舍内的起居等事宜早已由目犍连和须达多安排好了,他们不但将房间分配好,即连说法的日期与时间也都拟定好了。就像在竹林精舍一样,除了雨天,他们仍以室外修行为主,利用精舍的机会很少。

不几天,波斯匿王通过须达多的联系,来精舍拜访佛陀。

侍卫左右呼拥,其威势的确不是祇陀亲王所能比的。

波斯匿王身为一国之君,举止庄严,气宇轩昂,在求道方面也比任何人都热衷。

他来到佛陀面前,行过礼后坐下来。

「我很早以前就听闻过您的大名,今日能亲自候教,真是高兴万分。听说频婆娑罗主也皈依了您。

在敝国有许多婆罗门僧修习严酷的肉体苦行,但见他们一直未能悟道,而您!悉达多,年纪轻轻的刹帝利,为什么竟能悟道呢?您是否能为我指点迷津?」

他很注意自己大国之君的身份,小心翼翼地发出疑问,他又见佛陀不过与自己同年,竟然能悟道,实不可思议。

「大王!

悟道与年龄无关,而是由于累积了正确的思想和行为而来。因此我们不能轻视任何一个小市民,因为或许他们之中就有个伟大的王啊。

同样地,也不能轻视一个小沙弥,因为这个小沙弥一旦悟道时,就可以成为解救众生的大救世主。

我们也不能轻视星星之火。有道是星星之火可以燎原。小火转成大火时,美丽的都市和茂盛的树林一夜之间可以化为灰烬。

沙弥不久就能成为高僧大德,同理,只要保持心灵的清浄,好好修道,任何人都有悟道的可能。自己如能打开领悟之门,获得佛理,就能济渡其他尚在苦恼中挣扎的人们。

法、道无非教人要知足。排除心中一切愤怒、嫉恨等。什么是内心恐惧的的根源,就要设法拔除它。一个毁谤、破坏法事的人,他的罪过是很大的,因为他的行为将使自己内里的佛性遭到断伤。」

波斯匿王听了佛陀的一番教诲,深深感到人的心是一样难于控制的可怕的东西。

因为婆罗门修行者不断灌输他有关头颅七分一类的受罚的事。

他听说一个人诋毁婆罗门神,或破坏婆罗门祭事时,他的头会有撕裂的痛苦,最后会发狂致死。

过去他修习过婆罗门经典,因此很容易了解佛陀正法的涵义。

佛陀的言语,平易近人,但一字一句又都足以震憾人心。

佛陀静静凝视波斯匿王深思的神情。

「佛陀!请您指导我,做为一国的元首,我应该有什么样的心理准备和作为?」佛陀想起当初频罗娑罗王也曾提出同样的问题。他露出赞许的微笑。

「大王!

你要像爱你子女般地去爱人民,要坦诚无私地对待一切人。

权势和武力虽能控制人民的行动于一时,却无法永远束缚人们的心。

如果能以正确的心引导人民,则人民会乐于遵从你的法,而你将能达到和平治理国家的目的。

因此,即使是再小的孩子,也要加以爱护,因为这个孩子将来有可能学会如何控制己心,以步上中庸之道。

在人民的牺牲下构筑自己的幸福,是导致自己步上灭亡之途的主因。

你要向在苦役之下挣扎的人们伸出爱的援手,以慈爱的心来接触烦恼,就能轻易地祛除烦恼,当接触病患时,病患因而得以消失。

你虽贵为一国之君,但不可以自视过高,不要为左右亲信的巧言令色所迷惑。要努力远离苦恼,行正确的道,致力于悟的境地。

因此,首先要端正一心,任何生物无法息止于燃烧之物上,正确之道亦不存在于情欲的火焰中。愤怒的心更无法使理性伸张,因而远离忠告,被情欲支配的人,就无法了解事物的道理,终致国破家亡。

只知填塞知识,而不努力实践,就像观画中的芒果,永不知果中的真味。

知识唯有辅以实践才能发挥大用,就像种子应撒在肥沃的土壤才能发芽茁壮。

人生的旅途坎坷不平,生存的道上需要不懈地精进。求道的历程有甘有苦,但只要离开正确之道一步,苦恼就会接踵而至。

产生苦恼的原因不在他人而在自己本身,然而大多数人都不曾觉察此点而惯于归咎他人。因此要了解,是否感觉痛苦,原是你个人的责任。

为了断除苦恼,一定要找出苦恼的根芽,断除时必须要有智慧、勇气和毅力。在苦恼根除之后,就不要再播种苦恼的种子,如此才能安度偷快的人生。这个精舍也是先在牢固的基础上建立起来的,因此能耐风雨,人们也就能安心地居住于其间。

同样地,心要想安适,也需要以正确之法为基础。

法的基准,在于能改正自我中心的思想,使自己与他人在大自然中相互调和生存着。因此要虚心接受他人的意见,不要感情用事,或主观地曲解事实。

此外,不论听到什么,都不要轻易地动怒,更不要被其他低劣的情绪所控制。一旦有了傲慢、虚荣、怨恼等情绪时,你就已经在播苦恼的种子了。

在思想和言行方面也是一样,若能正确地想,正确地说,就不会产生不谐调的感情。

言语本在传达心意,一句不适宜的言语,不但阻碍了自己的上进之道,同时深深伤害到他人。

你们不可不注意啊!」

波斯匿王倾耳谛听佛陀震憾人心的言词。

他发现佛陀的话无懈可击,他的心灵在聆听法音的当儿渐趋澄浄。突然有种不可思议的感情在他心中奔流。

佛陀喝了一口弟子奉上来的茶水,润润喉咙,便继续说道:

「大王!

你应当知道,不正确的思想和行为,皆是导致苦恼的原因。

将慈爱的心置于光明世界中,这是朝向大调和的捷径。

心中存在着不自欺的善我的心,同时也存在着不调和的以自我为中心的假我之心。不用说,一切苦恼均由假我产生。

慈爱之心就像父母爱子女的那颗心,他们含辛茹苦地养育子女,却能像阳光一样不求回报,那是培育万物成为大调和之根源的神佛之心。

向他人夸耀自己,这种虚荣心,是求自我满足的最虚伪无常的心。有虚荣心的人是最愚不可及的,适足以表现出其心的贫脊。

不论你的权势有多大,你均无法支配人民的心,但如果能实行仁政,就能拥有人民纯朴尽忠的心,而建立起真正的和平。

此心善加思考,就能明辨善恶,因此王的真实的慈爱,必能使人民产生感恩图报的心。」

佛陀的话,不偏不倚地深中波斯匿王的心。

波斯匿王过去实施的政治,虽有许多是为人民着想,但仍不免是出于私心,所幸臣民都极拥戴他。

国王代表了国家,也代表了法。人民和臣下就是王的化身,王的手足。

国王可以主使一切,在当时认为,王的意志就是人民的意志,人民将意志与想法的行使权委托给王了,因此对王怀有敌意而企图背叛的人,也就是人民的敌人,他之曾遭到杀头的惩罚,也是应该的。

因此之故,当时的国君对人民具有生杀予夺的大权,在王的意志下,没有办不到的事情。

波斯匿王的政府,虽然以行仁政自期,但对于有背叛意图的人,则经常施展残酷。

具有人性的波斯匿王和身为一国之君的波斯匿之间,存在着极大的矛盾。现在听到佛陀的正法,他觉察到过去种种自以为是的作为,事实上潜藏着很深的罪孽。

「大王,人能觉察到自己的罪行时,他的罪已被原谅了一半。

人因为盲目而不知愚昧之所在,若能发现错误而即时改过,并不再重覆错误的行为,这便是真正有勇气的人。」

佛陀看出波斯匿王悔过的心意,因而这样鼓励他。

波斯匿王见佛陀直指自己的心事而很不好意思地想着:

「佛陀竟连我的心事都摸清了,并指示我今后该走的路,真是了不起。」

如此想着,他的神情非常肃穆而认真。

「大王!

心中的思虑,可通达善恶之境,若思念众生的幸福,就能为众生带来安乐。因此正确的思念是很重要的。不藉助他力而只凭一己的努力以慈爱之心来思考诸事,就会为自己的国度和人民带来光明与安乐。

正确的思念之外,将之实践出来,正是从政者应切实做到的。

工作时,就应循正确的方向做去,这是很重要的。

要知道,每日例行的工作是修行过程中重要的一环,不应发出埋怨、牢骚。要知道,你由于有健康的身体,所以能正确地工作,又,对一个能使自己安心工作的环境,也要抱持感谢的心,这是相当重要的。

真实的感谢,可转变为一种报恩的行为,没有发展出行为的感谢,就不算是真正的感谢。

为民众服务,是一国之君首要放在心上的任务。

人生的旅途并不需要你一味地前进,有时应当回过头来看看,看自己有没有走错路,否则目的地将无法抵达。

同时要正确地反省以往的思想与行为,正确的法不会使大自然的形态因人的知或意而改变。

所谓正确的法,就是中道,以没有偏颇的调和的心为基准,它不但是善我的基准,也是善意的第三者的心。

以合于中道的心一一地改正自己的思和行,使自己的心日趋丰润与广阔,才能期于自己免蹈错误之辙。

又端正自己的所思所行,心中充满光明,没有执着,以此心态进入禅定,则身心由于调适切恰而处于光明世界中。

在冥想中,可以与上苍互通心曲。

如果心中依然为一些低劣情绪所盘踞,则进入禅定时,极易为魔所支配而永不得安宁。这是非常危险的,千万要注意。

除了抱持正确的心之外,没有他途能使你安住于光明世界中。

人若能在生活中,将我所说的八个正确的道做为心行的尺度,则他可说是个真正的修行者。

欲从一切苦恼中解脱,也唯有实践这个道。」

波斯匿王聆听佛陀说法的当儿,肩头好像释下了重担,模模糊糊地感觉到一股暖流在心中扩散。

「佛陀,我仿佛回到了心灵的故乡,重见心灵的父母,心中充满了宁静与安适,真谢谢您啊!

今后希望您能继续引导我。

我知道我该走的路是,尽力实施毫无偏差的善政。……」

然后波斯匿王回过头对须达多长者说:

「须达多,

你引荐了一位良师,为憍萨罗国发掘到一件伟大的宝物,我打心底里感谢您。」

「不,不敢当,大王。我只是尽一点心意罢了。」

「不,你带给我们的是比任何财宝都要珍贵的东西啊。」

当他将视线转问佛陀时,就情不自禁地喊了一声「佛陀」,随即双膝跪至地面,问佛陀行最敬之礼。

佛陀说:

「大王!

有人从黑暗的涧窟中走到阳光辉耀的天地间来,但也有人故意从光明走冋黑暗,人生的航路是如此黑暗,伸手不见五指,而人们就在其间摸索前进。

有的人却在心中亮起了法灯,靠着这盏灯,他不但使自己从无常的现实界中斩断了一切执着解救了自己,使自己进入光明之境,同时他又能带领迷途的人脱离令人迷惘的黑暗。

这个肉体不是我们自己的东西,如果它确属我们,则我们必将永远年轻,永远保有它。

然而事实证明,这肉体终究要腐烂溃亡,然后回归大地。

无论你是一国之君,或是贫穷的奴工,均无法改变这个事实。因为肉体毕竟是一种无常的存在。

在心之外去寻求幸福,会产生痛苦。因为金银财宝、名誉、地位、情欲等均是无常的东西,单靠它们是无法获得真正的幸福的。当其中一样获得满足,心中必升起对另一样的欲求,如此辗转贪求,将无了时。

我们的心好比大自然一样,会不断向外延伸扩展的,一旦被欲望驾驭时,就永没有满足的一天。

但是如果我扪能在心灵深处谋求幸福的根源,则往往能获得真正的满足。因为知足的心能使人脱离执着。在生活中视法为一切的尺度,此心就不会为外物所动摇,如此才能享受到真正的安泰,才能安然进入悟境。

到那时,真正的佛国土就呈现在眼前了。」

波斯匿王听了这番话,高兴异常。

他过去虽曾受过许多婆罗门僧的指导,但只限于祭事和祈祷一类的事。心中从未真正领略过什么。他虽努力依教奉行,但不知问题出在那里,心中总有着不满足的不安感。

不但如此,他不知如何把握众生的心。佛陀的法,与仁者的王道政治相契合,是个灵活的政治之学。

他从此认识到,正确的法及其实践之道,正是使他的王国迈问调和的康庄大道。

\chapter{第七章\ 生为女人}\label{ch7}

\section{毗舍佉的布施}\label{sec7.1}

佛陀自来到祇园精舍驻讲后,憍萨罗国有了前所未有的光明景象。佛陀的声名因而更为响亮,皈依的人也不断地在增加。

不仅如此,波斯匿王与祇陀亲王也同时在佛门中言归于好。这对其他的婆罗门修行者而言,是个很大的剌激。

波斯匿王与祇陀亲王虽是兄弟,但不是一母所生,两人的关系很不好,两国间也仅止于礼貌上的往来。

弟弟波斯匿王非常嫌恶哥哥祇陀,经常借故侵入哥哥的领土,视哥哥为眼中钉,常欲拔除之而后快。

但是由于佛陀的慈悲,两人很快重拾孩提时的感情,心中的芥蒂也在不知不觉间烟消冰释。

对这个不可思议的事实,婆罗门修行者们都非常惊讶,更震摄于佛陀不可思议的感化力。

与此同时,他们也就非常恐惧,不知自己向来保有的地位和祭神方式会不会因佛陀的出现而受到影响。

事实上,波斯匿王并不因佛陀的光临而对婆罗门稍存冷淡或歧视。

因为波斯匿王深知婆罗门教的经典中也含藏了无数的真知灼见,和佛陀倡导的中道绝无相悖之处,只因婆罗门教中的种种作为在古老的传统中逐渐僵化,教义仅成为艰涩难懂的学问了,加之婆罗门修行者又惯于挟经典以自傲,炫示自己的特权地位。

婆罗门修行者的行迳令人反感而外,经典本身并没有错。也可说佛陀所说的法,就是婆罗门本身的法。

佛陀说:

「人类生而平等,应相待如兄弟。种姓制度使人迷惘,也使人堕落。不要将心意奔竞于外,要向内探讨本源----」

波斯匿王相信佛陀的法,坚信国家的政治必须以佛陀的慈悲为依归。

基于这样的体认,波斯匿王能以平静的心来对待大众,不论他是平民或婆罗门,丝毫没有差别待遇。

波斯匿王的这种想法虽引起部份婆罗门修行者的不满,但究非多数的婆罗门僧反对他,相反地,由于波斯匿王的转变,佛陀的声誉越来越高了。

祇园精舍比竹林精舍更为宽广,环境更幽雅,四周都有高可遮天的大树,大地明敞,绿意怡人。

佛陀的说法活动,虽有时在白天,但大多数是安排在晚间。

这是因为在家人,白天要工作,晚上才抽得出空暇。

太阳西沉后,天气十分凉爽舒适,这对说法的人或听法的人都是十分理想的时刻。

在说法的广场上,听众所操的灯火,宛如夜空中的星光,美丽地闪耀着。听众从各地聚集而来。

直到今日,每有佛事,即燃有蜡烛,就是这样流传下来的习惯。当时的灯火是由菜籽油所点燃,因为电灯的发明是近世纪的事。当时的照明设备非常差。由于人人在听法时都想感受到佛光,好像佛陀就在身边一样,同时也把灯火象征着永远在心中燃着的法灯。

夏虫由于火光的吸引都聚集而来,它们嗡嗡地绕飞在火光四周。灯火一经点燃,就见数十百只的飞虫纷纷齐集,有的甚至被烧死。

待天明后,只见夏虫的遗骸散布在广场的四处,佛陀每于说法后的翌晨,必为这些夏虫祈求冥福。

再微小的生命,也有生存的权利,也有它们的生存之道。它们虽微小,但终其一生也担负着一份使命与职责。夏虫的生命非常短暂,它们每在扑灯火时加速了死亡,佛陀为此很感悲悯。

自然的无常现象支配着整个宇宙。无论什么人都无法幸免,不管你是多么庞大,或是多么具有睿智,也无法抵抗宇宙的无常。

想要超越无常,唯有证悟法理,唯有洞悉永恒的真谛何在。

坐在讲坛上的佛陀。面对灯火照耀着的众生,开始了他在舍卫城的第一次说法。

「各位听众!看那些聚集在灯火周围的虫儿,虽然它们的生命是那么短暂,但他们为了追求光明,正竭尽全力在生存着。

他们却不知道接近灯火后身体将被焚烧的可怕,以致一个接一个地丧失了性命。在白天,它们是鸟类或其他动物的食物,提供了其他生物生存的资粮。它们的残骸,或成为土壤的肥料,或成为植物的营养剂。

动物们把排泄物散于大地,滋养了植物,而植物繁茂的枝叶,果实又可做为动物的粮食,像这样,自然界的生物生生相养,血、肉、骨交融在一起。

宇宙间的万事万物都彼此发生关连,彼此谐调地生活在一起,绝无例外。

可是各位听众啊!有许多人忘记了这条生存的法则,只注重个人的存在,忘记与他人共生共存的道理了。

我们看大自然,大自然教导我们正确的生存之道,教导我们和谐的相处之道。如果一块土地经久没有雨水的浸润,会有什么样的后果呢?草木枯萎而死,动物无以为生,人类也就无法继续生存。

雨水浸润大地,把肥料溶入土壤内,帮助植物生长,从山上流下来的水汇集成小溪流再聚成大河流,普遍滋润着大地,然后再汇流入大海。

无论是多么污秽的水,在流入大海后也会变得洁净了,然后藉着阳光的蒸发,遇冷后再度变为雨水,滋润大地。

这种轮回,将永不变更地持续着。

各位听众啊!

正视自己的存在!

人类本来拥有无比的力量,被赋予创造大地乐园的使命与能力,却只一味思索自己的事情,经常自陷于孤立,矛盾与痛苦之中。这样是对的吗?

自然界中的一切都那么和谐,唯有人类想独立生存。

藉种姓制度把人类加以区分,以不同的方式来对待,一心希望自己的种姓繁衍不息,其他的种姓沦亡灭迹。这样是对的吗?

藉权势、暴力来欺侮弱小,经常说谎、偷窃、发怒、埋怨、毁谤他人、嫉妒、贪着情欲、不知足等等,以致成为欲望的俘虏。这样是对的吗?

由于以上种种任性的作为,人类给自己带来苦恼,那播种苦恼种子而最后又加以收获的人,就是自己。

各位!你们的肉体看似是自己的,其实不然。如果肉体是自己的,就应该可以随心所欲地活动,然而肉体并不能依自己的心意活动。

尽管你不想生病,但肉体还是会有生病,虽然你不想步入老年,但肉体随着时间会逐渐老化。

不论你如何大量摄取营养,或食长寿药物,死亡终会降临。

不论你苦心积存了多少财宝,当你一口气不来时,分文也带不走。

虽然你拥有地位、名声,在死亡这个残酷的现实之前,也是无可奈何的。

你心爱的妻子儿女,也同样会遭到死亡的残酷的袭击。

大自然的一切都是无常的。

一切的事物都无法脱离无常的控制。

但是,各位不必害怕。

支配你们肉体的心,却是永存不灭的。

具有个性的灵魂,将永远存在着,没有所谓的死亡,它丝毫不受自然界无常法则的束缚。

不仅如此,灵魂不但存在于心中,它更是过去、现在转生时的伟大宝藏,随时等待机缘以谋贡献。

灵魂内含藏着无边智慧,远较现世所体验的人生经验,更为庞大而丰饶。今天我们所拥有的生活环境,乃是每个人为了能更广泛地锻炼灵魂而得到的。

因此,人不可因现前的立场或骄傲、或自卑。

务必要了解,无论是穷人或有钱人,都共处在使其灵魂达到更佳境界的修行过程中。

人类皆兄弟,皆平等,就是这个道理。

而且贫富的差异,并不能做为人生价值的标准。人的价值完全取决于心胸的宽窄与大小。

富有的人,应秉怜悯的心来布施贫穷的人。

贫穷的人应了解人生的价值,敞开心胸,努力行善。

在彼此相互了解之中,藉着正确的作为,将关系调和于大自然的大调和之中。发挥内在的智慧,以慈爱的心为根本,若能发展为服务大众的心,报恩与感谢的连结将会更密切,就能开创出光明、富饶的社会。

各位!

人生是短暂的,有限的。在这短暂的人生中,勿让丑恶的斗争和独占的欲望,毒害了你的心!

与正法一起迈着永远延续的步伐!在你们的心底,存有神佛严正的慈爱!

必须把这慈爱活用在自己的生活中。

无上的智慧就产生这种善意的生活中,智慧之门因此打开了,而能知道自己的前生,藉此得以更美满地生活在现世之中。

所谓的阿罗汉境地,就是这种心境和生活,任何人只要过着合乎法的生活,就可达到此种境界!」

佛陀的言词是平易近人的,给人一种踏实、信赖的感受,而深深地吸引了众生的心。

太阳西沉,四周渐渐昏暗下来,但广场上的灯火,却仿佛是佛陀说法的延伸,照亮了众生,更反映出佛陀。

佛陀忘情地、无休止地说着法,额头上沁出了汗珠,就像是一颗颗的珍珠,闪烁着耀眼的光芒。

对于那些头一次听到佛陀的人而言,佛陀的法,就像当头捧喝,使得醉生梦死般的众生心底,起了极大的振憾。

受佛陀感化的人,愈来愈多了!

佛陀所到之处,人潮汹涌。而那被感化的众生,又像潮水般将佛法向四方流传。

毗舍佉是某一大富豪的独生女,从小就被锦衣玉食地娇宠着。全家人均是虔诚的婆罗门信徒。虽然家里婢仆如云,但毗舍佉从不稍存歧视的心理。

她首次听到佛陀的名,是须达多长者捐献祇园精舍之时。

须达多长者和毗舍佉的父母私交很好,毗舍佉认为,想认识佛陀,经由须达多长者的介绍,是最怡当的了。

毗舍佉在得到父亲的许可后,便立刻动身前往须达多长者的住处。

年龄已过三十而仍然独身的毗舍佉,带了许多仆人去拜见须达多长者。她面容姣好,身穿纯白的衣裙,周身散发着青春、智慧和高雅的气息。

「我是毗舍佉,请长者多多指教。」

「哦!欢迎!欢迎!你父亲告诉我你要来,我一直在等候呢!令尊近来好吗?很久没有见到他了!」

「谢谢您!家人都很好。由于家父年纪大了,家中杂务多半都由我照料了。」「噢!这也难怪,府上没有接管杂事的男子。不过我看你是个有见地,有个性的女孩子,你父母一定很放心把事情交给你办的。你有什么事要我帮忙的吗?」

毗舍佉的父亲并未将毗舍佉此来的目的明说,只是请他多加照顾罢了。须达多以为毗舍佉是代替她父亲来谈生意的。

「我听说您把祇园精舍捐献给佛陀,感到非常惊讶。心想世上怎会有这么慷慨的人。后来听父亲说,您替许多孤苦的人们做过许多善事。我就一心想向您学习。」

「真不敢当!一点小事不算什么的。」

毗舍佉从须达多长者谦虚祥和的笑容中,认识到长者宽宏的胸襟,更由他这儿可以了解到佛陀是怎样一位伟大的人物了。

「您是否能带我去见佛陀呢?」

毗舍佉率直地道出了此行的目的。

须达多对于这一要求倒是出乎意料之外,他楞了一下,然后微笑道:

「你有这样的心,真是太好了。我会尽力为你安排,不过佛陀目前不在祇园精舍。」

「什么时候会回来呢?」

「嗯---」

须达多一时也不知道确切的日期。

「这样吧,等佛陀回来后,我再通知你,好吗?」

「真不好意思麻烦您了!」

「你听过佛陀说法吗?」

「听过,在舍卫城郊外的公园听过!」「你有什么感想吗?」

「嗯,刚开始的时候有些惶惑,但越听越平静。」

「嗯!」

「所以我一直盼望您能为我引见佛陀!」

「一定会的!」

说着须达多把手置于脑门,困惑地问道:

「你又为什么一定要我介绍呢?」

「我觉得自己一个单身女子,与佛陀素昧平生,突然前去求见,总不免太冒失了!」

「原来如此!」

「我要和先生您一样,向佛陀及僧团行布施。」

「你能这样想真是太好了。讲经说法的人是国家的宝,我们当努力帮助他们,使他们无后顾之忧。让我们为众生多做点有意义的事吧!」

须达多长者说话时,边注视着毗舍佉,边在心里想着:

「她看来那么柔弱,想不到心中却有了不起的念头!」

大多数的女子都是柔顺而没有主见的,她们需要被人爱护及照顾。

但毗舍佉就不一样,她跟男子一样地长于思考、计划,并急于实现自己的理想。

她虽然在富裕的环境中长大,却不会像一般被娇宠的女孩,常因一时冲动而做出难以想象的事情。从毗舍佉的言词与态度上可以看出,她做任何事都比男子更为冷静,绝非出于一时的冲动。

在这战乱无数的社会中,一个独身女子独掌偌大的家产,实在是很不简单的事。

须达多长者不禁想到,毗舍佉说不定有着较男子更强的办事能力。

毗舍佉在与须达多长者谈话时,双手自然放在膝上,很得体地问答着,姿态典雅,具有大家闺秀的风范,她偶尔将眼帘低垂,那眼神是那么善解人意,给人好感。

须达多长者说:

「令尊大人一向乐善好施,相信你也已承袭了他的这一美德。盼今后我们能合作,为社会多做些善事!」

说到这里,须达多长者执起毗舍佉的双手。

两人虽是第一次见面,但由于志趣相投,竟像是好朋友般,心中毫无任何年龄与身份上的芥蒂。

毗舍佉自从知道须达多长者的诸多义行善举之后,一直就对须达多长者非常景仰。再说今天的一席谈话中,更确立了她心中的想法。

毗舍佉暗自期许自己,要像须达多长者一样,尽自己的能力为众生做事。

后来,毗舍佉果然将鹿母精舍捐给佛陀的僧团。她对僧团的贡献,不亚于须达多长者曾做的。

毗舍佉离去后的十多天,须达多长者派人来通知她谒见佛陀的日期。

毗舍佉雀跃不已,立刻启程,赶到佛陀的驻足地。

\section{女人的业}\label{sec7.2}

由于毗舍佉加入了布施和服务众生的行列,祇园精舍的规模已远超过竹林精舍好几倍。

除了能容纳原有的僧众外,还可接受新的修行者。

祇园精舍位于一大片平缓的丘陵地带。气候温和宜人,距佛陀的故里—迦毗罗卫城,走路约一天的行程。

自佛陀迁驻于此,迦毗罗卫人民经常来访。他们太怀念佛陀了,有时虽然见不到佛陀,但能瞻仰到佛陀的住处也就心满意足了。

迦毗罗卫的人民,发现往日的悉达多太子,如今有着如许大的改变,都纷纷谈论著这件事。

大家都想皈依佛陀,注意心灵问题的人是越来越多了。

佛陀经常闭门静坐,审慎地注视着自己。

如果自己所说的法有错误,则众生和僧众就会远离正道。故佛陀每于说法前,

必反覆地思考,以整理出说法的重点。

但佛陀从不预定说法的内容与顺序。

佛陀注意听法者的反应,随时应机说教。

法是配合众生的心理需求而可以改变说法的,故而必须具有随机应变的说法能力。为此佛陀必须将大部份的时间用于冥想与禅定,以增进这方面的能力。

佛陀关心的是重点的叙述,所以有时说法的内容虽不是很有系统,但对他说法的目的毫无影响。

他以基本体验的方法,叙述法的真实性,这样足以打动听众的心。

为什么?因为人同此心,喜怒哀乐的感受人皆有之,这就是体验的本质所在。知识虽因人的贤愚而有不同的程度,但基本的生活体验则是人人共通的。佛陀把握了这个重点,他绝不是在玩头脑游戏,而是针对了人们对生活的真实体验来传他的道的。

「佛陀,毗舍佉来了!」

舍利弗进来报告道。

佛陀自冥想中转来,回答说:

「请她到这里来!」

舍利弗带领毗舍佉和同行的女伴,进入佛陀的房间。

须达多长者趋前介绍道:

「佛陀,我有个好友的女儿想来拜见您!」

毗舍佉及身后的女孩们,都深深行礼,并将头低垂着。

「欢迎你们。请抬起头来,不必拘礼!」

女孩子们听到佛陀温和的言语,都慢慢抬起头来。毗舍佉这才开口道:

「谢谢您!佛陀。我名叫毗舍佉,曾听您说过法。非常感激您的教诲。今日终于达到了拜见您的愿望。真谢谢您,也谢谢须达多长者。」

毗舍佉如此说时,露出了虔敬的笑容。

明亮的阳光照射在她脸上。

她微施脂粉的脸颊上泛着红晕。她似紧张,又兴奋,但强自镇静着。但凡一个人初见佛陀时,都会有些紧张不安的。

毗舍佉继续用带点紧张的音调说:

「因为想听佛陀说法,所以我们一家才搬到舍卫城来。听法时,我多半坐得很远,但内心十分平静,喜悦不住由心底升起来。我愿贡献一己的能力为僧团做点事,有什么事需要我们帮忙的,请佛陀吩咐一声!」

佛陀其实早已明了她的心意。

「你是个很奇特的女子,欢迎你来。我了解你布施的心意,不过我问你,你自小娇生惯养,经常在父母面前撤娇,你知道以什么回报你的父母,你知道如何善待仆人吗?」

佛陀的口气虽然严肃,但态度非常亲切。

毗舍佉一时不知如何回答,过了一会儿方说:

「我自从听您说法后,对父母非常顺从,同时对仆人的贡献存着感谢的心,经常想法子回报他们。」

「很好。千万别忘了!」

「谢谢您,佛陀。我一定牢记您的叮嘱!」

佛陀认为一个人不仅要知对僧团行布施,就是身边的事,该做的,仍应尽力去做到才是。

这正好直指毗舍佉的心坎。

布施不在大小,而在真诚与否。真诚的布施可以感动很多人,净化人们的心灵。

佛陀的法布施,能给人们带来心灵的喜悦与希望,并为人类社会奠下光明的基础。

因而佛陀之法得以流传。

佛陀说:

「毗舍佉,由于你的布施,佛团中僧伽们会衷心地感谢你,这功德足以为你的心灵带来光明。病人们每在吃药时,必会想到你的布施。他们的健康因你的布施康复了,真谢谢你。」

「佛陀,我布施并不求回报,我是因为要有所贡献而行布施的,我所能做的,也只有这些了!」

「毗舍佉,这就够了。」

太阳把光与热赐给众生,也没有要求谁回报,但众生的感谢是发自内心的,对方虽不求回报,他仍能感受到这功德带来的光明。

你家的庭院中有许多美丽的花朵,虽然它们有凋谢的一天,但每到春光明媚时节,它们又都绽放了。你撤下的种子,也将永远安稳地留在你心底。」

「佛陀,谢谢您,请您答应我,让我在我的家乡建一所精舍。那儿有一块土地,环境非常幽雅,是修行的好场所。我希望您的法,也能使我家乡的人得到好处,这是我的一番心意,请佛陀接受我的请求!」

佛陀当下答应了毗舍佉的要求。

佛陀并派目犍连主持该精舍的捐建事宜。

接着毗舍佉又小心翼翼地提出她的想法。

「我将每天供养僧团中的饮食,可以吗?」

「你为什么会有这个想法呢?」

「我见修行人既要修行,又要花时间出去寻食物,如果我按时供养他们饭食,他们就能节省时间在修行方面,并有更多的时间来引导众生了。」

「嗯,这想法很好。我们接收你这方面的布施了。」

「谢谢您。也请佛陀让我布施药食给生病的人,希望他们能早日恢复健康。」

「毗舍佉,你的布施将为你心中带来安慰与光明。」

「佛陀,谢谢您!对了,还有一件事要向您禀明。我在雨季中有一次在恒河边,看见有些僧侣赤身露体地越过恒河。岸边有些娼妓就对年轻的僧侣说:「不趁年轻的时候来享乐,还要等到年老吗?不妨现在跟我们一起玩乐,修行的事还是等老了以后再说吧!」我听了很难过,我要布施雨衣雨伞给僧侣们,以免他们再受到戏谑。」

说时,毗舍佉面现严肃的神情。

佛陀对她的一片心意,非常感动。

如果人人都像她一样具有布施的心意,这世界就会变成佛国的乐土。

佛陀不忍拂逆毗舍佉的一番心意,故而一一答应了她的请求。须达多长者在一旁也为毗舍佉的言词感动得连连点头称赞,大有自愧不如之慨。

由于毗舍佉的热心捐助,鹿母精舍很快地建造完成了。

鹿母精舍诚如其名,有许多鹿穿梭林间,微风拂面,是很理想的修行场所。毗舍佉很高兴自己的意见被佛陀接纳了。

自此以后,毗舍佉都把握机会听佛陀说法,尽心尽力地追求正法,并在布施的热诚下过着心安的日子。

另一方面,她家中产业的经营,并不因她的沉潜佛法而稍有怠忽,反而更为兴旺了。

由于毗舍佉平时待厚仁慈,家仆们存着感激的心思。即使毗舍佉不在旁督导,他们也能地把工作做好。

毗舍佉有时也从佛陀那儿得到工作上的指导。

「佛陀,我只是一个女子。却要劳您抽空给我指导,我很惭愧不安。如果我不是女孩子,情形将会比较好。」

「毗舍佉,虽然你是一个女孩子,却能以慈爱的心与他人交往。不贪、不怨、正正当当地做事。

你能时时顾到家仆们的生活,你的心地就像太阳一样的光明。由于家仆们能分工合作,你家的生意才如此兴隆。

大多数的女子,很容易为一点点小事就生气,同时意志不坚,欲望又多,不知满足,对为自己做事的人很不心存感谢。

还有些女子,虽然性情温柔、正直、不怨天不尤人,对身边的事能满足,但却不知如何以爱心待人。

另有一种女子,心地善良,凡事能满足,毫无怨尤,对别人拥有的幸福同样感到高兴,同时能以慈悲心来照顾困苦的人。

以上三种类型的女子,以最后这一类型最适合正法。

以法为心灵的一而生活着的女子,常能制裁心中的假我,远离一切执着,并生活在安逸之中。

男女在性格上虽然不一样,但刚柔功能的调和是很重要的。

女子在家庭中担负着使家庭美满幸福的功能。

男女在肉体上虽然不平等,但心灵应是平等的。也就是说,男女的性别没有优劣之分,二者的平等地位本是上天赋予的。

女子在结婚成家之后,不久就会生产子女。做妻子的,其责任就是在家中保护并养育子女。

夫妇之间的态度,对孩子具有教育的意义。彼此能互相协调扶持,才能养育出健康活泼的孩子。

这类家庭越多,越能促进世界的和谐。

女子既嫁至夫家,对公婆应尽心孝养,不论在何种情况下,都不可产生怨恼,不可忘记「忍辱」的美德,使一家人过健康快乐的生活,是非常重要的。

如果你因他人的言词与行动拂逆了自己的心意而心存怨恨,也就等于在自己的心田中种下了苦恼的种子,最后将无以自拔。从此家庭中再不存祥和之气,成员间不断在纷争之中,最后家庭的结构瓦解了。

所以在一个家庭之中,万不可有对立的现象出现。妻子应确实了解先生工作的性质,在精神上给他支持,并不断在生活细节上努力提高自己品行上的素养。这才是一个女子应做的事。

无论是对亲人或仆人,都应心存慈悲,亲切地待人接物。

先生在外劳苦奔波所得来的收入,做妻子的切不可任意花用,应有计划地储存之以备不时之需。

男女二人组织家庭,就是在共同负担起协调社会生活的责任。这绝不是偶然的或冲动的结合,它是在转生轮回的过程中,二人因深切的缘份而联结成的组织。

因此夫妻应相敬相爱,而这关系即将使缘生关系延续下去。

夫妻的因缘关系是依正进而存在着的。故二者的存在在协调的功能上具有极高的价值。

大多数的女子会因自身的美丽而生出骄傲心,但真正的美丽,往往表现在生活纯正、心思无私的女子身上。

女子由于自己的美貌而骄矜自满产生强烈的优越感,动辄瞧不起他人。

真正学习正法的男子,是不会被这种女子所诱惑的。那些被诱惑而致意乱情迷的男子,是最愚不可及的。

受骄矜、情欲等支配的男女,多属愚蠢而不知修身者。他们的一生也就充满了不幸。

许多女子,为了满足一己的虚荣,拼命夸张自己的美貌。殊不知美貌适足以使她深受其苦。

这样的女子很容易就沦为男子的玩物而致终日烦恼不安,终身都无法摆脱这种境遇。

再说,女子自幼即受双亲的刻意保护,没有充分的自由。即至长大成人嫁至夫家,又受先生的约束。生养子女后,又被子女剥夺了自由。

因此一个女子的诞生是不会受到家中成员欢迎的,这是与男子大不相同的地方。在婚前,做父母的为她将来的归宿操心,婚后,又怕她和先生相处不睦。同时又与她一起分担等待新生儿的焦虑。所以一个女子很难获得轻松愉快的生活境遇,内心也经常不得安宁。」

毗舍佉听完佛陀这一席对女子的评论,心头震憾不已。她从未深虑过自己身为女子的立场。

女人有女人特殊的工作,那是与男人大不相同的。

即使女人尽心力去做男子的工作,人们在潜意识中仍不会忘记她那女人的身份。在商场或旅途中,男人可以独来独往,女人却必需有男伴同行,才不致遭遇不测。

由于这先天上的不平等,许多女子在不知不觉间也就产生了依赖的心理,这种依赖的心理使她们更无法与男子争平等。

女子经常是处在被保护的状态下,她们在潜意识中也就要求男子的保护。基于这一现象,女子的行动也多属被动。

毗舍佉对佛陀的这一番见解深有所感。

她很想知道如何才能超越这一切呢?

自己一向都能做连男人都望尘莫及的事,心中并有着非凡的抱负。但是此刻,在佛陀面前,她感到自己很渺小,她只是一个微不足道的女人。

「那么我应如何做才能超越女人的障碍呢?」

她终于鼓足勇气提出这个问题。佛陀并没有回答她,只是面露微笑地注视着她。

毗舍佉生长在富豪之家,又是父母的掌上明珠。虽然在佛陀面前她自觉渺小,一直低垂着头,用心听道。但是在家仆或其他男性面前,她仍不免有着优越感,自视很高,常不经意就会使出命令或嘲讽的态度。

毗舍佉一向都能以慈悲的心怀来待人接物,但是她的慈悲心经常混杂着不自觉的优越惑,不时地流露出骄矜与自负。

要想超越女性特有的障碍,必先摒除这种优越感。舍掉自己高高在上的心理,把自己看得跟其他人一样,具有平等的对待关系。

\section{波斯匿王的盛情}\label{sec7.3}

在迦毗罗卫国,盛传着释迦牟尼佛陀的消息。王公贵族间群起骚动,因为有人说佛陀将于近日回归故里。

这消息来自听过佛陀说法的商贾们,一传十,十传百地,消息就这样传开了。净饭王自然也很快得到这个消息,于是派人四出打听,却没有结果。王想这大概是个毫无根据的传言吧!

但是想想自己的儿子出家至今也有十二年了,这日子不算短了,也应该是回家看看的时候了。净饭王由于年岁大了,不再执持以往的看法,他只盼望能再见见自己的儿子,他希望趁这个时候催佛陀返乡一趟。

佛陀当初出家时,曾留下一个儿子名罗候罗,如今已年满十二岁,是迦毗罗卫国王位的继承人。

年已七十的净饭王,对人生再也没有什么热望了。

他曾不止一次地派人去游说悉达多回国,非但悉达多没有一次如他愿回国,且大多数被派去的使者都跟随学佛而一去不返。

与十二年前的情形相比,豪门贵族出家修行的人数增加了许多,最近更因不断有贵族出家,使得迦毗罗卫国的政治人才损失不少。

净饭王终于想出一个权宜之策。

他恳求憍萨罗国的国王波斯匿劝悉达多回国。

他认为这个办法定能行得通。

他并将这想法与左右大臣商量,大家也都采一致的看法。

「那么派谁出使憍萨罗国呢?」

净饭王说着,注视周围的群臣,接着又说:

「憍萨罗国是个大国,我们不能失礼。要派个最适当的人选前去才是,须克鲁,有什么好意见吗?」

「我想这个任务,还是阿姆利德最合适。」

须克鲁推荐身边的阿姆利德。

阿姆利德抬起头,以清晰的语调说:

「我愿承担这个重任、同时我也想听听佛陀说法。我们应当请佛陀回到迦毗罗卫国来说法,使全释迦族都有机会获受法益。据说波斯匿王也常请佛陀去说法的。」阿姆利德是净饭王的弟弟。众人对于由他来担任使者,都无异议。

「阿姆利德,悉达多是你的侄子,这关系自又不同。这件事情请你无论如何要请波斯匿王帮个忙,劝佛陀回国。这也是全释迦族的心愿。我的年纪也大了,想在人世时能再见他一面。你一定要达成这个任务啊!」

净饭王的眼中流露着期盼。

「这件事包在我身上了,请您放心!」

阿姆利德的口气充满了自信,接着他建议道:

「还得派个人去太子那儿。我想优陀夷是个适当的人选。他是一国的大将,不会那么容易被动摇的。已有那么多人都跟太子出家了,真不可想象。」

然而过了几天,优陀夷穿着僧衣来到净饭的跟前。

优陀夷还想到回国,这与其他的使者不一样。这是后来的发展。

且说优陀夷被召到净饭王面前,得知自己承担的任务后一口答应了。净饭王不放心地问道:

「优陀夷,你确信没有问题吧!」

「大王,我可不像舍弟优陀尼耶,他当初去竹林精舍覚会一去不返,我是个领兵的人,一定会回来的。」

优陀夷挺着胸膛,口气坚决地说。

于是阿姆利德与优陀夷一同上路了。

二人骑在马上,有说有笑,状至轻松。

他们都有坚定的信念,认为定能达成任务,但是离迦毗罗卫城愈来愈远。优陀夷的心中竟有些起伏不定了。他似乎没有先前那么有把握了。

但他想想,除了他,还有王弟阿姆利德担当此任务,他并不需要担当全部的责任。虽这么想,他仍瑟瑟不安。

阿姆利德的性格十分爽朗,遇事多能镇定自如,愈接近目的地,他愈感振奋。优陀夷在后头问道:

「您觉得还好吧!」

「嗯,这跟到别国不一样,天气也不错,我的精神也很好。」

阿姆利德笑声朗朗,很高兴地回答道。

迦毗罗卫与憍萨罗国世代盟交,一向友好相处。憍萨罗国的边境卫士知道他俩是迦毗罗卫国的刹帝利,一些不留难地就放他们过关了。

憍萨罗是个大国,能轻而易举地平息任何战乱。不但国土面积大,军队的规模也大过迦毗罗卫国好几倍。

战士们不像今天有手枪等机动化的武器,全是以刀箭来作战。所以在当时,军队的人数是决定胜负的要素之一。当然小国也有小国的战术,他们常利用地形与战略来增加敌方进攻的困难度,或设计打击敌人的士气。不过尽管如此,就实力而言,仍要屈居大国之下。

憍萨罗位于印度北方,是众多邦国中的一个大国,小国如迦毗罗卫。都受她的疵护。

阿姆利德吩咐对优陀夷道:

「波斯匿王那边的任务达成后,你就一个人到祇园精舍去探望佛陀。」

阿姆利德说到这里,不禁想到哥哥的思子之情是如此之切,心中充满了感伤。他想,由优陀夷一人去去比较妥当。

但是这个想法错了,当初他若跟优陀夷一同前去,或可阻止优陀夷出家哩!二人越过丘陵,穿过树林,渡过河流,最后进入舍卫城。这时正是夕阳西下时分,浑红的太阳把城中的建筑物都染成了金红色。

二人绕到东门外,因为外来的人须由东门入城。

城堡坚固,高可接天。

城墙上有眺望台,正有士兵守卫着。

这座城真不简单,如果谁冒冒失失地来攻打,准会全军覆没,就是搭个梯子也上不去。」

「若想攻下这个城,您看用什么方法最好呢?」

「唯有断绝兵粮的输入,并用拖延时间的方法,不然就是躲在粮车内混进城去,趁夜间打开城门,用突袭战攻入。」

「这可得有周详的计划啊!」

「是啊,如果真要攻城的话。」

「哈哈哈……」

「哈哈哈……」

二人在东门入口处忘情地大笑起来。

阿姆利德拿出预先准备好的通行证。

「我是迦毗罗卫的使者,求见贵国国王。」

卫士们看看通行证,大概他们已事先得到通知,故很爽快地说:

「请由这边来,我们恭候很久了!」

二人被领至客殿。

由于两国是盟约国,向有亲善之举或为传送情报,故而没有交谊机构。二人来访的消息,早经由这个机构传达至波斯匿王了。

城中到处都有士兵做着警戒工作,宫殿的森严更可想而知了。

宫殿中的墙壁,走廊平时看起来没么特殊,然而一旦战鼓响起,壁、廊中可以隐藏许多的士兵,发射出千矢万箭。

防止敌人放火烧城,城壁内侧砌以红色砂岩,围墙也高高的有二、三重。这结构与气派也非迦毗罗卫城可比。

阿姆利德看到这种规模,不由得暗自叹息。

宫殿里的装潢摆饰,更令他叹为观止。

举凡圆柱、墙壁、空顶、地板等,都镶嵌了各式的红蓝宝石,光耀夺目。

殿堂广阔,可以容纳文武百官。

服式华丽鲜艳的卫兵,列队站在入口处,想来身手亦必不凡。在威武的仪表外,还有着一对锐利的眼目。

二人正浏览间,波斯匿王在侍从的拥卫下,进入大殿。

阿姆利德和优陀夷立刻趋前行礼问候。

「大王,我是迦毗罗卫国净饭王的使者阿姆利德,这位是优陀夷,在此向您致意了!」

波斯匿王端坐在雕工细腻的宝座上,说:

「请不必多礼,一路辛苦了!」

阿姆利德深深震慑于大国君主的威严。

「蒙您赐见,很感荣幸。祝大王政体康泰!」

「您是净饭王的弟弟吗?」

「是的。」

「净饭王近来好吗?」

「王兄的年纪大了,近来感情脆弱,容易感伤。他时时记挂着悉达多太子的事,希望他早日回国。」

年轻的波斯匿王听后,连连点头,似有所感。

「我在听过佛陀说法后,才知道人生的真谛,以及为王的使命。释迦族出此伟大的圣人,真教人尊敬!

我很高兴敝国与贵国的太子具有这样的缘份。听说佛陀出家至今尚未回国,他能以一刹帝利的身份而得到最高的悟境,必是个圣人无疑。我要向释迦族致最大的敬意!」

「谢谢大王的夸赞。但我释迦族人却从未听过佛陀说法,包括王兄在内,所有的人都深深企盼有这一天。」

「这是可想可知的情形。」

波斯匿现出同情的容色。

阿姆利德心想,一定要先由波斯匿王亲口提出佛陀返乡的建议,这样才能希冀他尽力帮忙。

对波斯匿王而言,佛陀简直是憍萨国的一块宝。一旦佛陀离开舍卫城,他将不知如何是好。所以他虽然同情迦毗罗卫国,却不热衷于佛陀是否回国的这个问题。

「你们此来有什么事吗?」

「我们就是特意前来央求大王帮忙的,希望大王能劝悉达多太子回国,这是净饭王和全迦毗罗卫人民的希望。请大王全力支持!」

波斯匿王回顾左右,陷入了沉思。他一时之间不知如何回答。「我非常明白你的意思,要知道佛陀我的良师,也是我们的国宝,他已不再是释迦族的独生子了,他是全世界伟大的导师,他是佛陀。我了解净饭王的心情,但教我如何去劝说呢?」

这番话分明是拒绝了阿姆利德的请求。

阿姆利德凝视地面美丽宝饰,一面思索着适当的措辞。他从而了解到悉达多太子的声望与力量。

「大王,我王兄和释迦族人都明白悉达多太子不会一直逗留在迦毗罗卫的,只因王兄年纪衰老,思子心切,在他有生之年希望能再见爱子一面,同时也想听听佛法。请大王多多谅察!」

由于波斯匿先前那种不熟衷的态度,阿姆利德只好退而求其次,不再强调希望佛陀永留迦毗罗卫国的话。

「我了解你的意思了,我并没有独占佛陀的意思。我会立刻请佛陀返乡,据我所知,他迟早总会回国一趟的。」

波斯匿王是个很明理的君主,经他如此一说,阿姆利德才松了一口气。

「大王,谢谢您,有劳您了!」

「明天一早我就到祇园精舍去,你们一路辛苦了,也该休息了。有什么事尽管吩咐宫人去做吧!」

二人深深向波斯匿王致谢。

随后宫仆带领二人前往客室。二人兴奋莫名,一时并无睡意。

第二天一早,波斯匿王一人骑着马来到祇园精舍。

「佛陀,波斯匿王来了!」

舍利弗趋至佛前通报道。

佛陀此时正在冥想,舍利弗的声音,把他从遥远的虚空界拉回现实。

「哦,舍利弗,国王来了吗?」

「是的。」

佛陀起身之际,顺便问道:

「你喜欢竹林精舍呢?还是祇园精舍?」

「对我来说,人生到处都是修行场所,也各有好处,是不应去做比较的。」这也正是佛陀询问的关键处,他似能看穿舍利弗的心意,故而出此一问。

人生到处是修行的场所,也到处都有指导自己的人、事、物,这种自觉是很重要的。环境本不是形成人格的要件,如果一个人易于受环境的摆布,就容易造下各种业。

祇园精舍就客观情形来说,是要比竹林精舍优点多。不但气候温和,即环境的整顿与设备也比较周全。但是僧团中的修行者却不可以沉醉于这种舒适的环境而忘了修道与渡众的重任,这是要牢记的一点。故而佛陀把握机会探询舍利弗的想法。

「波斯匿王此刻在那里?」

「在讲堂恭候您呢!」

「他一个人来的吗?」

「是的。他就把马拴在东门外。」

对此一现象,佛陀很感欣慰。因为想不到波斯匿王会无所疑惧地独自由王宫出行。

由王宫到祇园精舍虽没有多远。但路途上或有歹人埋伏袭击,同时修行所中也可能有敌人的间谍潜伏着,他们虽不一定住在祇园精舍中,但难保他们不乔妆成僧侣的模样来暗中行动。

自从佛陀驻锡于此,各类暴力事件已剧减不少,到处呈现着祥和、安溢的气氛。由于这一现象,波斯匿王也敢只身离宫出游了。

「大王,您一大早驾临,未能出迎。很抱歉!」

「佛陀,不敢当。我有件事要麻烦您!」

「请直说,不要客气。」

「这件事我想您一定办得到。虽然我并不愿意您去做,但是替他人想想,又不能不说。」

波斯匿王吞吞吐吐,不知如何道明来意。

其实佛陀早已知道波斯匿王的来意。此刻见波斯匿王为难的模样,就代他说出来了。

「我知道,我不会永远住在迦毗罗卫国的。那里虽然有我的双亲,也有与我缘份深厚的族人,但我于说法后,还是会回到祇园精舍的。」

波斯匿王听了佛陀这一番安慰的话,高兴得声音都哽咽了,他迫不及待的开口道:

「是啊。佛陀,如果您舍弃舍卫城,我该怎么办?我们听受佛法的日子是这么短。如果您离我们而去,我们会很难过的。」

「我知道我必须把正法的灯点燃,否则你们将无法真正得到解脱。」

波斯匿王对佛陀的善体人意很感高兴。

「佛陀,对,对,说实话,您的叔叔阿姆利德昨天来访,目的就在请您成全您父亲的心愿,赶快回国,意思是永远留在家乡,这使我很为难,如果您回去后不再来,将是我国人民多大的损失啊。佛陀,您一定得回祇园精舍来!」

「您的心意,我很感动。请问阿姆利德叔父回去了吗?」

「他还没有回去,不过我答应他先来见您的。」

「真谢谢您!」

「那么请告诉我,您预定什么时候启程?」

「我预定今年秋天回国,请大王转告来使好吗?」

「好的。」

波斯匿王说着,换了一个较轻松的坐姿,他为佛陀要回归故里一事,打从开始就很紧张。

佛陀已不是往日的悉达多太子了,他肩负着济渡众生脱离生死苦海的大任。因此佛陀不独仅为释迦族传道,也不是波斯匿王一人的导师。

然而目前的情形是,佛陀的肉身出自释迦族,眼前又受着波斯匿王的厚遇。

因此他仍要顾到现实,不能把自己高高地树于超人的地位。在现实世界中,我们必须时时考虑对方的立场,考虑人类实际的状况,这才合乎自然,也才能契合众生的心理而传扬正道。

所以面对波斯匿王,佛陀虽殷切地想到回归故里一事,一面又体恤到波斯匿王的一番心意。

佛陀和弟子们恭送波斯匿王离开精舍。波斯匿王回到王宫后,立刻把佛陀的计划告知热切盼望佳音的两个来使。

两人自然雀跃不已。

阿姆利德竟流下眼泪,说道:

「我这下可以安心回迦毗罗卫了,王兄多年来的心愿总算可以达成了。为这大好消息,真恨不能立刻飞回去。真感激您,大王,释迦族不会忘记您的恩惠。有需我们的地方,请随时吩咐。」

他兴奋得一再称谢。

「请您不要激动。说实在的,刚开始我还不知如何把话说明白,是佛陀自己把他要回迦毗罗卫的心意说出来的。我一句话都没说呢,这真教人吃惊。

佛陀了解我的心意,他也没有忘记家乡的父老。净饭王有这样伟大的圣人做儿子,真是一个有福气的人。」

他将手置于阿姆利德的肩头,不住点着头。

阿姆利德既有了波斯匿王的承诺,也就放心地先启程回国了,并命优陀夷到精舍去见佛陀。

回到迦毗罗卫,阿姆利德一五一十地把经过详细地向净饭王呈明。

净饭王惊喜不已。

他凝望宫顶,思念着佛陀。

十二年后,才得以再见爱子一面。

悉达多自小就很倔强,但此刻他对悉达多的倔强个性却有着莫名的怀念。那个倔强的太子,如今成为万人景仰的佛陀。在净饭王眼中,连波斯匿王都尊之为师,可想而知悉达多的改变是如何之大了。

他很早就想听自己的儿子说法了。

\section{优陀夷出家}\label{sec7.4}

优陀夷奉命去祇园精舍,原是想去和佛陀商量回国的细节问题。

「太子殿下,好久不见了。很高兴知道您要回国了!」

他的心中仍存着佛陀十二年前在宫中做太子的影像,所以不假思索地又说:「我算是没有白来舍卫城。太子您虽然变了,但往日的容貌还在,我真想念您哪!」

憍陈如在一旁听了,说道:

「优陀夷,这里没有悉达多太子,只有佛陀,你就称佛陀吧!」

「哦!憍陈如,真抱歉,佛陀,请原谅我的无礼!」

憍陈如经过这么多年的修行,外貌已有所改变,神情十分安泰,再也没有了往日那种颐指气使的模样。

佛陀注视着十二年没见的优陀夷。

「佛陀,我弟弟在这里吗?我也很想见见他。」

「优陀尼耶现在在摩竭陀国。」

优陀夷失望地点点头。

「优陀夷,你心中有没有苦恼?」

佛陀关切地询问优陀夷。

现在在优陀夷心中,只有弟弟修行和佛陀归国两件事。

对佛陀这一突如其来的问题感到很茫然。优陀夷,你没有苦恼吗?」

「没有,现在我没有任何苦恼!」

「是吗?但是你的目光锐利,像老鹰盯着猎物似的。由眼睛可以了解一个人的心,你的眼神非常猛烈。如果你的心中有斗争之火在燃烧,斗争之事就没有终止的一天。

老鹰在与猎物的搏斗中,渡过他的一生。

优陀夷,你是个武将,你几乎就在为争取战功而活着,你的心中永远没有安宁。

你看,憍陈如从前曾是你的上司,如今舍去了假我,皈依正法,他因为能摆脱执着,所以心倩很安逸。

你不久就会明白,你所寻求的东西都是无常不实的。

一国的君主在去世时,一样不能带走只瓦片土。

不仅如此,就是妻子、儿女、财宝,甚至自己的肉体,都没有办法带走。优陀夷,你认为这身体是属于你的吗?不是的,它是借来的。」

优陀夷很注意地听着,但没有听懂。

自己的身体为什么不是自己的呢?

「我认为这身体是自己的。当我拧捏自己的皮肤,我会感到痛,如果是捏别人,我就没有感觉。为什么您要说这肉体不是我们自己的呢?我实在不了解。」

「当你睡觉时,有个人在旁边不管怎么骂你,你也不会生气。这是什么原因,你想过吗?」

「那是因为睡着了的缘故啊!」

「睡着了,为什么就没有关系了呢?」

「因为没有意识了吧!」

「你的意识因为睡着就没有了吗?」

「这我就不知道了!」

「现在,你的肉体和心合为一体,所以能听我讲话,并以你的经验和知识来了解话中的意思。等你睡着之后,你的一切行为和感情都没有了,在这时候,你的肉体变成一无用处的东西,耳不能听,鼻不能闻。

肉体只能算是人生航路上的一条船,因此可以说,肉体是借自这个世界的一样东西,是带不去的。

优陀夷,你的肉体并不是你的,凡是向这世界借来的东西都是虚幻无常的。

假如这肉体可以由你支配,是属于你的,你就会永远年轻,不会衰老。但是事实是这样吗?」

「不错。我一直想要保持青春,但现在已四十五岁了,白头发越来越多了。

「优陀夷,人自生下来后,就免不了会生病,会衰老,最后是死亡,这正表现出肉体的无常。」

「这肉体真的是无常的。」

由于佛陀的指引,优陀夷总算有些明了了。

但如果是进一步的人生问题,他仍然莫名所以。

「优陀夷,你没有苦恼吗?」

佛陀再一次提出相同的问题。

「我想没有。只是不如意的时候会感到悲伤,也很容易生气。这是不是苦恼呢?」

「这就是苦恼了。当你心中有事情纠结时,就是一种苦恼了。

你有不如意的感觉,那是因为你有欲望。当欲望无法满足时,就产生苦恼了。欲望就是苦恼的根源,人常在苦恼中生存。

人的生活中充满了各式各样的苦恼。

优陀——你还认为你没有苦恼吗?」

「佛陀说的对,想来我周身都是苦恼。譬如现在就有苦恼。我担负着劝佛陀回国的责任,心中一直很担心。这担心也算是一种苦恼吗?」

「是。我因为已从一切苦恼中解脱了,所以要向众生宣说其中的道理。」

优陀夷的眼睛突然明亮了。

刚才如老鹰般锐利的眼光,一变而为虔敬的柔和。

「佛陀,我误会您了。我从前总认为您十二年不回国,让大王担心,是一个不孝顺的人。但是听了您的开示,我知道传扬正法来消除人间的苦恼是更重要的事。我现在总算明白了,谢谢您,佛陀。」

「优陀夷,你知道就好了。我最近要回国一趟,希望在回国后,你能帮助我。」

「那当然,我一定会的,佛陀。」

「你既知道自己有苦恼,你想不想除去它们呢?」

「谁都不会喜欢有苦恼的,我当然想除去。」

「那么你愿不愿意留在这儿一段时间,好好学习舍弃苦恼的法。」

「只要您允许,我一定努力学习。」

「不过祇园精舍的修行者,都得剃发,你这一身武士装扮也不适合,你要不要把衣服换-换?」

当此之时,憍陈如等已为优陀夷准备好僧衣,并准备为他剃发了。

阿舍婆誓把优陀夷带出佛陀的房间。

当阿舍婆誓为优陀夷剃发时,笑对他说:

「真是太好了,优陀夷。你也参加我们修行的行列了,我诚心欢迎你加入我们,过着安适的生活。

剃发出家是需要下定决心才做得到的,你认为如何?」

「阿舍婆誓,我知道。我已多少了解一点佛法。每个人都想脱离苦恼,我就是回迦毗罗卫后,也要继续修学正法,我会遵守修行的法规。」

「优陀夷,我们很高兴自己身为佛陀僧团中的一员。摩竭陀和憍萨罗等国的国王也都皈依了佛陀的正法。能在佛陀身边学习正道,是非常神奇的,你有没有这种想法呢?」

这世界尚有许多人在还没有听到佛法之前就去世了,那是多么可惜。

优陀夷,好好学习吧!你心中必能充满喜悦和感激的。我们也会尽力帮助你。」

「那真是太好了,谢谢你。想不到在离迦毗罗卫这么远的地方,竟有那么多故乡的修行者。我真高兴,我就像在迦毗罗卫一样的舒服。」

优陀夷道出了心中的惑受。他想,如果没有憍陈如和阿舍婆誓等这些同乡做伴的话,自己的心情就不会如此轻松愉快。他虽然相信佛陀说的每一句话,但他认为出家毕竟不是一件寻常的事。

他现在看到许多昔日在宫庭中的伙伴,因为有许多人都曾是佛陀在当太子时的随从,如今都跟佛陀出家修道了,他置身其间,好像仍在迦毗罗卫的宫中一样,心情很安稳。

优陀夷的头发已剃好了。

他并穿上僧衣,已是一个修行人了。

他摸摸自己的光头,笑望着阿舍婆誓。

「怎么样,还习惯吧?」

「很习惯,随时可以去游化了。」

「哈哈哈……」

「哈哈哈……」

从这一天开始,优陀夷就开始与阿舍婆誓作伴修行了。

第二天早晨,他和大家同时起床,并清扫房间,佛陀于身后唤他。

「优陀夷,你觉得怎么样,僧衣还合适吧?头也很凉快吧?要好好努力啊!」他回身向佛陀称谢道:

「托佛陀的福得以皈依正法,回迦毗罗卫后,我也一定要努力修习正法的。」

「这样很好。你可以把这里修行的情形告诉家乡的人,自己更不要忘了修行的事。」

「谢谢您,佛陀。」

他就这样暂时在精舍中过着僧团的生活。

在僧团中,白天多半出外游化,或在野外静坐冥想。有的僧人在街上为众生宣扬正法。

最初一、二日,优陀夷只是留在精舍中,担任清扫和看门的工作,以后就由阿舍婆誓陪同,在野外进行反省冥想的工夫。

优陀夷与阿舍婆誓并肩修习禅定之时,每每因为腰脚疼痛而无法继续。每隔十五分钟,他就会张开眼睛望望身旁的阿舍婆誓,而后者正是一动也不动如死尸般已沉入冥想。

看到这种情形,优陀夷惑到很惭愧。然后挪动一下座位,松松筋骨,再闭上眼睛。虽然如此,他的心仍觉不安地想要睁开眼睛。

有时候他干脆仰望天空,注视着悠悠的白云,以此打发时间。

他不禁暗想,反省自己竟是如此艰难的一件事。

这种日子继续了四、五天。到剃发的第七日,他拜别佛陀,启程回迦毗罗卫国。

「阿舍婆誓,我的衣服和刀、箭在哪里?我今天要回迦毗罗卫了,得整理行装了,麻烦你把我的东西拿来。」

「什么?你不是代表佛陀先回国做通报的吗?你的弓箭早就给城里的人了。」优陀夷这才明白,自己在不知不觉间已是修行僧了,自己带来的俗物都已不再属于自己了。

他无可如何地进入佛陀房中,向佛陀告辞。

「优陀夷,一路上可要小心啊!」

佛陀像是不知优陀夷心思似的,笑着对他说。

他低垂着头说道:

「我知道,佛陀。您对我的帮助很大。我将在迦毗罗卫等待佛陀的归来。谢谢您,佛陀。」

优陀夷的心情很复杂。

他在离开迦毗罗卫前,曾向净饭王保证自己不会像弟弟一样跟佛陀出家的。

但是现在既剃了头发,又穿着僧衣,如何向净饭王交代呢?他越想越不了解自己是怎么一回事。

当他走出佛陀的房间后,就径自离开了精舍。身上一无所有地踏上了回家的路子。

由于路途遥远兼有险阻,需要两天半的时间才可到达。

如今他两手空空,还必须在旅途中乞食,这是他从未有过的经验。

\section{迦毗罗卫城的骚动}\label{sec7.5}

他一早离开精舍,到了中午,肚子已耐不住饥饿了。

他站在一家农舍的屋檐下。一个老年妇人从里面出来,把芋粥盛入他的钵中。当他称谢后正要转身离去时,老妇人请他主持一项祭祀。因为这天是她先生的忌日。

优陀夷很感意外,但因为自己一副僧侣的打扮,也不好拒绝。

他只好依言进入屋子。在简陋的祭坛前,开始念诵婆罗门的咒文。

他念着连自己也不懂的咒文,偶尔眯着眼睛偷望老妇人。只见老妇人正合着掌虔敬地祈祷着。

他也就继续诵念咒文。

最初他的肚子饿得咕噜作响,可是当他一心念诵咒文时,也就忘了饥饿这回事。

就在这时候,他眼前出现佛陀的身影。

佛陀全身被金光围绕着,正法视着他。他觉得佛陀似乎把光传送给他,他为此吃了一惊。

他以为佛陀来了。

于是他口中的咒文不知不觉消失了,他合并的双掌自然靠近嘴边。这时他大大地吸了一口气,然后再把气吹进另一间黑暗阴沉的房间。

接着他的双臂向左右伸展开去。

这个动作重复了好几次。

祭祀完毕后,他回头望着老妇人。此时老妇人正低垂着头,泪流满面。

老妇人颤抖着身子,抬起头对他说:

「真感激您行这么隆重的祭祀,我的亡夫一定会很高兴。刚才您向房里吹气时,房中充满了耀眼的光,眼睛都没有办法睁开。我虽然不知是怎么回事,但我能觉到就在这房子里,真难得啊,我感动得眼泪都流出来了。实在感激您。」

优陀夷很感为难,他老实对老妇人说:

「刚才的祭祀其实不是我做的,我在念咒时出现的光,是佛陀带来的,是他为这屋子带来了光。我这才知道佛陀的伟大啊。」

「那么你是佛陀的弟子啰?」

「我只是其中的一个罢了。」

老妇人不禁又流下泪来。

「其他的修行人,如果我要求他们为亡夫祭祀时,都只是站在门口念念咒文,可是你却进到房中,还把佛陀的光也带进来,您的慈悲与恩惠,我永远也不会忘记。谢谢您,出家人。」

「不要谢我,这些都是佛陀带给您的。」他领受了老妇人的热诚款待后,就继续上路。

离开老妇人后,他重新回到自己原来的身份意识中。

经由刚才神秘的祭祀,他重新体认到佛陀的伟大。他在想,该如何向净饭王报告自己经历的一切呢。

强烈的阳光照射在身上,他拖着沉重的步子,默默地走着。

僧衣虽然轻便,他的心清却十分沉重。他此刻所有的,只是一袭僧衣,以及刚刚老妇人供养的粮食。

要是全身穿着武服,佩戴刀枪,就要神气得多了。即使碰到山贼也不怕了。目前一无所有,即使有两三条命,也不够赔的。

当他行至山间,太阳早已西沉,是准备就寝的时候了。

就在这时候,他看到四、五个面目狰拧,山贼模样的人物。

他们全都留着大胡子,其中一人,手持大刀,边喊边砍伐着身边丛生的草木。另一人看到优陀夷,但只注视了一会儿,知道只是个修行僧,就继续向山里走去。

虽只是一转眼的工夫,也足教优陀夷毛骨悚然了。

他正在想,如果他们来找他麻烦,他不是抵抗就是逃跑。没有其他法子好想。但是他们根本没有惊动他。

是简陋的僧衣救了他。

露宿野外的第一天,他就体验了修行者轻安的生活。人世之所以有苦恼,即在于人们对身边的物起执着,被身边的物所摆弄。

「你真的没有苦恼吗?人的苦恼来自五官和心灵对物的执着。……」

佛陀的话,重在他的耳边响起。

他用野火暧和身子,回想白天发生的每一件事情,并重新回味一周来在精舍的生活。想着想着,他躺下来。

在不知名的鸟鸣声中,他睡着了。

第二天,他把柴火打理好,重新上路。

翻过山头,他来到原野地带。

他将昨天老妇人供养的干粮取出来充饥。这是一种用火烧烤过的稻米,能够耐久不坏。

他将干粮放入口中,慢慢咀嚼着,米的香味渐渐透出来,他觉得非常好吃。

当天下午,他抵达迦毗罗卫城。

他的头脸与僧衣,布满了沙尘。

谁也没看出他就是优陀夷。

他每每要大声喊道:「是我,优陀夷!」别人才会恍然大悟。

大将领优陀夷如今变成一个沙罗门了。这个消息立刻传遍全城。

在当时,只要一点新奇的事,整个城立刻就会知道。

城中的人,都在谈这件事。

在进宫殿的大门前,他抖了抖身上的尘土,并在井边把手、脸、脚都洗干净。人们就聚集在远处看着他,不停地交头接耳,指指点点。

他清楚地感觉到来自身后的这一切骚动。但是此刻,他的心比任何时候都平静。

「不管他们说什么,想什么,我都不在乎了。」

他心中已有了决定。

他从容地经过注视的人群,进入宫中。

然后来到国王面前。

国王看到优陀夷这一身打扮,吃惊不小。

「大王,太子于今秋回国,恭喜您了!」

「优陀夷,你怎么了?你这种样子……你是什么时候出家的?一个大将领也会有苦恼吗?」

「是的。我在不知不觉中出家了。」

「不知不觉是什么意思?难道你是在梦中出家的吗?」

「不,不是的。太子,不,佛陀在说法时,我就像着了魔一样被他的伟大所感动,听他说法后,不知不觉就心安了,想法也就变了,这真是不可思议的现象啊!我本来打算暂时留在祇园精舍,以便多看看佛陀,如果得到佛陀回国确切的消息,就立刻赶回来。可是在这段期间,受到阿舍婆誓的指引,不知不觉就穿起僧衣,做修行人了。大王,请原谅我吧!」

「优陀夷,你跟我约定好的,说你绝对不会出家的,结果你还是毁约了。」

「我真的很抱歉。」

净饭王对优陀夷的行程,既感高兴,又有着失落的感觉。

他想,我的儿子悉达多,如今竟变成这样一位不可思议的人物,真万万想不到。

「大王,请您千万要听佛陀说法,我一直都误会佛陀了,如今才知道事实的真相。

现在的佛陀,举止、言语都是那样庄严优美,跟从前当太子时大大不一样了!」

净饭王默默无语。

这些话好像来自遥远的地方。连优陀夷都被悉达多迷惑了,这真不是一件简单的事。

如果负有保卫责任的武士们,一个个都出家了,那该怎么办呢?到那时,岂不是释迦族要灭亡了吗?自己的死亡,竟然牵连到种族的灭亡!

看看己经衰老的自己,净饭王禁不住起了这样的念头!

但是他很快又否定了这个假设。

一直连绵不绝的伟大的释迦族,怎可能就在自己这一代灭绝呢?

他在心中反覆着类似的肯定与否定,好不困扰。

他力使自己转移思绪,便问道:

「优陀夷,悉达多有没有什么话要转告我……」

问完,即迫切地凝视着优陀夷。

「有的。佛陀挂念大王的健康情形,而且还要来渡大王。」

「什么?度我?……也要我做沙罗门吗?」

「我想不是这个意思,不过我也不太清楚。」

净饭王闻言,不禁苦笑起来。

悉达多到底是什么意思呢?他实在想不透。

「他对罗睺罗的事都没提什么吧?」

「没有。」

「哦?……什么也没说?你以后都打算过修行的生活了吗?」

「我想是吧!我很适合这个样子。」

「现在再要你留头发,换上武士服,看样子是不可能的。反正悉达多要回来的,你就以这样子来迎接他好了。关于迎接佛陀的事,就由你跟阿姆利德共同商量进行吧!」

净饭王退朝下来,即来到波闍波提夫人的宫中,笑对她说:

「优陀夷出家了。你看到他的光头和僧衣了没有?悉达多他……」

波闍波提夫人只是默默注视着净饭王。

「长此以往,释迦族人可能全部都要出家了。我真不希望有这一天。……

去悉达多那儿的人,十有八、九都会出家。我想悉达多定有什么不可思议的力量。

那个波斯匿王竟然会舍不得悉达多回国,真是不可思议。净饭王眯着眼,不自觉地夸赞起自己的儿子。

「哦?优陀夷也出家了吗?」

「我看他刀箭都舍掉了,一副僧侣的打扮。」

「连优陀夷都出家了,我真不敢相信。」

「但是事实就是这样,没办法。他那一头漂亮的头发全都剃掉了,满头青青的,好像很凉快。我也去剃个光头,你看如何?哈哈哈………」

净饭王说完,捧腹大笑。

看在波闍波提夫人的眼里,这可不是什么好笑的事情。要是连国王都出家了,后果将如何?

国王出家了,就等于全释迦族人都出家了。

像波斯匿王,是以在家的身份来信仰佛法,但如果一国之君剃去头发,则表示他连国家都不要了。

她认为净饭王虽是开玩笑,但也不能太过份了。

一定要去听听儿子的说法,他有能力使优陀夷出家,一定有着不同凡响的地方,或许他真有不可思议的吸引力。

波闍波提夫人如此想着,心中不禁一阵恐慌。但潜意识中又有着前去听法的愿望。

净饭王接着又说:

「优陀夷曾当着那么多人的面跟我约好不出家的,看他怎跟人交代,我这儿子实在厉害啊!」

「大王,我想悉达多已经不是普通人了。他是世人口中的佛陀了。我也真想去听他说法。」

「什么?你也有这想法吗?」

「是啊,那孩子可是我养大的。听听孩子说法,总没有什么坏处吧!

我还希望在天臂城的哥哥也听听他说法呢!」

「夫人,这是你的心愿吗?你打算让天臂城的贵族们也出家吗?这可不行。」

这次轮到净饭王担心了,他极力反对夫人的主张。

如果连天臂城也受到佛陀的影响,那么整个释迦族就真的要灭亡了,刹帝利种姓将一个也不存在了。

绝不能这样。事情可不能闹大。净饭王脸上的笑容不见了。

佛陀要回国的消息,立刻轰动整个迦毗罗卫城。

人们聚集在一起,都是在谈有关佛陀的一切。

同时,人们还谈到出家的事情,许多人都因此而感到不安。

佛陀留在宫中的妻子耶输陀罗,如今从罗睺罗身上获得生存的安慰。她知道自己丈夫即将回国,心情更是复杂不安。

她很疑惑,自己该以怎样的身份来迎接佛陀,该当他是佛陀呢?还是自己的丈夫?

十二年,似很漫长,但转眼也消逝了。

这期间,有寂寞,有悲伤,有时又怀着希望。

自罗睺罗长大后,悉达多从未见过他。悉达多在听到自己孩子降临人世后,便给孩子取名为罗睺罗(意即「障碍物」),却很少来看孩子。

他如今已成为众人景仰的佛陀,但不知他见到这个顽皮的十二岁的儿子,会有怎样的感受呢?

他会很高兴吗?会像一般人一样地接受他吗?

自悉达多离宫后,她一直心怀恨意。

她恨他一句话也不留地就走了。

自那时起,她就搞不清楚悉达多究竟是自己的丈夫,抑或只是个陌生人。

一连好几夜,她为此困恼得无法成眠,她不知日子将会变成什么样子。

她有着无地自容的感觉。

她无法原谅自己的丈夫。

她甚至想到,如果有机会再见到悉达多,她必不放过他。

然而这憎恨的情绪,因着罗睺罗一天天的长大而渐渐消散了。

在佛陀行将回国的此刻,她决定以一个陌生人的方式来迎接他。

她告诉罗睺罗说:

「你父亲几个月后要回来了!」

「是吗?爸爸是什么样子?是像祖父?还是像祖母?我真想马上就到他。」

罗睺罗望着白色的穹顶,想象自己父亲的形像。

耶输陀罗望着天真烂漫的罗睺罗,忍不住流下泪来。

悉达多如果是因故去世,自己或许还能平静地安渡余年,他却偏偏是为了逃避什么而出家,使自己无法安定。

她的恨意又渐升起。

悉达多也是人子,为什么就忍心一去不返,不回来看看自己的骨肉,或跟他说说话呢?

佛陀的开悟,是牺牲了天伦之乐而得到的。

佛陀的开悟,是真是假,非亲眼见及是难以想象的。望着罗睺罗,耶输陀罗的心情十分复杂而矛盾。

罗睺罗为了将来继承王位所需,每天必须勤练剑术与弓法。

而且他在学习婆罗门教义之余,对宗教有着异乎寻常的关注。

他自得知自己的父亲即将回国的消息,脑中就不断浮现修行僧的模样。他似乎已能很清晰地描绘出父亲的形像与面貌了。



\chapter{第八章\ 佛陀返乡}\label{ch8}

\section{十二年后第一次回国}\label{sec8.1}

夏天最炎热的时期过了,秋意由喜马拉雅山麓传来。凉风袭击着迦毗罗卫城。佛陀一行,依约于此时启程回国。

行经憍萨罗国的各乡镇间,佛陀与弟子们依例是说法游化。

不久,到达迦毗罗卫城外的尼拘陀树林。僧众们仍像往常般,一早外出游化,过着规律的修行生活。

他们在林中静坐冥想,为了改掉心中的邪杂思绪,必须不断与内心的假我搏斗,因而要专心地反省自己。

苦恼的影子反现于伪我之中。一般人不能体察无常的真相,总误以为五官所形成的「我」便是自己。

不仅如此,周遭的人也都惯以那种眼光来审视你。从自己的肉体开始,举凡身边的一切,没有永久不变的。

你若依附无常来探求一切,最后得到的还是无常。

无常的本质就是变幻无常,人心专注于此,日久亦会变得虚幻寡情。

总括来说,假我即是一种无情的,虚幻的「我」。

是迷惑的「我」。

是苦恼的「我」。

佛陀严格地指导弟子们做反省工夫。

生活上的细节,佛陀多半由弟子们自己处理,很少过问,唯独在反省上不下工夫或经常颠倒是非时,佛陀才反覆地严加指正。

每当佛陀直指心事时,弟子们都无词以对,对佛陀的伟大感召,他们多半重又激起改过的勇气。

佛陀入城的日子,越来越近了。

几天后,佛陀一行进入迦毗罗卫国境。

村人们齐集于大街上,好奇地张望着衣衫肮脏的僧侣队。

佛陀率领一行人一村一镇地走过去。最后来到一片森林地带。雨季已经结束了,到处呈现青绿。小鸟清唱着,溪水潺流着。大家伏在溪边,浸润着干涩的喉咙。太阳俯照绿地,清洌的流水掠过静止的大地。

佛陀所到之处,都呈现着调和的自然景象。正法即在自然之中。

佛陀由悟而发现了自己的法身。佛陀是自然?抑或自然是佛陀?实在无法分辨。

这就是一种奥不可言的三昧境界。可说佛陀已和大自然合为一体了,一切的执着皆已断除,和法在一起时,佛陀就是法本身,而与万化冥和了。

与万化冥合,在法中发现自己时,大自然就动静自如,有着无穷的生命在焉。佛陀的手中有自然,自然的诸象,就像在掌心中,清晰可见。

三昧的境地,即是将自然纳于掌中的境地。

佛陀欲往何处,没有什么能阻挡得住的。往迦毗罗卫国的路程,在他眼前逐渐展开。

在这座森林中,佛陀与弟子们再度停下休息。

僧众中有属释迦族刹帝利种的,多半已达阿罗汉境地,是其他弟子修行时的指导者。

他们此次也是自出家后首度返乡,都相互商讨该如何让故乡的人了解佛法。憍陈如问阿舍婆誓道:

「你身为佛弟子,又已证阿罗汉果。回到家乡后,若说过去世的因缘诸法,恐怕很难教人相信。我认为应让他们了解生命的永恒性,以及转生轮回的事实,这是很重要的。人生不是只有今世,关于这一点,你有什么看法?」

阿舍婆誓擦拭着头上的汗珠,回答道:

「告诉人们有关转生轮回的体验,是很重要的。但我认为要人们了解实践正法的意义,当更重要。」

「那当然!了解并实践佛陀的法,可以使自己的心扉宽敞,体验到生命的奥秘。除了实践之外,没有更好的路子。迦叶、摩男俱利、跋提,你们的看法如何?」 

「我们也是这么想的!」

三人异口同声答道。

佛陀在旁静听五人的谈话。

待五人说完,佛陀接口道:

「你们一定很想念故乡吧!和亲戚朋友见了面,一定有说不完的话,可能连睡觉的时间都没有了……」

五人闻言,相视而笑。他们充满了兴奋之情,一心想着如何把自己悟道的经验述说出来,让乡人们也能了解正法。

憍陈如对佛陀说:

「佛陀,其实世界任何地方都是我们的住所,人与人之间有如手足。我认为在故乡的亲友们,也都是佛陀的弟子。所以,如何使兄弟们了解并实践佛陀的法,就是我们最关切的事。」

佛陀点点头道:

「憍陈如啊!正法需要靠自己确立起来。尤其在教化与自己关系亲近的人时,自己切身的体验与修行的事实才足以打动对方。因为体验后的正法能直达心灵,给心灵安泰。人心都是一样,你有这样的心,必能连带感动他人的心。

虽以知识的传授方式使对方接受,但无法使对方真正心领神通,同时修行人不能光靠知识的讲授,那也不是修行之道。只有在对正法了解后从而加以实践之,再从心灵深处涌现出智慧时,对方才能感动于你的传授,才能同样获得心灵的安泰与和谐。去实践正法,心灵才能充满光明,也才能打开悟道之门啊。……」

憍陈如与其他四人听了佛陀这一番叮咛,都高兴得向佛陀顶礼称谢。

\section{城中的乞食团体}\label{sec8.2}

迦毗罗卫国上下正在为佛陀的光临,做着种种准备。

大臣马歇向净饭王提出报告,他说:

「陛下,不知汁么缘故,近来城中到处有衣着肮脏的修行人。太子快要回城了,让他看到这情景,不大好,我想把这些修行人遣散到其他地方去,特此向陛下请示。」

「你的设想很周到,你就迅速去办这件事吧!」

于是马歇派出许多手下,持着净饭王的指令,分头利城中各角落去劝导那些衣冠不洁的出家人立刻离城。同时马歇也亲自出宫巡视。

当马歇正在四处巡察之时,不想背后有人拍他的肩膀。

「久违了,马歇,我是跋提!」

「啊?您是跋提?真是好久不见了呀。您怎么会在这里?……您怎么是这副模样?」

马歇不敢相信自己的眼睛,眼前的跋提,穿着破烂的僧衣,就跟乞丐一样。

「请问佛陀在那里?	」

「还在城外的林中休息。」

「那么城里的这些出家人都是佛陀的弟子吗?」

「不错!他们这时候应该都在城里乞食了。我们离开舍卫城已经有三天了。」

「哎呀,那不好了。我没有想到那些乞食的出家人是佛弟子。我为了佛陀回城时不发生意外,正令部下劝他们离城呢。现在必须下令停止这件事了。」

说完,立刻命左右传达撤令之意。

同时马歇不住地向跋提道歉,并请跋提与他一同回宫歇着。跋提则坚持要带领马歇去见佛陀。

于是马歇随跋提来到佛陀跟前。悉达多太子的容貌虽然没有变,但显然他已不是昔日在宫中享福的那位太子了。此刻见他身形宽阔,周身充满慈悲的气息,深深打动了马歇的心。

他伏身于地,低着头,对自己的失礼不断致歉。

「佛陀,请原谅在下的,实在不知道他们就是佛陀的弟子,正好在城中遇到跋提,故而没有造成大错。我真懊悔自己的鲁莾与愚蠢,请您原谅啊!」

「马歇!好久不见你了。你看来很好,这比什么都好哩!其他人也都好吗?」

「都很好,谢谢佛陀的关照。家人都托您的福,大家都很好。」

「那最好不过了。我的弟子们在城中惹麻烦了吗?」

「不是,没有,没有那回事……」

「我们不能拿衣着来评估一个人,他们都很知足,心中没有憎恶,不会发牢骚,也没有污秽的思想。如果我们只靠外表来评量人,必会忽视了那个人心中的美。要知道,心地美的人才真正称得上是人。」

「您说得很对,佛陀。我很惭愧,请原谅我!」

「我明天会回宫,请转知大王!」

「是!」

马歇满脸羞惭,战战兢兢地仍然低垂着头。

不用佛陀说,他也早就知道不能以服饰来评量人。明知此理,但他仍不免会犯这种错误,尤其当他一心一意替佛陀回城的安全问题担着心思之际。为了佛陀及佛弟子们的安全着想,反而赶走了佛弟子,这不是一大讽剌吗?

他辞别佛陀后,立刻赶回宫中报告此事。

「陛下,真糟糕!原来那些乞丐模样的出家人都是佛陀的弟子啊!我已向佛陀道歉了!」

「什么?佛陀的弟子是乞丐模样吗?我族中未曾出过乞食的人,如果真是这样,真是太丢人了!他们需要什么,尽管供给他们,城里的人知道他们是悉达多的弟子吗?」

「我想都知道的。」

「这有关释迦族的权威问题啊!」

「陛下,您不用操心。佛弟子们虽然穿着乞丐式的衣服,但是很有威严,那是乞丐所没有的。我在城中碰到跋提时,就有很强烈的感觉,跋提虽然衣着破烂,但是他步伐稳重,态度庄严,眼神慈和。我觉得以服装来评断人是错误的。」

「你是在对我说教吗?」

「陛下,在下不敢!这是在下从佛陀那儿得到的启示。一个人不管外表如何,心中若有不足,就和乞丐一样的贫乏了。光以外表来评价,就会远离事实的真相,以致失去了心中的平静。」

「马歇,你倒很会说嘛!」

「佛陀是一个伟大的人!」

「你?难道也想跟他出家吗?我们的国中,如果没有了刹帝利种,就会受到外地侵略了。到那时,释迦族就要绝灭了。这样子真对不起祖先,你们应该好好专精于武艺,不要再想别的了!」

「陛下,请放心。我拚了命也要保护迦毗罗卫城的。佛陀的教诲我会牢记,刹帝利的使命我也会达成的。」

「这就对了,一个人一定要有这样的决心。像你们这样武艺超群的将领不好好干,还指望手下的刹帝利吗?释迦族的延续,要靠你们了。……」

净饭王语重心长地说着。他迎接佛陀来归的心情是相当复杂的。

翌日,佛陀一行向迦毗罗卫城进发。

离国十二年,他如今带领着众弟子---包括舍利弗、目犍连、蒙迦罗闍、波萨罗、埋托勒呀、桠那等人----回到故里。

迦毗罗卫的城堡,远看就像浮在海上的战舰一般,钢造铁铸地屹立着,敌人无论由何处袭击,城内都能立即迎战。

所不同的是,这个战舰的周围披戴着繁花茂林,色彩夺目,这大约是迦毗罗卫城悠久的传统所使然。

紧张中透露着明媚与轻松,似乎迦毗罗卫是个把战争与和平揉合在一起的国度。

曾生长在这里,过着锦衣玉食生活的佛陀,现在一身轻松地回来了。迦毗罗卫在望,弟子们益发感受到佛陀人格的伟大。

净饭王、波闍波提夫人,以及难陀、罗睺罗等,全族上下都为了迎接佛陀,穿戴整洁华丽的服饰,齐集于东门口。而佛陀一行,则破衣素服,显示出强烈的对比。

金碧辉煌的王族团体出迎衣衫褴褛、风尘仆仆的乞食僧团。

给人的感觉,相当奇异。

净饭王轻声对身旁的波闍波提夫人说道:

「悉达多真让释迦族丢脸,为什么不能像其他婆罗门那样衣冠楚楚呢?既不乘象,也不骑马,实在有失体面。……	」

「大王,早知如此,应早点送些衣服到林中去。」

「他不会接受的。以他的想法,那些都是无用的东西。我真不懂他脑子里想些什么!」

波闍波提夫人听罢,默默地点点头,并注视着渐行渐近的佛陀一行人。

罗睺罗也站在母亲耶输陀罗身边,迎接父亲的归来。

眼前出现的父亲,和他平时想象中的父亲,有着相当大的距离。

罗睺罗是在多么优渥的环境中长成,他如何能想象王宫以外朴素的生活呢?他经常从身边的人们口中采取父亲的有关传闻。在他的小脑袋中,总认为父亲就是像婆罗门修行者一样的有着华丽炫目的容姿。

然而现在父亲的衣着有如乞丐,父亲身边的人又都黑黝黝的,配上褴褛的僧衣和疲惫的步伐,是显得如此的寒伧。

「他就是我的父亲吗?我该如何来迎接他呢?该如何称呼他呢?他给我的感觉是这么陌生。……」

罗睺罗在心中嘟喃着。

不一会儿,他又想道:

「虽然父亲是这副模样,但说不定有着什么魔力,不然为什么连优陀夷也被他说动了出家呢?我们不能光凭外表来看人,父亲说出来的话一定不简单的,我还是不要胡思乱想的好!」

耶输陀罗则自始未发一点声响,在一旁静默而含蓄地迎接佛陀。

她的眼睛一直紧盯在佛陀身上,身边人的交谈,她一句也没听到。

她目不转睛地望着那位与自己分别十二年的丈夫。

佛陀终于走近了。他的影像逐渐扩大。耶输陀罗的双眼弥漫着泪水,以致视线模糊,佛陀突在她眼底消失。

这么多年的等待,丈夫终于回来了。虽然人人称他佛陀,但他毕竟曾是自己的丈夫。常在她梦中出现的丈夫,如今就站在她伸手可及的地方。

她的泪水不听使唤地漱漱流下,她有一股放声痛哭的冲动,但是不得不强自压抑着。

就在这相见的一刹那间,她已在心中原谅了他。虽然她一向是怀抱着怨恨与不安,但一见到他,就再也寻不出一丝责备他的意愿了。

对耶输陀罗而言,丈夫的外表无关紧要,只要他能好端端地回来。

十二年来的空白,在这一瞬间消逝。怨恨的情绪也逃逸得无影无踪。她一直不知自己该以怎样的身份和地位来迎接这曾是自己丈夫的佛陀,这使她一直都处在心绪不宁的状态中。但是就只这一刻,她的不安早已不知去向。

当她的泪水溢出眼眶之时,她觉察到这不可思议的现象,心中惊异不置,深深感觉到人心的奥妙。

\section{与双亲、妻儿相会}\label{sec8.3}

佛陀带领着众弟子步履从容地来到东门外。

佛陀的心是如此宽广无边,态度是如此泰然而安详。

东门附近的景象依旧。

前来迎接的人们,依旧是那么令人怀念。

头戴王冠的父王,自然是人群中最显眼的。

脸上满布深痕的净饭王,面露微笑但神情迫切地注视着自己的儿子。

「父王,久久未能前来问安,深感歉疚!今天我和众弟子们要来叨扰您布施和照顾了!」

浄饭王看着佛陀,早已忘掉先前的不满与怨言,眼里充满了泪水,激动地说:

「悉达多,终于盼到你回来了。我听过许多传闻,想必你也经过一番清苦的修行吧!能见到你,我就心安了。你的寝宫还保持着原样,你就在那里好好休息吧!罗睺罗,快来拜见你的父亲,他如今已是憍萨罗王和摩竭陀王的师父,是一位悟了道的救世主。悉达多,这是你的独生子罗睺罗。」

净饭王迫切地引见父子俩。

他一心一意为自己的孩子谋求幸福,期望悉达多回宫,暂时忘了悉达多出家求道的目的何在。

自己的孩子悉达多已是万人的导师,正精神奕奕地站在自己面前,他忍不住泪流满面。

罗睺罗先是犹豫了一会儿,接着以清晰的声调喊了一声:「父亲大人!」

喊完,面带微笑地望着佛陀。

佛陀也笑望罗睺罗,挚起他的小手点头示意。

对每天勤习武艺的人来说,罗睺罗还稍嫌白了点,大概太受爱护的缘故吧!

接着佛陀轻抚罗睺罗的肩头说:

「你好吗?」

佛陀目不转睛地注视着自己的孩子。

十二年前,悉达多为自己的孩子命名为「罗睺罗」。

「罗睺罗」是障碍物的意思。那时候,悉达多一心一意想着出家求道,毫无为人父的打算。

正当他下定决心出家时,耶输陀罗产下一子,好像为他的出家带来了一个绊脚石。

如今面对罗睢罗,深深气自己未曾免尽父责而感到愧疚,但他同时知道,自己悟得的正法可以补偿他的这一过失。

佛陀一行与王宫贵人们,由东门开始向宫廷进发。净饭王喜于形色,不断向路边迎候的国人点头示意。

佛陀则亲切地牵着罗睺罗的手走着。

城里的民众,为了迎接太子归城,也都到处张灯结彩,富豪人家都穿上织锦华服,以示欢迎的盛意。

然而佛陀并未注意这些,他只感觉到自己亲骨肉的手中有股暖流正向他传送,他对自己与罗睺罗间十二年的空白有着异样的感受。

亲子间不可思议的连结感藉着罗睺罗手上的暖气传达给佛陀。

这是佛陀自来到这世上所感受到的另一奇妙的情景。

有了孩子以后的情境,唯有自己真正为人父母后才能了解到,那不是言语或理念所可表达的。

如此他也就可以想见净饭王对自己的感情,当是如何的深,如何的奥妙了。

父母对子女付出的爱,是不求报偿的,他们就希望子女平安地长大,幸福地生存。

但是对待子女以外的他人,就没有这么深刻了,总像中间隔着什么似的。这就是我们生存着的空间,因为有了亲子的亲密关系以致与他人反而产生了莫大的隔阂。如此说来,亲子关系的存在正是人人平等亲密交往中的一道围墙吧!

另一特殊现象攫住了佛陀的心。

那就是,佛陀虽有抚养自己长大成人的母亲,却没有生身的母亲,而罗睺罗则是没有父亲。

虽然两人都生活在爱的环境中,但却都不能在同时拥有双亲的境遇中成长。这使罗睺罗更有着惺惺相惜之感。

他深责自己是个未尽职责的父亲。

佛陀的母亲是在佛陀出世一个礼拜后去世的。

但罗睺罗的父亲则一直健在,随时可以伸出他大而有力的手来扶持罗睢罗。然而他竟没有这么做。

十二年,对他与罗睺罗的关系来说,是一场空白。

佛陀看看前面的父亲,父亲的肩头由于岁月的痕迹,略形狭窄。他看了也有说不出的感慨。

固然说他为了解脱人生的苦恼而出家,但出家一途未始不是一种逃避的行为,他当初就是为了逃避世俗的。

最初六年的苦行,逃避的心态成了他开悟最大的障碍。

他一味想从苦中脱离,以达乐境。但是就因为他执着苦乐的分别境遇,以致无法真正明了「苦」。一个人怎能希冀在乐境中了解着「苦」呢?

当我们真正了解了「苦」是什么,才能获得真正的安乐。苦乐不明,人才容易陷入迷惘与烦恼中。

佛陀想到这里,大大地叹了一口气。

十二年来,他为乡国带来了多么大的困扰,他要设法补偿这一切。

宫殿在望,佛陀加快了脚步。在进殿前,他将自己的双足洗涤干净。

净饭王直接将佛陀引进往昔佛陀身为太子时的寝宫。这座寝宫已空寂了好长一段日子了。

耶输陀罗也先一步走至宫门前恭迎。

净饭王对佛陀说:

「你先休息一会儿吧!」

说完,就留下他与耶输陀罗,自己亦径自返回内宫。

佛陀环顾了一下室内的设置,仍与他离宫时一样。耶输陀罗此时开口道:

「悉达多,真是好久没见了。这里有一件僧衣,是你在摩竭陀国修行时,我特为你做的。因为一直没有机会送给你,而一直留到今天,请现在就换上吧!」

耶输陀罗手中拿着喀西产的细绢所制的华丽僧衣。

「哦,耶输陀罗,谢谢你的好意。不过我不需要这么华贵的僧衣,你诚意的布施,我心领了。

这些年来,辛苦你了,把罗睺罗抚养长大,我非常感激你。」

佛陀自奉节俭,耶输陀罗从这番话中体味出来,一时不知再说些什么。

她觉得眼前的丈夫竟离自己那么遥远。

他已不是从前在宫中享乐的丈夫了,虽然给人一种说不出来的祥和温暖的感觉。但也不再有着亲密的情态。

宫女此时手捧盛满饮料和各色水果的盘碟进来。

耶输陀罗劝食道:

「请来用一点吧!」

「好的,谢谢!」

佛陀端坐下来,捧起茶碗。

茶叶的清香,阵阵扑鼻。这难以言喻的香味,是他久违了十二年的。

这是一碗加入蜂蜜的甜茶。

茶的原产地在中国,后来传至缅甸、泰国、印度等地。

由于东西文化的交流,今天西洋人也盛行喝茶,但多半喝改良过的红茶。

在当时的印度,尚未有特别精制的茶,但茶叶的芬芳,能使人消除疲劳,而成为极受欢迎的饮料。

但是只有上层阶级的人才可以饮茶。

佛陀边喝着茶,边品尝茶叶的芳香气。

因为佛陀不喝酒,故宫女早经关照而未曾送酒上来。

「你很疲倦了吧!」

「不,我因为已经习惯了山野中的生活,一点也不觉得累。」

「但是你比以前瘦了!」

「这些年来,我有时胖,有时瘦!」

「你的修行生活一定很辛苦吧!如果是我,不知如何过那种生活,我觉得能再见到你,已很感心满意足。我以为这一辈子再也见不到你了。我把全部精神都放在罗睺罗身上,无论如何,我要把他抚养长大,也好向你有个交代。那个孩子现在每天都和大人一起练弓、斗剑,还学摔角,在锻炼身子啊!」

「摔角?以他的身材,现在还没办法跟大人比吧?」

「但是他偏喜欢找大人来比划。不服输的脾气就跟你一样哩!」佛陀听了,很觉有趣。

「听说你的教团中也有女弟子……」

「……」

「假如我想出家,你愿意收我为弟子吗?」

「如果能在你身边照顾你,我该有多高兴啊!」

佛陀一语不发地望着窗外蔚蓝的晴空。

耶输陀罗见状,深觉自己说错了话,连忙闭口不语,并睁大眼睛望着佛陀。

四下一片沉静。

当此之时,宫女进报宴席已开。

佛陀即起身向外走去。

佛陀的众弟子先前蒙受过净饭王的布施,此时,已分批返回林间。

席间,佛陀的父王、王叔、母后波闍波提夫人、儿子罗睺罗、弟弟难陀等,都围坐一处,竭诚欢迎佛陀。

耶输陀罗随佛陀之后入席。

「悉达多,迦毗罗卫的生活很不错吧!在外面受过苦才会知道家的好处。你虽然已成为一个悟道的人,但也不能像乞丐那样空无所有吧!我希望你能好好再想一想,王宫里的生活是很舒适的。我实在看不惯你现在过的那种生活。」

净饭王仍不放弃说服佛陀回宫长住的念头,一面说着,一面自觉有理地点着头。

「谢谢父王的关爱。我的家族虽然显赫,但是我为了磨炼自己,必须过俭朴的出家生活。豪华舒适的生活或是山中苦修的生活,都无法使我领悟正法,我需要过的是一种少欲知足的合于中道的生活。」

「对了,你说释迦族很显赫,我们释迦族的祖先的确了不起,也很伟大。我说给你听。很早的时候,在阿佑提耶都城,有个伟大的王叫优迦伽,据说是太阳的化身。他广施仁政,获得人民的爱戴,使国家成为经济繁荣的一等大国。

国王有两个王妃。大王妃有两个儿子,二王妃有一个儿子。

大王妃的儿子在各方面都要比二王妃的儿子出色优秀。然而有一天,大王妃的儿子们对母亲说:

「我们希望由二娘娘的王子来继承王位,我们就在外面筑城来保护父王的城堡,这样做,将来可以免去因王位的争夺所造成的内战。为了使我们的亲族能和睦相处,巩固彼此的力量,并永远存续下去,我们应该到外地去建城。请母亲代为向父王请求。……」

国王最后答应了这项请求。

于是大王妃的王子们就在城外建造了一座城,取名迦毗罗卫,并迎父王前去。当父王一看到那座美丽坚固的城,就高兴的脱口而出:

「哦!释迦!」(译注:释迦即「能仁」之意)

两兄弟自那时起就自称释迦族,一直到今天。

「迦毗罗卫」有着如太阳般光辉宏壮之意,也叫做光之城。

我们的祖先是如此卓越,所以现在才有你们。

子孙应当敬仰祖先的成就,你能体会到这一点,实在是一个可喜的现象。

净饭王兴致勃勃地叙述着释迦的历史。

佛陀则在一旁默默地听着。

他知道父王所强调的祖先和自己所体悟到的祖先是有出入的。他说:

「父王,您知道吗?世上除了一代传一代的祖先之外,还有所谓的灵魂的祖先。

灵魂的祖先,即是永远存在于自己生命中的那一个。

我们在过去世中有过种种宝贵的体验。一一都存在于我们心中。父王,您也有的。那宝贵的东西,即称为智慧。我们一旦打开这智慧之门,就能明了世上的万事万相,我们的心就不会再迷惑、不安或烦恼了。

许多人之所以感受到现实的苦恼,即由于自己的愚昧,闭塞了内中的智慧之泉。

所谓愚昧的行程,如怨恨、嫉妒、毁谤、愤懑等是。尤其甚者,是那不知餍足的欲望,最能迷惑人心,能将智慧之门堵塞起来。

智慧之门一经开启,人本身就有着永远不死的生命,能够理解转生轮回的实相。

也就是说,能了解生老病死的苦因何在。

我注重的是如何去除苦恼之因,提供肉体的祖先是我们人生航路中提供交通工具者,与灵魂的祖先是大不相同的。

肉体的祖先是转生过程中有缘生关系的个体,我们对于他,则在于尽心力维护一健康的身体,并发展出美丽丰实的内在,以做为一种报答。

一般人只相信肉眼所及的狭窄的世界。如果我们能多运用心灵的眼,就能一目了然这世界是怎么一回事。

在心眼的观察下,人类所追遂的名誉、地位、财产等,无一不是虚幻无常的。

我,身为一国之君的儿子,正如大家眼见到的,正在过一种永恒生命中的修行生活。

若拿我和父王做比较,我必是个乞丐无疑。从外表来看,的确是那样,但是谁也不能否认我是父王您的儿子。

所以说,从外表是无法获得事实真相的。

受苦恼煎熬的人们,就是因为不知如何去除苦因,终日被苦恼所捉弄,以致迷惘不安。

有智慧的人,则为了去除苦因,而过着以法为依据的生活。

正法就是慈悲的光,像太阳一样,像空气一样,不分贫富贵贱,平均地赐予地上的生物。

我们若能除掉傲慢或谄媚的心,依着正法来生活,当我们进一步了解世事的真相时,我们的心自然就能安逸,而能沉浸在真正的幸福之中。

我就是领悟到,一个人要想谋求幸福,唯有实行正道。」

佛陀的叔伯们都聚精会神地听着这一番平易近人的道理,并深深感动着。

净饭王也只能保持缄默,无以反驳。

其实净饭王本身也学习过婆罗门的经典,对人神的问题一向都很重视,只是他一直对人与自然间的奥妙关系缺乏认知。

如今听得爱子的这一番说词,心中久存的结似乎渐被解开了。

大家都称悉达多为佛陀。

悉达多果真是佛陀啊!

净饭王若有所悟地想着。

他接着敲击着膝盖骨说:

「悉达多,从今天起让我也称你为佛陀吧!你确实是与众不同,你的这番见解,就是吠陀与奥义书中也没有办法找到的,我现在全都明白了。

想不到我也要成为自己儿子的弟子了,我们有亲子之谊,你要多加指教啊!

我真的生了一个很有出息的儿子,只有他能不费吹灰之力就使大国的国王也做他的弟子。

我们释迦族的声望更高了!

这真是可喜可贺!

佛陀今天的成就,真应了阿私陀仙人的预言了啊!」

净饭王很高兴地下了这样一个结论。

桌上佳肴成堆,但谁也没有心清光顾,只一心一意倾听佛陀的法音,同时反省自己的心思。

接着净饭王对佛陀说道:

「今天实在是值得庆贺的日子。来,请吃点东西吧,你不动手,别人也不敢动哩!」

佛陀称谢过后,即取了面前的水果放入口中。

其他的人这才跟着进食,放宽了心情,尽清享受面前的佳肴美食。

阿私陀仙人,是一位非常精进的婆罗门行者,早已谢世。当初佛陀初生之时,他就曾至王宫预言了佛陀成道的事实。阿私陀仙人当时预言过后,非常感伤,因为他希望自己能活着看到佛陀成道,并希望向佛陀学习正法。

净饭王忆起这位仙人,于是面露欣慰与满足的笑容望着眼前的佛陀。

宴会过后,净饭王传令希望全释迦族的子民踊跃学习佛法。

\section{解脱之道}\label{sec8.4}

第二天,佛陀在释迦族的王侯贵族及武士的集会上,做回乡后第一次的说法。

「各位乡亲父老,我看有许多人都盲目地过着日子,在王宫里,许多人的生活虽然舒适优裕,但心理上很不自由。然而在王宫之外呢,又有许多人为了活过这一天而操劳奔波,虽然大家都在同一个太阳底下生活,却由于种姓制度的限制,而造成种种不平等的现象。一个人是不能凭出身来决定价值的,而应该由他是否有慈悲的心行以及是否得到上苍赐予的喜悦来看他这个人的。

人的欲望是没有止境的。无穷的欲望陷人们于苦恼中,要想获得解脱,就必须设法从欲望的摆弄中脱身。

如何从欲望中脱身呢?应从怜悯他人不幸的遭遇开始,不要只是贪求个人一己的利益。

人与人相处时,由于心理上就有敌我之分,以致引起纷争,这都是因为个人有着自我保存的意念,同时被此意念左右了的缘故。如果不加警觉,就有在睡梦中被人杀害的可能。一个人只要一想到夜里可能会有人来加害于自己时,就没有办法安眠了。

我自出家以来,已不再为这问题所困扰。我没有了家国给我的束缚,我活得自由自在,好像天空中任意翱翔的飞鸟,这大自然到处是我的住所。

以武力支配他人,早晚也会被武力所支配。

这情况是因为偏颇的思想以及注重物质的态度,造成了一种因果循环的现象。

遭受侵犯的人,心中必时时燃着复仇的火把。受侮辱的不甘,形成他的意念,造成他的行为结果。

我们虽然时常想着以力量来支配他人的行动,但终究很难支配到人心。

一切的物皆属无常,武力只会消灭自已。

然而如果我们实时领悟心灵的价值,以及天地间不变的真理,虽是一个统领大国的野心之主,也能知道争执与破坏的虚妄面。

各位乡亲父老们,

人不是为了相互斗争、伤害而生存于世的。

人是为了拓展自己的心灵,以调和整个人类社会的这一神圣目的才降生的。一个人为了保护自己的权益,终其一生站在与他人为敌的立场,诚可谓愚不可及。

旧式的信仰,是无法解救自己的。

解救自己的,只有自己。

除了实行正道,过合乎正法的生活外,没有捷径可循。

在我们每个人的心中,都有「我」这样东西。我们首先要有这个自觉。

这个「我」,清清楚楚知道宇宙的创造之理,「我」以外,没有必要去依赖他人的自我。

正确的自己,就是可依赖的一切。

这个「我」,一无虚伪、憎恨,有如赤子般常使自己安泰,有着天真澜漫,毫不拘束的朴质的心。

然而这个「我」,常被一昧自我保存的「假我」所蒙蔽,经常迷惑于自己生存的环境以及所受的教育等,替自己制造了许多的苦恼。

于是这世界就如苦海一般。

这种苦恼的境界,实在是由自己制造出来的。」

佛陀如狮子吼般发出震人心魂的警语,在座的每一个人都深深被感动了。

佛陀静默了一会儿,然后环视大众,接着又说道:

「在此生死苦海中沉沦,即使死后,仍脱离不了苦海的羁绊。死后的苦境界,称之为地狱。能对轮回的现象加以了解的,就是开悟的先决条件。

所谓开悟,就是本身已脱离轮回的束缚,超越了时间与空间,觉悟到本性的伟大,看到了永不生灭的自我。

不论是地狱,还是极乐世界,都不是神所创造的,它们是人们的思想与行为所感应出来的境界。

不要害怕,也不要轻视自己,抱着希望,谦抑努力地生活在现前的环境中,上苍自会毫不吝惜地显示慈悲。

各位乡亲父老,

觉醒吧!打起精神来!

把我们的人生,变成有意义而值得回味的那种。」

佛陀的言语中充满慈悲、威严,打动了每一个人的心。

有的人因为深有所感而流下泪来,有许多妇女由于抑制不住激动的情绪而哭出声来。

净饭王是第一次领教到佛陀说法的威力,不禁低头合掌,向佛陀致敬。

他因为佛陀的这一开示,知道如何开启内在的心眼了。他开始对婆罗门的教义产生疑问。

过去从未听说过婆罗门教有解脱法门,虽然婆罗门教揭示轮回的思想,但无法使人做进一步的认知。

也就是说,人们囿于狭窄的轮回观念,以为人的命运是既定的事实,谁也无法改变。一般人只能寄望于来世,今生唯有勤加祭祀婆罗门神,才能求得神的呵护而获得快乐的来世。司祭者,婆罗门行者,就是引导人们至快乐来世的导师。

所以僧侣的地位高于一切。

这样的思想形态,很容易造成消极的人生态度。婆罗门的轮回思想没有救赎的功能。

佛法则不然,着重于当下的解脱之道,使人们当下能从转生轮回中解脱出来。

这在当时,是个惊天动地的说法。佛法中强调人们把眼光朝向佛陀之法,因为真正的幸福是得自于完全从苦况中脱离的事实。

现象界属物质世界,在其间无论你如何追求,也无法得到真正的幸福。我们只要环顾周遭的一切,就能很容易地了解这一点。

佛陀强调的解脱之道,至少有两方面。

其一是如前所述,从转生轮回中完全脱离。其二是学会不被现象世界的一切所束缚,以过一种自由自在的生活。

一个人一旦开悟而成为佛陀,则转生与否就全凭自己的意志了。

同时,在现实界中,也就能不受外物的束缚,心灵上逍遥自在,无所不乐了。

所谓生命的循环,就生命出现在现象世界,无法摆脱出循环之法以外的一种情形。

但是如果个人的灵魂,已完美净化至佛境界时,就能成为法的施行者,就能支配宇宙间的自然现象,诸如转生一事,也就可以听凭己意了。

一般人在未悟道前,总是受着业缘的循环现象所支配,苦乐持续着。

净饭王听法之余,最为感动的,就是知道人可以从转生轮回中解脱。

做为一国之君,他也是苦恼无穷。战争的恐惧时刻笼罩着他。

一个人只要活着,就无法避免死的一天。死亡的恐惧也是亦步亦趋地跟着他。然而接触佛法之后,知道正法即是一种从死亡中解脱的道,使人获得永恒的生命。一个人一旦知道这个事实,还有比这更高兴的吗?

等佛陀说法结束后,他就迫不及待地登上讲坛,对台下的听众说:

「我释迦族的大众啊,佛陀的法是真实不虚的。除了正法之外,没有其他的解脱之道。

今后除继承家族的长男之外,任何人都可自由加入佛陀的僧团。

我过去见许多武士们都跟佛陀剃度出家了,心里很为释迦族的前途悲伤,但是现在不同了,我深深明白佛陀的法可以指引我们走向真正解脱的大道。

我们要跟随佛陀,学习佛陀昭示的正法,你们说是不是这样呢?」

话声刚落,群众间即响起如雷的掌声,对净饭王的一番见解,大家都深表同感。

众人的情绪已因净饭王的诠释而高涨不已•虽然佛陀已离座他去•众人仍留在原位•舍不得散席。

佛陀说完法,由净饭王及罗睺罗护送回城外的林中。

佛陀深深为自己了了这一椿心愿而舒一口气。

他望向穹苍,眼前浮现出无数张充满法喜的脸孔,他也很感安慰,他知道今天在座的人,都将正式皈依佛法,成为宣扬佛法的先锋,教化城中的百姓。

由于净饭王的布施,尼拘陀树林中也多了一所精舍。

\section{王子们出家}\label{sec8.5}

跋提是佛陀最初的五个弟子之一,如今已达阿罗汉的境界。

他和佛陀一样,在去国十二年后,回到父母身边。

他对须克鲁•达那亲王说:

「父亲大人,佛法主张众生平等,具有无上的智慧与神通力。五百年前,您就是我的父亲。

您当时也是一族之长,为我们子孙祈福,父亲您还曾经是大婆罗门的领导人。

当时您所用的语言,是像这样……」

跋提说到这里,就流利顺畅地说出当时的语言。

须克鲁•达那十分惊异地倾听着,然后问道:

「这真是古代婆罗门使用的语言吗?真不可思议,我也使用那种语言吗?……」

「是的。」

古代的用语,他曾在有些婆罗门修行者那儿听过,如今却亲耳听见自己的儿子自然顺口地说出,他很感惊讶。

「真是太神奇了,你成为佛陀的弟子,真令人羡慕。我身为贵族武将已七十多年,心地很不清净,今世已没有办法再说那种语言了。

今天佛陀说的法,给我很大的感触,觉得非常奇妙,但要做到确立自我的话,也是相当困难的。」

父子交谈的当儿,席间还有都鲁•达那亲王之子阿难,以及阿姆利德•达那亲王之子那摩二人。

这两人听了跋提的话,也不住地点着头。

摩诃•那摩对跋提说:

「跋提,你的确是跟从前不一样了。我非常怀念我们小时候,你现在已从一切的执着中解脱了,想当初你目光炯炯地瞪视着猎物时的那种神情,不知到那里去了,你现在是那么稳重安祥,好令人羡慕啊!

我也想出家,但我是家族中的长男,负有继承之责,但我希望能做个在家居士,在日常生活中实践佛法,请你引导我吧!」

跋提回答道:

「那摩,要想在家修行以达涅槃境界,的确是比较困难的,不过,我们的第一步,不论在家或出家,都要做到确立自我,先消除内在的烦恼之火。也就是说,我们应试图过着思想与行为都合于法的生活。

我的职责即在于一方面寻求自我的确立之道,一面为迷惘的众人点燃法灯,绝不能稍有懈怠疏忽之心。」

阿难此时双目炯炯,口气坚定地对跋提说:

「我已下定决心于近日内出家。我今天就回去跟父亲商量,好尽快实现这个愿望。我不想再过这种盲目的生活了,到那时候还要你多指点指点我。」

跋提望着阿难说:

「修行的生活是很苦的,那和王宫里的生活完全两样。我们必须定时出去乞食,在王宫里面,会以为野外的生活别有一番趣味,但是真正到了荒郊野地,情形就不是那么单纯了。坦白说,我刚开始时,常会怀念王宫里的生活,不知是不是应该回家。

修行人的生活,实际上就是一种苦行的生活,没有坚定的决心是无法忍受的。」

跋提要让阿难进一步了解修行生活的实质情况,不希望阿难只是一时兴起,对修行怀着错误的憧憬。

「我很早就想出家做修行人了,今天听了佛陀的一番道理之后,就更坚定了我的决心。

我相信什么样的痛苦我都能忍受的。

至少我绝不会使释迦族丢脸的,替老前辈们洗脚,或打扫厕所等工作,我都愿意做的。

如果佛陀不肯收我做弟子,我就每天都到他那里,一直到他答应为止,就是死在路上也在所不惜。」

「嗯,如果有这样的决心,那当然就不成问题了。你到底是释迦族出身的啊!」

「跋提,请你务必为我请求佛陀。」

为了表示诚意,阿难低下头来,声声恳求着。

由于阿难一直就有出家的心愿,如今既经下定决心,就去和父亲商量,父亲也没有多加阻难,而且做父亲的因为听了佛陀说法之后,也深有所感,以致很快就答应了阿难的请求。

第二天阿难随即去访另一王子孙陀罗。

「孙陀罗,我决定出家修行了,父亲很高兴,我就要跟你分别了,宫里有关的事要托你照管了。」

「阿难,你已经决定了吗?如果你要托我,我也有苦衷,因为我也跟父亲商量过,如果你出家,我也要出家的。而且阿那律也说过要出家,在他还没改变心意之前,我们去约他,你看如何?」

两个年轻人当下就来到阿那律王子的住处。

「请通报阿那律王子我们来访。」

侍卫接令进入宫中。

不久,那摩出现在大厅。

「噢,阿难,昨晚失礼了。阿那律口口声声说要出家呢,不过还没做决定,你们有事找他吗?」

「我们就是为出家的事来找他的,他在吗?」

「他正在和父亲谈论佛法,上来说吧!」

那摩在前引导二人。

「父亲大人,阿难和孙陀罗来找阿那律。」

阿难与孙陀罗亦趋前致意道:

「叔父大人,好久不见了,您近来好吗?我们有事找阿那律商量。」

「哦,欢迎你们,家里长上都好吧?」

阿难笑着回答道:

「父亲很好,他自听佛陀说法后,还对我说,如果他再年轻一点,他也要出家呢!」

「佛陀之法的确神妙,能增加对人生信念。释迦族能出这样一位伟人,真是荣幸之至!我也正在和阿那律谈说佛法呢。

你们如果有事商量,我就不便在这里了。」

阿姆利德亲王说着从椅中站起。

「哦,叔父,您请不要离去,我们正好一同商量。」

阿难恳挚地说,然后转向阿那律道:

「阿那律,你真有出家的意思吗?」

「嗯,我是这么想。……昨天也跟孙陀罗谈过。」

阿难说:

「父亲已答应让我出家,孙陀罗也决定出家了。」

「啊,你们都……」

「是啊,我一直就在找寻这样一个机会。」

孙陀罗很坚定地说。

阿那律睁大了眼睛。

「那么就只有我还在犹豫了,要我马上放弃目前的生活方式,真有点舍不得。」

阿难等诧异地问道:

「为什么?」

「因为我喜欢武术,马术,我还想着将来做带兵的武将来保卫迦毗罗卫。但是听了佛法之后,又感到人生的渺茫,好像应该出家求道才是,所以我到现在还是拿不定主意。」

阿姆利德亲王在一旁劝解道:

「阿那律,这种不定的心情是不可以有的。武术的极致,根本上和佛陀之法是一样的,你以不定的心情,又怎能把武术练好呢?要想保卫迦毗罗卫,先要把心魔制伏才行。我看你不要出家了,让你的哥哥出家算了。」

那摩即接口道:

「父亲说得对,干脆我去出家,由你留下来辅助家业也是一样。」

「呀,请你们等一等,让我再想想。」

阿那律见哥哥如此认真,不由着急地说。

过了半响,阿那律似打定好主意,对父亲说:

「我决心抛弃一切,成为佛陀的弟子,来长养我的慧命。我发誓不再让您操心了,真是抱歉。」

阿那律就这样做了最后的决定。

由于阿难、孙陀罗、那摩,甚至包括自己的父亲在内,都由衷钦佩佛陀,倾心于佛法之下,他们都将出家修行视为无上神圣难得的事,在这一气氛之下,阿那律深深觉得留在家中是一种耻辱。

阿那律既做了决定,阿难像完成一件大事般说道:

「好了,从现在起,我们算是出家人了。我们应该赶快到尼拘陀树林去见佛陀。阿那律,你不妨再考虑考虑,不要勉强,我们先去见佛陀,你等考虑好后再来吧!」

阿难与孙陀罗立时起身向阿姆利德亲王告辞,正当离去时,阿那律赶紧说:

「当然我也一起去。我已经做决定了,怎么可以留下我一人就走了?」

阿难笑着说道:

「那么你就跟我们一块儿走吧!」

三人就结伴离开宫殹。

三人商议好先去把头发剃除,于最来到理发师优婆离这儿。

「优婆离,替我们把头发剃掉。」

「什么?为什么?」

优婆离瞪大了一双眼睛,不明究里地望着三位王子。

「我们三人都要出家了,所以要把头发剃掉。」

「你们真会开玩笑啊!」

「是真的,我们今天就要到尼拘陀树林去出家。」

优婆离真的是呆住了。

「你怎么还不快动手吗?」

阿那律坐在优婆离面前的椅子上催促着。

「你们是说真的吗?真的把头发剃掉没有关系吗?亲王知道这件事吗?万一他们怪罪下来,我有几个脑袋也不够的呀!我不能做这种事,恕我不能为你们剃头发。」

「优婆离,你放心好了。我们都已得到同意,什么都不用操心,如果你不肯,小心我们拿下你的脑袋。」

优婆离面有难色。

「我们不是故意来为难你的,也不会连累你的,要知道带着发髻不好出家,就请你赶快动手剪吧!」

「我剪了,真的会没事吗?」

他见三人的态度如此认真,不像是骗他的,因此颤抖着双手,开始为阿那律剃发。

阿那律摸着光秃秃的脑袋,情绪非常复杂。他是三人中最后下决心的人,却最先剃掉头发,一时有着难以适应的感觉。

当三人都已把头发剃掉后,阿难对优婆离说:

「我们想在你这儿换上僧衣,我们脱下来的衣服就送给你了,那都是喀西产上等料子,喜欢的话,你就穿上身。」

阿难说着,带头脱掉衣服,同时把脱下的衣服交给优婆离。

优婆离哭丧着脸,有着依依难舍的神情,见阿难把质地上好的衣服送给他,很不敢当地说:

「王子殿下,我这种身份不配穿这种衣服。」

阿难回答道:

「总之,这些衣服全是你的了,随你如何支配,若是拿去卖掉,也可以,一切麻烦你了。我们要上路了,祝你身体健康,生意兴隆。再见了。」

「王子殿下,也祝你们一路平安。」

三人离开理发店,在优婆离的目送下向尼拘陀树林进发。



\chapter{第九章\ 活在真理中}\label{ch9}

\section{一周来的反省}\label{sec9.1}

三人边走边谈论世俗中种种的生活,并且互相勉励,自今而后,一切的世俗活动都将停止,好好向解脱之道迈进。

正当谈得起劲时,他们已来到街市的尽头。有个年轻的姑娘等在街边,想招呼他们,三人目不斜视地从她身边经过,然而有一个声音自身后响起:

「孙陀罗殿下,请等一下!」

正是那个姑娘的声音。

孙陀罗不得不停下脚步,并对同伴说:

「对不起一下,我马上回来。……」

说完,立刻奔至女郞身边。

阿难与阿那律相互交换了一个会心眼色,露出微笑。

孙陀罗与姑娘来到树荫底下,孙陀罗很认真地对姑娘说:

「真抱歉,你还是对我死心了吧!我已决定出家,你去找一个更好的对象吧!」

「如果你出家,我也跟着出家。」

「请你不要为难我。」

「你就这样狠心丢下我吗?」

姑娘抬高声音说道,孙陀罗怕路人听见,神情十分尴尬,一时不知如何是好。

姑娘来自富豪之家,身边还有好多人侍候呢。她跟孙陀罗大约在半年前相识,姑娘长得娇小玲珑,有着圆圆的脸蛋,虽常透着忧郁,但很逗人喜欢。孙陀罗就是被她的可爱所吸引。

照当时的习俗,两人的阶层不同,是不能论及婚嫁的,但孙陀罗却肯定地告诉女孩子,他必说服双亲来娶她过门,女孩子就以迫切的心情等待着佳期的来临。

正当这时候,佛陀归城说法,孙陀罗在听完佛法后,就像变了一个人似地把女孩子抛诸脑后,一心一意想着出家修行。

自女孩子知道孙陀罗出家的决定后,就守候在这里,希望能改变孙陀罗的心意。

女孩子情绪激动,知道孙陀罗出家的心意已定,悲痛得哭出声来。

「请你原谅我,事先没有跟你商量就做这决定,都是我不好。你要知道,自我听了佛法后,一点也没有结婚成家的念头了,我这样说,实在很抱歉,我真的是已经下定决心出家了。」

「你难道就不为我想一想吗?」

「我不是不为你想,我也考虑过要告诉你,但是我知道,如果我告诉你,你也不会答应的,所以我就……」

「你未免太残忍了,你跟我是有约在先的,今天就想这样一走了之,你们男人都是这样狠心吗?」

孙陀罗只得低垂着头,连声说着抱歉。

在当时,女人只是男子的附属品,一夫多妻是正常现象。故女孩子虽然如此哀恳着,也拿他没有办法。孙陀罗这样一声不响地抛下女友出家,当时的社会不会责难他,而且两人身份也不同,一般人只会认为女孩子在无理取闹罢了。

孙陀罗也许深受当时风气的影响,所以会这样不告而别而不以为忤。现在看到女孩子如此伤心,他也很感抱歉,觉得自己伤害到对方了。

就这样,两人都陷入沉默中,时间一分一秒地溜过去,孙陀罗想到还等在路边阿难及阿那律,心中非常焦急。

他知道这样拖下去,不能把事情解决,正焦虑着。姑娘似乎平静了一点,用着清晰的语调对孙陀罗说:

「我知道,我也没有立场先阻止你,我只有让你走,因为我只能这样……」

说到这里,又抽抽搐搐地哭起来。

孙陀罗轻抚姑娘的双肩,柔声地说道:

「你务必请原谅我,你成全的好意,我将铭记在心。」

女孩子这时力持镇定地对孙陀罗说:

「再见了,我不知什么时候才会原谅你。」

说完,深深望了孙陀罗一眼,然后头也不回地快步离去。

孙陀罗望着女孩子渐去渐远的背影,直到影子在人群中消失,他在心底不断地祝福着她。

阿难和阿那律因久等孙陀罗不来,索性躺在草地上,边谈话,边眺望天空的白云。

孙陀罗向两人道歉后,即并肩继续上路。

三位王子终于抵达目的地。

阿难首先找到跋提。

「噢,阿难,你已决心出家了吗?」

「嗯,孙陀罗和阿那律也一起来了。」

「哦?他们现在在那里?」

「在那边林中等着。」

「那我们过去吧!」

跋提轻快地跟着阿难进入林中。

「阿那律,孙陀罗,你们很适合理光头嘛!你们都得到家里的同意了吗?」

「是的,我们都是得到同意才出家的。」

「这样很好。不过剃了头发并不表示马上就可以成为佛弟子了,我们的僧团规定,在正式皈依三宝前,一定要做好彻底的反省,也就是说,要像第三者一样审视自己过去以来的一言一行,以找出出错的地方以及导致错误的原因,并努力改正它们。而且你们要诚心诚意地过这样的生活,否则会承受不了修行生活所带来的苦楚,你们会对自己说慌吗?阿难,你说说看。」

「我从不对自己说假话的。」

「那么对他人呢?」

「有时候会。」

「为什么?」

阿难想了一下说:

「自己是很可爱的,怎么忍心骗自己呢?」

「那么,孙陀罗,你一个人在房中时,是如何对待自己呢?」

「我对自己一向是很坦白的,绝不会对自己说谎。」

跋提点头说道:

「不错,当自己独处时,多半是很坦率而正直的,那颗坦白的心是很重要的,可上通于神。佛陀教义的重点,即在于要我们把那颗心活用于日常生活中,以使自己及他人得到真正的平安。我们只要看看小孩子,就能明白这一点。小孩子是很直爽而没有邪念的,所以心地再诡谲的人,一旦在天真烂漫的小孩面前,也会被感染得心无杂念而纯真自然了。」

跋提一面说,一面注视三人的脸,三个人全神贯注地听跋提说着。

「阿那律,你家中有许多种田的奴工吧,你知道他们如果不注意季节的变换,农作物会如期收获吗?」

「我平常并不太注意这类事情,不过我知道农工们都是配合季节播种收成的。」

「这就是了。这叫做自然的法则。佛陀所教给我们的法则,也就是一种要我们顺应自然而生存的法。

自然的状态是不变的,你们当获取自然的启示,坦诚地审视自己的心,并努力修正之,这是很重要的步骤。你们有七天的时间可以从事反省,把出生以来的所有言行加以反省。你们若能真正审察出过去的错误,头部四周就自然会泛现金光,因为心中的阴暗被驱除,以致充满了佛光的缘故。等那种光出现后,我就带你们去见佛陀。」

三个人不由得摸着自己的头异口同声地说:

「这不是很困难吗?如果随便想想是不会起作用的吧?」

「所以你们应该认真地反省自身,不要存着敷衍的想法。」

跋提说着,定睛望着三人。

阿那律忍不住嘀咕道:

「这可怎么办,如今头发也剃了,若反省不成功,也不能马上再恢复刹帝利的身份了。」

说着,以眼瞄视孙陀罗,希望有人与他的看法一致。

孙陀罗则故意避开阿那律的视线,目光落在前面的草丛间。

阿那律又忍不住问道:

「孙陀罗,你有这个自信?我真怕万一佛陀不收我们,我们如何再回城?」

孙陀罗并不想回答这个问题,故仍保持沉默。

阿难和孙陀罗都了解阿那律的心意有些动摇,认为他的意志不够坚定。

阿难这时说道:

「我们的头,因为是刚刚剃好的,所以有些光亮,不过再过一个礼拜,就会变暗了。我想到自己过去,有好几次都仗着自己家世的方便,伤过好几个女人的心,常为女人的问题苦恼,也每次都让母亲为我担忧,故我的心情一直很郁闷。」

这情形与孙陀罗的一样,孙陀罗听后,心情上稍稍舒解了一些。

跋提开口道:

「阿难,你正可以好好地把事情的本末反省一下,究竟是什么原因使你做出那样的事来,反省过后,就不要再犯同样的错误!

孙陀罗也是,如果你曾使某个女人哭泣,也要追究其中的原因,衷心向那受到伤害的女子道歉。你若替对方的立场多想一想,同情与歉疚的心就会油然而生。希望大家都好好地洗涤自己的心灵吧!」

孙陀罗被跋提的这番话吓了一跳,听跋提讲话的神情与口气,就像是他刚刚已看到他与那女孩分手的那一幕情景了。

其实一个人只要常行正道,自然而然就会有这类灵感来揣测事情的。

跋提并不清楚孙陀罗的爱清生活,但当他说话时,自然就会切中对方的心理,好像有某种透视能力似的。

这是由于跋提的守护灵透过了跋提的意识,使他不知不觉间说出一些切中时机的道理,守护灵则是透过意识界而了解到孙陀罗的意识状态,故而能把握到孙陀罗的心情与现况。

一个人的心灵不到澄澈清明的地步,是无法有这种神通的,有时即使守护灵一再指示自己这样,那样,但当此人心灵昏昧不明时,他无法接受到讯号的。

孙陀罗见跋提好像什么都知道,觉得有些畏惧,他再三警惕自己,今后若想过精舍的生活,做佛陀的弟子,千万不可存苟且,必须怀着如童稚般坦率真诚的心才是。

跋提见他们已不再提出意见,就说道:

「那么请你们在七天中好好反省自身吧!」

说完就回精舍去了。

三人就在原处埋头努力地反省本身。

他们不分昼夜地探索,冥想。到第三天,带来的粮食所剩无几,他们只得到村镇上去乞食了。

乞来的粮食十分粗糙,自然跟宫中的不能比,所以三个人到了第四天,都两眼无神,体力不继了。

阿那律又忍不住埋怨道:

「出家真是痛苦!就是肚子饿了,也不能随便吃东西。每天都吃一样的东西,真不知道能不能再捱下去,精舍里的生活也是这样的吗?」

阿难说:

「有什么办法呢?我也快受不了,没想到修行是这么痛苦的事。」

「阿难,你也有这感觉吗?」

「是啊!孙陀罗,你觉得怎么样?」阿难回过头去询问孙陀罗。

「也是一样。不过这样每天反省的结果,我发现自己的确有许多缺点,真感到羞愧。我认为今后能不能成为佛陀的弟子,才是一个大问题呢!」

「哦,你倒是反省得很彻底,可以不用担心了!」

「看!孙陀罗的脑后好像有金光出现了!」阿那律此时回头笑着说。

孙陀罗直觉地回头看着,然后说:

「什么也没有啊?难道你会透视?」

「孙陀罗,回头是看不到的,因为后光是在你眼睛后面啊!」

阿那律说完,捧腹大笑。

这时孙陀罗听出阿那律是跟他开玩笑,因此有点不高兴。

阿难对阿那律说:

「你不要开玩笑了,孙陀罗是很认真在反省自己的。我们应该互相鼓励,真诚地讨论,一定要在这限期内成为佛陀的弟子。」

「我不以为然,像这样的反省,实在太严格了。反省自己固然重要,但太过于虐待自己的话,反而违背了佛陀的教义,跋提不是也说过吗?只要知道有错,就像对方道歉就好了。」

「虽不错,但只凭道歉就够了吗?人只要对自己的错误道声歉,就能被原谅吗?如果真这样,我倒有些不明白了!」

阿难发出了疑问。

孙陀罗开口道:

「反省的目的,主要在于重新安排往后的生活吧!我所担忧的不是反省本身,而是反省后的生活,也就是说,一般人在反省过后,会对自己失去信心哩!

一个人在反省过后,总会发掘出自己许多缺点,所以反省过后,是不是能再面对自己,去过有自信的生活?

我想,反省的积极意义是不再犯同样的错,所以一周的反省,即在发掘错误的原因,以致对未来的生活反而失去了信心。」

阿难和阿那律觉得孙陀罗的话很有道理。

如果一个人在反省过后,还是马马虎虎地生活着,那即使他是如何认真地反省,也是毫无价值的。

因此两人人也感染到同样的忧惧了。三个人都默默地进一步沉思这一个问题。虽然天色已晚,应该做些入晚的准备了,但三个人因为都有心事,而把起火等工作扔在一边,继续沉思着。

阿难首先开口道:

「说到意志,我可能比孙陀罗更薄弱哩!过去我们碰到不好的事情,只会说声『不好』,从来不去分析其间的因果关系。

今天我们都想对自身的缺点和作为,做一番认真的探讨。我想单就这一种反省的态度,对以后的生活应该是很有助益了。未来的生活毕竟还是一个未知数,没有必要去苦思苦想,你们觉得怎样?错误的原因固然形形色色,但仔细想来,日常生活中有许多琐碎的小错,多半由于自己未能把持镇定,常被当时的环境和气氛所操纵了。所以如果我们能在日常生活中,处处保持冷静,遇事必不致心忡忡的了!」

「阿难说得对极了!」

孙陀罗以明快的口气回答道。诚如阿难所说,自己以前从不知反省自己,这次为了要入佛门,才勉强自己来反省,一且觉察出自己的许多弱点与缺点后,又有着难以忍受的屈辱感。

人生究竟是怎样的,我们不身体力行某些道理,就无法得知。在迈开脚步前就左思右想,踌躇不前,终究无法开出道路来。由于阿难的提醒,孙陀罗有了重新探索的依据。

阿那律也有同感,他是把哥哥留在宫中担当家业而出家的,如今万万没有再厚着脸皮回宫的道理,他笑着说:

「我们要继续加油!」

说着,轻快地起身到林中去捡柴。

他们过的是真正的野宿生活,既没有遮蔽雨露的帐篷,也没有野营的炊事用具。所以当他们到农家乞食时,布施的人多半了解这情形而供给他们熟食。

没有帐篷,如何避雨呢?好在印度的气候非常明确。除雨季外,是不下雨的。因而不太需要帐篷。白天虽然酷热难熬,入晚后就转为凉爽宜人,很适宜修禅定。一般的农家在日出前即开始农忙,到上午十一点钟即收工回家,在傍晚四点以前的一段时间为午休时间,然后再工作一两小时,这是一天工作的大概情形。

佛陀僧团中或一般婆罗门的修行僧们,多半选在晚间禅定,也就是这种风土气候造成的现象。

三人的反省禅定,也多半集中在夜晚。由于只有七天的期限,他们也把握白天的时间,这时就去寻找林中荫凉之处。

三个人虽都年轻,但这种简陋辛劳的生活,实在不是一向在宫中养尊处优的他们所能习惯的,所以疲态毕露,状至狼狈。

起初的几天,三人还能在讨论中持续着反省的工夫,到了最后两三天,几乎没有了说话声。

第七天来临时,三人的身体都不听使唤了。

\section{慈悲的佛陀}\label{sec9.2}

在第七天的下午三点钟,三人拖着疲惫的身子,怀着瑟瑟不安的心情向尼拘陀树林出发。

在下山途中,阿那律被石头绊了一下,阿难伸手去扶他,不料二人的脚又绊在一起而滚下斜坡。

阿那律发现自己的衣服破了,手脚也擦伤,并渗出血丝,两人都灰头土脸的。孙陀罗气急败坏地赶过来。

「呀,不要紧吧!有没有受伤……」

他急切地想安慰二人。

三个人你望望我,我望望你,都不觉苦笑起来。孙陀罗说:

「真是惊险哪!这兆头不好!」

三人继续顶着烈日,朝佛陀的所在进发。

脚下的尘土又干又烫,每踏一步,就有灰砂飞扬。如此默默地走着,他们只觉得出一步,就与佛陀的距离拉近了一步。阿难一想及马上可以见到佛陀了,就不禁喜上心头,虽然脑中不断思索着不知该对佛陀说些什么。

快抵达目的地时,阿难才注意到周遭的情形,望望满是灰土的手脚,对阿那律说:

「阿那律,我们这样子去见佛陀,非常失礼吧!找个地方洗洗手脚,你看如何?」

「当然好啦!孙陀罗,你知道这附近有没有河?」

孙陀罗笑道:

「我也不太清楚,再过去一点,好像有条小河哩!」

孙陀罗因为经常跟父亲出城游玩,所以对这一带地形比较熟。

果真在树林之后,有一条小小的河流,河水虽不清澈,但总可以洗脸洗脚了。不一会儿工夫,三人就像变了样似地都很干净了。他们摸着光溜溜的头,不觉笑出声来。

孙陀罗问道:

「阿难,见到佛陀时,还是由你代表问候吧!」

阿难不置可否。阿那律接着说:

「三个人一块儿问候不是很好吗?」

「不知怎的,心情十分紧张。阿难,你呢?」

「我心里也是噗嗵噗嗵地跳着,记得在城中见到佛陀时,就觉得他好有威严,但是又觉得他很可亲近哩!」

「你有没有跟他说话?」

「没有。是那次法会结束后。我听到他和蔼地跟大家交谈。而留下这样的印象。」

「你出家是因为佛陀的缘故吗?」

「那是原因之一……」

「我是因为看到一位真正舍弃了虚伪面貌的人而深深被他伟大的人格所感动,一心也想变成像他那样的人,所以想到出家修行。像我这么任性的人,居然在见到佛陀的那一刹那,像是被什么一下子点醒了似的。」

孙陀罗说着,凝望虚空,露出了坚定的神情。

「好了,我们该上路了!」

阿那律催促着,于是三人迈开大步,向目的地前进。

不一会儿,尼拘陀树林已经在望。

进入林中,他们首先遇到目犍连。

目犍连身形瘦削,但眼睛圆亮,目光锐利。三人一眼就认出他是谁,只听他说:

「佛陀正在等你们,你们安心地去见他吧!」

阿那律闻言,悄声对阿难说:

「我高兴的脚直发抖呢!」

三个人终于见到佛陀了,佛陀周围还有其他的人。孙陀罗迫不及待地说:

「让我走前面……」

说着就奔至佛陀跟前,跪拜下去。

阿难和阿那律跟在后头,也低垂着头,以示敬意。

「请抬起来!」

佛陀用充满慈祥的声调说道。

三人战战兢兢地抬起头,接触到一双柔和而慈悲的眼睛。

「你们已下定决心,现在很累了吧?今天可要好好休息。」

三个人不断叩着头。

「阿难……」

阿难听到佛陀在唤他,立刻抬起头来,仰望着佛陀。

「经过这一个礼拜来的反省,有何感想?」





「发现了从前所疏忽的种种缺点和人格上的污点。」

佛陀微微点头,和蔼地说道:

「在正道上最重要的一件事,就是此心能不为感情所动。没有感情的人,固然也算不得是个人,但是太容易感情冲动的人,是很可怜的。一个人动不动就生气、忌妒、骄傲等,都是因为他允许感情来摆布他,变成一个不自由的人。这样的人,心境很难安定,也没有办法正确地判断事理,既使自己苦恼,又使他人悲哀。

所谓心的调养,就是在设法使自己的心趋于平静,获到舒泰,避免处于失常的感情状态中。

如果想拨正波涛般起伏的感情,首先要了解它的导因,处于较超然的立场来分析之。有的人虽能了解感情波动的原因,但仍不免受其操纵,这可说就是他的「业」在作祟。当我们与他人为某事起冲突时,最好避免注视对方。因为此时对方的情绪变化与面部激动的表情,都会直接影响自己的情绪与判断能力。尤其当对方怒不可遏时,切不可继续与之辩争,应极力回避之。等对方的情绪恢复平静时再与之讨论事理,就能收到预期的效果,自己也就不致受到伤害。

人之会苦恼,多半起于对物的执着。丢开执着,就能以冷静的态度看清事物的真相。经常保持这样的心态,真理之光自会在心中显现。」

佛陀说到这里,用心地注视三个人。

佛陀虽然是对阿难说了上面这段话,其实这话中的涵义适用于所有血气方刚的年轻人。

「阿那律!」

佛陀叫到阿那律,阿那律不由得吃了一惊,抬起头凝视着佛陀。

「你今后要在心性方面多下工夫,培养出一颗专注的心,勿再胡思乱想,犹疑不定。精神涣散的人,皆不能成就大事。事业之成就在于心性之集中,一念之心能通达万事。

你有爬山的经验吧?站在山顶时,我们的视野就较开阔,脚下的世界尽收眼底。我们可以从许多地方登上山去,一旦上得山顶,就能享受这不寻常的景致。修道也是一样,只要择定一条路,执持一念努力往前进,终有到达悟境的一天。站在山顶上,可估量出许多条道路,而不论是去那一条,你都能把握住方向,也就是说,不论在修行的过程中是如何,一且你抵达悟境后,你就能掌握住所有事物的真实情况了。凡精通一样技艺的人,也就能触类旁通地知道其他事物的道理,其原因即在此。只要有机会站在山顶上,周边什么事都看得清清楚楚的了。」

阿那律惭愧地回答道:

「我今后一定要从这方面下工夫了。」

佛陀赞许地点点头。

「孙陀罗!」

这一次佛陀喊到孙陀罗。孙陀罗正在为一周前分手的那个女孩子而感觉困扰时,因为那女孩子的脸庞与神情正不断浮现他眼前,使他深深不安。

他听见佛陀在喊他,也不由得吃了一惊,紧闭着双眼,以恭敬的态度接受佛陀的教诲。

「你应把心扉打开,不要再那么执着地紧闭着它。任谁都免不了会犯错,问题是如何实时改过而做为他日行事的参考。一般人都不愿面对现实,但是自己是不会受骗的,自欺的人势必信受欺骗之苦。

端正己心,忠实地生活下去,这才是正道。当然,如果为了忠于自己,一丝不苟,战战兢兢地生活着,也非善道,那样会把自己局限在一个小天地中,心量变得更狭窄了,因为此心为忠实之一念所束缚了。想要做得切恰,不致为一固定的观念所拘束的话,就要学习那中天辉耀的太阳,在自己心中画上圆亮而宽广的心,把那心当做自己的心,如此心量会逐渐扩大,不再为琐事所羁绊,也不会因此疏忽掉琐事。

正法就是合于中正之道的法,不脱离为人之道,也不随意阿谀附合不合理的事相,是能使万物复苏的一种法则。

人是同时以肉体和心灵而生存着的,你若有意注重任何一者而疏忽了另一者,都会产生苦恼。因为肉体与心灵本为一体,二者能调和,才合乎中正之道。肉体若受到极度的痛苦,心灵也会跟着生病,而心灵如果遭受打击,食欲就会受到影响。当我们了解到生命永恒的本质之后,就可觉察到心中无明所引致的苦恼,多半来自对肉体五官的贪着,这贪着之念因而束缚了心灵的自由,故我们应努力于除去心灵的束缚,方能领悟到中道的真实。」

孙陀罗听了佛陀这一番慈悲的开示,眼中不知不觉渗出泪水。

佛陀开示的当儿,除他们三人外,另有许多修行者在旁倾听着。

当此之时,阿难觉得内心深处好像有什么正在急切地朝上涌出,他想加以抑制,但终于忍不住放声大哭出来。

三个人由于佛陀的开示,立时霍然开通,脑后皆浮现出黄光,佛陀看在眼里,深感安慰。

阿难后来一直随侍佛陀身边,照料佛陀身边的一切琐事。

而且阿难有着过人的记忆力,凡经佛陀说过的法,他都能过耳不忘。在佛陀灭度后,他负责就记忆所及,把佛法记载下来,故至今我们看到许多经典都是以「如是我闻」为开端而流传于世。

在当时,文字的书写极不方便,有许多经典是借口口相授的方式传下来的,直到阿育王时代,才有较完善的文字记载。

阿育王乃是一位虔诚信佛的君主,为了想使正法传诸后世,而以自己全部的资产来召集佛教的高僧大德编纂佛陀的教义。这是佛经第二次的大结集。

第一次大结集是在佛陀灭度后不久,由摩诃•迦叶所召集,这次结集中,阿难是主要的纂辑人,一来由于他记忆力非凡,对佛陀说过的法都能记诵详实,二来,他经常跟随着佛陀,亲聆过无数次的说法,由他来纂辑佛经,自是非常合适。

口传经义自然会发生错误,但以当时文字书写之不便的情况而言,一般人对口授得的东西多半有较强的记忆力,同时对这种传授方式也习以为常,故而学得的东西,其可靠性不是现代人所能比拟的。

再说,一旦有了文字记载以后,人们就习惯于依赖文字而惰于思考,这是人类的弱点之一。没有文字时,人们磨炼了自己的记忆力,并能将错误减至最少。

希腊的圣哲柏拉图就曾谈到人类的记忆,说人类自有文字记事以来,记忆力就开始衰退,变得非常容易遗忘。

进一步说,古代人由于物质文明的不够发达,虽然知识水平较现代人差,但在精神方面,都有着现代人所无法想象的优秀面,所以古代人似乎经常有奇迹式的成就出现,这是现代人所望尘莫及的。

譬如像玛雅文明、希腊文明、古中国文明,以及金字塔所象征的埃及文明等等,都是很好的例子。

所以由这一点看来,当时佛法几乎可说是毫无遗误地被传至阿育王时代。

当言语变成文字后,就被人以眼来接触,读者也就容易加上自己的见解,因而形成各式各样的解释,错误之处也就无形增多了。

当一个人说话时,通常伴随着感情的作用使他的话产生预期的效果,也就是说言语加上了当时的场合、气氛、声调及其他因素而能成为有生命的东西跃动着。若把言语转变成文字,除非有生动的词藻来使它生动化,否则它就是一片没有感性的叙述,说服力也就很有限。所以写下文字的人,应有着驾驭文字的能力,并加入许多想象式的描述,以增加文字的魅力。文字的表达需有特殊技巧,这就造成了文字转述的障碍所在。

佛陀的言教变成文字传于后世时,意义与叙述方式就变得十分繁杂而真伪莫辨了。

现在言归正传。阿难在当时诸弟子中,由于记忆力特强,担当了编纂佛经的大任,故佛经一开始的「如是我闻」就成了佛经的基础形式。

且说佛陀对三人开示完毕之后,舍利弗就来带领三人入内。三人擦拭着纵横满面的泪水,跟在身躯庞大的舍利弗之后。

到了休息地点,舍利弗回过身来对他们说:

「你们如今已成为佛陀僧团的一员了,在人生旅途上算是真正得救了,希望你们把握机会好好充实自己,并能把自己获得的喜悦分享他人。」

三个人以炯炯有神的眼光注视舍利弗,并高兴地向他道谢。

此时晚霞已染红整个山头,薪火的白烟正袅袅上升,在大自然中,三个人只像是点缀其间的三个小颗粒。

阿那律小声对阿难说:

「我们可得加油啊,不然就是出家了,也得不到正果。」

阿难没有搭理,阿那律就用轻触他的肩头,继续说道:

「阿难,你怎么不吭声?愿望达成了,你难道不高兴吗?」

阿难这才回过头来说:

「当然高兴啦!我正在回味佛陀刚才的那番话,你就请安静一点吧!」

阿难的口气中有些不耐。

正说间,有一个人走近他们,他们抬头一看,原来是佛陀,正微笑着注视他们。

三人一惊,下意识地端正了坐姿。

佛陀微微点着头,好像在赞许他们。

「阿难,你在想迦毗罗卫吗?」

「没有,佛陀。我正在想,我现在能陪侍在佛陀身边,是多么幸福,以后还望佛陀多加指点哩!」

阿难低着头,恭敬地回答道:

「孙陀罗,你还是跟小时候一样,那么结实。」

佛陀又笑对孙陀罗说。

「阿那律,你的肩膀怎么样了?还痛吗?」

「不,已经不碍事了。」

阿那律困惑地回答道。

佛陀离开之后,三个人都舒了一口气。

「我们从山上滚下来的事,佛陀好像知道哩!」

「是啊,真奇怪,我们又没说什么,身上也没有伤痕,他怎么看出来的呢?」

孙陀罗大惑不解地思索着,然后问阿难:

「你觉得怎样?」

但答话的人,却是不知什么时候又出现在眼前的佛陀,他说:

「阿那律没注意绊了一交,阿难去扶他,然后两个人就像石头般滚落下来。现在还痛吧?」

三个人都张大了眼和嘴看着佛陀。

佛陀则把药草交给阿难,继续说:

「快把这药草贴在阿那律的肩上吧!」

阿难遵照吩咐地为阿那律贴上药草。

原来佛陀刚才离开他们,是去找治伤的药草。

三个人为佛陀此一悲悯的举动深深感动了。

佛陀将手轻轻地按在阿那律肩上,然后又轻按手脚等部位。

当佛陀的手一触及患部,阿那律就感觉一股暖流由患部扩散至全身。

阿那律舒服得快要睡着了,但他赶紧睁开眼睛,不敢稍有懈怠。

「你感觉怎么样?」

佛陀以柔和的眼光望着他说。

「真是舒服极了!」

「你站起来走走看。」

阿那律依言站起身来,来回走了两三步,奇妙的是腰脚的关节处已不再酸痛了。身体已比受伤前更轻快灵活了,心上有种说不出的爽快。

他匐伏在佛陀的脚前,感激地说:

「托您的福,我身上的疼痛全都消失了,现在觉得轻快无比。」

佛陀也快慰地点点头。

阿难和孙陀罗也接近阿那律身边,为他这一现象感到高兴。

「阿难,你的脚痛吗?把脚伸出来让我看看。」

佛陀边说,边以手按在阿难的脚踝上。

一个悟了道的人,能以心灵之眼透视对方的身体状况,就像照X光一样。

若身体的某部位不适时,会呈现出黑块或白块,因为该处的细胞未能正常发生功能的缘故。

此时,有灵力的治病者,只需集中意识,以手按住患部,就能使之逐渐透明,细胞恢复生机,疼痛自然就消失了。

为什么集中意识以手按住患部会有治疗作用呢?因为人的手掌会藉意识的集中而增加电流的放射,以此电流的剌激,可以治愈患处。平时意识不加集中时,一般人的手掌就能放射微量的电流。

这电流会产生一种光,光会射向患处。

细胞未能发挥正确功能,是因为推动细胞运作的电作用变弱,所以以手按抚,有充电作用,而能使细胞恢复常态。

当然,这并不是说任何疾病都可以用这种方式治疗。

这与细胞的定数、疾病的轻重,以及治疗者手光的强弱都有关系。光的强弱又与其本人的精神调和度和念力的强弱有关。

通常使用药用外,再加上按抚,最能加速病情的恢复。

由于佛陀的按抚治疗,阿那律和阿难身上的酸痛一刹那间消失无踪了。佛陀见两人已没有问题了,就面露微笑地离开了。

「真是不可思议啊!不可思议啊!」

阿那律不住地赞着,并端视自己的脚部。

孙陀罗见状,怀疑地问道:

「是真的好了吗?你可当真?」

「一点也不假,完全好了!」

阿那律说着,跳了两三下。

「阿难,那么你呢?」

孙陀罗还是不相信。

「我也好了,由于佛陀这一手,我已恢复原样了!」

「哦?真是这样子吗?」

孙陀罗若有所思地,与阿难、阿那律并肩躺下。

三个人从今天起,就正式由王子的身份一变而为佛弟子,开始过严酷的修行生活。

\section{优婆离加入僧团}\label{sec9.3}

自从优婆离替三位王子剃发并送他们后,他一整夜都未曾合眼。

他一直怀着惶恐不安的心情,怕自己替王子剃发之一举会惹来杀身之祸。他一心想着免掉灾难的法子,自己以一奴工的身份,即使逃到外国,也很快会被人送回来,那时仍难免一死。

「我也出家吧!」

他迷惑地想着。

他早先也一直有出家修行的心愿,只因自己以奴工的身份,是身不由己的。他在城中是数一数二的名理发匠,但虽然名气大,所用的工具也仅有剪刀、剃胡刀和木梳而已,家中更无妻儿的,一无牵挂。

他听说佛陀和一般婆罗门修行者的主张不一样,佛陀主张众生平等,没有贫富贵贱之分,无论什么阶层的人,只要一心一意把心灵修养成功了,一样能得道。当然谁都能成为佛陀的弟子。

当佛陀回城说法的那天,他在远处看着佛陀,听到佛陀所说的正法,不禁泪如雨下,从那时起,他就对佛陀僧团心存向往。

「就这样决定了,出家去!从佛法中必能了解人生许多矛盾的真相。」

下定决心后,他第二天就抱着三位王子留下的衣物,离开迦毗罗卫城。

来至尼拘陀树林,他竟犹豫起来,他有些害怕,不知自己有没有下错决定。正在此时,一位身材高大的修行者迎面而来,他一看来者是舍利弗。

于是他战战兢兢地趋前问候道:

「舍利弗尊者,我叫优婆离,是个理发匠,请允许我为弟子吧!」

舍利弗停下来,看着瘦小的优婆离,以温和亲切的态度说道:

「你是理发匠吗?我曾看到你在大家后面很热心地听佛陀说法呢!」

「是的。听法的这机会实在难得,以前我也听过婆罗门修行者说法,但都不像佛陀能打动我的心,我心中的高兴到今天都没办法平息。」

「噢,是这样吗?这表示你跟佛法有缘。你手上拿的是……」

「您是指这些衣服吗?」

「很漂亮的衣服哩!」

「这是宫中三位王子出家时放在我那儿的东西,我拿这些东西也没有用,所以想把它们还给王子,然后我自己也要出家。舍利弗尊者,请收我为弟子吧!」

说完,优婆离立刻向舍利弗躬身拜下去。

「你的心情我很了解,但你要知道,我们过的是很清苦的生活,既无住处,也无收入,你能忍受吗?我劝你还是安心地在家修习佛法,那也是一样的,等你信心更坚疋的时候再出家也不迟,千万不要凭一时的冲动。到你确信自己的信心不再动摇时,就请来这里找我吧!」

「今天不行吗?」

「希望你好好考虑,下定决心后再说。」

优婆离自认自己的决心已够坚定,但他见舍利弗拒绝了自己,知道多说无益。

只好说:

「那么到时候请您多多帮忙!」

舍利弗因为还有事待办,听优婆离回答后就离开了。

优婆离无可奈何地呆望着舍利弗渐去渐远的背影。

过了几天,他把理发店的门关了,就出发往佛陀所在的尼拘陀树林。

他剃掉自己的头发,穿上一件脏僧衣,他坚信修行者的生涯逢很适合他的。一到尼拘陀树林,他就马上去见舍利弗。舍利弗看到优婆离的头上发出淡淡的金光,真怀疑自己的眼睛,他奇怪优婆离未经七天的反省,头上就已发光了,这真不可思议。

这金光似乎与优婆离的脸相重叠,看来他虽然未经七天的反省,他此刻的心境已具备了入门的资格。

舍利弗定睛望着优婆离说:

「要入佛陀的僧团之前,必须先做七天的反省,这是此地的规定,你愿意吗?

「当然,我会按一切吩咐去做。」

优婆离坚定地点头说道,然后就去寻找一适当的场所,开始静坐。

舍利弗为优婆离迫切求道的心情所感动,楞楞地看着他。

七天很快过去了。

舍利弗一天都在等优婆离前来,但直到晚上还不见优婆离的影子。

他不知发生了什么事,赶紧到优婆离禅定的地方来探看究竟。短小的优婆离依然定定地闭着眼睛坐在原处。

脑后泛现着较七天前更明亮的光。

舍利弗想唤他出定,但又怕打扰了他,就悄悄离开了。

优婆离的个性十分倔强,在自己的反省过程中还有不了解的地方就不愿轻易停止反省。

虽然他知道自己只需反省七天,但在自己没有把握之前,即使过了十天、二十天,他也不想离座起来。

他这种行为是很认真严谨,但那足以使自己的心灵趋于狭窄。

这种个性,形成了他日后修行时以戒律为主的一个事实。

佛陀在世时,并没有特意制定什么戒律。他并没有一一指陈弟子该这样、该那样地做着细节的规定,他仅是对弟子们脱轨的行为加以提醒,让弟子们自己去领悟个中的道理。

佛陀认为修身的最大依归,即在于本心的觉悟。

人心本有着自由的本质,这本质的神圣,是谁都无法侵犯的。

因此,一个人在言行方面,要尽量避免将自由的心灵束缚住了,应从心性悠游自在的状态中开始修养,以领悟本心的伟大本质。同理,我们也不应去束缚他人的心,或任意责备他人的言行。

佛陀的教义即着重于对心灵宇宙的了解,并确认肉体只是一种心灵的反映罢了。

宇宙的主人就是个个灵魂,宇宙内的一切细胞活动就是各灵魂属性的分身。四海之内皆兄弟的一种观念,即源于人类皆有同一完美属性的灵魂的一种观念。

具体而言,当我们束缚或责备他人的言行,间接就是在责备和束缚自己的自行。

自己和他人同为一体,这就是佛陀所领悟到的。

因此佛陀从不强迫弟子们如何,也从不制定死板的条文戒律。

佛陀灭度后,由于人们见解的纷歧而使佛法变质了,佛法只成为部份人士能领悟的道理而离开了大众。

或者可说是由于玄妙的哲理和严格的戒律,而把佛法从众生身边拉开了。

佛陀谢世后,由于优婆离的极规律的生活和严厉审视自己的态度,渐渐形成了佛教重成律的风气。

在他看来,佛陀已不在身边,为了守住正法,必须崇尚戒律,并非为了建立一种戒律式的佛法。

戒律是一种有边有角有范畴的存在,它使事情没有圆融和合的余地,为保持纯正,必须排斥不纯正。然而所谓纯正与不纯正的界限又何在?

佛法的神髓,即在使一切悲悯升华,以致超越了纯与不纯的区别。

就广大无边的佛法来看,不善之事只如一小小的黑点,不必定要以嫉恶之心来把它自佛去中分离。

佛法具有广大的视野,也有包容的慈悲,嫉恶如仇的心态是离悟道有着相当距离的。

不过话说回来,优婆离求道的真挚态度,实难能可贵,那使他日后成为一位守戒律的模范。

一直到第九天,优婆离才去见舍利弗。

这九天中,他不饮不食。

然而意识却格外清晰明朗,意志力成功地支撑了肉体。

「我已彻底反省过了,请您多多指正!」

优婆离认真诚挚的态度,再度感动了舍利弗。

优婆离身后的光更明亮了。舍利弗激动地说道:

「优婆离!请快跟我来,我带你去见佛陀。」

「啊,真是太好了,专者,我一定要好好修行,不辜负您的期望。」

舍利弗把优婆离带至佛陀跟前。

优婆离紧跟在后,唯恐稍有落后。

经过九天的反省,他的心境静如止水。

又由于马上就可见到佛陀而能一偿宿愿,心情更是愉悦轻快。

「佛陀!优婆离来了!」

「优婆离……」

佛陀沉思了一会儿,很快地领悟道:

「哦,是城里的理发匠啊!请到这边来!」

优婆离快步趋前,在佛陀脚旁礼拜下去。

佛陀注视着优婆离,慈祥地说:

「请把脸抬起来……」

「啊,是!是!佛陀,真是久违了。我在城内听过您说法,像我这样卑贱的人,若也能成为您的弟子,那我真是太幸运了。

那天佛陀的话深深感动了我,不断在我心中回响着,我都忍不住哭了。

这个世界的确是无常的,除了佛陀之外,没有人能扭转这个事实。我常想,像我这样的人,如果也能成为佛陀的弟子,那该有多好啊!所以我请求舍利弗收留我!」

佛陀面露微笑地听优婆离絮絮叨叨地说着。

「欢迎你来。我本来就主张众生平等。不过在求道之先,应了解自己。若不了解自己,就不知道什么是真,什么是假。反省就是了解自己的第一步骤。

你不只反省七天,已有九天了吧	」

「啊,是的。」

「都反省了些什么?」

「哦,为了要做您的弟子,我不允许自己有不纯正的念头,所以一直在反省这一点。」

「噢,什么不纯正的念头呢?」

「我以前一直就有逃避奴隶生活的念头,后来三位王子要我剃发,我又为了怕被处死而想到出家。我就是在反省这样的念头。」

佛陀看得出优婆离认真精进的心态,他知道优婆离出家的动机并非单是为了憧憬受人尊敬的修行者的身份以脱离下阶层的悲惨境遇。

如果优婆离有这样的念头,身后就不会有黄金色的光。

不仅如此,优婆离已能坦荡落地不再惧怕死亡了。

在佛弟子,有不少的人存有一种虚荣,喜欢当佛陀的弟子,喜欢成为修行人。所以佛陀对优婆离的坦诚率直,很感慰藉。

「佛法的目的在使这个世界趋于谐调,使每个人都能了解自己的心,从而创造出更丰盈的心。此心若有不平不满的情况时,行为就会趋于乖戾,心灵就不再安泰,不再能认清事实的真相。

你已好好地审察了自己,接下去的课题是如何拓展自己的心灵。严以责己,宽以待人,就是合于佛法的做为。一般人既对自己严厉,也对别人严厉,因为他们将统御自我的意识加诸他人了。所谓自我统御,只应用于自我,以控制自己的自行,并非为了去统御他人。

与人相处时,可以由对方看到自己,因此对自己严苛的人,必定对他人也严苛。但是,是不是对自己宽容的人教别人也宽容呢?正好相反,一个对自己百般宽容的人,他的自我保存意识必也十分强烈,这样的人更会苛待他人。

不管怎么说,当你反省自身时,若仍意识着他人的言行正确与否,就很难发现正确无误的自己。他人毕竟只是端正己心的一种参考而已,不能将他人视为自己可驾驭的对象。虽然人心一体,但每个人都是以单一的主体而修行着的。因此,虽然对自己严厉,对他人却不得不宽容。这个道理,你能了解吗?」

佛陀以慈悲的眼神注视着优婆离。优婆离说:

「我能了解。我会把您今天的开示做为我今后修行的准则,我一定拚命去努力!」

「佛法向来不离自然,知道吗?希望你以法为凭依,好好努力于正道。」

「请让我加入僧团吧!」

「当然!你现在看着我!」

优婆离依言抬头望向佛陀,这一惊非同小可,他倏地匐伏在地上。

因为佛陀的身后,现大光明,像是梵天的光,正灿烂地辉耀着,那万丈光芒,真无以名状。

「你超越生死的念头而来到这里,你是一个出色的修行人。」

佛陀语声甫落,优婆离高兴得放声痛哭起来。

在旁伫立良久的舍利弗,眼眶也湿润了,他很感激佛陀的悲悯。

阿难、阿那律不知何时出现的,他们看到优姿离,感到很惊讶。但见优婆离度真挚,神情肃穆,二人的内心也为之悸动不已。

他们过去与优婆离有着非常悬殊的身份关系,从今以后就是修行的同伴了,不能再存有阶级的意识。

所以他们默默在心中惕励自己,今后一定要以平等的心性态度来对待优婆离,一同在正道上奋勉共进。

\section{冒充者}\label{sec9.4}

关于僧团中的生活,本文早已提过,似无再述的必要,但是由于佛陀的声望日隆,皈依的人急剧增加,几乎是络绎不绝。因此想再将僧团的生活概况叙述一番,以供参考。

佛陀的僧团自从来到迦毗罗卫之后,又增加了阿难、优婆离等众,修行僧的人数已超过两千。

最初,入门的人都曾经佛陀一一接见审视,但因为人数激增,就改由舍利弗、目犍连、迦叶等大弟子代为负责其事。

这些弟子都已达阿罗汉的境地,已能透视他人的言行心性状态,故足可担当起此一任务。

加入僧团的人既增多了,就有增建精舍的必要,否则大众处于露天之下,显得杂乱无章,行止无法统一。

当时主要的精舍只有两处,一是祇园精舍,一是竹林精舍。

祇园精舍本已尽善尽美,但对人数日增的僧团而言,就有了不够容纳的顾虑了。

为了解决此一问题,就有了许多临时构筑的简陋小屋,也有人利用洞穴暂时栖身。

大的精舍则用来供佛陀说法,或供弟子们讨论佛法之用。虽然佛陀的声音宏亮清晰,但在荒郊野外对着成千上百的人演说,其效果究竟不如室内,故精舍一直被视为宣扬佛法不可或缺的所在。

加之印度有漫长的雨季,当雨季一到,低地就经常洪水为患,山地又成为瀑布区,不可能再过野宿生活。

所以僧团中的修行人在雨季中,多半足不出户,利用这段期间加强内省的修行工夫。

祇园精舍中除佛陀说法的宽敞讲堂外,还有能容几千人禅定的隔间。其他有佛陀的居室,佛陀大弟子们的房间,并有厨房、盥洗室等等。

竹林精舍的厕所设在屋外,前文已提过,即在屋外挖一长二十公尺,深二公尺的坑道。坑上铺以木板,周围绕以稍可遮人的围墙,再加一块板以成屋顶。非常简陋的设置,虽离精舍颇远,但因风向不定,臭味仍不时飘至精舍。

祇园精舍的盥洗室则建在精舍内,有着良好的通风设备,给了精舍中人不少的方便。

居位于精舍内的人,都尽量利用屋内炊煮食物,在屋外个别进行炊事,人一多就容易发生山林火灾等事,极不安全。

精舍是由石头和泥土的混合建材所造,极适合避暑和避雨。

后来不断有信众捐建精舍,替修行者整顿出易于修行的环境。但如前所述,精舍的主要用途在于说法和讨论佛法,来听法者,或是修行者,或是在家信徒,一般修行者也不住精舍中,多流动在外,故修行者在街头说法,是常见的情形。

每逢佛陀说法,就有人不远千里,翻山越岭地前来听讲。精舍中的讲堂也就常常为之爆满,有许多弟子反而退居于屋外。

当时的精舍,可说就是今日寺庙的雏形。佛陀灭度后,佛教由南向北传至中国、日本等地,寺院被着上一层厚的历史色彩,渐由开放转为封闭,最后变成为死者诵经做佛事之所在。

僧侣的职责本在于把正法传给世人,但是今天,他们只是死者灵魂的守护者7。他们的存在意义•已与从前佛陀在世时;侣大异其趣了。当时的精舍,是僧俗授受佛法的所在,到处呈现着盎然的生气。

今天,僧侣们的生活是靠死者亲属的布施来维系着,在传法时,难免有仰人鼻息之类的牵制。

从前的僧侣靠乞食来修行,目的即在断绝执着,透过乞食行脚以审视人己之心,这种严格的修行方式,在今日几已不可见。今日的僧侣为何出家修行,也为一般世人所难以了解,他们诵念着经文,给人一种徒具着修行的形式的印象罢了。

从另一方面来说,传法的对象应以生存着的人为优先考虑,死者只是其次。

因为死者的灵魂安份与否,是与生者有密切关系的。此处所指的灵魂,特别是指执迷不悟的灵魂而言。当这些灵魂生前在人世间时,即是执着心重,贪恋世物者,离开人世间后,就羁留于地面,而无法超生。

既然不得超生,只有依赖生者以维系灵命。

如果此时被依赖的生者本身言行亦属不成熟者,则该灵魂便永无脱离执迷的一天了。

本来僧侣为死者诵经,目的是在超渡亡灵,使他们不致留连人世间而能往生天界。然而有许多寺庙,未待超渡,即先行葬礼,使得亡灵依然觉得自己还在人世间,必须靠家属生活着。

因此往往有人被亡灵占据着肉体而自己则身不由己。拙作「恶灵」,就提到许多这类例子。

如果说执迷不悟的亡灵向生者捱近了,但只要这个人是一持守正法的人,能依正法生活,亡灵也无法占据他的肉体来骚扰他,同时亡灵可从他的行为中了悟自己行为的谬误而能获得解脱。这种情况,生者虽无法意识到,却是极可能发生的。

脱离了肉体束缚的亡灵是很容易了悟一些道理的,他们自会选择较光明的去处。但这必须赖生存着的人能理解正法,以给亡灵某种程度的启示,方可达到的。当人世间一片祥瑞时,地狱也会跟着趋于谐调。

尤其当亡灵的能是靠吸取生者的能而成立时。

地狱界是个不能透光的世界,故而到处黑暗片片。若不以现象界的人为媒介时,就无法保住地狱中的能。地狱界的能又来自人心发出的恶念。自我保存所产生的念力是地狱界中生命的源头。

由此得知,当人世间的罪恶猖獗时,地狱界也就跟着非常活跃,同时更使得人世间鸡犬不宁。

从这一角度来看,使人世间生存着的人了解正法,岂不是比让亡灵了解要来得更重要吗?

佛陀住世时的精舍,是为了济渡生者而设,不是为了死者。

但今天正好相反,把济渡的对象转向死者了。

更令人悲哀的现象是,有许多担任济渡重任僧侣,实际上并未具备这种能力。他们只知讲经、诵经,自己并未能实践经中的法理。

我们很了解自己的心,知道自己的心不纯正时,经文与牌位对我们都是毫无用处的。

寺院应充份被做为济渡生者之用,应以它为传道的中心,使生者都知道以法为准则而生活着。

二千五百余年以来,佛教的教义由活泼泼的圣教依据转为死寂刻板的条文。

我们面临到末法时代,所谓末法,就是法的死亡。

现代的确是末法时代。

佛教已完全降格为死人的宗教,有时寺庙只供活人观光之用。曾经光芒四射的佛教,如今只剩下一具形骸,好似没有生命的木乃伊。

观乎佛陀住世时精舍的全貌,能不有所警觉而思拨正今日寺院的畸形发展吗?

且说由于僧团人数的激增,佛陀以及负责指导的大弟子们已渐无暇顾及每一加入僧团的份子。在当时,僧团中的修行以个人的开悟为主题,至于修行上的细节与方法,则多听任自由。因此佛陀僧团中有许多修行方法。似乎跟其他教派的修行法相混淆了。

唯一可资区别的,则是服装方面,婆罗门修行者多着丝绸等上好料子的僧衣。佛陀僧团中的修行者则不注重衣饰,有时破旧肮脏的僧服使人误以为路边的乞丐。

与乞丐不同的是,他们不会蹲在街角,或徘徊街头,一旦乞食完毕就径自离开,回到精舍,或回到林间,继续禅定与反省的工夫。

僧团中的另一特色是,年轻的修行僧很多,老年人很少,这一点跟婆罗门教也很容易区别。

虽然僧团中的年轻僧侣多穿着粗陋的僧衣,但由于气质的非凡,人们还是可以马上看出他们是佛陀的弟子。

佛陀的伟大是有口皆碑,因此他的弟子们也连带受到民家的欢迎,人们都乐意布施衣食给他们。有的人听说是佛弟子来了,不管识与不识,立刻延入家中,请教佛法或探询佛陀的近况。

因此在当时,造成了一种风气,都认为与其在家中辛苦地工作,不如早点加入僧团做个修行人,既少烦恼,又能免去生活上的担子。

僧团中的人一向很严格地自律着,除乞食外,绝不额外接受在家人的供养,因此他们跟在家人很少发生纠葛。

但是有一天,发生了一件事。

一位五十来岁的商人模样的人来拜访目犍连。







他一见到目犍连,就诚惶诚恐地赞扬佛法,并请求让他学习正法。

目犍连为此人突如其来的行迳大惑不解。

之后,那人说要迎目犍连的某一位弟子回去做养子,以继承家业,并声称该弟子已与他的独身女有过婚约,现在他的女儿已告怀孕。

目犍连对所提到的名字很感陌生,于是答应要调查一下,再作答覆,然后把那人送回家。

他左右为难,不知道该不该把这件事告诉佛陀。继之一想,还是先由自己调查清楚了,再请佛陀裁决。

当天晚上,他将自己负责指导的弟子们召集而来,说明事情的经过。

弟子们都讶异地面面相觑,彼此交头接耳道:

「事情可要闹大啦!」

目犍连闻言,很不自在地说道:

「这件事是谁做的,自己一定清楚,请不要顾忌,可以来找我商量,我不会责怪你的。虽然已经出了家,但如果与那位姑娘情缘未了的话,相信佛陀也不会留难的。修行不必一定出家,只是这件事的做法很令人遗憾,让姑娘和她的家人如此担忧,是很不应该的。这真是修行人不应该有的事。」

他边说边以锐利的目光扫视全场,同时身子向前倾,好像已看透是谁做了这件事。

在场之人,谁也没做亏心事,否则一定会心惊肉跳,现出不自在的神色。

目犍连继续提高声音道:

「以为不吭声就没有人知道了吗?」

「我们之中没有人会做这种事。」

前面有一个人很自信地说道:

「那么是那个商人在说谎吗?」

「不知道他是不是说慌。我们这里没有这样的人。」

座中有人如此回答道。

目犍连继续问道:

「你们有没有听过类似的消息?……」

大家都一致摇头回答说没有。

目犍连很感困惑。

如果说他的弟子中没有人做出这种事,那会是谁呢?难道有人想恶意中伤僧团的名誉,冒充僧团的人去欺骗人家吗?

这个人又竟然冒充了目犍连的弟子去行骗,实在令人气恼。

「你们之中,真的没有人做这种事吗?」

「绝对没有,这种行为会玷辱佛陀之名的。」

目犍连再度审视了众人的脸,确定座中没有这样的人之后,才稍稍有些心安。但他随及想到,如果不找出这当事,是不足以洗刷僧团蒙上的羞辱的。

这事很明显的是某流浪者冒充了僧团之名而做的,他直觉到这个人一定在听了佛法之后,懂得了佛法的一点皮毛后,就假扮成修行人去诱骗女子及其家人。

他想亲自解决这个问题,所以从第二天起,就藏身于该女子的住处附近,等候那个冒充的人出现。

他由清晨等到日暮,甚至空着肚子,就守候在草丛间。

如此过了三天。

当日影西斜,夜幕渐渐低垂之际,目犍连突然看见一个皮肤白晳,个子高挺的僧侣立在那商贾的门前,不一会儿,影子就消失在门里。

这个僧侣的长相和动作,他都觉得十分陌生。

目犍连直觉到他就是那个冒充者。

他压抑住心中的怨恼,来到商贾家中。

那个年轻的僧侣正状至熟络地跟姑娘以及姑娘的母亲谈笑着。

两个出家人的目光相遇了。

年轻的僧侣立时怔在那儿。

目犍连随即向主人翁探问来人是谁,而经主人翁的介绍,正是他要找的那个冒充者。

此时那个年轻僧侣知道自己的身份已暴露了,立刻跪下来,求目犍连原谅他这种冒充的行迳。

目犍连见此人很诚意地向自己道歉,看出他不是一个罪不可赦的大坏蛋。于是他问年轻人为什么要说谎。

年轻男子齿牙哆嗦,嘴唇发紫地说明了原委。

原来他私心倾慕这个姑娘已经很久了,一直苦无机会接近,为此他曾懊恼地想以死来求解脱。正当这时候,佛陀的影像自他心中出现,他灵机一动,何不妆扮成出家人,就有机会和这姑娘见面谈话了。于是他瞒着家人做了僧衣,并剃掉头发。

他终于得偿宿愿,跟姑娘甚至私订了终身。

年轻人是木匠出身,住在隔山的小镇上,他不想继承家业,只想改行做布疋买卖,曾想着与姑娘家做生意上的来往。

他并打着如意算盘,如果能将姑娘娶到手,那就是人财两得,岂不称心如意?

目犍连听完这一番说词,不禁觉得好笑,想这男子竟能把商贾一家骗得如此服贴,着实不是等闲之辈。正因为他有头脑。才不甘心一生沦为木匠。而想着做买卖赚大钱。金钱实在会诱人步入歧途啊。

他很不满这年轻人假藉佛陀之名来行骗。并声称自己有此一举是佛陀的影像浮现心中而得到的灵感所致,这说法会引起人们的误会。佛陀一向严禁僧徒说谎,而这个人竟为了推卸责任而把佛陀搬出来,实在非常可恶!

为了这个理由,目犍连开始训诚他。

「你的意思是佛陀教你来骗这姑娘的吗?这分明是你自己的欲心在作祟,佛陀绝不会教人做这种事的,说谎的罪恶,你应该知道才是!」

「我没有这个意思,我绝不敢说是佛陀教我的,我也不敢那样想!」

「那么你为什么要强调是因为佛陀出现在你心中,你才想到要冒充出家人的呢?」

「啊!您误会了。不如说佛陀是我的救命恩人,要不是他在我心头出现,我早就已经去寻死了。我能活到今天,并且还能跟心上人说话,完全是……」

年轻人手指着姑娘,激动得说不下去了。

目犍连为这情景所感,一时也楞在那里。

他也就不再追究下去,眼前这个男子不过是个痴情人罢了,为了想接近心上人而出此下策,差点酿成大祸。看他的眼神,如果姑娘的家人肯接受他,也可以有个圆满的收场了。

此时目犍连的情绪已恢复平静,先前的激愤已化为乌有,他反而为了男子的前途而向商贾请求,请他就把这年轻人收为养子。

于是这件事就不着痕迹地掩盖过去而获得圆满的解决。

目犍连的想法没有错,这个痴清男子后来替岳家积聚了不少的财产,后来还成僧团的大施主。

后来目犍连不幸被狂徒以乱石袭击而死,他竟也悲愤地随后绝食而死。

由于佛陀僧团的声望日高,假藉名义蛊惑在家人的假修行人也日益增多,因此僧团中的修行人越发惕厉自己专注于沸法的研习,以免玷污佛陀的令名。

当初僧团中,除了埋托勒呀等四个女修行外,全都是男众。

自佛陀回到迦毗罗卫,释迦族中有许多女子也纷纷要求出家,然而佛陀对这件事一直保持沉默,未置可否。

由于妇女的加入,会增加僧团中男众修行时的困扰。出家修行,本在觅一清净之地以断除情欲等引致烦恼的苦因,女子一旦加入,这环境势将变得复杂了。

婆罗门修行者中有女修行,而佛陀僧团中不宜有女修行,这情况是有理可循的。因婆罗门的游化僧皆已是上了年纪的人,他们这一阶段的修行旨在断绝对世事的执着,以求幸福的来生。然而佛陀的僧团中充满了为求正道的年轻僧侣,既是年轻人,皆值血气方刚之龄,难保不会惑于美色而动摇了求道的信念。故佛陀对女子出家修行一事,为了顾全整个僧团的成就,一向不做鼓励的表示。

但是佛陀既主张众生平等,也没有理由坚持不让女子加入,故而后来僧团虽然由于允许女子加入而达到了平等的理想,但问题也就从此开始产生,许多年轻僧侣由于受不住情欲的诱惑而弃道还俗,而僧团中也屡屡有败坏风气的事传出来。

\section{真实行道的人}\label{sec9.5}

阿难、阿那律、陀孙罗三位王子,以及优婆离等人的出家,给迦毗罗卫带来很大的震撼。在短短的期间内,就有那么多人相继出家,的确是迦毗罗卫有史以来少有的大事。佛陀的吸引力是那么大,人们不分年龄、阶级和性别,都被他深深吸引着。

阿姆利德亲王来到宫中谒见净饭王,并就这件事与大王商量对策。

「情形这样发展下去,可不是好现象。现在在城里,到处可以听到『我要出家』的呼声哪。尤其是我家的那两个孩子也吵着要跟在阿那律后头出家。我看现在有许多人都在家中冥想静坐,想修行的人不断在增加,这样下去,以后迦毗罗卫就没有可以打仗的人了。释迦族说不定要灭绝了!」

「……」

「您想,连优陀夷都出家了,凡是听过佛陀说法的人都像着了魔似的变成佛陀的俘虏了。其实我自己也是一样,在听佛陀说法的后几天,做什么事都不对劲,佛陀可说是伟大的人,但是释迦族的人都变成修行人了,那迦毗罗卫怎么办呢?我看现在该赶紧想法子阻止武士们出家了,请王说说您的高见吧!」

阿姆利德急切地注视着净饭王。

他见净饭王一直没有开口说话,只是闭着眼睛在思索什么,他不知净饭王在想什么,故而很焦急,希望他能开口说句话。

净饭王的身体沉在椅子里,仍然是三缄其口。他内心比阿姆利德还要苦恼。他虽曾宣布过释迦族的人可以跟佛陀学法,但他并不是要他们都出家,只是希望他们能以佛法为依据来生活,他那里想到现在出家修行变成一种风气。

武士们的生命本来也是没有保障,为了保家卫国,随时有抛头颅的可能。有时与邻国的一个小冲突,也会断送掉好几条人命。目前释迦族的背后虽有憍萨罗国撑腰,但南方的敌国随时有来攻的情势,一旦爆发战争,就将死伤无数了。

像这种国境四周弥漫着无比紧张气氛的情形下,怎不让他们兴起出家的念头呢?

净饭王久久才睁开眼睛,以沉重的语气问道:「边境的守备情况如何?没有什么大变化吧!」

阿姆利德对此一出乎意料之外的问题,一时不知如何作答,他微楞了一下,才回答道:

「是的,目前没有什么大变化。据斥候通报,敌人的军团没有移动。其他一些小冲突,不过是在剌探我们的情形罢了。王请放心。

「我觉得最近的情形很奇怪,他们有时会故意来挑琢。既没有演变成战争,倒可以暂时安心,不过……」

「所以我们也不敢大意,打算继续增加守备,以防万一,现在正在重编兵队,以便随时接应。」

「……」

「虽然我们已有了周全的防范计划,但最近许多武士们闹着要出家,实在是头痛。关于这件事,大王,您觉得……」

「听说不只是男人,连女人也想出家,而且不只是没孩子的寡妇,许多少女也想出家。夫人说有许多宫女想皈依佛陀。」

「我也听说过了,我想这不成问题,因为佛陀好像只收男众,这可以不必担忧!」

「是这样吗?」

「听说佛陀不允女子加入僧团。」

「但僧团中不是已有女修行了吗?」

「那是从婆罗门来皈依的,好像一般人就不可以了。」

「既然都是女人,为什么婆罗门的可以,释迦族的不可以,绝没有这个道理吧?佛陀一向提倡平等,连理发匠都可以入门,女人当然也可以的,你以为呢?」

「这当然是很有道理的。」

「夫人说宫女中想出家的就有二十几个,她自己最近也像是想出家的样子,我实在觉得问题很严重。」

「大王,您看怎么办?是不是下道命令不准他们出家?」

「不然的话,这会造成严重影响的。」

「不,不能这么做——」

「为什么呢?」

「我也不知道,就因为不知道,所以很困扰,不知如何是好。自从听了佛陀的法,知道世上没有比浑浑噩噩,只重名利的人生更无意义的事了。如果我不是一国之君的话,我也会出家哩!每天吃着要经人尝过的食物,真希望自己也能自由自在地选择自己爱吃的东西而不必怕有人会下毒,就是一次也好,我这种心情,你能了解吧!」

「当然了解,但是大家都出家的话,以后怎么办呢?释迦族会变成什么样子?」

「你说得不错!但正法不能被歪曲,出家的愿望是无法加以压制的。」

净饭王双手抚着头,眼睛瞪着天花板,不一会儿,又闭上眼睛,似乎十分烦恼。

阿姆利德也无法可想,把葡萄一颗颗送入嘴中,试图压抑心中的烦躁。

屋里的气氛十分凝重。二人就这样默默相对,这时候,波闍波提夫人走进来。

「你们有什么事吗?这样愁眉不展的?」

波闍波提打破了屋中的沉寂,并给二人递上热茶。

阿姆利德忙从椅中站起,对波闍波提夫人行了礼,然后再入座。他说:

「啊!没什么。是关于释迦族的问题,特来向大王讨教。夫人,您对大家争相出家一事有何看法?」

「我没有什么特别的意见,也不知大王的看法如何。我认为想出家的就让他们出家,没有什么关系。你只要想想僧团中的生活情形,就知道迦毗罗卫是不会溃亡的,想出家的人,还是不要拦阻他们吧!」

阿姆利德听了波闍波提夫人这番见解,很感意外地瞪大了眼睛。

他想,夫人就是这样令人困惑,对事物的判断全凭直觉,下结论也很简单直率。她对现况有没有通盘的了解?她对国内可能遭遇的困境是如何看法?

这时夫人继续笑着说:

「当初悉达多抛妻别子地跑去出家,但他是经过考虑后,在没有后顾之忧的情况下才出家的。像阿难、阿那律等人,也是在不必负担家业的情况下出家的,所以出家的人虽然增多了,对整个社会而言,是用不着担忧的。」

「那么您对最近边境常闹事的看法是怎样的?如果年轻的战士个个厌战出家了,那国内不就只剩下老弱妇孺了吗?」

「你似乎把修行看成逃避现实了,你要知道,害怕战争的人,一定不能出家的,因为修行的生活比作战还要严酷。想以出家来逃避战争的人,也是无法打仗的,人数再多,如果只是乌合之众,对国家也没有用处,由真正可靠的人来防卫迦毗罗卫也就够了,现在阿那律的哥哥那摩不是到边境去守卫了吗?他比阿那律更想出家,但为了保卫释迦族,他还是留守在家中了。放着该做的事不做,只因为怕打仗而想出家来逃避。这种人留着也没用,你们说是吗?」

阿姆利德无词以对。

真正害怕打仗的人是不可能想到出家的。因为他只要看看佛陀的生活情形也就明白了。破旧的僧衣,粗糙的饭食,简陋的住宿,是何等严酷的生活。

净饭王这时从椅中站起来说:

「夫人说得很好,对于想出家求道的人,我们不应该制止。悉达多成为佛陀之后,将连带使得我释迦族的名声传至永远。佛陀的法将传遍全世界,法灯在人心中永不熄灭,我们应该从旁维护这盏法灯,使它永不熄灭。如果出家的人增加了,战争相对的也会减少了,那不是很好吗?夫人,是不是这样呢?」

「不错,我对王的看法非常同意,以后的事自有以后的发展。」

迦毗罗卫的安全仍受着南邻极大的威胁,但是经过波闍波提夫人的一番剖析,净饭王已不再强把出家问题与此纠结在一起而衍生不必要的烦恼。当然,此刻城内希望出家的人仍不断涌现,削掉头发,往尼拘陀树林求见佛陀的人不绝于途。

有许多武士削发之后来求佛陀收为弟子,经佛陀开示,修行之道不在出家,若能在家克尽了为人的义务与职责,是和出家修行有同等功德的。所以有许多武士很快又回到迦毗罗卫恢复了原来的身份。

净饭王对佛陀的用心非常感激,益发感觉出佛法的浩瀚与博大。

另一方面,佛陀对妇女的要求出家很感为难,站在正法慈悲与平等的立场来看,对真正想求解脱的人,无论其身份如何,都是不能拒绝的。

迦毗罗卫的妇女们得知佛陀的此一心意时,都雀跃不已而争相走告,于是往尼拘陀树林的路上又多出了许多求道心切的妇女。

她们通常是结伴前来,因此尼拘陀树林中的修行场突然间热闹起来。

被人抛弃的商贾妇,或奴隶阶层的子女们都陆续而来。

她们入门后所达的要求并不比男子严格。女子在社会上的地位一向很低微,一般只被认为是男子的附属品罢了。佛陀慈悲,想使她们从旧有的桎梏中获得解脱,因而对她们很宽容,以示尊重。

加之僧团中的修行者在修行过裎中,首要摒除一切阶级意识,力使自己怀持众生平等的观念与作为,所以女子一旦加入了僧团,即享有和男子同等的地位。

当然,她们如果有败坏僧团风气的情事发生,她一样是要遭到逐出僧团之类的惩罚的。

大致说来,她们的修行方式和男众差不多,白天到城里乞食游化,并经常做反省与冥想的工夫,但是她们不能跟男众一样,露宿于荒郊野地中。

在婆罗门的沙罗门中,有结伴修行游化的例子,然而佛陀僧团中的比丘尼多属年轻女子,若散居在荒郊野在中是极不安全的。因此佛陀规定她们住宿于精舍内,若精舍容纳不下,也都是尽量让男众至郊外露宿,以腾出空位。

男众与女众分别在不同的地方修行。比丘尼则由埋托勒呀负责指导。

自从僧团中允许女子的加入后,人数显著地又在急剧增加着。

女子的思绪比男子敏捷,故而在沉思冥想的过程中,女子感受灵力的现象较男子迅速而普遍,她们藉此修行法门多能很快了解到过去世的种种经历,这中间也不乏人被恶灵操纵了,以致迷失了正道。如此一来,埋托勒呀的工作变得较繁重了,因而佛陀的大弟子如迦叶、耶萨和阿舍婆誓等人就来辅助她,为她分担指导的责任。

由于接触的机会增多了,有许多比丘尼就暗暗地喜欢上像耶萨这样皮肤白晳,面容英挺的比丘了。每当耶萨在台上说法时,她们都格外地聚精会神,同时眼睛还闪烁着异样的光芒,情绪煞是激动。因此只要一听说耶萨要说法,比丘尼们纷纷丢下身边的工作,齐集一堂,渴求一瞻耶萨的丰采。

其他的比丘尼们渐渐注意到比丘尼中的这种心理倾向了,因而心神也跟着动摇不已。由于比丘尼心性上多情的波动,直接影响到他们清修的心志。他们憧憬着与异性交往的奇妙情景。而另一方面呢,有些比丘尼虽然倾慕耶萨,但见耶萨志行坚定,毫不为情所动,只好心上带着耶萨的影子,转而求取其他比丘的垂顾。

这就形成一个现象,男女两众每当到镇上乞食时,就有意无意间地聚集在一起。有的人像这样的接触,时日一多,难免生起非份之想。

佛陀非常清楚这一情势的发展,而且这也早就在他的意料之中,这正是他不愿意女子加入僧团的主要原因之一。如今碰到这种局面,他反而不便当着众人的面说破这一点,他也知道当面训诫,将产生不了效果,因男女的情愫在孳长的过程中,是十分盲目而执着的,只有待情动之后的空虚感来临时,他们说不定自己就会了悟个中的道理。

本来僧团中都以男女两众互相回避为行止的原则,因为男女交往会造成修行上的大障碍,如果不加回避,而又时相来往的话,日久就会发生问题。

出家的目的,不只是在建立自我,同时也在将自己悟得的道理加以发挥,以接引其他人来过合于正法的生活,所以出家有着救渡人的神圣使命在焉。凡有悖于此一目的的行为,都在禁止之例。

现在由于女众的加入,使得僧团在一夜之间风气大为转变,这真是出人意表之外的事。

佛陀在说法时,常委婉地就这一点来规诫大众。有些比丘尼听后,常惭愧得无地自容,也有些比丘尼在听的当儿很觉惭愧,但法会结束后,又立刻故态复萌。

当时的社会不像现在,有许多可供人们消遣的游乐活动以调剂生活的紧张。当时娱乐的项目很有限,譬如到郊外采蜂蜜、欣赏角力或弓射比赛、饮酒、赶庙集等。如果想寻求进一步的剌激,就是追逐异性了。所以可说,追逐异性在当时是很容易被人们想到的一种消遣活动。

但是对修行者而言,这类的活动全都必须避免,以免妨碍正道。许多年轻的修行者,基于生理上的需求,往往很难抑制自己。

有一位年轻的僧侣,在听完佛咜的一番规诫后,心情抑郁,无以复加,于是产生了自虐的心理。一日,他忍不住以小刀割切自己的前胸,由于难忍痛楚而在地上来回打着滚,同伴赶来时,他已因流血过多而面色惨白。

虽经医生的抢救,最后仍因失血过多,于第四天在高烧中死去。

自这件事发生后,僧众们受的打击很大,都藉此事的教训惕励自己,一时间僧团的风气好转不少。

但是没有多久,又有了男女僧众互相追逐的风声传出。

佛陀对大众说:

「如果座中有人正因为感情的问题而受着苦恼与动摇的话,不妨还俗吧!要知道,我们这个僧团的存在,在便利大家有个和乐的环境共同修善,既要使自己悟道,将来还要负起责任去传布佛法,使他人悟道。现在因为复杂的感情问题而影响了僧团的声誉,个人也违背了出家修行的目的。

如果有心修行,在哪里都是一样,万不要对出家抱执着的态度。如果出家了,都不好好精进于正道,这罪过是很大的。也不要认为自己既已出家了如今又还俗回家是一件可耻的事,这样反使心胸狭窄了。人各有其所,并不是使有出家修行才会悟道。

当然,在僧团中藉着互相砥砺的力量,是更容易近道而比在家人受惠较多的,但是如果生活在僧团中,却忘了自己是靠许多人的布施而维护修行之道时,那心中的障碍将比在家人为甚,因而也就无法悟道了。

你们今天能齐集一堂共度修行生活,都是靠着生生世世的缘份的,如果忽略了缘份的可贵而沉迷在五官的逸乐中,将堕回生死的道上,不断受转生的痛苦。我们当保持一念之觉,随时改掉不良的习气,否则就没有真正了悟的一天。

正法存在于真正去实践它的人们心中,你们不能只重视外表与形式。一个人活得是否有价值,全在于他心性的活动与外在的言行是不是合于正法。你们应该成为正法的实践者。而且时时当警觉自己是想求法以获得解脱的人。如果不知不觉迷惑于五官的享乐,成为六根的奴隶,虽然身着僧衣,也不能算是一个出家人了。

有的人将问题的产生迁怒于肉体,试图百般虐待自己的肉体以脱离它的纠缠,这这种做法是毫无用处的,因为身体五官的支配者仍在于我们的心灵,此心不端正,任你如何鞭笞,身体是不会自动归正的。你的心,才是问题的核心所在,对肉体执着与否,全凭一念之思。

我们的肉体是以双亲为缘而产生的,它是以适应地球的生存环境而存在着,只要调理适当,它始终可以健康地运作着。使它趋于不健康与痛苦的,是我们的心。

所以使心归至正确方向,是我们必须努力的目标。我们当学会正确地判断事理,正确地表达意见,正确地固守工作岗位,正确地维持生计,正确地向正道精进,正确地执持意念,正确地进入静定状态。这是人生旅途上的八条正道,你们当依道努力去实践之。

八正道的中心即是个人的心,此心与神佛之光紧密相连,我们藉着对八正道的实践,可以显现神佛之光,这就是实践正法的意义所在。佛法本为众生而存在,不在于只求个人的解脱。我们应把法悦与慈悲推及于他人,人人都感受到法的慈光,这人世间的乐土就可以实现了。乐土(佛国土、净土)必须先构筑在每一个人的心上,才可期望有实现的一天。

你们修行时,即是在心上先构筑一片净土,以达到悟的境界。你们的目的在越过生死苦海以达于不生不灭的彼岸,你们目前的生活方式,就是到彼岸的一条该走的道路,万不能比在家人放逸,这是你们必须出家的原因之一。

基于此理,如果无法忍受僧团中清修的生活,就应该考虑还俗,不要妨碍了他人的修道。有想还浴的念头,原是可以谅解的,不要多所顾忌,尽管向我提出来,我绝不会留难你们的,最重要的是,千万不要欺骗自己。」

由于佛陀的这一番开示,许多僧众由于自己的荒谬行为而感到惭愧不已,都低垂双目细细咀嚼佛陀话中的涵义。那些行为不检点的比丘尼们更惊觉到此地是一个比战场还要冷酷的陶冶自我的地方。当初她们为何出家,都因近来的放荡行为而差点忘记了。于是都默默低垂着头,发誓再也不让佛陀操心了。

佛陀是一个懂得人情事故的人。僧众中有破坏风纪的人,但他很少做驱逐出门的惩戒。即使有人还俗回家,也都是自思无法克尽僧职时自动离去的。

(本文完)


\end{document}